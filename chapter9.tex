\chapter{Hypothèse : Réflexions et interrogations}

\section{Evaluation du travail fait en musicothérapie : }

Apprendre à écouter, c'est un travail et des résultats peuvent être
visibles. Nous utilisons un outil qui est le son. Nous accompagnons
le patient d'un point A pour aller au point B : que s'est -il passé
dans son écoute? Nous pouvons apporter des résultats visibles et tangibles
d'une forme d'apprentissage de l'écoute, d'une transformation de la
perception.On se base sur un graphique résultant d'un test de reconnaissance
de sons qui permet de visualiser une transformation psychologique
de l'écoute. 
\begin{itemize}
\item Il y a des résultats : nous pouvons constater soit un changement,
un statisme, un apprentissage,ou un refus d'apprendre et de se transformer.
Ce sont des données qui peuvent servir à mieux comprendre le patient
et à l'accompagner dans son cheminement.
\item Des questions sous-jacentes peuvent émerger comme celles-ci :
\end{itemize}
\begin{enumerate}
\item Quelle est la part d' objectivité ? de subjectivité?
\item S'il n'y a pas de changement visible dans le test , quelles conclusions
peut-on en tirer ? le changement va-t-il toujours de pair avec le
patient? synchronisé ou différencié dans le temps?
\item Est-ce normatif? par cette démarche, il y a le risque de catégoriser
et de paralyser le patient dans son parcours. Mais, cela peut aussi
l'aider dans son travail, son évolution. Ces deux possibilités sont
intrinsèques à tous les tests.
\end{enumerate}
\begin{itemize}
\item Nous sommes confrontés de plus en plus à donner des rapports aux caisse-maladies.
S'il y a une constatation de changement, de progression, le résultat
n'enfermera pas le patient dans une catégorie psychologique, qui,
transmise à celles-ci, pourrait lui être négative pour la poursuite
de son cheminement professionnel, via la vie active. 
\item Est-ce que ce test pourrait être un outil pour les thérapeutes et
les patients ? 
\item Avoir un support réel, visible car graphique pourrait-il être d'une
quelconque utilité pour le patient et pour le thérapeute ?
\item Est-il possible, à partir de deux tests d'écoute, de tirer des hypothèses
sur l'impact du son, de la musicothérapie, du soin par le son, sur
un patient ?
\item Le patient reste au centre de nos préoccupations.
\item Serait-ce un moyen, une façon de démontrer par ce moyen simple (autre
que l'Irmfct) que représente le test d'écoute l'utilité de la musicothérapie
? et ainsi de permettre une plus large acceptation et diffusion de
ce type de thérapie dans plus de milieux hospitaliers ou autres ?
\end{itemize}

Comme l'exprime à juste titre André Malraux : ``\emph{Le monde de
	l'art n'est pas celui de l'immortalité , c'est celui de la métamorphose.''}
De même, la musique est un art produit par l'homme et qui a un impact
sur lui-même. Les deux interagissent, s'interpénètrent et s'auto-transforment
au cours des siècles. Ce que nous pourrons constater lors de l'aboutissement
d'une thérapie n'est pas de trouver une autre personne mais une transformation
de la perception de celle-ci par rapport au monde qui l'entoure. Selon
ce que nous vivons, nous nous transformons mais continuons à être
soi. Nous continuons à ``être soi'' mais autrement. Nous ne perdons
pas notre identité.

\section{La musicothérapie et la méthode Tomatis : }
