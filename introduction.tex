\chapter{Introduction et hypothèse}

\label{jeSuisLaMusique:viret}
\begin{quotation}
\emph{<<\,\emph{Par le Son, le Silence du Non-Être vient à l'Être}. [\dots] 
\textsl{Je suis}
	\emph{la musique que je fais ou écoute}. [\dots]\,>>
[\ldots] \emph{la musique a la capacité d'harmoniser
les composantes d'une entité psychophysique pour qu'il soit ``bien
dans sa peau'' et ``bien dans son âme.}''}\, \autocite[ch. 1,  p. 8]{viret:b}
\end{quotation}





\section{Introduction : Un test pour mesurer les trans\-for\-ma\-tions de l'écoute}

L'écoute est-elle universelle ou diférente selon l'individu?
 Chaque être humain est constitué anatomiquement comme son prochain;
 nos oreilles ont  une anatomie similaire. Néanmoins, comme le sont 
les empreintes digitales, l'écoute et l'approche des sons est unique
et personnelle à chacun. En partant de ce concept, chaque oreille
fonctionne différemment et va donc entendre différemment. On
introduira ensuite le concept de perméabilité au
changement. L'environnement, le temps, les ressentis apportent leur
part au façonnement constant de l'oreille et ainsi de l'écoute. Ces
changements sont quantifiables, mesurables et se retrouvent liés à
certaines expériences de vie. Il nous est possible de visualiser ces
changements via certains tests et de catégoriser certains affects de
l'écoute du patient au fil de la thérapie par le biais de
ceux-ci. Cette quantification nous permet d'évaluer objectivement
l'évolution de l'écoute du patient au cours de la thérapie  mais aussi
de constater la présence ou l'absence de corrélation entre le
traitement et l'évolution psychique du patient.



\section{L'hypothèse}


   Constat: on manque d'outils pour évaluer objectivement les changements
en musicothérapie.

L'objectif de cette étude est :
-de savoir si le test d'écoute Tomatis est un outil valable et utilisable en musicothérapie.
-de savoir s'il permet de déceler une possible corrélation avec l'état
psychique du patient.
-s'il amène un éclairage différent sur les tests psychologiques
utilisés en musicothérapie ainsi que des éléments autres.

 
 \begin{itemize}
 \item  Question:
   Est-ce que l'écoute est visible et quantifiable par
          l'analyse d'un test?
 
 \end{itemize}
 	
 \begin{itemize}
 \item Question:
   Dans l'affirmative, y-a-t-il transformation?
   
\end{itemize}
 
\begin{itemize}
          
 \item Questions:
   Si cette modification existe, possède-t-elle un lien direct avec une prise en charge
  en musicothérapie?
  
  \item Dans l'affirmative, 
   si cette transformation est visible, est-elle en 
 	concordance 
 	avec celle de l'état psychique du patient?
        
 \end{itemize}

 






 

\section{Méthode d'intervention}

	Nous utiliserons deux tests différents : 
	un test d'écoute spécifique quantitatif et qualitatif 
	et un autre test, le WHOQO-Bref qui est qualitatif.

		
        \subsection{Test d'écoute}
        
Avec le test d'écoute spécifique, nous obtiendrons des
         critères d'observations qui donnent la 
	représentation générale des courbes d'écoute (équilibre, déséquilibre, harmonie), ainsi que les 
	chiffres qui représentent les seuils d'écoute selon les
        fréquences et le volume. Il s'en suivra une interprétation
        psychologique selon les zones de fréquences relevées.
	A cet effet, nous utiliserons l'appareil test
        d'écoute conçu à partir de 1950 par Alfred Tomatis, médecin
        O. R. L., choisi car puisant ses sources en audiologie: le Hearing Test,ou TLST, testant
        l'écoute.
	Le test est réalisé en début et en fin de thérapie
        afin de recueillir les résultats des
        deux tests et de procéder par comparaison.
        Nous spécifions qu'aucun support de la méthode conçue par
        Tomatis n'interviendra pendant les séances de musicothérapie.
        Cela n'impliquera ni 
\textsl{Oreille
	électronique} ni musiques préparées et filtrées. Leur
      fonctionnement et leur utilisation seront expliquées lors du
      chapitre sur la méthode, mais nous n'en ferons aucun
      usage. L'objectif est de mettre à profit ce test et de
      se limiter à ce support graphique, tel un ``dessin'',
      une image qui nous fournira des critères d'analyse.
       
	
	\subsection{WHOQO-Bref}
        
   Le WHOQO--Bref un test d'évaluation de la qualité de vie, issu du
	programme de la santé mondiale, l'OMS.
	Ce test est réalisé en parallèle supposée rempli par les patients eux-même  avant et après la thérapie.
	Il s'agit d'une vérification qualitative. Ce test nous permet
        d'avoir l'opinion des patients sur leur processus de travail.
        Il est utilisé pour voir s'il existe un changement au cas où
        il n'y en aurait pas avec le test d'écoute.
	
	
	 
\section{Plan du travail}

Nous aborderons d'abord l'aspect théorique : la musicothérapie, l'écoute, le son, l'oreille, le 
test d'écoute, les différents tests d'écoute en musicothérapie.  Ensuite, nous 
expliquerons  la méthode Tomatis
et puis, beaucoup plus en détails, son test d'écoute.
 
Puis, ce sera l'aspect clinique : les tests d'écoute réalisés  avec deux groupes 
de patients en parallèle, l'un de contrôle et l'autre d'intervention.

Et finalement suivront la vérification de l'hypothèse, les conclusions et 
interrogations. 


	


\chapter{La musicothérapie}

La musicothérapie est une pratique ancestrale qui existe depuis des
temps immémoriaux dans les mythologies et dans les rites. Dans l'Antiquité, que ce soit en Chine ou dans le monde arabe 
médiéval, nous relate François-Xavier Vrait,  	 
les vertus de la musique sont reconnues et considérées comme une évidence depuis 
fort longtemps. 
  \textquote{La théorie des nombres 
(``l'\underline{\underline{harmonie}} peut se calculer") est intégrée et infléchie dans la 
philosophie et les traités musicaux [\dots] et les formes d'utilisation 
thérapeutique de la musique sont décrites dans un ``traité de politique", 
``Kitab as Siyasa" [\dots] document syrien ou sabéen de la fin du 
\textsc{viii}\ieme\ siècle.}\autocite[ch. III, p. 
96]{vrait_musicotherapie_2018}. \footnote{Le terme "harmonie" reviendra lors de l'analyse des tests Ch. 6}
La musique, la matière sonore, était  reconnue en quelque sorte comme étant d' utilité
publique, 
par les  politiciens  et les philosophes dans le but d'atteindre un
équilibre personnel et une harmonie civique.
 	 
A juste titre, l'Association Suisse de Musicothérapie suisse met particulièrement en relief 
la communication dans sa définition. Elle l'a décrit comme un 
 
\begin{itemize}
\item \textquote{processus thérapeutique, pour entrer en communication
    avec soi-même et pouvoir ensuite mieux percevoir le monde qui nous
    entoure, communiquer et s'exprimer.\autocite{site_musitherapy}}
\end{itemize}

La communication avec soi-même et avec l'autre est en effet un des
points 
d'une extrême importance dans l'équilibre psychique de l'homme, reflet
de 
l'harmonie entre l'intérieur et l'extérieur. L'harmonie est un des  
critères révélateurs qu'il nous sera donné de visualiser par le test
d'écoute ( dont l'équilibre entre l'écoute aérienne et osseuse) et sur
lesquels 
nous appuyerons notre hypothèse  
dans ce travail.


 Au fil des siècles, de nombreux autres 
textes évoquent les liens de la musique avec la médecine, de sa place dans les 
rituels thérapeutiques et notamment en psychothérapie fin \textsc{xix}\ieme, 
début \textsc{xx}\ieme\ siècle. \footnote{Voir \ref{musicothEtpsycho},
  p. \pageref{musicothEtpsycho}.}.
Selon Aurelia Sickert-Delin, la musicothérapie 
psychologique doit être différenciée de celle dite médicinale qui 
\enquote{\emph{exerce une action 
énergétique, physiologique}} [\dots] avec \enquote{\emph{des effets curatifs}}  
ainsi que de celle dite \enquote{\emph{musicale, artistique}}. 
% \enquote{\emph{L'artiste-musicien éveille l'\,``artiste intérieur'' que l'être 
%en souffrance porte en lui, pour lui permettre de s'auto-guérir [\dots] par 
%l'écoute, l'expression et la création.}}\autocite[ch. 1,  p. 14, du texte 
%inédit communiqué par A. Sickert-Delin, musicothérapeute à Alersheim, rapporté à J. 
%Viret]{viret:b}

 De fonctionnelle, analytique, mo\-da\-le,  à 
struc\-tu\-rale, la musicothérapie se retrouve actuellement 
 à un tournant décisif où elle devient 
 \emph{intégrative} tout en conservant ses racines séculaires. Elle est 
intégrative dans le sens où elle permet un travail d'élaboration psychique dans une perspective de structuration identitaire \autocite[ch. III, p. 53, 
105]{vrait_musicotherapie_2018} et dans celui de l'intégration des données 
neuroscientifiques.
 
% Selon M. 
%Schneider \blockquote{elle cherche à sauvegarder et à fortifier la pure 
%substance sonore de l'homme\autocite[Voir tome I, pp. 202--203]%
%	[M. Schneider, <<Le rôle  de la musique dans la mythologie et les rites 
% des civilisations non européennes>>]{schaeffner.ea:histoire}.}


 
 
Les musicothérapeutes sont souvent musiciens mais conjugent plus
rarement dans leur profession la
médecine et la science. De leur côté, les neuroscientifiques appuient
et renforcent la crédibilité de l'action du son sur notre cerveau, via
l'oreille, en démontrant ses effets par un moyen technique
\textbf{visuel} que représente l'IRMfct. Mais sans être musicien ou
musicothérapeute, leur découverte est plus rarement intégrée
directement dans leur pratique avec l'aspect intuitif de cette prise
en charge et le contexte
relationnel. Il est ardu d'être très compétent dans ces deux domaines
de pointe à
la fois.

Nous pouvons nous imaginer que l'avenir apportera d'autres outils
directement accessibles telle une version différente d'IRMfct\footnote{Imagerie par Résonance Magnétique
   Fonctionnelle} afin
 de, si nécessaire, visualiser directement l'effet en temps réel de la musique sur le
 cerveau:  imager couramment et facilement la manière d'ouïr de chaque
 patient dans le but de mieux le connaître et de 
 prendre les décisions adéquates en musicothérapie. (l'image de la
 physiologie de l'audition personnalisée à chaque séance!). Pourquoi
 pas?  Mais faut-il encore avoir 
 les compétences d'analyse de la lecture d'un tel appareil.
 
 Ainsi, revenons à la réalité: il existe des moyens simples tel l'enregistrement:
 
 La pratique de l'enregistrement sonore des séances 
permet d'avoir un support concret et solide, comme le pratiquent Edith Lecourt, ou Carole et Clive 
Robbins. \footnote{``Les Art-thérapies'', pp88--117, Ed.Armand Colin}
Les séances sont écoutées, filmées pour une
analyse la plus objective possible; on donne 
une forme et un sens aux sons recueillis pour les retracer dans le
parcours du patient (avec ou sans sa présence). Cette analyse  restera
tout de même très subjective de par sa nature ( le son ne laisse pas
de trace) et de par la nécessité de la présence d'autres formes d' écoutes(
celles du thérapeute).


Car l'aspect fugace du son, de la musique, de ce médium volatil et
intemporel par
définition, n'amène pas le
même aspect concret que peuvent témoigner des supports
graphiques. Ceux-ci sont des 
reflets d'un espace-temps du travail d'élaboration
psychique d'un patient, sur lesquels l'art-thérapie s'appuie et trouve
ses sources. 
Vivant dans un monde très visuel, les preuves sous cette forme sont
validées pour soutenir l'argumentation du bien-fondé d'une thérapie. \footnote{
	\pdfcomment{Faire une phrase}les critères de l'EBM (evidence based medecine, médecine basée sur des 
        preuves) F. X. Vrait, ch. II, pp. 105--106 }.

     On veut voir pour croire. Est-ce 
notre esprit formaté cartésien depuis quelques centaines d'années qui nous 
empêche de penser différemment? 
Actuellement, c'est une nécessité due à notre époque pour crédibiliser l'impact 
du son sur notre être. 

Par conséquent, notre hypothèse est que  la 
musicothérapie a un impact certain sur la façon d'écouter en la
modifiant et qu'elle peut perdurer après une thérapie; qu'elle est 
démontrée et démontrable \textsl{objectivée}, sous la forme d'un test, comme saisie par 
l'\oe il neutre de l'objectif d'un appareil
photographique.


