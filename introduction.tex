\chapter{Introduction}



\begin{quotation}
 \textit{\textbf{Epigraphe  }   ``La musique vient dans la chair comme un produit immatériel
 qui vient travailler la zone a soigner. Je pompe de la
 guérison.
Depuis le début des écoutes j'ai la sensation physique et
 psychique de la
 transformation.
 La musique est équilibrante et guérisseuse, ma zone
 anesthésiée se remet à vivre, elle est remise en activité.
 Il y a comme un consentement cellulaire.
La béance s'estompe, cette
partie redevient comme les autres. (...)
Apaisement. Consentement. Réconciliation.''}
% (...). [\dots]\, >>

\end{quotation}

Nous avons été très sensible au témoignage de cette patiente dont le
processus a été porté par le son et l'écoute. Le son l'a touchée intégralement et a eu un impact sur sa transformation physique
et psychique.
Avec l'image évoquée du
consentement cellulaire, nous pourrions faire le
parallélisme entre l'entrée des sons dans la sphère d'écoute et la variation de la 
perméabilité cellulaire en cytologie.\autocite[ch. 3 pp. 70--76]{marieb:biologie}  Si les sons réussissent à pénétrer dans la
cellule psychique du patient, il peut y avoir amélioration des
échanges, une 
communication, une forme d'homéostasie,  \autocite[ch. 1
pp. 10]{marieb:biologie} qui reflète un état d'équilibre dynamique.

Mais comment détecter la façon d'entendre du patient?
Comment comprendre les raisons pour lesquelles il y a imperméabilité aux
échanges, un refus des sons et une fermeture au monde si ce n'est
peut-être  en testant
son écoute?  Son écoute pourrait-elle nous donner certaines clés dans sa
compréhension? C'est une hypothèse.
D'autre part, serait-ce  possible  par un test d'écoute
de marquer et souligner l'importance du processus musicothérapeutique? 
Il jouerait un
rôle de
révélateur de cette notion si abstraite et si primordiale qu'est
l'écoute dans ce domaine.
Ce sont les questions auxquelles nous allons tenter de
répondre dans ce travail.


Être au diapason, en harmonie avec soi et les autres
nécessite une écoute afin de nous accorder ou réaccorder à l'univers.
Car en définitive, comme le dit si poétiquement David Elbaz, nous sommes tous les
descendants de la cristallisation de la musique primordiale de
l'univers. \autocite{delbaz_recherche_2016} \footnote{David Elbaz, astrophysicien, chef de laboratoire au CEA et Alain
Destexhe, chercheur en neurosciences intégratives et computationnelles
à l'Institut  NeuroPsi de Paris Saclay} 

 



Si la musicothérapie a pour objectif d'apporter un soin aux patients, notre  approche se différencie par le fait de
vouloir mieux la mettre en valeur par des preuves
autres que par exemple des témoignages de patients. Fort nous a été de
constater un manque d'outils pour son évaluation. Car quelle que soit la technique utilisée, quel que soit
le traitement sonore, on espère une modification, on la suppose, la constate
 mais
 on ne la quantifie que difficilement. C'est la raison pour laquelle
 nous nous sommes servis d'un test d'écoute
 spécifique de la méthode d'Alfred Tomatis, choisi car puisant ses
 sources en audiologie.
 Ce test particulier  nous servira à
 souligner l'importance de la musicothérapie 
 sur la transformation de l'écoute.
 Car ce n'est qu'après plusieurs
années de pratique et d'expérience que nous avons commencé à saisir
l'essentiel de la validité de ces théories.
\footnote{Nous sommes tout à fait conscients des
 grandes divergences d'opinion entre les adeptes d'une musicothérapie
 traditionnelle et cette méthode, bien que le lien qui les unit est
 la musique, mais ce ne sera pas l'objet de notre travail.}

Quoique nous n'ayons pas pu réunir toutes les données nécessaires
aux tests réalisés, - car nous ne sommes pas dupes qu'un vrai travail
nécessite beaucoup plus de précisions-, il nous a été possible toutefois d' étayer
les résultats obtenus, de recueillir quelques considérations hypothétiques par rapport à un
travail thérapeutique et de nous ouvrir à des réflexions. 
Grâce à Sandra Lutz Hochreutener,  \footnote{Dr. Sandra Lutz
  Hochreutener. Lehrt Musiktherapie und in der Weiterbildung – Tätig
  im Departement Musik. Funktion Co-Leitung und Dozentin Bereich
  Dossier, ZhDK}
 nous avons été encouragés à toutes les énoncer, pour pouvoir mettre un jour un terme à ce travail!


 

  


\section {L'hypothèse}


   La    multiplicité des suppositions ayant trait au constat d'un manque
   d'outils d'évaluation objectif des résultats issus de la
   musicothérapie  nous a  permis de relever les questions
   suivantes:  



 
 \begin{itemize}
 \item  Question:
   Est-ce que l'écoute est visible et quantifiable par
          l'analyse d'un test?
 
 \item Question:
   Dans l'affirmative, y-a-t-il transformation?
   
 \item Questions:
   Si cette modification existe, possède-t-elle un lien direct avec une prise en charge
  en musicothérapie?
  
  \item Dans l'affirmative, 
   si cette transformation est visible, est-elle en 
 	concordance 
 	avec celle de l'état psychique du patient?
        
 \end{itemize}
 

\section{Méthode d'étude}

	Nous utiliserons deux tests différents : 
	le test d'écoute spécifique d'Alfred Tomatis
	et le test-questionnaire, le WHOQO-Bref; tous les deux sont
        qualitatifs et quantitatifs.

        
        Le test d'écoute détecte la manière de recevoir
        l'information. Nous obtenons une  
	représentation graphique générale des courbes de l'écoute
        (équilibre, déséquilibre, harmonie) à partir des seuils d'écoute
        calculés selon les fréquences et le volume que le sujet entend. Il s'en suivra une interprétation
        selon un procédé de lecture des zones de fréquences relevées.
	A cet effet, nous utiliserons l'appareil conçu à partir de 1950 par Alfred Tomatis, médecin
        O. R. L.: le Hearing Test,ou TLST, testant
        l'écoute.
	Nous procéderons en début et en fin de thérapie
        afin de recueillir les résultats des
        deux tests dans le but d'établir une comparaison.
        Nous spécifions qu'aucun support de la méthode conçue par
        Tomatis n'interviendra pendant les séances de musicothérapie.
        Cela n'impliquera ni 
\textsl{Oreille
	électronique} ni musiques préparées et filtrées. Leur
      fonctionnement et leur utilisation seront expliquées lors du
      chapitre sur la méthode, mais nous n'en ferons aucun
      usage. L'objectif est de mettre à profit cette façon de  tester pour constater
      s'il existe un changement dans l'écoute du sujet et de
      se limiter à ce support graphique, tel un ``dessin'',
      une image qui nous fournira des critères d'analyse.
       
	
	
        
   Le WHOQO--Bref ( World Health
   Organisation Quality of Life Assessement). Il s'agit d'un test d'évaluation de la qualité de vie, issu du
	programme de l'Organisation Mondiale de la Santé, l'OMS.
	Ce questionnaire est réalisé en parallèle supposée, rempli par
        les patients eux-même  avant et après la thérapie, qu'elle soit
        musicothérapeutique ou lors d'un suivi classique en
        clinique psychiatrique.
	Il s'agit d'une vérification qualitative qui nous permet
        d'avoir l'opinion des patients sur leur processus de travail
        et  pour constater s'il y a une correspondance dans 
        les résultats obtenus avec le test d'écoute. 
	
	
	 


\section*{Plan du travail}


Nous aborderons en première partie l'aspect théorique : la musicothérapie, l'écoute, le son, l'oreille, le 
test d'écoute, les différents tests d'écoute en musicothérapie.  Ensuite, nous 
exposerons le test d'écoute Tomatis avec un bref aperçu de sa méthode.

 
En deuxième partie,ce sera l'aspect clinique : les tests d'écoute réalisés  avec deux groupes 
de patients en parallèle, l'un de contrôle (GC) et l'autre d'intervention.(GE)

Et finalement suivront la vérification de l'hypothèse, les conclusions et 
interrogations. 





\section{Prémisses: communication, harmonie, dépression,
  musicothérapie}





La \textbf{communication}, l'\textbf{harmonie} et la \textbf{ dépression}
sont des concepts primordiaux en
musicothérapie.


La \textbf{communication} se définit, avec `cum
municare' signifiant ``mettre en commun'', `partager'. Selon le Petit
Robert 1, 1995, c'est le fait d'établir une
relation, un lien, un rapport, un échange avec quelqu'un. Passage ou échange de message entre un sujet émetteur et un
sujet récepteur au moyen de signes, codes. 
 L'Association Suisse de Musicothérapie retient l'idée d'un 
\begin{itemize}
\item \textquote{processus thérapeutique, pour entrer en communication
    avec soi-même et l'autre afin de mieux percevoir le monde (...).\autocite{site_musitherapy}}
\end {itemize}
Encore selon le Petit Robert, le mot \textbf{harmonie} comporte  ( étymol.:
 <gr.='assemblage'>);
 des sons assemblés, combinaison, ensemble de sons perçus de manière agréable, accord.
 Dans la mythologie grecque, \textbf\textit{{Harmonie}}  était l'épouse de Cadmos,
 introducteur de l'alphabet, et elle-même était une nymphe
 douce et éprise de paix, fille d'Arès et d'Aphrodite. 
L'harmonie avec soi-même et avec l'autre est synonyme de l'équilibre
psychique. C' est ainsi que par le test d'écoute de Tomatis, on peut
relever les deux formes de perception utilisable dans notre travail,
notifiant au chapitre 4 un concept plus convenable à l'analyse
comparative des résultats individuels.




Le terme polysémique de  \textbf{`dépression'} nécessite quelques
éclaircissements sur ses significations dans des contextes différents;
dans le langage courant,\autocite[Petit
Robert 1, p. 10, 1990] on entend 1. un `abaissement ou enfoncement' produit par une pression de
haut en bas ou par toutes autres causes; par extension, enfoncement,
concavité, creux (\textit{phys. et géogr.}), 2. un terme équivalent en
météorologie désigne un abaissement barométrique (baros=pression);
3. dans le domaine économique, on entend le fléchissement de la
consommation, la chute des cours avec dépréciation des marchandises et
ralentissement des affaires (crise, récession) 4. en psychopathologie, on l'assimile à des signes de lassitude, de
découragement, de faiblesse, de l'anxiété, dont les synonymes sont
`asthénie', `mélancolie' , `la déprime'.
Dans la définition
psychologique citée dans le \autocite {Doron et
Parot, 2019}, sans évoquer en détail les différentes acceptations, on
retient la dépression comme étant une protection du système psychique.
En psychanalyse, il peut être comparé à un phénomène d'agressivité
inconsciente forte retournée contre soi-même et qui met dans un état
de souffrance.

Cela étant, la définition de cette pathologie n'est pas l'object de
notre travail mais plutôt celui de 
constater si, dans l'écoute, certaines
caractéristiques de la dépression peuvent être améliorées par la
restructuration de la sensibilité perceptuelle.

\chapter{Aspects musicothérapeutiques et éléments théoriques}

Les vertues de la musique sont reconnues depuis une pratique
ancestrale aussi 
dans les mythologies et dans les rites (Chine, et dans le monde arabe
médiéval). 
Les formes d'utilisation 
thérapeutique de la musique figurent même dans un ``traité de politique",
``Kitab as Syasa'' remontant à des documents syriens ou saabéens datant de  [\dots] de la fin du 
\textsc{viii}\ieme\ siècle.  \textquote{La théorie des nombres 
permettait de calculer l'harmonie} intégrable dans la 
philosophie et les traités musicaux. \autocite[ch. III, p. 
96]{vrait_musicotherapie_2018}.
En définitive, la reconnaissance par les  politiciens  et les
philosophes de la \textit{matière sonore}, comme d' utilité
publique laisse entendre que l'équilibre personnel
peut contribuer à une forme d' ``harmonie civique''.

 

\section{Parcours musicothérapeutique}


 Au fil des siècles, de nombreux autres 
textes évoquent les liens de la musique avec la médecine, de sa place dans les 
rituels thérapeutiques et notamment en psychothérapie fin \textsc{xix}\ieme, 
début \textsc{xx}\ieme\ siècle.\footnote{Voir \ref{musicothEtpsycho},
  p. \pageref{musicothEtpsycho}.}
Selon Aurelia Sickert-Delin, la musicothérapie 
psychologique doit être différenciée de celle dite médicinale qui 
\enquote{\emph{exerce une action 
énergétique, physiologique}} [\dots] avec \enquote{\emph{des effets curatifs}}  
ainsi que de celle dite \enquote{\emph{musicale, artistique}}. 
% \enquote{\emph{L'artiste-musicien éveille l'\,``artiste intérieur'' que l'être 
%en souffrance porte en lui, pour lui permettre de s'auto-guérir [\dots] par 
%l'écoute, l'expression et la création.}}\autocite[ch. 1,  p. 14, du texte 
%inédit communiqué par A. Sickert-Delin, musicothérapeute à Alersheim, rapporté à J. 
%Viret]{viret:b}

 Ainsi, de fonctionnelle, analytique, mo\-da\-le,  à 
struc\-tu\-rale, la musicothérapie se retrouve actuellement 
 à un tournant décisif où elle devient 
 \emph{intégrative} tout en conservant ses racines séculaires. Elle est 
intégrative dans le sens où elle permet un travail d'élaboration psychique dans une perspective de structuration identitaire \autocite[ch. III, p. 53, 
105]{vrait_musicotherapie_2018} et dans celui de l'intégration des données 
neuroscientifiques.
 
% Selon M. 
%Schneider \blockquote{elle cherche à sauvegarder et à fortifier la pure 
%substance sonore de l'homme\autocite[Voir tome I, pp. 202--203]%
%	[M. Schneider, <<Le rôle  de la musique dans la mythologie et les rites 
% des civilisations non européennes>>]{schaeffner.ea:histoire}.}


 
\section{Crédibilité actuelle de l'approche de la musicothérapie }


L'alliage de la musicothérapie avec les données actuelles de pointe en
science est-il faisable? Comment le réaliser pour obtenir plus de pertinence
dans sa crédibilité?

 
Les musicothérapeutes sont souvent musiciens mais conjugent plus
rarement dans leur profession
médecine ou neurosciences. De leur côté, les neuroscientifiques appuient
et renforcent la crédibilité de l'action du son sur notre cerveau, via
l'oreille, en démontrant ses effets par un moyen technique
\textbf{visuel} que représente par exemple l'IRMfct. Mais, sans être musicien ou
musicothérapeute, leur découverte est plus rarement intégrée
directement dans leur pratique car hors contexte relationel d'une
séance, sans l'aspect intuitif et impalpable de cette forme de prise
en charge.
Il est malheureusement ardu d'être très compétent dans ces deux domaines
de pointe de façon simultanée.   \footnote  {Il existe toutefois certains chercheurs
comme Emmanuel Bigand ou Hervé Platel qui font exceptions.}

Car l'aspect fugace du son, de la musique, de ce médium volatil et
intemporel par
définition, n'amène pas à tout un chacun le
même aspect concret que peuvent témoigner des supports
graphiques. Ceux-ci sont des 
reflets d'un espace-temps du travail d'élaboration
psychique d'un patient, sur lesquels l'art-thérapie, par exemple, s'appuie et trouve
ses sources.
Néanmoins, il y a l'enregistrement sonore des séances qui 
permet peut-être d'avoir un support plus concret et solide, comme le pratiquent Edith Lecourt, ou Carole et Clive 
Robbins. \footnote{``Les Art-thérapies'', pp88--117, Ed.Armand Colin}
Les séances sont écoutées, filmées pour une
analyse la plus objective possible; on donne 
une forme et un sens aux sons recueillis pour les retracer dans le
parcours du patient (avec ou sans sa présence). Par contre, cette analyse  restera
tout de même très subjective de par sa nature (le son ne laisse pas
de trace) et de par la nécessité de la présence d'autres formes d'
écoutes (celles du thérapeute).

Peut-être pouvons-nous nous imaginer que la technologie future apportera d'autres outils
directement accessibles pendant les séances, afin
 de, si nécessaire, visualiser directement l'effet en temps réel de la musique sur le
 cerveau:  imager couramment et facilement la manière d'ouïr de chaque
 patient peut se révéler important, complémentaire et intéressant pour
 l'anamnèse et la prise en charge. Ce serait l'image de la
 physiologie de l'audition personnalisée à chaque séance!

 
Vivant dans un monde très visuel, les preuves sous cette forme sont
validées pour soutenir l'argumentation du bien-fondé d'une thérapie.\footnote{
	\pdfcomment{Faire une phrase}les critères de l'EBM (evidence based medecine, médecine basée sur des 
        preuves) F. X. Vrait, ch. II, pp. 105--106 }
On veut voir pour croire. Est-ce 
notre esprit formaté cartésien depuis quelques centaines d'années qui nous 
empêche de penser différemment? 
Actuellement, c'est une nécessité due à notre époque pour crédibiliser l'impact 
du son sur notre être.


Si la 
musicothérapie a un impact certain sur la façon d'écouter en
entraînant sa 
modification, peut-elle être  démontrée et démontrable, \textsl{objectivée},
simplement sous la forme d'un test, comme saisie par 
l'\oe il neutre de l'objectif d'un appareil
photographique?

Cette hypothèse formulée va être donc l'objet de notre étude qui
débutera par la notion d'écoute que nous allons revisiter.
