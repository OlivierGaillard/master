\chapter{Introduction et hypothèse}
\begin{quotation}
\emph{<<\,Par le Son, le Silence du Non-Être vient à l'Être. [\ldots] Je suis
la musique que je fais ou écoute. (...)a\,>> La musique a la capacité d'harmoniser
les composantes d'une entité psychophysique pour qu'il soit ``bien
dans sa peau'' et ``bien dans son âme.''}\, \autocite[ch. 1,  p. 8]{viret:b}
\end{quotation}

\section{Introduction}

L'auteur exerce le métier de musicienne professionnelle et a  choisi de compléter sa formation en devenant musicothérapeute  ainsi que consultante à la méthode Tomatis.  



Elle exerce en cabinet privé et en cliniques psychiatriques. L'utilisation des différents outils de travail varie et s'adapte selon les lieux. En institution, elle est exclusivement musicothérapeute. En cabinet
privé, elle a la liberté d'utiliser l'une des deux techniques ou de les fusionner. Celles-ci se sont modifiées grâce à  l'expérience acquise, ont évolué  
à travers chaque cas particulier.
 Se lancer dans cette écriture, c'est creuser et approfondir ses propres recherches. C'est prendre de la distance, observer, se remettre en question, être plus précis dans ses observations, plus systématique dans la manière de travailler. D'où ce questionnement présenté ici : 


\section{Hypothèse: un test pour mesurer les transformations de l'écoute}

L'écoute est-elle universelle ou personnelle ?
 Chaque être humain semble  globalement pareil à un autre; toutes les oreilles paraissent  avoir une anatomie similaire. Et pourtant, si l'on observe de près nos empreintes digitales, on s'aperçoit de l'unicité de chaque être. De même, chaque oreille peut être différente, différenciée et  fonctionner d'une autre façon;  par conséquent notre  écoute pourrait être à chaque fois particulière nous appartenir de façon singulière tout en étant perméable au changement.
Le temps imprime une évolution dans notre être intérieur et extérieur, nous sommes vivants dans nos mouvements psychologiques et physiques. Des traces devraient en être aussi visibles dans l' écoute. Il est plausible de s'en poser la question.
En admettant ainsi qu'il n'y ait pas de statisme mais au contraire une transformation dans l'écoute, ce changement devient identifiable, \textit{remarquable}, unique (dans le sens personnel) à travers nos expériences de vie.  Un test pourrait être un moyen d'évaluer les modifications supposées de cette écoute. 
Donner une forme à l'écoute, une image, une   \emph{vision de l'écoute}, telle une fenêtre qu'il nous est possible d'ouvrir, peut être une source d'informations pour  objectiver et évaluer d'une part 
les changements, la transformation de l'écoute, en début et en fin de thérapie, et d'autre part pour constater s'il y a un parallélisme dans la transformation psychique  de la personne prise en charge.

L'objet de cette étude est de faire un constat
qui pourrait donner matière à réflexion.


\subsection{L'hypothèse}

Qu'est-ce qui est proposé ? 
	
	Nous utiliserons deux tests différents : 
	un test d'écoute spécifique que l'on peut considérer comme quantitatif, 
	et un autre test, le WOQUO-Bref qui est qualitatif.
	
Objectif : constat de l'évolution de l'écoute à travers le travail fait en musicothérapie.

A cet effet, nous utiliserons un test spécifique à Tomatis, qui se fera dans le cadre d'une musicothérapie. Aucun support de cette méthode n'interviendra, ni \textsl{Oreille
	électronique} ni musiques préparées et filtrées. Nous expliquerons la raison de   leur importance mais nous n'en ferons aucun usage. L'idée est de mettre à profit cette forme de test et de  s'en tenir à ce support
graphique, visible, un``dessin'', une image utile, utilisable, tangible,
presque ``palpable'', avec des critères
d'interprétations.
	



 



\subsubsection{La musicothérapie}
La musicothérapie est une pratique ancestrale. Il suffit de considérer la très grande  place de la musique dans les mythologies et dans les rites.  Selon M. Schneider
\begin{quote}
	elle cherche à sauvegarder et à fortifier la pure substance sonore de l'homme\autocite[Voir tome I, pp. 202--203]%
	[M. Schneider, <<\,Le rôle  de la musique dans la mythologie et les rites des civilisations non européennes\,>>]{schaeffner.ea:histoire}
\end{quote}

La musique, la matière sonore, est utilisée dans le but d'un 
\begin{quote}
	processus thérapeutique, pour entrer en communication avec soi-même et pouvoir ensuite mieux percevoir le monde qui nous
	entoure, communiquer et s'exprimer\footnote{%
		\href{http://www.musictherapy.ch/fr/musicotherapie/quest-ce-que-la-musicotherapie/}{musictherapy.ch}}.
\end{quote}

 Selon Aurelia Sickert-Delin, on peut aussi différencier la musicothérapie dite médicinale qui \emph{"exerce une action énergétique, physiologique" }(...) avec \emph{"des effets curatifs"}  ainsi que celle dite\emph{"musicale, artistique"}. 
 	 \emph{"L'artiste-musicien éveille l'"artiste intérieur" que l'être en souffrance porte en lui, pour lui permettre de s'auto-guérir(...) par l'écoute,l'expression et la création."}
 	 
 	 
Les vertus de la musique sont reconnues et considérés comme une évidence depuis fort longtemps dans l'Antiquité, que ce soit en Chine ou dans le monde arabe médiéval, nous relate François-Xavier Vrait.
 \footnote{François-Xavier Vrait, "\textit{La musicothérapie"} Ed.Que sais-je? 2018} 
 
  
L'aspect fugace du son, de la musique, de ce médium volatil par
définition, ne semble pas apporter, comme en art-thérapie, le
même aspect concret que peuvent témoigner des supports graphiques,
visuels, reflets d'un espace-temps lors d'un travail d'élaboration
psychique d'un patient. 
Si nous ne sommes pas formés en tant que médecins et musicothérapeutes ni issus d'un milieu scientifique, nous ne pouvons pas prétendre non plus à l'utilisation des outils
 tel que l'IRMfct\footnote{Imagerie par Résonance Magnétique Fonctionnelle}; en contre-partie, toutes ces recherches en
neurosciences éclairent, appuient et renforcent la crédibilité de l'action
majeure du son sur notre cerveau, via l'oreille, en démontrant ses effets par cet intermédiaire technique visuel. Nous faisons ici l'hypothèse que l'action de la musicothérapie a un impact certain sur la façon d'entendre et  pourraient être perçus, d'une certaine façon plus
clairement ou plus simplement, \textsl{objectivés}, sous la forme d'un test, comme saisis par l'\oe il de l'objectif d'un appareil
photographique.


Démontrer, prouver.
Serions-nous comme St Thomas qui voulait voir pour croire ou est-ce notre esprit formaté cartésien depuis quelques centaines d'années qui nous empêche de penser différemment? Ce peut être aussi une nécessaire complémentarité due notre époque scientifique pour crédibiliser l'impact du son sur notre être. 


\section{Plan du travail}

Dans la première partie, nous aborderons l'aspect théorique : l'écoute, le son, l'oreille, le test d'écoute, les différents tests d'écoute en musicothérapie.  Ensuite, nous expliquerons  la méthode Tomatis
et puis, beaucoup plus en détail, son test d'écoute.

En deuxième partie ce sera l'aspect clinique : les tests d'écoute réalisés  avec deux groupes de patients en parallèle.

Et finalement suivront la vérification de l'hypothèse, les conclusions et interrogations. 
Quelques réflexions autour notre propre expérience en cabinet privé, étayé d'un cas d'étude.
