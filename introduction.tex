\chapter{Introduction et hypothèse}

\label{jeSuisLaMusique:viret}
\begin{quotation}
\emph{<<\,\emph{Par le Son, le Silence du Non-Être vient à l'Être}. [\dots] 
\textsl{Je suis}
	\emph{la musique que je fais ou écoute}. [\dots]\,>>
[\ldots] \emph{la musique a la capacité d'harmoniser
les composantes d'une entité psychophysique pour qu'il soit ``bien
dans sa peau'' et ``bien dans son âme.}''}\, \autocite[ch. 1,  p. 8]{viret:b}
\end{quotation}







\section{Introduction : Un test pour mesurer les trans\-for\-ma\-tions de l'écoute}

L'écoute est-elle universelle ou diférente selon l'individu?
 Chaque être humain est constitué anatomiquement comme son prochain;
 nos oreilles ont  une anatomie similaire. Néanmoins, comme le sont 
les empreintes digitales, l'écoute et l'approche des sons est unique
et personnelle à chacun. En partant de ce concept, chaque oreille
fonctionne différemment et va donc entendre différemment. On
introduira ensuite le concept de perméabilité au
changement. L'environnement, le temps, les ressentis apportent leur
part au façonnement constant de l'oreille et ainsi de l'écoute. Ces
changements sont quantifiables, mesurables et se retrouvent liés à
certaines expériences de vie. Il nous est possible de visualiser ces
changements via certains tests et de catégoriser certains affects de
l'écoute du patient au fil de la thérapie par le biais de
ceux-ci. Cette quantification nous permet d'évaluer objectivement
l'évolution de l'écoute du patient au cours de la thérapie  mais aussi
de constater la présence ou l'absence de corrélation entre le
traitement et l'évolution psychique du patient
Si nous faisons l'hypothèse que notre écoute n'est pas fixe, figée mais au contraire qu'elle 
peut subir une 


L'objectif de cette étude est d'effectuer  un constat de l'évolution de l'écoute au travers du travail fait en 
musicothérapie et de déceler une possible corrélation entre ces
évolutions et l'état psychique du patient.


\section{L'hypothèse}



\begin{enumerate}
\item Est-ce  qu'un test d'écoute permet-il de mesurer  la capacité 
  d'écoute?
  

 
\item Existe-t-il une modification de l'écoute?

 
  
\item Si oui, cette modification possède-t-elle un lien avec une prise en charge
  en musicothérapie?
  
\end{enumerate}






 

\section{Méthode d'intervention}

	Nous utiliserons deux tests différents : 
	un test d'écoute spécifique quantitatif et qualitatif 
	et un autre test, le WHOQO-Bref qui est qualitatif.

		
        \subsection{Test d'écoute}
        
Avec le test d'écoute spécifique, nous obtiendrons des
         critères d'observations qui donnent la 
	représentation générale des courbes d'écoute (équilibre, déséquilibre, harmonie), ainsi que les 
	chiffres qui représentent les seuils d'écoute selon les fréquences et le volume. 
	A cet effet, nous utiliserons l'appareil test
        d'écoute conçu à partir de 1950 par Alfred Tomatis, médecin
        O. R. L., choisi car puisant ses sources en audiologie: le Hearing Test,ou TLST, testant
        l'écoute.
	Le test est réalisé en début et en fin de thérapie
        en nous permettant de recueillir les résultats comparatifs des
        deux tests et de procéder par comparaison.
        Nous spécifions qu'aucun support de la méthode conçue par
        Tomatis n'interviendra pendant les séances de musicothérapie.
        Cela veut dire, ni 
\textsl{Oreille
	électronique} ni musiques préparées et filtrées. Nous
      expliquerons leur fonctionnement et leur utilisation lors de
      l'explication de la méthode, mais nous n'en ferons aucun
      usage. L'idée est de mettre à profit cette forme de test et de
      s'en tenir à ce support graphique, visible, comme un ``dessin'',
      une image apportant des critères d'interprétations.
       
	
	\subsection{WHOQO-Bref}
        
   Le WHOQO--Bref un test d'évaluation de la qualité de vie, issu du
	programme de la santé mondiale, l'OMS.
	Ce test est réalisé en parallèle supposée rempli par les patients eux-même  avant et après la thérapie.
	Il s'agit d'une vérification qualitative. Ce test nous permet d'avoir l'opinion des patients sur leur processus de travail.
	
	
	 
\section{Plan du travail}

Nous aborderons d'abord l'aspect théorique : la musicothérapie, l'écoute, le son, l'oreille, le 
test d'écoute, les différents tests d'écoute en musicothérapie.  Ensuite, nous 
expliquerons  la méthode Tomatis
et puis, beaucoup plus en détails, son test d'écoute.

Puis ce sera l'aspect clinique : les tests d'écoute réalisés  avec deux groupes 
de patients en parallèle.

Et finalement suivront la vérification de l'hypothèse, les conclusions et 
interrogations. 


	


\chapter{La musicothérapie}

La musicothérapie est une pratique ancestrale. Il suffit de considérer la très 
grande  place de la musique dans les mythologies et dans les rites. 
% Selon M. 
%Schneider \blockquote{elle cherche à sauvegarder et à fortifier la pure 
%substance sonore de l'homme\autocite[Voir tome I, pp. 202--203]%
%	[M. Schneider, <<Le rôle  de la musique dans la mythologie et les rites 
%des civilisations non européennes>>]{schaeffner.ea:histoire}.}
Selon la définition donnée actuellement  par l'association de musicothérapie suisse, 
la musique, la matière sonore, est utilisée dans le but d'un 
\textquote{processus thérapeutique, pour entrer en communication avec soi-même 
et pouvoir ensuite mieux percevoir le monde qui nous
	entoure, communiquer et s'exprimer.\autocite{site_musitherapy}}
La communication est un des points sur lequel nous nous attacherons plus particulièrement 
dans ce travail. En effet, il sera mis en évidence lors de la comparaison et des lectures des 
tests.
D'autre part,  
 selon Aurelia Sickert-Delin, on peut aussi différencier la musicothérapie 
psychologique, de celle dite médicinale qui 
\enquote{\emph{exerce une action 
énergétique, physiologique}} [\dots] avec \enquote{\emph{des effets curatifs}}  
ainsi que de celle dite \enquote{\emph{musicale, artistique}}.
% \enquote{\emph{L'artiste-musicien éveille l'\,``artiste intérieur'' que l'être 
%en souffrance porte en lui, pour lui permettre de s'auto-guérir [\dots] par 
%l'écoute, l'expression et la création.}}\autocite[ch. 1,  p. 14, du texte 
%inédit communiqué par A. Sickert-Delin, musicothérapeute à Alersheim, rapporté à J. 
%Viret]{viret:b}
 	 
 	 
Les vertus de la musique sont reconnues et considérés comme une évidence depuis 
fort longtemps dans l'Antiquité, que ce soit en Chine ou dans le monde arabe 
médiéval, nous relate François-Xavier Vrait.  
 Il nous précise ainsi que \textquote{la théorie des nombres 
(``l'\underline{\underline{harmonie}} peut se calculer") est intégrée et infléchie dans la 
philosophie et les traités musicaux [\dots] et que les formes d'utilisation 
thérapeutique de la musique sont décrites dans un ``traité de politique", 
``Kitab as Siyasa" [\dots] document syrien ou sabéen de la fin du 
\textsc{viii}\ieme\ siècle.}\autocite[ch. III, p. 
96]{vrait_musicotherapie_2018}. \footnote{Le terme "harmonie" reviendra lors de l'analyse des tests Ch. 6}
Mais au fil des siècles, de nombreux autres 
textes évoquent les liens de la musique avec la médecine, de sa place dans les 
rituels thérapeutiques et notamment en psychothérapie fin \textsc{xix}\ieme, 
début \textsc{xx}\ieme\ siècle. Nous nous étendrons plus à ce sujet un peu plus 
loin\footnote{Voir \ref{musicothEtpsycho}, p. \pageref{musicothEtpsycho}.}.
 
 La musicothérapie est passée de fonctionnelle, analytique, mo\-da\-le,  à 
struc\-tu\-rale pour se retrouver actuellement 
 à un tournant décisif où elle devient 
 \emph{intégrative} tout en conservant ses racines séculaires. Elle est 
intégrative dans le sens où elle permet un travail d'élaboration psychique dans 
une perspective de structuration identitaire \autocite[ch. III, p. 53, 
105]{vrait_musicotherapie_2018} et dans celui de l'intégration des données 
neuroscientifiques.
C'est ce domaine qui nous interpellera dans cette étude.


 
  
L'aspect fugace du son, de la musique, de ce médium volatil par
définition, ne semble pas apporter, comme en art-thérapie, le
même aspect concret que peuvent témoigner des supports graphiques,
visuels, reflets d'un espace-temps lors d'un travail d'élaboration
psychique d'un patient. Néanmoins, l'enregistrement sonore des séances
est un moyen parfois utilisé systématiquement par certains musicothérapeutes, comme le pratiquent Edith Lecourt, ou Carole et Clive 
Robbins \footnote{``Les Art-thérapies'', pp88--117, Ed.Armand Colin}
pour retracer ce travail en l'analysant.


En général, les musicothérapeutes sont musiciens mais rarement
médecins ou scientifiques. Il eût été intéressant, si l'objectif
 était 
 de visualiser l'effet immédiat de la musique sur le cerveau, de
 pouvoir tirer des informations ou des suggestions de travail  avec
 des outils tels que l'IRMfct\footnote{Imagerie par Résonance Magnétique
   Fonctionnelle}.
Les recherches en neurosciences éclairent, appuient et renforcent la crédibilité de l'action
majeure du son sur notre cerveau, via l'oreille, en démontrant ses effets par cet intermédiaire technique \textbf{visuel} ; elles apportent une 
reconnaissance de la musicothérapie dans le monde scientifique. Nous apporterons néanmoins une certaine nuance à cette remarque par le fait que cette observation par ce type de technique est réalisé sur un un laps de temps pendant l'action. En fait, cela ne sous-entend pas que celle-ci, bénéfique s'il en est, va perdurer dans le temps. 

Par conséquent, nous  faisons ici l'hypothèse que l'action de la 
musicothérapie a un impact certain sur la façon d'écouter, qu'elle peut perdurer après une thérapie en étant 
perçue plus
distinctement, \textsl{objectivés}, sous la forme d'un test, comme saisie par 
l'\oe il neutre de l'objectif d'un appareil
photographique.
\pdfmargincomment{"Car il nous faut..": Nouveau paragraphe?}
Car il nous faut constamment démontrer, prouver\footnote{
	\pdfcomment{Faire une phrase}les critères de l'EBM (evidence based medecine, médecine basée sur des 
preuves) F. X. Vrait, ch. II, pp. 105--106 }. On veut voir pour croire. Où est-ce 
notre esprit formaté cartésien depuis quelques centaines d'années qui nous 
empêche de penser différemment? 
Ce peut être aussi une nécessité due à notre époque pour crédibiliser l'impact 
du son sur notre être. 



