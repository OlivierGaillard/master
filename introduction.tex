\chapter{Introduction et hypothèse}

\label{jeSuisLaMusique:viret}
\begin{quotation}
\emph{<<\,\emph{Par le Son, le Silence du Non-Être vient à l'Être}. [\dots] 
\textsl{Je suis}
	\emph{la musique que je fais ou écoute}. [\dots]\,>>
[\ldots] \emph{la musique a la capacité d'harmoniser
les composantes d'une entité psychophysique pour qu'il soit ``bien
dans sa peau'' et ``bien dans son âme.}''}\, \autocite[ch. 1,  p. 8]{viret:b}
\end{quotation}

\section{Introduction}

\pdfcomment{Faut-il te présenter? Moi je laisserais tomber ces 2
paragraphes}

L'auteur exerce le métier de musicienne professionnelle et a  choisi de 
compléter sa formation en devenant musicothérapeute  ainsi que consultante à la 
méthode Tomatis\footnote{Alfred Tomatis, médecin oto-rhino-laryngologue, 
concepteur de la méthode nommée audio-psycho-phonologie, inventeur de l'Oreille 
% OGA: tu veux faire référence à 3 endroits où Tomatis est cité: Chapitre 3.3, 
% chapitre 4 et chapitre 5? Si oui précise les endroits en y mettant
% des \label{oreille_elec_1}, \label{oreille_elec_2} et \label{oreille_elec_3}
% Moi j'ai retrouvé ces références au chapitre 4.
Electronique. \pdfcomment{Vérifier 3.3.} Chap. 3.3; \ref{outil_oreille_electro},
 \ref{travail_sous_oreille_electronique} et 5.}.  



Elle exerce en cabinet privé et en cliniques psychiatriques. L'utilisation des 
différents outils de travail varie et s'adapte selon les lieux. En institution, 
elle est exclusivement musicothérapeute. En cabinet
privé, elle a la liberté d'utiliser l'une des deux techniques ou de les 
fusionner. Celles-ci se sont modifiées grâce à  l'expérience acquise, ont 
évolué  
à travers chaque cas particulier.
 Se lancer dans cette écriture, c'est creuser et approfondir ses propres 
recherches. C'est prendre de la distance, observer, se remettre en question, 
être plus précis dans ses observations, plus systématique dans la manière de 
travailler. D'où ce questionnement présenté ici : 


\section{Hypothèse: un test pour mesurer les trans\-for\-ma\-tions de l'écoute}

L'écoute est-elle universelle ou personnelle ?
 Chaque être humain semble  globalement pareil à un autre; toutes les oreilles 
paraissent  avoir une anatomie similaire. Et pourtant, si l'on observe de près 
nos empreintes digitales, on s'aperçoit de l'unicité de chaque être. De même, 
chaque oreille peut être différente, différenciée et  fonctionner d'une autre 
façon;  par conséquent notre  écoute pourrait être à chaque fois particulière 
et nous appartenir de façon singulière tout en étant perméable au changement.
Le temps imprime une évolution dans notre être intérieur et extérieur, nous 
sommes vivants dans nos mouvements psychologiques et physiques. Des traces 
devraient en être aussi visibles dans l' écoute. Il est plausible de s'en poser 
la question.
En admettant ainsi qu'il n'y ait pas de statisme mais au contraire une 
transformation dans l'écoute, ce changement devient identifiable, 
\textit{remarquable}, unique (dans le sens personnel) à travers nos expériences 
de vie.  Un test pourrait être un moyen d'évaluer les modifications supposées 
de cette écoute. 
Donner une forme à l'écoute par une image, une \emph{vision de l'écoute}, en 
quelque sorte, telle une fenêtre qui nous ouvre sur des informations qui nous 
permettent d'objectiver et évaluer d'une part 
 l'évolution de l'écoute, en début et fin de thérapie, et d'autre part pour 
constater s'il y a un parallélisme dans la transformation psychique  de la 
personne prise en charge.

L'objectif de cette étude est de faire un constat de l'évolution de l'écoute à travers le travail fait en 
musicothérapie. 


\subsection{L'hypothèse}
\pdfcomment{Plutôt mettre ici <<un test pour mesurer etc.>>?}

Notre hypothèse est que le processus d'écoute en musicothérapie améliore la capacité d'écoute.
	
Nous utiliserons deux tests différents : 
un test d'écoute spécifique quantitatif, 
et un autre test, le WOQUO-Bref qui est qualitatif.

\begin{description}
	\item[Utilisation d'un test d'écoute spécifique] dont les critères d'observations sont : représentation générale des courbes (équilibre, déséquilibre, harmonie), ainsi que les chiffres qui représentent les seuils d'écoute selon les fréquences et le volume. 
		
	Un outil :  un appareil, le Hearing Test,ou TLST, testant l'écoute; le test est réalisé en début et en fin de thérapie. 
	Vérification partiellement\footnote{partiellement  car les tests de par leur configuration ne sont pas purement quantitatif; il y a une part de qualitatif.} quantitative des résultats avec comparaison des deux tests. 
	A cet effet, nous utiliserons l'appareil test d'écoute conçu à partir de 1950 par Alfred Tomatis, médecin O. R. L., choisi car puisant ses sources en audiologie.
	
	
	
	\item[WHOQO--Bref] Le WHOQO--Bref un test d'évaluation de la qualité de vie, programme de la santé mondiale, OMS.
	Ce test est réalisé en parallèle supposée rempli par les patients eux-même  avant et après la thérapie.
	Il s'agit d'une vérification qualitative. Ce test nous permet d'avoir l'opinion des patients sur leur processus de travail.
	
	
	 
\end{description}


Nous spécifions qu'aucun support de la méthode conçue par Tomatis n'interviendra pendant les séances de musicothérapie.
Cela veut dire, ni 
\textsl{Oreille
	électronique} ni musiques préparées et filtrées. Nous expliquerons la 
raison de   leur importance lors de l'explication de la méthode, mais nous n'en ferons aucun usage. L'idée est de 
mettre à profit cette forme de test et de  s'en tenir à ce support
graphique, visible, un ``dessin'', une image utile, utilisable, tangible,
presque ``palpable'', avec des critères
d'interprétations.
	


\subsubsection{La musicothérapie}
\pdfcomment{Effectivement ce point mérite une section}
La musicothérapie est une pratique ancestrale. Il suffit de considérer la très 
grande  place de la musique dans les mythologies et dans les rites. 
% Selon M. 
%Schneider \blockquote{elle cherche à sauvegarder et à fortifier la pure 
%substance sonore de l'homme\autocite[Voir tome I, pp. 202--203]%
%	[M. Schneider, <<Le rôle  de la musique dans la mythologie et les rites 
%des civilisations non européennes>>]{schaeffner.ea:histoire}.}

La musique, la matière sonore, est utilisée dans le but d'un 
\textquote{processus thérapeutique, pour entrer en communication avec soi-même 
et pouvoir ensuite mieux percevoir le monde qui nous
	entoure, communiquer et s'exprimer.\autocite{site_musitherapy}}

 Selon Aurelia Sickert-Delin, on peut ainsi différencier la musicothérapie 
psychologique, de celle dite médicinale qui 
\enquote{\emph{exerce une action 
énergétique, physiologique}} [\dots] avec \enquote{\emph{des effets curatifs}}  
ainsi que de celle dite \enquote{\emph{musicale, artistique}}.
% \enquote{\emph{L'artiste-musicien éveille l'\,``artiste intérieur'' que l'être 
%en souffrance porte en lui, pour lui permettre de s'auto-guérir [\dots] par 
%l'écoute, l'expression et la création.}}\autocite[ch. 1,  p. 14, du texte 
%inédit communiqué par A. Sickert-Delin, musicothérapeute à Alersheim, rapporté à J. 
%Viret]{viret:b}
 	 
 	 
Les vertus de la musique sont reconnues et considérés comme une évidence depuis 
fort longtemps dans l'Antiquité, que ce soit en Chine ou dans le monde arabe 
médiéval, nous relate François-Xavier Vrait.  
 Il nous précise ainsi que \textquote{la théorie des nombres 
(``l'\underline{\underline{harmonie}} peut se calculer") est intégrée et infléchie dans la 
philosophie et les traités musicaux [\dots] et que les formes d'utilisation 
thérapeutique de la musique sont décrites dans un ``traité de politique", 
``Kitab as Siyasa" [\dots] document syrien ou sabéen de la fin du 
\textsc{viii}\ieme\ siècle.}\autocite[ch. III, p. 
96]{vrait_musicotherapie_2018}. \footnote{Le terme "harmonie" reviendra lors de l'analyse des tests Ch. 6}
Mais au fil des siècles, de nombreux autres 
textes évoquent les liens de la musique avec la médecine, de sa place dans les 
rituels thérapeutiques et notamment en psychothérapie fin \textsc{xix}\ieme, 
début \textsc{xx}\ieme\ siècle. Nous nous étendrons plus à ce sujet un peu plus 
loin\footnote{Voir \ref{musicothEtpsycho}, p. \pageref{musicothEtpsycho}.}.
 
 La musicothérapie est passée de fonctionnelle, analytique, mo\-da\-le,  à 
struc\-tu\-rale pour se retrouver actuellement 
 à un tournant décisif où elle devient 
 \emph{intégrative} tout en conservant ses racines séculaires. Elle est 
intégrative dans le sens où elle permet un travail d'élaboration psychique dans 
une perspective de structuration identitaire \autocite[ch. III, p. 53, 
105]{vrait_musicotherapie_2018} et dans celui de l'intégration des données 
neuroscientifiques.
C'est ce domaine qui nous interpellera dans cette étude.


 
  
L'aspect fugace du son, de la musique, de ce médium volatil par
définition, ne semble pas apporter, comme en art-thérapie, le
même aspect concret que peuvent témoigner des supports graphiques,
visuels, reflets d'un espace-temps lors d'un travail d'élaboration
psychique d'un patient. Les enregistrements sonores des séances peuvent tout de 
même retracer ce travail, comme le pratique \pdfcomment{j'ai mis le prénom entier}Edith Lecourt, ou Carole et Clive 
Robbins. \footnote{à rechercher réf.précises}On revient sur les séances en écoutant avec le patient et un dialogue s'installe.


D'autre part, si nous ne sommes pas formés en tant que médecins-musicothérapeutes ni issus 
d'un milieu scientifique, nous ne pouvons donc  pas prétendre à l'utilisation 
des outils
 tels que l'IRMfct\footnote{Imagerie par Résonance Magnétique Fonctionnelle};\pdfcomment{je n'abrègerais pas `Imagerie' car sinon `cet intermédiaire technique' n'est pas clair} 
en contre-partie, toutes ces recherches en
neurosciences éclairent, appuient et renforcent la crédibilité de l'action
majeure du son sur notre cerveau, via l'oreille, en démontrant ses effets par cet intermédiaire technique \textbf{visuel} ; elles apportent une 
reconnaissance de la musicothérapie dans le monde scientifique. Nous apporterons néanmoins une certaine nuance à cette remarque par le fait que cette observation par ce type de technique est réalisé sur un un laps de temps pendant l'action. En fait, cela ne sous-entend pas que celle-ci, bénéfique s'il en est, va perdurer dans le temps. 

Par conséquent, nous  faisons ici l'hypothèse que l'action de la 
musicothérapie a un impact certain sur la façon d'écouter, qu'elle peut perdurer après une thérapie en étant 
perçue plus
distinctement, \textsl{objectivés}, sous la forme d'un test, comme saisie par 
l'\oe il neutre de l'objectif d'un appareil
photographique.
\pdfmargincomment{"Car il nous faut..": Nouveau paragraphe?}
Car il nous faut constamment démontrer, prouver\footnote{
	\pdfcomment{Faire une phrase}les critères de l'EBM (evidence based medecine, médecine basée sur des 
preuves) F. X. Vrait, ch. II, pp. 105--106 }. On veut voir pour croire. Où est-ce 
notre esprit formaté cartésien depuis quelques centaines d'années qui nous 
empêche de penser différemment? 
Ce peut être aussi une nécessité due à notre époque pour crédibiliser l'impact 
du son sur notre être. 


\section{Plan du travail}

Nous aborderons d'abord l'aspect théorique : l'écoute, le son, l'oreille, le 
test d'écoute, les différents tests d'écoute en musicothérapie.  Ensuite, nous 
expliquerons  la méthode Tomatis
et puis, beaucoup plus en détails, \pdfmargincomment{j'ai mis détails au pluriel} son test d'écoute.

Puis ce sera l'aspect clinique : les tests d'écoute réalisés  avec deux groupes 
de patients en parallèle.

Et finalement suivront la vérification de l'hypothèse, les conclusions et 
interrogations. 
