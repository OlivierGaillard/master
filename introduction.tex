
\chapter*{\ Préface}
préambule, lat.\textit{praefari}: ``dire d'avance''
avant-propos, avertissement, avis, notice, prolégomène, prologue, prodrome
Ma première idée m'est venue, chronologie
Citer Eckert...à réfléchir..

\chapter{Introduction}



\begin{quotation}
 \textit{\textbf{Epigraphe  }   ``La musique vient dans la chair comme un produit immatériel
 qui vient travailler la zone à soigner. Je pompe de la
 guérison.
Depuis le début des écoutes j'ai la sensation physique et
 psychique de la
 transformation.
 La musique est équilibrante et guérisseuse, ma zone
 anesthésiée se remet à vivre, elle est remise en activité.
 Il y a comme un consentement cellulaire.
La béance s'estompe, cette
partie redevient comme les autres. (...)
Apaisement. Consentement. Réconciliation.''.} \textbf{Témoignage d'une
patiente lors d'une séance.}
% (...). [\dots]\, >>

\end{quotation}

Nous avons été très sensibles au témoignage de cette patiente dont le
processus a été porté par le son et l'écoute. Le son a eu un impact
sur les aspects physiques
et psychiques de la patiente.
Avec son image très personnelle évoquée du
\textit{``consentement cellulaire''}, nous pourrions faire le
parallélisme entre l'entrée des sons dans la sphère d'écoute et la variation de la 
perméabilité cellulaire en cytologie \autocite[ch. 3 pp. 70--76]{marieb:biologie}. Si les sons réussissent à pénétrer dans la
cellule psychique du patient, il peut y avoir amélioration des
échanges, une 
communication, une forme d'homéostasie,  \autocite[ch. 1
pp. 10]{marieb:biologie} qui reflète un état d'équilibre dynamique.

Mais comment détecter la façon d'entendre du patient?
Comment comprendre les raisons pour lesquelles il y a imperméabilité aux
échanges, un refus des sons et une fermeture au monde si ce n'est
peut-être  en testant
son écoute?  Son écoute pourrait-elle nous donner certaines clés dans sa
compréhension? C'est une hypothèse.
D'autre part, serait-ce  possible  par un test d'écoute
de marquer et souligner l'importance du processus musicothérapeutique? 
Il jouerait un
rôle de
révélateur de cette notion si abstraite et si primordiale qu'est
l'écoute dans ce domaine.
Ce sont les questions auxquelles nous allons tenter de
répondre dans ce travail.


Si le but de la musicothérapie est d'apporter un soin aux patients,
notre  approche consiste à mettre en évidence ses effets de manière
plus objective qu'à travers de simples ``témoignages'' de patients (ces derniers, bien qu'importants dans le processus thérapeutique,
n'étant que des données subjectives et donc peu exploitables d'un
point de vue scientifique). Fort nous a été de
constater un manque d'outils pour son évaluation. Car quelle que soit la technique utilisée, quel que soit
le traitement sonore, on espère une modification, on la suppose, la constate
 mais
 on ne la quantifie que difficilement. C'est la raison pour laquelle
 nous nous sommes servis d'un test d'écoute
 spécifique de la méthode d'Alfred Tomatis, choisi car puisant ses
 sources en audiologie.
 Ce test particulier  nous servira à
 souligner l'importance de la musicothérapie 
 sur la transformation de l'écoute.
 Car ce n'est qu'après plusieurs
années de pratique et d'expérience que nous avons commencé à saisir
l'essentiel de la validité de ces théories.
%Quoique nous n'ayons pas pu réunir toutes les données nécessaires
%aux tests réalisés, - car nous ne sommes pas dupes qu'un vrai travail
%nécessite beaucoup plus de précisions-, il nous a été possible toutefois d' étayer
%les résultats obtenus, de recueillir quelques considérations hypothétiques par rapport à un
%travail thérapeutique et de nous ouvrir à des réflexions. 
%Grâce à Sandra Lutz Hochreutener,  \footnote{Dr. Sandra Lutz
 % Hochreutener. Lehrt Musiktherapie und in der Weiterbildung – Tätig
%  im Departement Musik. Funktion Co-Leitung und Dozentin Bereich
  %Dossier, ZhDK}
% nous avons été encouragés à toutes les énoncer, pour pouvoir mettre un jour un terme à ce travail!


 

  


\section {Hypothèses théoriques et questions}




   Le constat d'un manque
   d'outils d'évaluation objective des résultats issus de la
   musicothérapie  nous porte aux questions
   suivantes:  
\begin{itemize}
 \item Est-ce que l'écoute est quantifiable par
          l'analyse d'un test?
        \item Dans l'affirmative, observe-t-on une transformation?
         \item Y a-t-il  un lien avec l'approche musicothérapeutique?
  \item Est-ce que la transformation est proportionnelle avec l'état psychique du patient?
   \end{itemize}
 

	 


\section*{Plan du travail}


Nous aborderons en première partie les aspects théoriques : la musicothérapie, l'écoute, le son, l'oreille, le 
test d'écoute, les différents tests d'écoute en musicothérapie. Ensuite, nous 
exposerons le test d'écoute Tomatis avec un bref aperçu de sa méthode.

 
La deuxième partie de ce travail se focalisera sur les aspects
cliniques, à savoir les tests d'écoute réalisés  avec  nos patients. 

Pour finir, nous examinerons la validité de nos hypothèses, ouvrirons
une discussion sur les résultats obtenus ainsi que les limites de ce
travail, et finalement aborderons les perspectives futures que
laissent entrevoir nos résultats.



\section{{Aspects musicothérapeutiques et éléments théoriques}}


\paragraph{Prémisse}

Les vertues de la musique sont reconnues dans la mythologie et le
monde des rites depuis les temps ancestraux (Chine, monde arabe
médiéval). 
Les formes d'utilisation 
thérapeutique de la musique figurent même dans un ``traité de politique",
``Kitab as Syasa'' remontant à des documents syriens ou saabéens datant de  [\dots] de la fin du 
\textsc{viii}\ieme\ siècle.  \textquote{La théorie des nombres 
permettait de calculer l'\textbf{harmonie}} intégrable dans la 
philosophie et les traités musicaux. \autocite[ch. III, p. 
96]{vrait_musicotherapie_2018}.
En définitive, la reconnaissance par les  politiciens  et les
philosophes de la \textit{matière sonore}, comme d'utilité
publique laisse entendre que l'équilibre personnel
peut contribuer à une forme d' ``\textbf{harmonie }civique''.




Les concepts de \textbf{communication} et  d'\textbf{harmonie} 
sont essentiels en
musicothérapie et nous nous appuyons brièvement
sur leur définition puisque utilisés ultérieurement.
L'éthymologie du mot  \textbf{communication} dérive du latin  \textit{`cum
  municare'} qui signifie ``mettre en commun, partager''.
%Bibliographie  Petit Robert 1, 1995
Le Petit Robert la définit comme une
relation, un lien, un rapport, un échange de message entre un sujet émetteur et un
sujet récepteur au moyen de signes et/ou de codes.

L'Association Suisse
de Musicothérapie retient l'idée d'un\textit{ processus thérapeutique }pour entrer en \textbf{communication} avec soi-même et avec 
l'autre dans le but d' une meilleure perception du
monde. (...)\autocite{site_musitherapy}.
Le concept de \textbf{communication} prendra son plein sens 
dans le chapitre 4.5 développant les différentes ``zones du test''.
\paragraph{}
Le mot \textbf{harmonie} ( étymol.:
<gr.=\textit{'assemblage'}>) comporte 
 des sons assemblés, des combinaisons, un ensemble de sons perçus de
 manière agréable, un accord.
 
 %Dans la mythologie grecque, \textbf\textit{{Harmonie}}  était l'épouse de Cadmos,
% introducteur de l'alphabet, et elle-même était une nymphe
 %douce et éprise de paix, fille d'Arès et d'Aphrodite. 
 L'harmonie avec soi-même et avec l'autre se présente comme  synonyme d'équilibre
psychique, traduisant les formes d'écoute intérieure et
extérieure,  visibles à travers les courbes des tests étudiées
dans l'Etude clinique.(Cf. p.27)
%Cela étant, la définition de cette pathologie n'est pas l'object de
%notre travail mais plutôt celui de 
%constater si, dans l'écoute, certaines
%caractéristiques de la dépression peuvent être améliorées par la
%restructuration de la sensibilité perceptuelle.



Ces remarques nous permettent d'anticiper
l'application utile du test de Tomatis dans notre travail, relatant l'analyse
comparative des résultats individuels.
%relever les deux formes de perception utilisable 

 
\subsection{Crédibilité actuelle de l'approche de la musicothérapie }


Au fil des siècles, de nombreux autres 
textes évoquent les liens entre musique et médecine, de sa place dans les 
rituels thérapeutiques et notamment en psychothérapie fin \textsc{xix}\ieme, 
début \textsc{xx}\ieme\ siècle.
%\footnote{Voir \ref{musicothEtpsycho},
  %p. \pageref{musicothEtpsycho}.}
Selon Aurelia Sickert-Delin, la musicothérapie 
psychologique doit être différenciée de celle dite médicinale qui 
\enquote{\emph{exerce une action 
énergétique, physiologique}} [\dots] avec \enquote{\emph{des effets curatifs}}  
ainsi que de celle dite \enquote{\emph{musicale, artistique}}. 
% \enquote{\emph{L'artiste-musicien éveille l'\,``artiste intérieur'' que l'être 
%en souffrance porte en lui, pour lui permettre de s'auto-guérir [\dots] par 
%l'écoute, l'expression et la création.}}\autocite[ch. 1,  p. 14, du texte 
%inédit communiqué par A. Sickert-Delin, musicothérapeute à Alersheim, rapporté à J. 
%Viret]{viret:b}

 Ainsi, de fonctionnelle, analytique, mo\-da\-le,  à 
struc\-tu\-rale, elle se retrouve actuellement 
 à un tournant décisif où elle devient 
 \textbf{ musicothérapie intégrative} tout en préservant ses racines
 séculaires dans le sens où elle permet un travail d'élaboration psychique dans une perspective de structuration identitaire \autocite[ch. III, p. 53, 
105]{vrait_musicotherapie_2018} et dans celui de l'intégration des données 
neuroscientifiques.
 
% Selon M. 
%Schneider \blockquote{elle cherche à sauvegarder et à fortifier la pure 
%substance sonore de l'homme\autocite[Voir tome I, pp. 202--203]%
%	[M. Schneider, <<Le rôle  de la musique dans la mythologie et les rites 
% des civilisations non européennes>>]{schaeffner.ea:histoire}.}


 


L'alliage de la musicothérapie avec les données actuelles de pointe en
science est-il faisable? Comment le réaliser pour obtenir plus de pertinence
dans sa crédibilité?
Les musicothérapeutes sont souvent musiciens mais conjugent plus
rarement dans leur profession
médecine ou neurosciences. De leur côté, les neuroscientifiques appuient
et renforcent la crédibilité de l'action du son sur notre cerveau, via
l'oreille, en démontrant ses effets par un moyen technique
\textbf{visuel} que représente par exemple l'IRMf. Mais, sans être musicien ou
musicothérapeute, leur découverte est plus rarement intégrée
directement dans leur pratique car hors contexte relationel d'une
séance, sans l'aspect intuitif et impalpable de cette forme de prise
en charge.
L'aspect fugace du son, de la musique, de ce médium volatil et
intemporel par
définition, ne semble pas amener à tout un chacun le
même aspect concret que peuvent témoigner des supports
graphiques. Ceux-ci sont des 
reflets d'un espace-temps du travail d'élaboration
psychique d'un patient, sur lesquels l'art-thérapie, par exemple, s'appuie et trouve
ses sources.
Néanmoins, il y a l'enregistrement sonore des séances qui 
permet d'avoir un support plus concret et solide, comme le pratiquent Edith Lecourt, ou Carole et Clive 
Robbins. \footnote{``Les Art-thérapies'', pp88--117, Ed.Armand Colin}
Les séances sont écoutées, filmées pour une
analyse la plus objective possible; on donne 
une forme et un sens aux sons recueillis pour les retracer dans le
parcours du patient (avec ou sans sa présence). Par contre, cette analyse  restera
tout de même très subjective de par sa nature (le son ne laisse pas
de trace) et de par la nécessité de la présence d'autres formes d'
écoutes (celles du thérapeute).
Comme l'exprime si justement Chistophe André \autocite[154]{van_eersel_cerveau}, le travail sur le psychisme du patient peut se  représenter tel un
explorateur du XVème siècle navigant sur des flots inconnus et
s'engageant sur d'autres terres, plus souvent guidé par des résultats
concrets que par des théories, même si le renfort d'études
scientifiques est irremplaçable.

