% !TEX root = ./master.tex
%\addcontentsline{toc}{chapter}{Préface}
%\chapter*{Préface}
%l'humeur dont l'impact de l'application de a musicothérapie au moyen d'un test d'écoute. 

%reste encore et toujours un mystère, une vraie énigme, au point que cette  
 %fascination %qui s'exerce  sur moi depuis longtemps, 
 %, sous forme du 
% présent essai.
% L'impact du son sur l'humain  m'a conduite à me dédier à ce travail.
% Fort de mon bagage de musicienne, j'ai choisi en 2009  de suivre deux formations en parallèle,  
% la musicothérapie et l'enseignement de  Tomatis, les deux passionnantes au point de ne pas éluder et 
% sacrifier l'une au profit de l'autre, et permettant ainsi une expérience double et nouvelle sur mon 
 %cheminement professionnel musical.
 %Ma pratique e
 %différente sur un 
 %chemin de musique. Les deux façons de travailler me passionnent et ne me permettent pas 
 %d'éluder l'un au profit de l'autre. 
 %J'exerce  la musicothérapie en clinique dans le domaine psychiatrique et suis consultante de la 
 %méthode Tomatis® en cabinet privé. M'apercevant que les 
 %Au fil du temps, je me suis aperçue que je mêlais de plus en plus les deux, les 
 %acquis de l'une des approches me servaient  pour l'autre, et réciproquement,
 % le moment était venu de 
 %prendre du recul pour mieux 
% il m'a semblé important de définir ma position, en m'interrogeant  sur la pertinence et le bien fondé 
% d'allier les deux pratiques  ou au contraire, de maintenir  leurs caractères initiaux différenciés. Cette 
%étude m'a mobilisée vers une  recherche et une élaboration comparative  des idées, visant à un degré 
%supérieur de clarté et d'efficacité dans ma pratique.
 % poussée à chercher, creuser, comparer, 
 %élaborer et 
% confronter des idées afin d'obtenir plus de clarté.
% Le test d'écoute représente principalement mon fil rouge avec la  reconnaissance de sons 
 %spécifiques (de 8000 à 250 Hz) par émission à faible volume avec augmentation progressive.
 %Chaque réponse au son est noté l'un après l'autre de manière systématique à l'instant de la réaction  
 %gestuelle du sujet, illustrant  sa capacité d'écoute (Cf. Ch. 3, 3.2.2). 
 %L'ensemble des séances individuelles de musicothérapie fait  l'objet d'observation afin d'analyser 
% leur possible impact associé à une éventuelle modification de leur capacité d'écoute.
 %  ont été observées afin de L'application d'une seule technique musicothérapeutique Afin de cibler les 
 %soins 
 % dont la trame se caractérise par un souci 
 %permanent de rigueur, étayée davantage de repères et de précisions  
% pour cibler les 
 %soins auxquels a droit le patient.
 %, parfois utilisé en 
 %musicothérapie mais très souvent chez Tomatis.  
 %Voici donc un aperçu du cheminement que je souhaite présenter, incluant doutes et incertitudes,
 %Obtient-on  plus de rigueur, davantage de repères, plus d'indices et de précisions pour cibler les 
 %soins auquel a droit le patient, si on l'observe à travers l'oeil d'un test d'écoute?
 % Voici le cheminement que je présente aujourd'hui avec les doutes traversés et les interrogations 
%  auxquelles j'ai 
 % voulu répondre, 
%  laissant à l'esprit la liberté et l'initiative de quelques tentatives de constructions de réponses.
%Ecouter, s'écouter, communiquer: la musicothérapie permet de développer la communication en
  %travaillant sur l'écoute. Notre corps est
  %tout entier influencé, y compris notre oreille.
%Dans le Département Universitaire Psychiatrique où j'exerce, la musicothérapie appartient aussi au 
%domaine de l'inconscient, non -- mesurable.
%Je vous emmène dans un chemin, mon chemin.

\chapter{Introduction}


%Nous approfondirons cette forme d'
% ``évidence", avec le test spécifique d'écoute Tomatis, dans le sens premier du terme : la \textit{visualisation} de l'écoute et de sa
% transformation.
 %Nous pourrons constater
 %l'évolution du
 %patient par des graphiques sur les différences pré -- et post -- traitement en
 %mettant en exergue les différentes écoutes, tels des ``clichés photographiques",  en
 %relation avec l'état psychique du patient et grâce au processus suivi en musicothérapie.

 %Il est à noter ici qu'il ne s'agira pas d'analyser la matière musicothérapeutique susceptible d'amener une telle transformation, sujet sur lequel il serait aussi passionnant de s'étendre.


\begin{quotation}
 \textit{``La musique vient dans la chair comme un produit immatériel
 qui vient travailler la zone à soigner. Je pompe de la
 guérison.
 %Depuis le début des écoutes
 (...)
 %j'ai la sensation physique et
% psychique de la
% transformation.
 La musique est équilibrante et guérisseuse, ma zone
 anesthésiée se remet à vivre, elle est remise en activité.
 Il y a comme un consentement cellulaire.
La béance s'estompe, cette
partie redevient comme les autres. (...)
Apaisement. Consentement. Réconciliation.''}

Témoignage d'une patiente
% (...). [\dots]\, >>

\end{quotation}



Nous avons été très sensible au témoignage de cette patiente dont l'accompagnement thérapeutique a 
été porté par le son et l'écoute, touchée autant dans son physique que dans son psychisme.
A partir de son image très personnelle évoquée du \textit{``consentement cellulaire''}, nous pourrions 
faire le
parallèle entre l'entrée des sons dans la sphère d'écoute et la variation de la
perméabilité cellulaire en cytologie \cite{marieb:biologie}. Si les sons réussissent à pénétrer dans la
\textit{cellule physique et psychique} du patient,  on constate, par 
imprégnation et sensibilisation aux ondes,
une amélioration de son état %sous  un apaisement, 
sous une forme d'harmonie et d'homéostasie,  
\autocite[10]{marieb:biologie} 
reflet d'un équilibre psychique dynamique.  
%Une circulation d'énergie  fluctue, toujours dans 
Cette  constante recherche d'équilibre 
correspond  à celle pour laquelle on tend pour obtenir  une bonne qualité de vie.
%Comment définir cet état? 

La qualité de vie représente un concept important en médecine pour évaluer l'état de bien-être subjectif
et objectif. Selon la définition de l'Organisation mondiale de la Santé (OMS) «la santé est un état de 
complet bien-être physique, mental et social, et ne consiste pas seulement en une absence de maladie 
ou d’infirmité» \footnote{«Préambule à la Constitution de l'Organisation mondiale de la Santé, tel 
qu'adopté par la 
Conférence internationale sur la Santé, New York, 19 juin -22 juillet 1946; signé le 22 juillet 1946 par les 
représentants de 61 Etats. (Actes officiels de l'Organisation mondiale de la Santé, n°. 2, p. 100) et entré 
en vigueur le 7 avril 1948»}.
La  qualité de vie est 
l'ensemble 
des ressources sociales, 
personnelles et physiques nécessaires à un individu pour réaliser ses aspirations et satisfaire ses 
besoins. Il existe des instruments de mesure que l'on nomme mesures génériques, telle le 
questionnaire  WHOQOL 
que 
nous avons employé dans le projet de recherche.
%est le  ressenti issu d'un bien être physique %sans maladie et d'une  
%bonne santé  psychique.
L'aspiration à une bonne qualité de vie correspond par ailleurs  à l'ouverture des sens, dont l'ouïe qui 
permet d'entrer 
en contact avec le monde.
Lorsque ce concept s'effondre, des murs de protection s'échafaudent.
%confronté à de nombreusesperturbations, on se ferme dans la douleur, on crée des murs de protection 
%autour de soi, tant et si bien que l'on n'y voit plus rien et que l'on entend plus rien.
% C'est une recherche qui devient constante. 
%Comment tendre vers  
%une amélioration et une meilleure qualité de vie pour essayer de retrouver cet équilibre fragile?
Dans le travail de musicothérapie en clinique psychiatrique, l'auteure a été interpellée par cette 
fermeture au monde, l'imperméabilité à 
 toute 
 communication et refus des sons par certains patients. L'observation de la capacité d'écoute  ne 
 pourrait-elle  pas être un moyen important pour rejoindre le patient dans son univers, l'aider à se 
 retrouver et entrer en relation avec son entourage?  Quelle est donc la perception sonore du patient 
 dans 
 son état?
   %d'abord comment un patient perçoit-il encore un son dans l'état renfermé dans lequel il se trouve? 
 %Que peut-il encore  percevoir? 
 %Et surtout comment écoute-t-il ?   % Nous nous sommes 
 %interrogés  
 %sur la manière de s'en sortir.
   Peut-on observer une modification de la capacité d'écoute relative à la variation de l'état psychique du 
   patient?
  % Est-ce que sa faculté d'écoute se modifie à mesure que  son  état psychique empire ? 
 % Ainsi nous en sommes venus 
  %à l'idée de tester la faculté d' écoute du patient.
  \\
  En nous focalisant sur la façon dont un patient intègre les sons dans son univers, %(Cf. Ch. 5: étude 
  %clinique)
  % En comprenant mieux son univers sonore et sa capacité d'écoute en la testant,
   on obtiendrait plus de renseignements et repères afin de l'aider à 
  retrouver  une 
  meilleure qualité de vie.
 %Est-ce que la musicothérapie, active ou passive, pourrait-elle  
% contribuer à une 
%amélioration de la qualité de vie d'un patient? 
%Si tel était  le cas, un lien étroit entre une amélioration de la 
%qualité d'écoute  et une meilleure qualité de vie  d'un patient serait envisageable.
%L'hypothèse émise serait qu'une amélioration de la qualité de vie est la conséquence d'une 
%musicothérapie adaptée. Celle-ci serait en correspondance avec une meilleure qualité d'écoute, 
%observable au moyen d'un test de capacité d'écoute.
%Peut-on observer un lien  ?% si ce n'est peut-être en testant son 
%écoute?  Celle-ci pourrait-elle nous donner certaines clés de compréhension?
%vec l'attention
%compréhension
%que lui porte le thérapeute?
%De plus, serait-ce possible ainsi
%e marquer et souligner l'importance du processus musicothérapeutique?
 \\
Ainsi, l'objectif de ce travail est de vérifier l'hypothèse suivante:  la  mesure de la  
transformation de la capacité d'écoute révèlerait l'impact  du traitement 
musicothérapeutique sur la qualité de vie.
%la capacité d'écoute et sur la qualité de vie.

\section*{Questions de recherche}

%Nous allons procéder de différentes façons avec l'observation et l'interprétation  des séances de 
%musicothérapie, ainsi que celle des graphiques illustrant la qualité d'écoute mis en lumière par le test 
%d'écoute Tomatis\textsuperscript \textregistered 
%formes d'écoute et leur modification  
%et les réponses au 
%questionnaire WHOQOL. 
Nous tenterons à 
répondre aux questions de recherche suivantes: 

\begin{itemize}
	\item L'impact de séances en musicothérapie  peut-il être mesurable par le test 
	d'écoute Tomatis\textsuperscript \textregistered?
	%Une musicothérapie ciblée sur l'écoute  est-elle en lien avec une modification de la qualité d'écoute 
	%mesurable par le test d'écoute Tomatis?
	\item La transformation de la capacité d'écoute est-elle  proportionnelle au changement 
	du sentiment de qualité de vie du patient?
\end{itemize}

  Notre étude se déroule en milieu psychiatrique, 
auprès de patients souffrants de troubles de l'humeur. Elle consiste en une triangulation de données 
récoltées à partir de plusieurs observations.  
1° l'observation de la capacité d'écoute, révélée par un test 
d'écoute selon la méthode Tomatis\textsuperscript \textregistered, test effectué deux fois, en début et 
fin de thérapie;
2° l'observation du ressenti subjectif de la qualité de vie au moyen du 
questionnaire WHOQOL, effectué deux fois, en début et fin de séjour en clinique;
3° l'observation des séances  individuelles en musicothérapie. 
\\
Une amélioration du tracé des courbes d'écoute en fin de séjour allant dans la tendance d'une courbe 
%harmonieuse (Cf. 2.1) 
dite idéale selon Tomatis va corroborer une amélioration de la qualité d'écoute; cette dernière pouvant 
être 
possiblement corrélée avec une amélioration du ressenti du patient quant à sa qualité de vie, ressenti 
observé lors des séances en musicohérapie et 
mesuré au moyen du questionnaire WHOQOL.
%Il est constitué 
%de deux groupes dont l'un est témoin; chaque patient a un suivi individuel en musicothérapie.  
%Nous  nous pencherons d'une part sur l'observation de séances de musicothérapie, 
%d'autre part sur celle de la qualité d'écoute révélée par un test d'écoute selon la méthode Tomatis. 
%Ensuite sur leu
%(avec les critères) 
 %et aussi leur lien avec la qualité de vie en s'appuyant sur les réponses aux questionnaires  
%WHOQOL.  
%Le questionnaire WHOQOL, rempli par le patient en début et fin de séjour, qui permet de 
%comparer le 
%essenti subjectif de la qualité de vie.
%L'ensemble des séances individuelles de musicothérapie ont fait  l'objet d'observation afin d'analyser 
%eur possible impact sur l'humeur des patients, associé à une éventuelle modification de leur faculté 
%d'écoute.

%avec leur 
%impact sur la sensibilité de perception.



%Le test d'écoute ainsi considéré jouerait un rôle à la fois révélateur de la faculté d'écoute singulière et 
%propre à chacun dont le  pouvoir de transformation attesterait  et soulignerait l'importance du 
%%%processus musicothérapeutique.
%Il se jouerait alors un
%rôle de
%révélateur de cette notion si abstraite et si primordiale qu'est
%\textbf{l'écoute} dans ce domaine.



%Car, comme l'affirme T. Stegemann:
%\begin{quotation}
%	\begin{german}
%	``Das Ohr ist das empfindlichste
   % Sinnesorgan des Menschen und das wichtigste Diagnostikinstrument
  %  eines Musiktherapeuten'' \autocite[p. 44]{seminar_zuerich}.
%\end{german}
% \end{quotation}


%Considérant que le but de la musicothérapie est d'apporter un soin aux patients,
%notre  approche consiste à mettre en évidence ses effets de manière
%plus objective. Les ``témoignages'' de patients, bien qu'importants dans le processus thérapeutique,
%s'avèrent être des données subjectives moins exploitables d'un
%point de vue scientifique. %Force nous a été de
%constater parfois un manque d'outils pour son évaluation.
%Car quelle que soit la technique utilisée, quel que soit
%le traitement sonore, on espère une évolution, un changement chez le patient, on peut supposer, 
%constater
% mais
 %difficilement quantifier.
 
% \begin{german}
% ``(\ldots) schwierig erfassbar mit Worten, Fakten oder messbaren 
% Veränderungen''
% \end{german}  \autocite[175]{hegi_improvisation_1993}: dans un processus 
% 	musicothérapeutique, il  
% est ardu d'obtenir et de détecter  des changements mesurables, 
%  affirme F. Hegi.
 %Toutes ces raisons nous ont amenés à nous servir d'un test d'écoute
 %spécifique.%de la méthode d'A. Tomatis.
 

 
 %Afin d'observer cette capacité d'écoute et de démontrer un changement ou une modification lors d'un 
% traitement en musicothérapie, %un test est un instrument adéquat.
% \textbf { Le test d'écoute}  peut être un instrument démontrant le changement d'écoute du 
% patient
% lors d'un traitement en musicothérapie. 
 %Pour y procéder,
  Nous  avons fait le choix du test d'écoute Tomatis\textsuperscript \textregistered  car il 
 puise ses racines  en 
 audiologie avec utilisation des sons purs.
 %; en ce sens, en première lecture, il se démarque des autres 
% tests en apportant plus de neutralité. 
 Nous verrons sur quels points il se différencie des bilans musicaux 
 ou 
 d'autres formes de tests liant psychologie et musique, dont la recherche était importante à nos yeux, 
 avec une liste, 
 évidemment non exhaustive, dressée au Chapitre 3. 
 %Cette technique de test se base sur l'emploi particulier du son pur.
 %Celui-ci, dans son essence même, est plus neutre, permet davantage d'objectivité, hors de tout 
% contexte 
 %personnel, biographique, historique. 
  %Au préalable, nous voulons rendre attentif au fait que
  Ce test sera considéré ici 
 comme un outil d'évaluation et uniquement comme tel, sans prétention quelconque de défendre une 
 théorie ou une 
 méthode. Nous expliquerons le cheminement de A. Tomatis, nécessaire dans ce contexte, en  nous 
 centrant sur les 
 %Nous nous sommes centrés sur l'apport de 
 résultats obtenus afin de les mettre 
 au profit du patient.
 % qu'il a une prétention trop universelle. 
 %l'utilisation de ce test pour voir s'il nous apporte un résultat 
 
 %D'autre part, nous avons choisi le questionnaire WHOQOL pour avoir un autre type 
 %d'évaluation, celui 
 %de la qualité de vie, donnée complémentaire, d'abord  pour confirmer ou infirmer le rapport avec une  
 %transformation d'écoute, ensuite pour savoir s'il est implicite  ou non d'un lien avec la musicothérapie.
 % Est-ce que le patient ressent 
 %Y-a-t-il concordance entre  %le questionnaire qualitatif  
 %les résultats observables en musicothérapie et ceux des  graphiques issus du test d'écoute?
 %, susceptibles d'apporter une meilleure qualité de vie vie 
 %au patient transformation.
 % L'objet et le but de notre étude est  de savoir quel  type de 
  %musicothérapie 
  %agit le plus sur l'écoute %mais à différencier l'impact de celle-ci sur l'écoute.
  % Nous nous sommes donc 
 % tenus uniquement  à l'évaluation de l'écoute et nous nous sommes pas attachés  aux nombreuses 
%techniques, réceptives et/ou actives, susceptibles de la modifier.

% Où y aura-t-il le plus de transformation d'écoute ? dans le groupe de musicothérapie  ou celui du 
% groupe de contrôle? Nous avons conscience de la forte influence des médicaments sur les patients  et 
 %de leurs ajustements difficiles et douloureux ainsi que de l'extrême variabilité psychologique qui en 
 %découle.
 %Nous n'avons malheureusement pas pu  tenir compte des autres for\-mes de thérapies créatives, du 
% suivi médical ou 
 %psychologique, de 
% la prise de médicaments, tout en ayant conscience de leur importance.
 

%2 hypothèses: 
% L'hypothèse émise serait qu'une amélioration de la qualité de vie est la conséquence d'une 
% musicothérapie 
% adaptée. Celle-ci serait en correspondance avec une 
 %eilleure qualité d'écoute, observable au moyen d'un test de capacité d'écoute.

 
%L'impact de la musicothérapie sur l'écoute est mesurable par le test Tomatis

 	%peut servir à
% souligner  l'importance 

 %par le constat de la transformation de la capacité d'écoute.
%Nous avons tenu à mettre ces résultats en lien avec le questionnaire  WHOQOL, dont les réponses, 
%bien que subjectives, peuvent les confirmer ou les infirmer. %Car ce n'est qu'après plusieurs
%années de pratique et d'expérience que nous avons commencé à saisir
%l'essentiel de la validité des théories.


%Quoique nous n'ayons  pas pu réunir toutes les données nécessaires et scientifiques
%aux tests réalisés, il nous a été possible toutefois d'étayer
%les résultats obtenus, de recueillir quelques considérations hypothétiques et de nous ouvrir à des 
%%%réflexions.
%Grâce à Sandra Lutz Hochreutener,  \footnote{Dr. Sandra Lutz
 % Hochreutener. Lehrt Musiktherapie und in der Weiterbildung – Tätig
%  im Departement Musik. Funktion Co-Leitung und Dozentin Bereich
  %Dossier, ZhDK}
% nous avons été encouragés à toutes les énoncer, pour pouvoir mettre un jour un terme à ce travail!










%Synthèse des résultats pré/post thérapie

\section*{Plan du travail}
Notre travail s'est centré sur l'oreille et l'écoute.
Nous aborderons en première partie les aspects théoriques : la musicothérapie, l'écoute, le son, l'oreille, 
le
test d'écoute, les différents tests en musicothérapie. Ensuite, nous
exposerons le test d'écoute d'A. Tomatis\textsuperscript \textregistered avec un bref aperçu de sa 
méthode.


La deuxième partie de ce travail se focalisera sur les aspects
cliniques, à savoir l'analyse des séances en  musicothérapie,  des tests d'écoute et des résultats aux 
questionnaires  réalisés  avec  nos patients.



Pour finir, nous examinerons la validité de notre hypothèse, ouvrirons
une discussion sur les résultats obtenus ainsi que les limites de ce
travail, et finalement aborderons les perspectives que
laissent entrevoir nos résultats.


\chapter{Problématique: aspects musicothérapeutiques et éléments théoriques}
Les concepts de `communication' et  d'`harmonie'
sont essentiels en
musicothérapie. Puisqu'ils auront tout  leur sens sous le chapitre de l'interprétation des tests
d'écoute, et en considération des séances faites en clinique, nous nous appuierons brièvement
sur leur définition. 
\section{Communication et harmonie }
%et lavec développant les différentes ``zones du test'' à interpréter.
L'étymologie du mot `communication' dérive du latin  \textit{`cum
  municare'} qui signifie ``mettre en commun, partager'' \autocite{dicpetitrobert}.
%Bibliographie  Petit Robert 1, 1995
Il est défini comme étant une
relation, un lien, un rapport, un échange de message entre un sujet émetteur et un
sujet récepteur au moyen de signes et/ou de codes, que ce soit de nature biologique (système nerveux), technologique ou sociale. En psychologie, on distingue la communication verbale comportant "des éléments voco-acoustiques et visuels, de la non-verbale avec la posturo-mimo-gestualité" \autocite{doronparot}.
L'Association Suisse
de Musicothérapie (ASMT) retient ce concept important dans sa définition de la musicothérapie, avec 
l'idée  d'un\textit{ "processus thérapeutique }pour entrer en `communication' avec soi-même et avec
l'autre dans le but d' une meilleure perception du
monde.(...) "\autocite{site_musitherapy}.
%\paragraph{}

De même avec le terme `harmonie' ( étymol.:
<gr.=\textit{'assemblage'}>) qui nous renvoie à
 des sons assemblés, des combinaisons, un ensemble de sons perçus de
 manière agréable, un accord. L'idée de perfection, de beauté et d'harmonie était déjà rattachée, comme 
 nous le fait remarquer F. Vrait, aux  vertus de la musique reconnues dans la mythologie et le
 monde des rites, depuis les temps ancestraux, que ce soit en Chine, dans le monde arabe
 médiéval ou dans l'Antiquité grecque. La musique cosmique s'alliait  à la musique terrestre pour une harmonie parfaite.
 %Dans la mythologie grecque, \textbf\textit{{Harmonie}}  était l'épouse de Cadmos,
% introducteur de l'alphabet, et elle-même était une nymphe
 %douce et éprise de paix, fille d'Arès et d'Aphrodite.
%Cela étant, la définition de cette pathologie n'est pas l'object de
%notre travail mais plutôt celui de
%constater si, dans l'écoute, certaines
%caractéristiques de la dépression peuvent être améliorées par la
%restructuration de la sensibilité perceptuelle.
\textquote{ Les formes d'utilisation
thérapeutique de la musique figuraient dans un ``traité de politique",
``Kitab as Syasa'' %\autocite [80]{vrait_musicotherapie_2018}
remontant à des documents syriens ou saabéens datant de  [\dots] de la fin du
\textsc{viii}\ieme\ siècle.} L'auteur relate que \textquote{la théorie des nombres
permettait de calculer l'`harmonie' intégrable dans la
philosophie et les traités musicaux } \autocite[80]{vrait_musicotherapie_2018}. De plus, de la musique 
était jouée en relation avec des questions débattues afin de rendre le plus juste des jugements.
Ainsi, les  politiciens  et les
philosophes considéraient non seulement la matière sonore  sous l'angle de la valeur du 
nombre mais la reconnaissaient autant nécessaire pour  des 
soins 
que d'utilité
publique, car l'équilibre personnel contribuait à une forme d' ``harmonie' civique.
 %Le travail avec le sonore n'était pas considéré comme un luxe mais une nécessité.
 \\
Ce que nous allons rechercher ici, c'est la représentation de ce concept d' ``harmonie', avec soi-même 
et 
avec l'autre qui se présentera dans ce contexte comme synonyme d'équilibre
psychique. Elle traduira et quantifiera la forme d'écoute et de communication (intérieure et
extérieure), la rendra visible selon les graphiques %selon notre hypothèse,
 des courbes analysées, %qualifiées d'harmonieuses 
 où on distinguera en outre l'aérienne de l'osseuse grâce aux tests de capacité 
 d'écoute (Cf. l'Etude clinique).
Ces remarques nous permettent mieux d'anticiper
l'application utile du test Tomatis dans notre travail, relatant l'analyse
comparative des résultats avec les deux formes de perception.
%relever les deux formes de perception utilisable


\section{Réflexion sur la crédibilité de la musicothérapie }

Penchons-nous brièvement sur la reconnaissance actuelle de la musicothérapie.
Au fil des siècles ont été évoqués les liens entre musique et médecine, de sa place dans les
rituels thérapeutiques et notamment en psychothérapie fin \textsc{xix}\ieme,
début \textsc{xx}\ieme\ siècle.
Elle a de plus en plus sa place dans un cadre médical et s'implique dans le contexte thérapeutique, tout 
en devant être encore plus différenciée, selon A. Sickert-Delin, lorsqu'elle est appliquée à la  psychologie.
%\footnote{Voir \ref{musicothEtpsycho},
  %p. \pageref{musicothEtpsycho}.}
% la musicothérapie appliquée à lapsychologie devrait être différenciée de celle dite médicinale qui
Elle \enquote{\emph{exerce une action
énergétique, physiologique}} [\dots] avec \enquote{\emph{des effets curatifs}}
ainsi que de celle dite \enquote{\emph{musicale, artistique}}.
\enquote{\emph{L'artiste-musicien éveille l'\,``artiste intérieur'' que l'être
en souffrance porte en lui, pour lui permettre de s'auto-guérir [\dots] par
l'écoute, l'expression et la création.}}\autocite[14] {viret:b},
texte inédit.
% Selon M.
%Schneider \blockquote{elle cherche à sauvegarder et à fortifier la pure
%substance sonore de l'homme\autocite[Voir tome I, pp. 202--203]%
%	[M. Schneider, <<Le rôle  de la musique dans la mythologie et les rites
% des civilisations non européennes>>]{schaeffner.ea:histoire}.}
\\
On évoque des liens, on constate des effets, mais de manière générale, le monde médical peine à reconnaître la réelle place de la musicothérapie et reste souvent dubitative, sceptique vis-à-vis d'elle.
 Comment l'allier avec les données actuelles de pointe en
science pour obtenir plus de pertinence
et la doter de plus de crédibilité?
%De plus, il existe encore ce fossé entre l'aspect clinique et scientifique.
Les musicothérapeutes sont souvent musiciens mais conjugent plus
rarement dans leur profession
médecine ou neurosciences. De leur côté, les neuroscientifiques appuient
et renforcent la crédibilité de l'action du `son' sur notre cerveau, via
l'oreille, en démontrant ses effets par un moyen technique
visuel que représente par exemple l'IRMf. 
%Mais, sauf quand ils sont musicien ou
%musicothérapeute, leur découverte est plus rarement intégrée
%directement dans leur pratique car hors contexte relationel d'une
%séance, sans l'aspect intuitif et impalpable de cette forme de prise
%en charge.
Faut-il donc cumuler plusieurs professions, en étant autant médecin, neuroscientifique,  musicien que 
musicothérapeute 
pour confirmer le potentiel de la musicothérapie? 
Le  Prof. A. Jaschke (Amsterdam)
réunissant de 
manière peu commune toutes ces qualifications, a étudié  
 l'effet de la musicothérapie active sur l'oxygénation, prouvant une forte diminution 
de stress sur des
bébés prématurés, avec l'utilisation des
Biomarkers ("Le cerveau enchanté": Premier Symposium Européen 
NeuroTechSymphony, au Centre Hospitalier Universitaire Vaudois (CHUV) (18/19.9. 2019)). Il est 
réjouissant d'obtenir ce type de résultats 
%avec l'observation scientifique; 
donnant confirmation de la valeur de ce travail en séance.
 %souvent vécu et perçu avec des effets positifs.%, comme nous le verrons dans le Ch. Etude Clinique. 
\\
Néanmoins, %tout un chacun n'étant pas Jaschke,
 ne pourrait-on pas obtenir la même crédibilité mais plus 
simplement? D'où la réflexion qui suivra sur le visuel par rapport au sonore, sur la visualisation du 
sonore, recherche datant déjà 
du siècle passé.
% 
%individuellement, 
%grâce au test sur la qualité d'écoute    %--quoique ne pouvant être effectué qu'en 
%différé, et non pendant 
%la séance même--
 % 
%, prouvant l'impact des séances de musicothérapie. 
%Fort heureusement, de nombreuses perspectives s'ouvrent, entr'autres celles évoquées lors du
% NeuroTechSymphony, une première en Europe, qui a eu lieu au CHUV le 18 et 19 septembre 2019, il 
% nous a été donné d'apercevoir l'ampleur de la grande avancée technologique
% et de l'émergeance entre l'interface de la musique, la technologie, la
 %création de jeux interactifs spécifiques avec leur fort impact sur la
% réhabilitation. Avec entre autres le Prof. A. Jaschke, en qualité de musicien, médecin et 
% neuroscientifique, nous a  présenté l'étude en cours avec utilisation de
%Biomarkers en Neuromusicology sur des
%bébés prématurés prouvant l'effet de la musique ``en live'' sur leur
%oxygénation, et donc sur leur diminution de stress.  
% Pour obtenir plus de crédibilité dans notre domaine, il faut continuer à faire encore 
%plus d'études scientifiques afin d'obtenir des chiffres qui seront percutants dans le monde politique 
%pour 
%avoir, de par leur intermédiaire, une plus large reconnaissance.

\subsection{Le visuel par rapport au sonore}  
%Nous sommes entourés d'images et nous mesurons chaque jour leur portée. Le visuel pourrait-il se 
%mettre au service du son? C'est dans ce sens que nous nous sommes intéressés à représenter des 
%graphiques dequalité d'écoute pour  visualiser la façon dont un sujet écoute afin de mieux comprendre 
%son univers sonore.  
 %	la façon dont un sujet écoute ou de l'intérêt de représenter des graphiques de
%qualité d'écoute: 
%nos efforts se sont concentrés dans l'idée d'une tentative de la représentation de la  qualité d'écoute. 
Le but de visualiser la  qualité d'écoute   % la façon d'écouter des patients 
grâce  des 
graphiques permettrait, par là même, d'argumenter le bien-fondé du sonore. %en 
%prenant appui d'obtenir des éléments visuels décrivant le sonore.
%avons rassemblé  %Par rapport à différentes types de thérapie, comme l'art-thérapie,
En effet, l'aspect fugace du son, de la musique, ce médium volatil et
intemporel par
définition n'apporte peut-être pas le
même aspect que peuvent amener d'autres supports, comme en art-thérapie, 
reflétant plus concrètement un espace-temps du travail d'élaboration
psychique.
%S'être interrogé sur le fort impact du visuel par rapport au sonore n'est pas récent.
 Au XVIIIème siècle, le 
physicien E. Chladni avait déjà cherché à interpréter les ondes sonores sous des formes visibles,
%a rendu l'interprétation des ondes sonores sous des formes visibles, nous 
rapporte O. Dewhurst-Maddock. De même, poursuit l'auteure, dès 1960, H. Jenny, 
médecin, physicien et musicien étudia  ``la science de l'énergie ondulatoire, la cymatique, pour exprimer 
et expliquer les analogies entre les géométries et formes visibles de la nature avec celles inhérentes au 
son''\autocite [30] {Dewhurst}. %Cela reste encore un sujet d'exploration passionnant. 
Lors d'un concert, si nous pouvions visualiser les sons qui
s'échappent de l'orchestre, ce serait un chatoiement de cercles qui se
répandraient tout autour de nous, comme
la propagation des
ronds dans l'eau suite à un ébranlement de sa surface.
Dans cette même idée, mais vu sous un autre angle, le  son perçu peut 
être figuré par un graphisme, des courbes donnant lieu à une 
interprétation de la capacité d'écoute.
%d'écoute perçu de façon  individuelle 
%....le graphisme des courbes relevés lors 
Ainsi celle-ci représenterait possiblement  un 
outil de validation des effets de séances en musicothérapie. 
%que pourrait avoir le visuel pour rendre crédible
%Revenons à une plus modeste échelle.
% comment crédibiliser et visualiser des résultats simples en musicothérapie?  obtenir des repères, des façons de se guider et guider le patient.
% d'un patient, sur lesquels l'art-thérapie, par exemple, s'appuie et
%ses sources.
%Avec la technique de l'enregistrement sonore des séances, comme le pratiquent E.
%Lecourt, ou C. et Cl.
%Robbins \autocite {lecourt_les_2017}, on y tend déjà et s'en rapproche par la saisie d'éléments pris sur 
%%%l'instant pour une
%analyse la plus objective possible du sonore.
\\
 Le défi ou la difficulté avec la musicothérapie, comme le relate Vrait, 
 c'est qu'elle  "se constitue à partir 
 de l'analogique et tente d'aller vers le digital'' 
 \autocite[24]{vrait_musicotherapie_2018}.
% citant Ducourneau à Toulouse qui s'appuie sur les travaux psychanalytiques de G.Rosolato.
Dans le développement qui suivra, on découvrira que le test d'écoute permet ainsi d'obtenir des   
\enquote{signifiants, des sons, des éléments de mesure}; on donne
une forme et un sens aux sons recueillis pour les retracer dans le
parcours du patient.
\\
%Par contre, cette analyse  restera
%tout de même très subjective de par sa nature et de par la nécessité de la présence d'autres formes d'écoutes (celles du thérapeute).
%Aux HUG, il y a depuis 6 ans une étude très précise dans le même domaine mais avec des musiques 
%enregistrées.
%Alors, de fonctionnelle, analytique, mo\-da\-le,  à
%struc\-tu\-rale, la musicothérapie se retrouve actuellement
%à un tournant décisif où elle devient
%\textbf{ musicothérapie intégrative} tout en préservant ses racines
%séculaires dans le sens où elle permet un travail d'élaboration psychique dans une perspective de structuration identitaire \autocite[p.105]{vrait_musicotherapie_2018} et dans celui de l'intégration des données
%neuroscientifiques.
%Si nous nous sommes longuement interrogés sur la pertinence en musicothérapie de l'utilisation du 
%test 
%d'écoute Tomatis,% c'est que ce procédé simple et facile d'accès,
% relie le sonore et le visuel.
% Il pourrait être aussi 
%susceptible d'apporter un lien modeste entre l'aspect scientifique -- raisons pour lesquelles leurs 
%fondements ont été relatés au Ch. 3.  -- et l'aspect clinique, caractéristique  non seulement présente 
%en musicothérapie mais aussi chez Tomatis.
 %Qu'est-ce que l'écoute?  Qu'est-ce que le son?
 %Replongeons dans leur définition.
%raisons pour lesquelles nous avons entrepris ce travail.



