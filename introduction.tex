% !TEX root = ./master.tex
\addcontentsline{toc}{chapter}{Préface}
\chapter*{Préface}
%Ce travail est un essai. Depuis longtemps, le son exerce sur moi sa fascination.
 L'impact du son sur l'humain reste encore et toujours un mystère, une vraie énigme, au point que la 
 fascination exercée  sur moi depuis longtemps, m'a conduite à me dédier à ce travail, sous forme du 
 présent essai.
 Fort de mon bagage de musicienne, j'ai choisi en 2009  de suivre deux formations en parallèle,  
 la musicothérapie et l'enseignement de  Tomatis, les deux passionnantes au point de ne pas éluder et 
 sacrifier l'une au profit de l'autre, et permettant ainsi une expérience double et nouvelle sur mon 
 cheminement professionnel musical.
 %Ma pratique e
 %différente sur un 
 %chemin de musique. Les deux façons de travailler me passionnent et ne me permettent pas 
 %d'éluder l'un au profit de l'autre. 
 J'exerce  la musicothérapie à la Clinique Privée de Meiringen ainsi qu'au 
 Département 
 Universitaire Psychiatrique de Berne, tout en poursuivant  la méthode 
 Tomatis en cabinet privé à Genève. M'apercevant au fil du temps que les 
 %Au fil du temps, je me suis aperçue que je mêlais de plus en plus les deux, les 
 acquis de l'une des approches me servaient  pour l'autre, et réciproquement, le moment était venu de 
 prendre du recul pour mieux 
 définir ma position, en  m'interrogeant  sur la pertinence et le bien fondé d'allier les deux pratiques 
 ou au contraire, de maintenir  leurs caractères initiaux différenciés.  Cette étude m'a mobilisée vers une 
 recherche et une élaboration comparative  des idées, visant à un degré supérieur de clarté et d'efficacité, 
 aussi dans ma pratique.
 % poussée à chercher, creuser, comparer, 
 %élaborer et 
% confronter des idées afin d'obtenir plus de clarté.
 Le test d'écoute représente principalement mon fil rouge dont la trame se caractérise par un souci 
 permanent de rigueur, étayée davantage de repères et de précisions  pour cibler les 
 soins auxquels a droit le patient.
 %, parfois utilisé en 
 %musicothérapie mais très souvent chez Tomatis. 
 Voici donc un aperçu du cheminement que je souhaite présenter, incluant doutes et incertitudes,
 %Obtient-on  plus de rigueur, davantage de repères, plus d'indices et de précisions pour cibler les 
 %soins auquel a droit le patient, si on l'observe à travers l'oeil d'un test d'écoute?
 % Voici le cheminement que je présente aujourd'hui avec les doutes traversés et les interrogations 
%  auxquelles j'ai 
 % voulu répondre, 
  laissant à l'esprit la liberté et l'initiative de quelques tentatives de constructions de réponses.
%Ecouter, s'écouter, communiquer: la musicothérapie permet de développer la communication en
  %travaillant sur l'écoute. Notre corps est
  %tout entier influencé, y compris notre oreille.
%Dans le Département Universitaire Psychiatrique où j'exerce, la musicothérapie appartient aussi au domaine de l'inconscient, non -- mesurable.

%Je vous emmène dans un chemin, mon chemin.

\chapter{Introduction}


%Nous approfondirons cette forme d'
% ``évidence", avec le test spécifique d'écoute Tomatis, dans le sens premier du terme : la \textit{visualisation} de l'écoute et de sa
% transformation.
 %Nous pourrons constater
 %l'évolution du
 %patient par des graphiques sur les différences pré -- et post -- traitement en
 %mettant en exergue les différentes écoutes, tels des ``clichés photographiques",  en
 %relation avec l'état psychique du patient et grâce au processus suivi en musicothérapie.

 %Il est à noter ici qu'il ne s'agira pas d'analyser la matière musicothérapeutique susceptible d'amener une telle transformation, sujet sur lequel il serait aussi passionnant de s'étendre.


\begin{quotation}
 \textit{\textbf ``La musique vient dans la chair comme un produit immatériel
 qui vient travailler la zone à soigner. Je pompe de la
 guérison.
 %Depuis le début des écoutes
 (...)
 %j'ai la sensation physique et
% psychique de la
% transformation.
 La musique est équilibrante et guérisseuse, ma zone
 anesthésiée se remet à vivre, elle est remise en activité.
 Il y a comme un consentement cellulaire.
La béance s'estompe, cette
partie redevient comme les autres. (...)
Apaisement. Consentement. Réconciliation.''.}

Témoignage d'une patiente
% (...). [\dots]\, >>

\end{quotation}



Nous avons été très sensible au témoignage de cette patiente dont le
processus de guérison a été porté par le son et l'écoute, touchée
autant dans son physique
que dans son psychisme.
A partir de l'image très personnelle évoquée du
\textit{``consentement cellulaire''}, nous pourrions faire le
parallèle entre l'entrée des sons dans la sphère d'écoute et la variation de la
perméabilité cellulaire en cytologie \autocite[70--76]{marieb:biologie}. Si les sons réussissent à pénétrer dans la
\textit{cellule psychique} du patient (au propre comme au figuré),  il peut y avoir amélioration par imprégnation et sensibilisation aux ondes, pour un apaisement, une forme d'harmonie et d'homéostasie,  \autocite[10]{marieb:biologie} où se reflète un état d'équilibre psychique dynamique.

Mais comment détecter la sensibilité d'\textbf{oreille} du patient?
Comment comprendre les raisons pour lesquelles il y a imperméabilité à toute communication, refus des sons et fermeture au monde si ce n'est
peut-être en testant
son écoute?  Celle-ci pourrait-elle nous donner certaines clés de compréhension?
%vec l'attention
%compréhension
%que lui porte le thérapeute?
%De plus, serait-ce possible ainsi
%e marquer et souligner l'importance du processus musicothérapeutique?
Le test d'écoute ainsi considéré jouerait un rôle à la fois révélateur de la faculté d'écoute singulière et 
propre à chacun dont le  pouvoir de transformation attesterait  et soulignerait l'importance du processus 
musicothérapeutique.
%Il se jouerait alors un
%rôle de
%révélateur de cette notion si abstraite et si primordiale qu'est
%\textbf{l'écoute} dans ce domaine.
Ce sont ces questions de recherche auxquelles nous allons tenter de répondre dans ce travail.


Car, comme l'affirme T. Stegemann,

\begin{quotation}
	
	`\textit{`\textbf{``Das Ohr ist das empfindlichste
    Sinnesorgan des Menschen und das wichtigste Diagnostikinstrument
    eines Musiktherapeuten.''}''\autocite [44]{seminar_zuerich}}
\footnote{ \textit{"L'oreille est l'organe le plus sensible des sens de l'humain
et l'instrument de diagnostic  le plus important du
musicothérapeute}`'' Seminar  ZhDk, 2018,
Univ.-Prof. Dr. med. Thomas Stegemann, Universität für Musik und darstellende Kunst Wien, traduction libre}.
 \end{quotation}


Considérant que le but de la musicothérapie est d'apporter un soin aux patients,
notre  approche consiste à mettre en évidence ses effets de manière
plus objective. Les ``témoignages'' de patients, bien qu'importants dans le processus thérapeutique,
s'avèrent être des données subjectives moins exploitables d'un
point de vue scientifique. %Force nous a été de
%constater parfois un manque d'outils pour son évaluation.
Car quelle que soit la technique utilisée, quel que soit
le traitement sonore, on espère une évolution, un changement chez le patient, on peut supposer, constater
 mais
 difficilement quantifier.
 
 `\textit{` \enquote{ (...) schwierig erfassbar mit Worten, Fakten oder messbaren 
 Veränderungen } \footnote { \enquote{ (...) difficilement évaluable avec des mots, des faits ou des 
 changements  
 	mesurables}, traduction libre} \autocite[175]{hegi_improvisation_1993}} : dans un processus 
 	musicothérapeutique, il  
 est ardu d'obtenir et de détecter  des changements mesurables, 
  affirme F. Hegi.
 Toutes ces raisons nous ont amenés à nous servir d'un test d'écoute
 spécifique.%de la méthode d'A. Tomatis.
 
 
 
 
 \textbf { Le test d'écoute}  peut être un instrument prouvant et démontrant le changement d'écoute du 
 patient
 lors de l'aboutissement d'un traitement en musicothérapie.
 Pour y procéder, nous  avons fait le choix du test d'écoute Tomatis pour les raisons suivantes:  ce test 
 puise ses racines  en 
 audiologie avec utilisation des sons purs; en ce sens, en première lecture, il se démarque des autres 
 tests en apportant plus de neutralité. Nous verrons sur quels points il se différencie des bilans musicaux 
 ou 
 d'autres formes de tests liant psychologie et musique, dont la recherche était importante à nos yeux, 
 avec une liste, 
 évidemment non exhaustive, dressée au Chapitre 3. 
 %Cette technique de test se base sur l'emploi particulier du son pur.
 %Celui-ci, dans son essence même, est plus neutre, permet davantage d'objectivité, hors de tout 
% contexte 
 %personnel, biographique, historique. 
  Au préalable, nous voulons rendre attentif au fait que ce test sera considéré ici 
 comme un outil et uniquement comme tel, sans prétention quelconque de défendre une théorie ou une 
 méthode. Nous expliquerons le cheminement de Tomatis, nécessaire dans ce contexte, tout en  sachant 
 que sa méthode  est 
 encore sujet à débat, ce qui ne retiendra pas ici notre intérêt. Nous nous sommes centrés sur l'apport de 
 résultats obtenus à mettre 
 au profit du patient.
 % qu'il a une prétention trop universelle. 
 %l'utilisation de ce test pour voir s'il nous apporte un résultat 
 
 D'autre part, nous avons choisi le questionnaire \textbf {WHOQOL} pour avoir un autre type 
 d'évaluation, celui 
 de la qualité de vie, donnée complémentaire, d'abord  pour confirmer ou infirmer le rapport avec une  
 transformation d'écoute, ensuite pour savoir s'il est implicite  ou non d'un lien avec la musicothérapie.
 % Est-ce que le patient ressent 
 Y-a-t-il concordance entre  le questionnaire qualitatif  et les résultats graphiques issus du test d'écoute?
 
  \textbf{La musicothérapie :}  a été considérée dans son ensemble sans 
 différencier les 
 techniques utilisées.
  L'objet et le but de notre étude n'étaient pas de savoir quel  type de 
  musicothérapie 
  agit le plus sur l'écoute mais à différencier l'impact de celle-ci sur l'écoute. Nous nous sommes donc 
  tenus uniquement  à l'évaluation de l'écoute et nous nous sommes pas attachés  aux nombreuses 
  techniques, réceptives et/ou actives, susceptibles de la modifier.

 Où y aura-t-il le plus de transformation d'écoute ? dans le groupe de musicothérapie  ou celui du 
 groupe de contrôle?
 Nous n'avons malheureusement pas pu  tenir compte des autres for\-mes de thérapies créatives, du 
 suivi médical ou 
 psychologique, de 
 la prise de médicaments, tout en ayant conscience de leur importance.
 

 
 

 Ainsi, l'hypothèse que nous avons posée est celle-ci: \textbf{le test Tomatis, par la  mesure de la  
 transformation de la capacité d'écoute, révèle  l'importance et l'impact  de la musicothérapie sur 
 l'écoute}.
%L'impact de la musicothérapie sur l'écoute est mesurable par le test Tomatis

 	%peut servir à
% souligner  l'importance 

 %par le constat de la transformation de la capacité d'écoute.
%Nous avons tenu à mettre ces résultats en lien avec le questionnaire  WHOQOL, dont les réponses, 
%bien que subjectives, peuvent les confirmer ou les infirmer. %Car ce n'est qu'après plusieurs
%années de pratique et d'expérience que nous avons commencé à saisir
%l'essentiel de la validité des théories.


Quoique nous n'ayons  pas pu réunir toutes les données nécessaires et scientifiques
aux tests réalisés, il nous a été possible toutefois d'étayer
les résultats obtenus, de recueillir quelques considérations hypothétiques et de nous ouvrir à des réflexions.
%Grâce à Sandra Lutz Hochreutener,  \footnote{Dr. Sandra Lutz
 % Hochreutener. Lehrt Musiktherapie und in der Weiterbildung – Tätig
%  im Departement Musik. Funktion Co-Leitung und Dozentin Bereich
  %Dossier, ZhDK}
% nous avons été encouragés à toutes les énoncer, pour pouvoir mettre un jour un terme à ce travail!







\section* {Questions de recherche}


Nous disposons de deux outils, un test d'écoute et un questionnaire WQ, qui vont nous permettre de 
répondre aux questions de recherche suivantes: 

\begin{itemize}
%\item Est-ce que la capacité d'écoute est quantifiable par
%          l'analyse d'un test?
        \item Si la capacité d'écoute est quantifiable par
                 l'analyse d'un test,
          une transformation est-elle observable?
          %après thérapie?
         % observe-t-on une transformation?
        \item Y aurait-il un lien entre la transformation de la capacité d'écoute observée  et un traitement 
        musicothérapeutique?
        %La transformation de cette capacité d'écoute observée aurait-elle un lien
          %avec le traitement musicothérapeutique?
          %Existe-t-il  un lien avec l'approche musicothérapeutique?
  \item Est-ce que la transformation de la capacité d'écoute est proportionnelle au changement 
  d'état psychique du patient?
   \end{itemize}



%Synthèse des résultats pré/post thérapie

\section*{Plan du travail}
Notre travail s'est centré sur l'oreille et l'écoute.
Nous aborderons en première partie les aspects théoriques : la musicothérapie, l'écoute, le son, l'oreille, 
le
test d'écoute, les différents tests en musicothérapie. Ensuite, nous
exposerons le test d'écoute Tomatis avec un bref aperçu de sa méthode.

La deuxième partie de ce travail se focalisera sur les aspects
cliniques, à savoir les tests d'écoute et les questionnaires  réalisés  avec  nos patients.

Pour finir, nous examinerons la validité de notre hypothèse, ouvrirons
une discussion sur les résultats obtenus ainsi que les limites de ce
travail, et finalement aborderons les perspectives que
laissent entrevoir nos résultats.


\chapter{Problématique}

\section{Aspects musicothérapeutiques et éléments théoriques}
Les concepts de \textbf{`communication'} et  d'\textbf{`harmonie'}
sont essentiels en
musicothérapie. Puisqu'ils auront tout  leur sens sous le chapitre de l'interprétation des tests
d'écoute,  nous nous appuierons brièvement
sur leur définition. 
\paragraph{Communication et harmonie}:


%et lavec développant les différentes ``zones du test'' à interpréter.
L'étymologie du mot  \textbf{`communication'} dérive du latin  \textit{`cum
  municare'} qui signifie ``mettre en commun, partager'' \autocite{dicpetitrobert}.
%Bibliographie  Petit Robert 1, 1995
Il est défini comme étant une
relation, un lien, un rapport, un échange de message entre un sujet émetteur et un
sujet récepteur au moyen de signes et/ou de codes, que ce soit de nature biologique (système nerveux), technologique ou sociale. En psychologie, on distingue la communication verbale comportant "des éléments voco-acoustiques et visuels, de la non-verbale avec la posturo-mimo-gestualité" \autocite{doronparot}.
L'Association Suisse
de Musicothérapie (ASMT) retient ce concept important dans sa définition de la musicothérapie, avec l'idée  d'un\textit{ "processus thérapeutique }pour entrer en \textbf{`communication'} avec soi-même et avec
l'autre dans le but d' une meilleure perception du
monde.(...) "\autocite{site_musitherapy}.
%\paragraph{}

De même avec le terme \textbf{`harmonie'} ( étymol.:
<gr.=\textit{'assemblage'}>) qui nous renvoie à
 des sons assemblés, des combinaisons, un ensemble de sons perçus de
 manière agréable, un accord. L'idée de perfection, de beauté et d'harmonie était déjà rattachée, comme 
 nous le fait remarquer F. Vrait, aux  vertus de la musique reconnues dans la mythologie et le
 monde des rites, depuis les temps ancestraux, que ce soit en Chine, dans le monde arabe
 médiéval ou dans l'Antiquité grecque. La musique cosmique s'alliait  à la musique terrestre pour une harmonie parfaite.
 %Dans la mythologie grecque, \textbf\textit{{Harmonie}}  était l'épouse de Cadmos,
% introducteur de l'alphabet, et elle-même était une nymphe
 %douce et éprise de paix, fille d'Arès et d'Aphrodite.
%Cela étant, la définition de cette pathologie n'est pas l'object de
%notre travail mais plutôt celui de
%constater si, dans l'écoute, certaines
%caractéristiques de la dépression peuvent être améliorées par la
%restructuration de la sensibilité perceptuelle.
\textquote{ Les formes d'utilisation
thérapeutique de la musique figuraient dans un ``traité de politique",
``Kitab as Syasa'' %\autocite [80]{vrait_musicotherapie_2018}
remontant à des documents syriens ou saabéens datant de  [\dots] de la fin du
\textsc{viii}\ieme\ siècle.} L'auteur relate que \textquote{la théorie des nombres
permettait de calculer l'\textbf{`harmonie'} intégrable dans la
philosophie et les traités musicaux } \autocite[80]{vrait_musicotherapie_2018}. De la musique était jouée en relation avec des questions débattues afin de rendre le plus juste des jugements.
Ainsi, et ce n'est pas négligeable, les  politiciens  et les
philosophes reconnaissaient déjà la \textit{matière sonore} comme nécessaire non seulement pour  des soins mais étant aussi d'utilité
publique.
 L'équilibre personnel contribuait à une forme d' \textbf{`harmonie'} civique.
 
 
%Le travail avec le sonore n'était pas considéré comme un luxe mais une nécessité.
L'harmonie avec soi-même et avec l'autre se présentera ici comme synonyme d'équilibre
psychique, traduisant les formes d'écoute et de communication (intérieure et
extérieure, visibles, selon notre hypothèse,
 à travers les courbes (courbes aérienne et osseuse) des tests de capacité d'écoute rapportées
dans l'Etude clinique.
Ces remarques nous permettent mieux d'anticiper
l'application utile du test de Tomatis dans notre travail, relatant l'analyse
comparative des résultats avec les deux formes de perception.
%relever les deux formes de perception utilisable


\section{Crédibilité actuelle de l'approche de la musicothérapie }

Auparavant, penchons-nous brièvement sur la reconnaissance actuelle de la musicothérapie.
Au fil des siècles ont été évoqués les liens entre musique et médecine, de sa place dans les
rituels thérapeutiques et notamment en psychothérapie fin \textsc{xix}\ieme,
début \textsc{xx}\ieme\ siècle.
Elle a de plus en plus sa place dans un cadre médical et s'implique dans le contexte thérapeutique.
%\footnote{Voir \ref{musicothEtpsycho},
  %p. \pageref{musicothEtpsycho}.}
Selon A. Sickert-Delin, la musicothérapie appliquée à la
psychologie devrait être différenciée de celle dite médicinale qui
\enquote{\emph{exerce une action
énergétique, physiologique}} [\dots] avec \enquote{\emph{des effets curatifs}}
ainsi que de celle dite \enquote{\emph{musicale, artistique}}.
\enquote{\emph{L'artiste-musicien éveille l'\,``artiste intérieur'' que l'être
en souffrance porte en lui, pour lui permettre de s'auto-guérir [\dots] par
l'écoute, l'expression et la création.}}\autocite[14] {viret:b},
texte inédit.
% Selon M.
%Schneider \blockquote{elle cherche à sauvegarder et à fortifier la pure
%substance sonore de l'homme\autocite[Voir tome I, pp. 202--203]%
%	[M. Schneider, <<Le rôle  de la musique dans la mythologie et les rites
% des civilisations non européennes>>]{schaeffner.ea:histoire}.}
On évoque des liens, on constate des effets, mais de manière générale, le monde médical peine à reconnaître la réelle place de la musicothérapie et reste souvent dubitative, sceptique vis-à-vis d'elle.
 Comment l'allier avec les données actuelles de pointe en
science pour obtenir plus de pertinence
et la doter de plus de crédibilité?
De plus, il existe encore ce fossé entre l'aspect clinique et scientifique.
Les musicothérapeutes sont souvent musiciens mais conjugent plus
rarement dans leur profession
médecine ou neurosciences. De leur côté, les neuroscientifiques appuient
et renforcent la crédibilité de l'action du \textbf{`son'} sur notre cerveau, via
l'oreille, en démontrant ses effets par un moyen technique
\textbf{visuel} que représente par exemple l'IRMf. Mais, sauf quand ils sont musicien ou
musicothérapeute, leur découverte est plus rarement intégrée
directement dans leur pratique car hors contexte relationel d'une
séance, sans l'aspect intuitif et impalpable de cette forme de prise
en charge.
Faut-il donc cumuler plusieurs qualifications, médecin neuroscientifique,  musicien et musicothérapeute pour crédibiliser la musicothérapie?
%Fort heureusement, de nombreuses perspectives s'ouvrent, entr'autres celles évoquées lors du

 Lors du \textbf{1° Symposium
 NeuroTechSymphony}, une première en Europe, qui a eu lieu au CHUV le 18 et 19 septembre 2019, il nous a été
 donné d'apercevoir l'ampleur de la grande avancée technologique
 et de l'émergeance entre l'interface de la musique, la technologie, la
 création de jeux interactifs spécifiques avec leur fort impact sur la
 réhabilitation. Avec entre autres le Prof. A. Jaschke, en qualité de musicien, médecin et 
 neuroscientifique, nous a  présenté l'étude en cours avec utilisation de
Biomarkers en Neuromusicology sur des
bébés prématurés prouvant l'effet de la musique ``en live'' sur leur
oxygénation, et donc sur leur diminution de stress. Nous l'avons interviewé et d'après lui, il n'y a pas 
d'autres solutions, pour obtenir plus de crédibilité dans notre domaine, que de continuer à faire encore 
plus d'études scientifiques afin d'obtenir des chiffres qui seront percutants dans le monde politique pour 
avoir, de par leur intermédiaire, une plus large reconnaissance.


S'être interrogé sur le fort impact du visuel par rapport au sonore n'est pas récent. Au XVIIIème siècle, le 
physicien E. Chladni a rendu l'interprétation des ondes sonores sous des formes visibles, nous 
rapporte O. Dewhurst-Maddock. De même, poursuit l'auteure, dans les années 60, H. Jenny, 
médecin, physicien et musicien étudia  ``la science de l'énergie ondulatoire, la cymatique, pour exprimer 
et expliquer les analogies entre les géométries et formes visibles de la nature avec celles inhérentes au 
son. ''\autocite [30] {Dewhurst}
%que pourrait avoir le visuel pour rendre crédible
%Revenons à une plus modeste échelle.
% comment crédibiliser et visualiser des résultats simples en musicothérapie?  obtenir des repères, des façons de se guider et guider le patient.
Car, par rapport à différentes types de thérapie, comme l'art-thérapie,
l'aspect fugace du son, de la musique, de ce médium volatil et
intemporel par
définition, ne semble pas apporter à tout un chacun le
même aspect concret des supports
graphiques
reflétant un espace-temps du travail d'élaboration
psychique.
% d'un patient, sur lesquels l'art-thérapie, par exemple, s'appuie et
%ses sources.
Avec la technique de l'enregistrement sonore des séances, comme le pratiquent E.
Lecourt, ou C. et Cl.
Robbins \autocite {lecourt_les_2017}, on y tend déjà et s'en rapproche par la saisie d'éléments pris sur l'instant pour une
analyse la plus objective possible du sonore.


 Le défi ou la difficulté avec la musicothérapie, comme le relate Vrait, 
 c'est qu'elle  `se constitue à partir 
 de l'analogique et tente d'aller vers le digital'' 
 \autocite[24]{vrait_musicotherapie_2018}.
% citant Ducourneau à Toulouse qui s'appuie sur les travaux psychanalytiques de G.Rosolato.
Dans le développement qui suivra, on découvrira que le test d'écoute permet aussi d'obtenir des   
\enquote{signifiants , des sons, des éléments de mesure}; on donne
une forme et un sens aux sons recueillis pour les retracer dans le
parcours du patient.
%Par contre, cette analyse  restera
%tout de même très subjective de par sa nature et de par la nécessité de la présence d'autres formes d'écoutes (celles du thérapeute).

%Aux HUG, il y a depuis 6 ans une étude très précise dans le même domaine mais avec des musiques enregistrées.

%Alors, de fonctionnelle, analytique, mo\-da\-le,  à
%struc\-tu\-rale, la musicothérapie se retrouve actuellement
%à un tournant décisif où elle devient
%\textbf{ musicothérapie intégrative} tout en préservant ses racines
%séculaires dans le sens où elle permet un travail d'élaboration psychique dans une perspective de structuration identitaire \autocite[p.105]{vrait_musicotherapie_2018} et dans celui de l'intégration des données
%neuroscientifiques.
Si nous nous sommes longuement interrogés sur la pertinence en musicothérapie de l'utilisation du test 
d'écoute Tomatis, c'est que ce procédé simple et facile d'accès, reliant le sonore et le visuel, est 
susceptible d'apporter un lien modeste entre l'aspect scientifique -- raisons pour lesquelles leurs 
fondements ont été relatés au Ch. 3.  -- et l'aspect clinique, caractéristique  non seulement présente 
en musicothérapie mais aussi chez Tomatis.

 Qu'est-ce que l'écoute?  Qu'est-ce que le son?
 Il nous a semblé indispensable de replonger dans leur définition.
%raisons pour lesquelles nous avons entrepris ce travail.



