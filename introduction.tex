% !TEX root = ./master.tex

%\chapter*{\ Préface}

\chapter{Introduction}



\begin{quotation}
 \textit{\textbf ``La musique vient dans la chair comme un produit immatériel
 qui vient travailler la zone à soigner. Je pompe de la
 guérison.
 %Depuis le début des écoutes
 (...)
 %j'ai la sensation physique et
% psychique de la
% transformation.
 La musique est équilibrante et guérisseuse, ma zone
 anesthésiée se remet à vivre, elle est remise en activité.
 Il y a comme un consentement cellulaire.
La béance s'estompe, cette
partie redevient comme les autres. (...)
Apaisement. Consentement. Réconciliation.''.}

Témoignage d'une patiente
% (...). [\dots]\, >>

\end{quotation}

Nous avons été très sensible au témoignage de cette patiente dont le
processus de guérison a été porté par le son et l'écoute, touchée
autant dans son physique
que dans son psychisme.
A partir de l'image très personnelle évoquée du
\textit{``consentement cellulaire''}, nous pourrions faire le
parallélisme entre l'entrée des sons dans la sphère d'écoute et la variation de la
perméabilité cellulaire en cytologie \autocite[ch. 3 pp. 70--76]{marieb:biologie}. Si les sons réussissent à pénétrer dans la
cellule psychique du patient, il peut y avoir amélioration des
échanges, apaisement, une forme d'harmonie et d'homéostasie,  \autocite[ch. 1
p. 10]{marieb:biologie} où se reflète un état d'équilibre dynamique.

Mais comment détecter la sensibilité auditive du patient?
Comment comprendre les raisons pour lesquelles il y a imperméabilité
aux
échanges, refus des sons et fermeture au monde si ce n'est
peut-être en testant
son écoute?  Celle-ci pourrait-elle nous donner certaines clés de sa
compréhension?
De plus, serait-ce possible ainsi
de marquer et souligner l'importance du processus musicothérapeutique?
Il se jouerait alors un
rôle de
révélateur de cette notion si abstraite et si primordiale qu'est
\textbf{l'écoute} dans ce domaine. Ce sont les interrogations auxquelles nous allons tenter de répondre dans ce travail.


Car, comme l'affirme T. Stegemann,

\begin{quotation}
	`\textit{`\textbf{``Das Ohr ist das empfindlichste
    Sinnesorgan des Menschen und das wichtigste Diagnostikinstrument
    eines Musiktherapeuten.''}''\autocite [p.44]{seminar_zuerich}}
\footnote{ \textit{"L'oreille est l'organe le plus sensible des sens de l'humain
et l'instrument de diagnostic  le plus important du
musicothérapeute}`''
Univ.-Prof. Dr. med. Thomas Stegemann, Universität für Musik und darstellende Kunst Wien, traduction: V. Gaillard}.
 \end{quotation}


Considérant que le but de la musicothérapie est d'apporter un soin aux patients,
notre  approche consiste à mettre en évidence ses effets de manière
plus objective. Les ``témoignages'' de patients, bien qu'importants dans le processus thérapeutique,
s'avèrent être des données subjectives et donc moins exploitables d'un
point de vue scientifique. Force nous a été de
constater parfois un manque d'outils pour son évaluation. Car quelle que soit la technique utilisée, quel que soit
le traitement sonore, on espère une modification, on la suppose, on la constate
 mais
 on ne la quantifie que difficilement.
  C'est la raison pour laquelle
 nous nous sommes servis d'un test d'écoute
 spécifique de la méthode d'Alfred Tomatis, choisi car puisant ses
 sources en audiologie. D'une part, cette technique de test, comme nous le verrons précisément au Ch. 3. 3, se base sur l'emploi particulier du son pur.
 Celui-ci, dans son essence même, est plus neutre et permet plus d'objectivité en étant hors de tout contexte personnel, biographique, historique ou d'associations d'idées. En ce sens, le test Tomatis se différencie des bilans musicaux comportant des oeuvres musicales ou d'autres formes de tests liant psychologie et musique, éventail de recherches dont il nous a semblé important de dresser une liste (Cf. Ch. 2.), évidement non exhaustive.


 Ainsi l'hypothèse que nous avons posée est de savoir si ce test Tomatis pourrait  servir à
 souligner l'importance de la musicothérapie
 par le constat de la transformation de la capacité d'écoute.
Nous avons tenu à mettre ces résultats en lien avec le questionnaire  WHOQOL, dont les réponses, bien que subjectives, peuvent les confirmer ou les infirmer. %Car ce n'est qu'après plusieurs
%années de pratique et d'expérience que nous avons commencé à saisir
%l'essentiel de la validité des théories.
Quoique nous n'ayons pas pu réunir toutes les données nécessaires
aux tests réalisés (Cf. Ch. 4), - car nous ne sommes pas dupes qu'un vrai travail
nécessite beaucoup plus de précisions -, il nous a été possible toutefois d'étayer
les résultats obtenus, de recueillir quelques considérations hypothétiques et de nous ouvrir à des réflexions.
%Grâce à Sandra Lutz Hochreutener,  \footnote{Dr. Sandra Lutz
 % Hochreutener. Lehrt Musiktherapie und in der Weiterbildung – Tätig
%  im Departement Musik. Funktion Co-Leitung und Dozentin Bereich
  %Dossier, ZhDK}
% nous avons été encouragés à toutes les énoncer, pour pouvoir mettre un jour un terme à ce travail!







\section {Questions de recherche}




\begin{itemize}
 \item Est-ce que la capacité d'écoute est quantifiable par
          l'analyse d'un test?
        \item Dans l'affirmative,
          peut-on observer une transformation de sa capacité?
          %après thérapie?
         % observe-t-on une transformation?
        \item Si oui, est-ce que la transformation de cette capacité d'écoute observée aurait un lien
          avec le traitement musicothérapeutique?
          %Existe-t-il  un lien avec l'approche musicothérapeutique?
  \item Est-ce que la transformation de la capacité d'écoute peut-elle être proportionnelle avec le changement de l'état psychique du patient?
   \end{itemize}

Remarque importante: l'objet de notre étude n'est pas de différencier les techniques musicothérapeutiques susceptibles d'apporter ou non une modification de la capacité d'écoute.

%Synthèse des résultats pré/post thérapie

\section*{Plan du travail}

Notre travail s'est donc centré sur l'oreille et l'écoute.
Nous aborderons en première partie (Ch. 1. 2. 2./3.)les aspects théoriques : la musicothérapie, l'écoute, le son, l'oreille, le
test d'écoute, les différents tests en musicothérapie. Ensuite, nous
exposerons le test d'écoute Tomatis avec un bref aperçu de sa méthode.

La deuxième partie de ce travail se focalisera sur les aspects
cliniques, à savoir les tests d'écoute réalisés  avec  nos patients.

Pour finir, nous examinerons la validité de nos hypothèses, ouvrirons
une discussion sur les résultats obtenus ainsi que les limites de ce
travail, et finalement aborderons les perspectives que
laissent entrevoir nos résultats.



\section{{Aspects musicothérapeutiques et éléments théoriques}}






Les concepts de \textbf{`communication'} et  d'\textbf{`harmonie'}
sont essentiels en
musicothérapie et nous nous appuyons brièvement
sur leur définition puisqu'ils seront utilisés ultérieurement.


L'étymologie du mot  \textbf{`communication'} dérive du latin  \textit{`cum
  municare'} qui signifie ``mettre en commun, partager'' \autocite{dicpetitrobert}.
%Bibliographie  Petit Robert 1, 1995
Il est défini comme étant une
relation, un lien, un rapport, un échange de message entre un sujet émetteur et un
sujet récepteur au moyen de signes et/ou de codes.
L'Association Suisse
de Musicothérapie (ASMT) retient l'idée d'un\textit{ processus thérapeutique }pour entrer en \textbf{`communication'} avec soi-même et avec
l'autre dans le but d' une meilleure perception du
monde. (...)\autocite{site_musitherapy}.

Si nous avons retenu les concepts de \textbf{`communication'} et d'\textbf{`harmonie'}, c'est qu'ils prendront leur plein sens
dans le chapitre 6. 4., pp.77 -- 78,  développant les différentes ``zones du test''.
\paragraph{}
Ainsi le mot \textbf{`harmonie'} ( étymol.:
<gr.=\textit{'assemblage'}>) renvoie, comme nous le savons, à
 des sons assemblés, des combinaisons, un ensemble de sons perçus de
 manière agréable, un accord.
 %Dans la mythologie grecque, \textbf\textit{{Harmonie}}  était l'épouse de Cadmos,
% introducteur de l'alphabet, et elle-même était une nymphe
 %douce et éprise de paix, fille d'Arès et d'Aphrodite.
%Cela étant, la définition de cette pathologie n'est pas l'object de
%notre travail mais plutôt celui de
%constater si, dans l'écoute, certaines
%caractéristiques de la dépression peuvent être améliorées par la
%restructuration de la sensibilité perceptuelle.
Les vertues de la musique sont reconnues dans la mythologie et le
monde des rites depuis les temps ancestraux (Chine, monde arabe
médiéval et Antiquité grecque avec l'idée de perfection, de beauté et d'harmonie  qui est rattachée).
Les formes d'utilisation
thérapeutique de la musique figurent même dans un ``traité de politique",
``Kitab as Syasa'' remontant à des documents syriens ou saabéens datant de  [\dots] de la fin du
\textsc{viii}\ieme\ siècle.  \textquote{La théorie des nombres
permettait de calculer l'\textbf{`harmonie'}} intégrable dans la
philosophie et les traités musicaux \autocite[ch. III, p.
96]{vrait_musicotherapie_2018}.
En définitive, et ce n'est pas négligeable, la reconnaissance par les  politiciens  et les
philosophes de la \textit{matière sonore} comme étant d'utilité
publique, laisse entendre que l'équilibre personnel
peut contribuer à une forme d' \textbf{`harmonie'} civique.


L'harmonie avec soi-même et avec l'autre se présente par conséquent comme synonyme d'équilibre
psychique, traduisant les formes d'écoute intérieure et
extérieure: elle pourrait être, selon notre hypothèse, visible,
 à travers les courbes (courbes aérienne et osseuse) des tests de capacité d'écoute étudiées
dans l'Etude clinique (Cf. Ch.4).

Ces remarques nous permettent mieux d'anticiper
l'application utile du test de Tomatis dans notre travail, relatant l'analyse
comparative des résultats individuels.
%relever les deux formes de perception utilisable


\subsection{Crédibilité actuelle de l'approche de la musicothérapie }


Au fil des siècles, de nombreux autres
textes ont évoqué les liens entre musique et médecine, de sa place dans les
rituels thérapeutiques et notamment en psychothérapie fin \textsc{xix}\ieme,
début \textsc{xx}\ieme\ siècle.
%\footnote{Voir \ref{musicothEtpsycho},
  %p. \pageref{musicothEtpsycho}.}
Selon Aurelia Sickert-Delin, la musicothérapie
psychologique doit être différenciée de celle dite médicinale qui
\enquote{\emph{exerce une action
énergétique, physiologique}} [\dots] avec \enquote{\emph{des effets curatifs}}
ainsi que de celle dite \enquote{\emph{musicale, artistique}}.
\enquote{\emph{L'artiste-musicien éveille l'\,``artiste intérieur'' que l'être
en souffrance porte en lui, pour lui permettre de s'auto-guérir [\dots] par
l'écoute, l'expression et la création.}}\autocite {viret:b}, Ch. 1 p. 14,
inédit de A. Sickert-Delinh.

 Ainsi, de fonctionnelle, analytique, mo\-da\-le,  à
struc\-tu\-rale, elle se retrouve actuellement
 à un tournant décisif où elle devient
 \textbf{ musicothérapie intégrative} tout en préservant ses racines
 séculaires dans le sens où elle permet un travail d'élaboration psychique dans une perspective de structuration identitaire \autocite[p.105]{vrait_musicotherapie_2018} et dans celui de l'intégration des données
neuroscientifiques.

% Selon M.
%Schneider \blockquote{elle cherche à sauvegarder et à fortifier la pure
%substance sonore de l'homme\autocite[Voir tome I, pp. 202--203]%
%	[M. Schneider, <<Le rôle  de la musique dans la mythologie et les rites
% des civilisations non européennes>>]{schaeffner.ea:histoire}.}





L'alliage de la musicothérapie avec les données actuelles de pointe en
science est-il faisable? Comment le réaliser pour obtenir plus de pertinence
dans sa crédibilité? Comment la doter de plus de crédibilité?
Les musicothérapeutes sont souvent musiciens mais conjugent plus
rarement dans leur profession
médecine ou neurosciences. De leur côté, les neuroscientifiques appuient
et renforcent la crédibilité de l'action du \textbf{`son'} sur notre cerveau, via
l'oreille, en démontrant ses effets par un moyen technique
\textbf{visuel} que représente par exemple l'IRMf. Mais, sauf quand ils sont musicien ou
musicothérapeute, leur découverte est plus rarement intégrée
directement dans leur pratique car hors contexte relationel d'une
séance, sans l'aspect intuitif et impalpable de cette forme de prise
en charge.
L'aspect fugace du son, de la musique, de ce médium volatil et
intemporel par
définition, ne semble pas apporter à tout un chacun le
même aspect concret que peuvent témoigner des supports
graphiques. Ceux-ci sont des
reflets d'un espace-temps du travail d'élaboration
psychique d'un patient, sur lesquels l'art-thérapie, par exemple, s'appuie et
trouve
ses sources.
Néanmoins, l'enregistrement sonore des séances
permet d'avoir un support plus concret et solide, comme le pratiquent Edith
Lecourt, ou Carole et Clive
Robbins.  \autocite {lecourt_les_2017}
Les séances sont écoutées, filmées pour une
analyse la plus objective possible; on donne
une forme et un sens aux sons recueillis pour les retracer dans le
parcours du patient (avec ou sans sa présence). Par contre, cette analyse  restera
tout de même très subjective de par sa nature (le son ne laisse pas
de trace) et de par la nécessité de la présence d'autres formes d'
écoutes (celles du thérapeute).
Comme l'exprime si justement Chistophe André \autocite[154]{van_eersel_cerveau},
même si le renfort d'études
scientifiques est irremplaçable, le travail avec le patient peut se représenter
tel un
explorateur du XVème siècle navigant sur des flots inconnus et
s'engageant sur d'autres terres, plus souvent guidé par des résultats
concrets que par des théories.
