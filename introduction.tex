\chapter{Introduction et hypothèse}
\begin{quotation}
\emph{Par le Son, le Silence du Non-Être vient à l'Être. Je suis
la musique que je fais ou écoute. La musique a la capacité d'harmoniser
les composantes d'une entité psychophysique pour qu'il soit ``bien
dans sa peau'' et ``bien dans son âme''}\,\footnote{Jacques Viret, \emph{B. A-BA de la musicothérapie}, \cite{Viret2007}.}.
\end{quotation}

\section{Introduction}

L'auteur exerce le métier de musicienne professionnelle et a  choisi de compléter sa formation en devenant musicothérapeute  ainsi que consultante à la méthode Tomatis.  



Elle exerce en cabinet privé et en cliniques psychiatriques. L'utilisation de ces deux axes de thérapies varie et s'adapte selon les lieux de travail. En institution, elle est exclusivement musicothérapeute. En cabinet
privé, elle a la liberté d'utiliser l'une des deux techniques ou de les fusionner.
C'est ainsi qu'à travers chaque cas particulier,
elle a modifié ses techniques de soins, et les ai fait évoluer, animée par beaucoup de curiosité! Se lancer dans cette écriture, c'est creuser et approfondir ces recherches. C'est prendre de la distance, observer, se remettre en question, être plus précis dans ses observations, plus systématique dans la manière de travailler. D'où ce questionnement présenté ici : 


\section{Hypothèse: un test pour mesurer les transformations de l'écoute}

L'écoute est-elle universelle ou personnelle ?
 Chaque être humain semble  globalement pareil à un autre; toutes les oreilles paraissent  avoir une anatomie similaire. Et pourtant, si l'on observe de près nos empreintes digitales, on s'aperçoit de l'unicité de chaque être. De même, chaque oreille peut être différente, différenciée et  fonctionner d'une autre façon;  par conséquent notre  écoute pourrait être à chaque fois particulière nous appartenir de façon singulière tout en étant perméable au changement.
Le temps imprime une évolution dans notre être intérieur et extérieur, nous sommes vivants dans nos mouvements psychologiques et physiques. Des traces devraient en être aussi visibles dans l' écoute. Il est plausible de s'en poser la question.
En admettant ainsi qu'il n'y ait pas de statisme mais au contraire une transformation dans l'écoute, ce changement devient identifiable, \textit{remarquable}, unique (dans le sens personnel) à travers nos expériences de vie.  Un test pourrait être un moyen d'évaluer les modifications supposées de cette écoute. 


\subsection{L'hypothèse}

		\begin{itemize}
			\item Qu'est-ce qui est proposé ? 
			Utilisation d'un test d'écoute 
			\begin{itemize}
				\item Constat de l'évolution de l'écoute à travers le travail fait en musicothérapie
			\end{itemize}
		\end{itemize}

Un test spécifique d'écoute serait  un 
moyen de donner  forme à l'écoute, comme une fenêtre avec une  image, en quelque sorte une \emph{vision de l'écoute}. Avec  des séances de musicothérapie, dont  le son est l'outil, on pourrait en mesurer l'impact sur la façon d'écouter du patient. 
Il  donnerait selon des paramètres de référence, des indices différenciés, propres à chacun sur la perception du son, avec un comparatif en début et fin de thérapie. Il serait  une source d'informations. Il permettrait de visualiser plus objectivement
les changements, la transformation de l'écoute et constater s'il y a un parallélisme dans la transformation psychique  de la personne prise en thérapie.


 L'objet de cette étude est de faire un constat
qui pourrait donner matière à réflexion.
La problématique : 

A cet effet, nous utiliserons un test spécifique à Tomatis, qui se fera dans le cadre d'une musicothérapie. Aucun support de cette méthode n'interviendra, ni \textsl{Oreille
électronique} ni musiques préparées et filtrées. Nous expliquerons la raison de   leur importance mais nous n'en ferons aucun usage. L'idée est de mettre à profit cette forme de test et de  s'en tenir à ce support
graphique, visible, un``dessin'', une image utile, utilisable, tangible,
presque ``palpable'', avec des critères
d'interprétations.

La musicothérapie est une pratique ancestrale. Il suffit de considérer la très grande  place de la musique dans les mythologies et dans les rites. Selon M.Schneider
			
\textsl{elle cherche à sauvegarder et à fortifier la pure substance sonore de l'homme."}\footnote{M.Schneider," Le rôle de la musique dans la mythologie et les rites des civilisations non européennes", Histoire de la musique, tome I, Paris, Gallimard, Encyclopédie La Pléiade, 1960, pp.202-203}

\paragraph{La musique, la matière sonore} est utilisée dans le but d'un "processus
thérapeutique, pour entrer en communication avec soi-même et pouvoir ensuite mieux percevoir le monde qui nous
entoure, communiquer et s'exprimer"\footnote{%
 \href{http://www.musictherapy.ch/fr/musicotherapie/quest-ce-que-la-musicotherapie/}{musictherapy.ch}}.

L'aspect fugace du son, de la musique, de ce médium volatil par
définition, ne pourra apporter, comme en art-thérapie, le
même aspect concret que peuvent témoigner des supports graphiques,
visuels, reflets d'un espace-temps lors d'un travail d'élaboration
psychique d'un patient. Nous gageons et faisons l'hypothèse que l'action et l'impact de la
musicothérapie sur la façon d'entendre pourraient être perçus, d'une certaine façon plus
clairement, \textsl{ objectivés}, comme saisis par l'\oe il de l'objectif d'un appareil
photographique.
Si nous ne sommes pas médecins ni issus d'un milieu scientifique, en tant que musicothérapeute, nous ne pouvons pas prétendre à l'utilisation des outils
 tel que l'IRMfct; en contre-parti, toutes ces recherches en
neurosciences éclairent, appuient et renforcent la crédibilité de l'action
majeure du son sur notre cerveau, via l'oreille, en démontrant ses effets par cet intermédiaire technique visuel.  Emmanuel Bigand, chercheur, professeur de
psychologie cognitive à l'Université de Bourgogne, relève l'aspect
paradoxal de la musique, la complexité de sa structure sonore sans
fonction biologique précise mais faisant réagir fortement l'être
humain.\footnote{\cite{AuteurInconnu2011}, chap.~3 p.~35, "Vous avez l'oreille musicale".}.  Notre cerveau peut être activé autant par
la musique que par la nourriture ou la drogue. Et pourtant la musique,
qui est un élément artificiel en soi, n'a aucun rôle dans notre survie ni dans
notre nutrition.


\section{Plan du travail}

Dans la première partie, nous aborderons l'aspect théorique : l'écoute, le son, l'oreille, le test d'écoute, les différents tests d'écoute en musicothérapie.  Ensuite, nous expliquerons  la méthode Tomatis
et puis, beaucoup plus en détail, son test d'écoute.

En deuxième partie ce sera l'aspect clinique : les tests d'écoute réalisés  avec deux groupes de patients en parallèle.

Et finalement suivront la vérification de l'hypothèse, les conclusions et interrogations. 
Quelques réflexions autour notre propre expérience en cabinet privé, étayé d'un cas d'étude.
