\chapter{Introduction et hypothèse}
\begin{quotation}
\emph{Par le Son, le Silence du Non-Être vient à l'Être. Je suis
la musique que je fais ou écoute. La musique a la capacité d'harmoniser
les composantes d'une entité psychophysique pour qu'il soit ``bien
dans sa peau'' et ``bien dans son âme''}\,\footnote{Jacques Viret, \emph{B. A-BA de la musicothérapie}, \cite{Viret2007}.}.
\end{quotation}

\section{Introduction}

J'exerce le métier de musicienne professionnelle depuis de nombreuses
années. Par la suite, j'ai complété ma formation en devenant musicothérapeute et également consultante à la méthode Tomatis, 
études suscitées par mon intérêt toujours plus croissant au sujet du développement de la personne à travers l'outil 
que nous offre la musique : le son.



Tout en poursuivant mon activité de musicienne, selon les lieux de travail (cabinet privé, cliniques), les deux dernières formations ont pris des valeurs différentes dans la pratique. En institution, je serai exclusivement musicothérapeute. En cabinet
privé, j'aurai la liberté de choix entre les deux méthodes.
C'est ainsi qu'à travers chaque cas particulier,
j'ai modifié mes techniques de soins, et les ai fait évoluer,
pour parfois les fusionner. 
Ce qui m'a animée ? ma curiosité! Se lancer dans cette écriture, c'est creuser et approfondir mes recherches. C'est prendre de la distance, observer, se remettre en question.  Pourrai-je être plus précise dans mes observations? Comment clarifier et systématiser ma manière de travailler?


\section{Hypothèse: un test pour mesurer les transformations de l'écoute}

L'écoute est-elle universelle ou personnelle ? Chaque être humain semble  globalement pareil à un autre; toutes les oreilles peuvent sembler pareilles. Et pourtant, si l'on observe de près nos empreintes digitales, on s'aperçoit de l'unicité de chaque être. De même, chaque oreille peut être différenciée, et par conséquent aussi l'écoute. Le temps imprime des changements dans notre être intérieur et extérieur, et nous sommes vivants dans nos mouvements psychologiques et physiques. Des traces devraient en être aussi visibles dans l' écoute.
En admettant ainsi qu'il n'y ait pas de statisme mais au contraire une transformation dans la façon d'écouter, ce changement devient identifiable, \textit{remarquable}, unique (dans le sens personnel) à travers nos expériences de vie.  Un test pourrait être un moyen d'évaluer les modifications supposées de cette écoute. 


\subsection{L'hypothèse}

Un test spécifique d'écoute serait  un 
moyen de donner  forme à l'écoute, comme une fenêtre avec une  image, en quelque sorte une \emph{vision de écoute}. Avec  des séances de musicothérapie, dont  le son est l'outil, on pourrait en mesurer l'impact sur la façon d'écouter du patient. 
Il  donnerait selon des paramètres de référence, des indices différenciés, propres à chacun sur la perception du son, avec un comparatif en début et fin de thérapie. Il serait  une source d'informations. Il permettrait de visualiser plus objectivement
les changements, la transformation de l'écoute et constater s'il y a un parallélisme dans la transformation psychique  de la personne prise en thérapie.


 L'objet de cette étude est de faire un constat
qui pourrait donner matière à réflexion.

A cet effet, nous utiliserons un test spécifique à Tomatis, qui se fera dans le cadre d'une musicothérapie. Aucun support de cette méthode n'interviendra, ni \textit{\textsl{Oreille
électronique}} ni musiques préparées et filtrées. Nous expliquerons la raison de   leur importance mais nous nous y étendrons que brièvement. L'idée est de s'en tenir à un support
graphique, visible, presque ``palpable'', avec des critères
d'interprétations, un``dessin'', une image utile,
utilisable, tangible.


\paragraph{La musique, la matière sonore}est utilisée dans le but d'un processus
thérapeutique, pour entrer en communication avec soi-même et pouvoir ensuite mieux percevoir le monde qui nous
entoure, communiquer et s'exprimer\footnote{%
Voir ce lien sur \href{http://www.musictherapy.ch/fr/musicotherapie/quest-ce-que-la-musicotherapie/}{musictherapy.ch}}.
L'aspect fugace du son, de la musique, de ce médium volatil par
définition, ne pourra apporter, comme en art-thérapie, le
même aspect concret que peuvent témoigner des supports graphiques,
visuels, reflets d'un espace-temps lors d'un travail d'élaboration
psychique d'un patient. Nous gageons et faisons l'hypothèse que l'action et l'impact de la
musicothérapie sur la façon d'entendre pourraient être perçus, d'une certaine façon plus
clairement, \textsl{ objectivés}, comme saisis par l'\oe il de l'objectif d'un appareil
photographique.
En tant que musicothérapeute, nous ne pouvons pas prétendre à l'utilisation des outils
scientifiques tel que l'IRMfct; par contre, toutes ces recherches en
neurosciences appuient et renforcent la crédibilité de l'action
majeure du son sur notre cerveau.  Emmanuel Bigand, professeur de
psychologie cognitive à l'Université de Bourgogne, relève l'aspect
paradoxal de la musique, la complexité de sa structure sonore sans
fonction biologique précise mais faisant réagir fortement l'être
humain.\footnote{\cite{AuteurInconnu2011}, chap.~3 p.~35, "Vous avez l'oreille musicale".}.  Notre cerveau peut être activé autant par
la musique que par la nourriture ou la drogue. Et pourtant la musique,
qui est un élément artificiel en soi, n'a aucun rôle dans notre survie ni dans
notre nutrition.


\section{Plan du travail}

Dans la première partie, nous aborderons l'aspect théorique : l'écoute, le son, l'oreille, le test d'écoute, les différents sortes de tests d'écoute en musicothérapie.  Ensuite, nous expliquerons  la méthode Tomatis
et puis, beaucoup plus en détail, son test d'écoute.

En deuxième partie ce sera l'aspect clinique : les tests d'écoute réalisés  avec deux groupes de
comparaison.

Et finalement suivront la vérification de l'hypothèse, les conclusions et interrogations. 
Quelques réflexions autour notre propre expérience en cabinet privé, étayé d'un, voire deux cas d'étude.
