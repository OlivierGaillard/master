%\chapter{Alfred Tomatis}


\subsection{ 
Le test d' Alfred Tomatis}

\paragraph{Le test d'écoute TSLT: conçu par Tomatis,
   cet appareil, appelé ``\emph{Hearing Test}'', permet de traduire l'écoute par 
   un graphique. La
   qualité de l'écoute est objectivée.}

 
est basé sur des sons purs et leur
 reconnaissance, ce qui permet d'objectiver la qualité de l'écoute; il a été créé dans les années 50, avec un
  appareil contenant un générateur de fréquences qui émet des sons
  purs s'étalant de \SIrange{125}{8000}{\Hz}, d'octave en octave, en passant par les valeurs
\SIlist{1500;3000;6000}{\Hz}, et dont l'intensité, peut varier de 5 en \SI{5}{\dB}, de \SIrange{10}{100}{\dB}. 
Ceux-ci sont transmis au moyen d'une
  transmission aérienne (avec un casque) et osseuse (avec un vibrateur). Ces sons sont à identifier et à
  signaler par le patient en levant la main du côté où il l'entend. On
  varie le volume (de très faible à fort). En procédant de manière
  simple, aucune performance, aucune note à transcrire, n'est demandée
  au patient, ni une 
  recherche de sens ou d'association d'idées.
 
 Détecter un son précis, l'entendre, reconnaitre sa présence et même à très
 faible intensité, le situer dans l'espace, à droite, à gauche, au
 milieu ou ailleurs,  apporte une particularité qui est celle de  donner simultanément  \textbf{une information
   physiologique et psychologique} sur le patient.
 C'est pour cette raison qu'il a été nommé
 audio (oreille)-psycho-phonologique (voix). \footnote{\footnote{Avec le professeur Tomatis: formation suivie dès 1995, Boulevard de Courcelles, Centre de l'écoute 
Tomatis à Paris; puis en 2009/11/13/15 avec V. Gas, V. Drouot et J.P. Granier, formateurs et consultants. Source: site internet officiel: \cite{tomatis.com}.}
.}
  
  Dans son ouvrage \emph{Éducation et
    Dyslexie}\autocite{tomatis:education} il
  a présenté ce test d'écoute comme étant très important car il détermine les
  possibilités d'écoute du sujet : auto-écoute et écoute de
  l'autre\footnote{<<\,Considérations sur le test d'écoute\,>>. Propos
  	recueillis au cours du \textsc{iii}\ieme\ congrès international
  	d'audio-psycho-phonologie (Anvers 1973) à la suite d'un entretien recueilli par B. Auriol
  	avec le professeur Tomatis. \autocite{auriol_stress}.}. 
   Les procédures de passation du test semblent
  se rapprocher d'un audiogramme classique mais il n'en est rien car
  le but d'un audiogramme classique  est de mettre en évidence un
  trouble de l'audition.\footnote{Cf. \ref{passation}, p. 
    \pageref{passation}.}.
  

  On détecte donc si le patient désire ou non se servir des sons
  qu'il a à sa disposition. Il a peut-être la possibilité d'entendre un large spectre de
  sons mais ne souhaite pas, ne veut pas les écouter. Les raisons sont multiples et en général d'ordre psychologique (traumatismes,
  expériences négatives). Le cerveau aura le
  pouvoir d'assourdir certaines fréquences, de les masquer jusqu'à les faire disparaître peu à peu de
  son champ d'écoute. Par protection, par réflexe de survie, le
  patient choisit de les
  annihiler alors que les sons sont là, réels, et que  l'oreille peut physiquement les collecter. Le cerveau crée ce
  que l'on appelle des distorsions
  d'écoute\autocite{tomatis:education}.

  
  Ce test est un test spécifique destiné à fournir une traduction
graphique de l'écoute, et d'en objectiver la qualité. Plus précisément,
il  est fait pour :
\begin{itemize}
\item constater la posture d'écoute de la personne ainsi que de vérifier
la fonction de dynamisation, la fonction vestibulaire,
et la fonction d'écoute.
\item observer les modifications et les évolutions des courbes
  aériennes et osseuses au cours
de la thérapie.
\end{itemize}

Cette mise en évidence des seuils d'écoute est une forme d'objectivité
--- quoique cette notion comme dit précédemment, est très complexe
avec le son ---; mais en un même temps, et cela peut paraître paradoxal, il
est possible d'analyser par ces résultats le potentiel d'écoute de
chaque patient en particulier.




Tomatis a défini la «courbe d'écoute idéale», courbe qui correspond à l'oreille absolue
des chanteurs et des musiciens,  avec  le ténor italien Enrico
Caruso (1873--1921) dont il a analysa la voix à partir des enregistrements
de ses vocalises sur vinyles. Caruso représentait la courbe auditive
optimale dont il décida de se référer.

\begin{figure}
	\centering
	\includegraphics[width=0.7\linewidth]{images/courbeideale.jpg}
	\caption{Courbe idéale}
	\label{fig:courbeideale}
      \end{figure}

      
Sur le plan de la physique pure, elle indique les réponses de l'oreille
lorsque celle-ci fonctionne bien. Elle répond en fait à la courbe
de Wegel dite ``courbe en citron", inversée.\footnote{%
		Voir l'annexe \ref{acoustique} p. \pageref{acoustique}
		 pour cette partie technique.}.

               

L'acquisition de cette courbe idéale correspond à l'\textsl{harmonisation}
du jeu de deux muscles de l'oreille moyenne. Ce jeu
permet de régler en permanence la pression interne au niveau du
labyrinthe.
Lorsque l'interprétation des informations transmises à l'oreille est
erronée, il s'agit donc  de
distorsions d'écoute, déjà citées plus haut. Cette distorsion est liée au dysfonctionnement
de ces deux muscles de l'oreille moyenne dont le rôle est de permettre l'arrivée
harmonieuse du son dans l'oreille interne, puis au cerveau. Car, lorsque
le message sensoriel est altéré, le cerveau se protège en déclenchant
des mécanismes d'inhibition de l'écoute. On naît
avec ce potentiel mais celui-ci s'altère parfois avec les difficultés
de la vie et on introduit des distorsions
pour se défendre contre certaines agressions du monde extérieur. 

Sur le plan du test d'écoute, on remarquera
alors des distorsions, des manques par rapport à la courbe dite 
idéale.
Et lorsqu'il n'y a pas de distorsions, on parle d'harmonie. L'harmonie
est la régulation des émotions, l'équilibre entre son écoute
intérieure et extérieure. On la visualisera sous la forme de
courbes continues et parallèles.
Ces paramètres sont importants et nous reviendrons plus en détail sur le test d'écoute :(8.3.2)
%\enquote{\emph{L'oreille a un
%psychisme\autocite[{tomatis:loreille}.}} 




  




 




  


Pour aller plus avant dans le noyau du thème abordé, nous considérons utile
l'approfondissement de certaines notions.
\section{Méthode et test d'écoute}

Par {\textit{l'audio-psycho-phonologie}}, on entend l'étude des
phénomènes auditifs, phoniques et psychologiques et leurs anomalies.
De ces dernières,  dérive la mise en place d'un processus pédagogique
et/ou thérapeutique pouvant
utiliser plusieurs techniques.
Une de ces techniques,
  appelée
\label{outil_oreille_electro}
\textit{Oreille Electronique}, utilise
un système appelé \textit{ la
bascule} \autocite{escera-key},comme déjà cité plus haut (Cf.3. 1) permettant de créer une alternance entre deux conditions perceptives
du même message sonore, avec un passage soudain et imprévu de fréquences graves à des
fréquences aiguës.
Cette application favorise une amélioration naturelle \emph{d'interprétation du message
sensoriel}, répondant à des objectifs rééducatifs, par ailleurs en
interaction avec la psycho-neuro-immunologie (\gls{PNI})\footnote{Cf. Glossaire.}, elle-même sensible à
l'impact des événements psychiques sur le système immunitaire.
Cette conception intégrative de l'homme met en interaction toutes les
dimensions corporelles et psychologiques, dont les émotions et les cognitions.


``L'effet Tomatis'' est constitué par les principes suivants:
\begin{itemize}
	\item La voix est soumise à l'oreille, c'est-à-dire la voix ne contient que ce que l'oreille entend.
	\item Toute modification de l'audition implique immédiatement
          et inconsciemment une
          modification de la voix.
	\item Il est possible de transformer l'émission vocale par une stimulation
auditive
		entretenue pendant un certain temps (loi de
               ``\gls{rémanence}'')\footnote{Cf. Glossaire.}.
%   \begin{enumerate}
%\item lualatex master
%\item makeglossaries master)
%\item lualatex master
%\end{enumerate}

\end{itemize}

Dans sa globalité, l'``effet Tomatis'' se manifeste par une action
simultanée sur les fonctions de
l'oreille en touchant le système nerveux central (SNC) (coordination
                motrice et équilibre), par l'intermédaire du système
                vestibulaire.
                De même, cet ``effet Tomatis'' agit aussi sur certains troubles
                neurophysiologiques et  joue un rôle de dynamisation cérébrale et corporelle
                par des fréquences spécifiques.


Il serait important d'offrir une vision plus ample de
l'articulation entre l'approche de Bekésy et l'approche de Tomatis,
raison pour laquelle nous présenterons les différences conceptuelles de
base entre les deux chercheurs.
En effet, le concept Tomatis prouve le bien-fondé de ce test que nous
avons choisi pour ce travail.
Il est l'un des aboutissements de
ses recherches car il puise ses racines dans l'audiologie et s'en
démarque pour les raisons que nous allons voir succintement.
\paragraph{Les différences conceptuelles de la physiologie auditive
  entre Bekésy et Tomatis}.

En bref, dans  l'approche de von\textbf{ Bekésy} (Budapest 1899 -- Honolulu 1972,
physicien américain d'origine hongroise) ses
recherches en acoustique concernant les techniques de communication
téléphonique l'amenèrent à s'intéresser au problème de l'audition et à
élaborer des modèles de fonctionnement de l'oreille. Il élucida en
particulier le rôle de la membrane basilaire, et ses découvertes
permirent d'améliorer les traitements de la surdité (PN, prix
Nobel de physiologie 1961).
Sa vision affirme que la fonction principale de l'oreille (voir
Fig. 3.1)
consiste à transmettre les sons de manière passive, au même titre qu'un micro et le rôle des osselets
est limité à sa simple transmission du
son. Il avait déjà énoncé cette loi en 1923, et elle a été adoptée
universellement dans les sciences physiologiques.


En divergence avec G. Békésy, \textbf{Tomatis} oppose la conception de la
physiologie auditive comme \textbf{active} (Cf. Fig. 3.2) et non
passive.\footnote{Cf. Annexe A. 1. 1. Anatomie de l'oreille et sa physiologie}
Son originalité réside ainsi dans la transmission du son
au niveau de l'oreille moyenne et interne:
le \textbf{tympan}, dans son rôle de transmetteur dans l'oreille
          moyenne, effectue --grâce aux muscles de l'étrier et du marteau--
		un\textbf{ travail de visée} en ciblant les sons. Il
se tend
		pour se mettre en résonance avec les sons à percevoir.
                Il fait aussi un autre travail qui est celui de \textbf{sélectionner des
sons
		pour se protéger}. Ainsi le tympan se tend et se détend,
              amortit et adapte
l'intensité
		sonore inondant  l'oreille interne.


\begin{figure}
	\centering
	\includegraphics[width=1.0\linewidth]{images/Cochleederoule_bas.jpg}
	\caption[Modèle de Békésy]{Cochlée selon Békésy/ Tomatis Développement SA, 2012}
	\label{fig:cochleederoulebas}
\end{figure}


 \begin{figure}
	\centering
	\includegraphics[width=1.0\linewidth]{images/Cochleederoule_haut.jpg}
	\caption[Cochlée selon Tomatis]{Cochlée selon Tomatis}
	\label{fig:cochleederoulehaut}
      \end{figure}
      Tomatis attribue une grande importance à l'analyse
          fréquentielle au niveau de la\textbf{ cochlée}:
l'onde acoustique arrivant par le canal auditif
jusqu'au tympan  excite la membrane tympanique, donc l'os de la caisse
du tympan \autocite {tomatis_conf1972}.%auriol:cle
 A l'instar d'une
peau de tambour qui fait chanter le bois auquel elle est attachée,
c'est toute la boîte crânienne qui est inondée de sons et en particulier
l'oreille interne. La cochlée, de par sa grande densité, capte les sons
et résonne comme du cristal.\footnote{La transmission du son par l'os est de
5000 $m/s$.}.
Les fréquences qui forment les sons vont ainsi exciter les cellules
ciliées la tapissant, tel un piano enroulé.
Chaque fréquence se dirige \textbf{instantanément }et
naturellement vers la cellule ciliée correspondante grâce à la
forme du limaçon, produisant ainsi un tri fréquentiel
instantané.


Le rôle des tourbillons est de \textbf{s'adapter aux bruits}
et non de transmettre les sons.
Lorsque l'intensité des sons aug\-men\-te,
l'ex\-ci\-ta\-tion des cellules ciliées provoque des perturbations liquidiennes
dans l'oreille interne, c'est-à-dire des tourbillons. Ceux-ci se propagent
et sont amortis par l'étrier. Si les sons atteignent une intensité
dangereuse pour les cellules ciliées, l'étrier réagit fortement et
entraîne une réaction du marteau qui modifie la tension du tympan.
A son tour, le tympan, relâché, amortit le volume sonore transmis
à l'oreille interne, comme la paupière qui se ferme quand la lumière
est trop intense.


\begin{quotation}
	``Le tympan se met dans un certain état de tension pour jouer le
	rôle d'un diapason qui fait vibrer toute la boîte crânienne
	par l'intermédiaire du \emph{sulcus tympani}.
	\emph{C'est toute la boîte crânienne qui vibre et qui transmet le son à
la vésicule labyrinthique et non à la chaîne ossiculaire que l'on a l'habitude
de considérer comme le véhicule du son.} La chaîne ossiculaire est un ensemble
qui
	joue le rôle d'\textbf{adaptateur, de régulateur et non de transmetteur}.'' \autocite {tomatis_conf1972}

\end{quotation}


\paragraph{Evolution des hypothèses inhérentes au système d'écoute}



Dans la chronologie des découvertes scientifiques,  l'hypothèse de la \textbf{``batterie de
résonateurs''} de Von Helmholtz (1863)  avait été remplacée par la
théorie dite de  --``l'onde propagée'' -- ou --``des
tourbillons''-- de Bekésy (1928).
Cependant, les travaux de Leipp (1970, 1976), Tomatis (1972), Sellick et coll.,(1982), Wilson (1983),
  Johnstone (1986), Dancer et
  Franke (1987) ont revalidé la position de Von Helmholtz.\autocite[p 24---28,
  ch. 1]{auriol:cle}

  Parmi les nombreux apports de Zwicker, \autocite[p 84]{auriol:cle} nous retenons que notre système d'écoute peut rendre
\textit{attentif} ou\textit{ sourd} à certaines fréquences ou à certains patterns
spectraux, mais peut aussi construire des sons fantômes comme dans les --- \textit{``sons de Zwicker''} --- (1964).
Ainsi nous disposons de quelques éléments-clés pour la compréhension
sur le phénomène de la \textbf{``distorsion audiométrique''} \autocite
{auriol:cle}, sujet déjà abordé et dont il sera également question dans l'approche
clinique.



De même, l'équipe de chercheurs menés par Christine Petit (2012, 2019) relève
actuellement
 le rôle encore très mystérieux de la \textbf{cochlée} sur notre audition:
\nomenclature{cochlée}{Anatomie : organe de l'audition, appareil sensoriel,
en forme de spirale, la cochlée, incluse dans l'os du rocher,
forme le limaçon membraneux, se situe dans l'oreille interne et
permet de déceler des sons extrêmement faibles, de discriminer des fréquences
et de masquer des sons faibles par des sons forts.}
<<\,C'est une sorte de minuscule appareil électroacoustique capable
de discréminer des sons extrêmement faibles, capable de \emph{masquer
les sons faibles par des sons forts}, pouvant \textbf{distordre les
sons,} et \emph{capable d'élaborer un traitement extrêmement
sophistiqué des sons} ``. \,>> \autocite{petit_lookscience}   \footnote{Christine Petit, titulaire de la chaire Génétique et
Physiologie cellulaire au Collège de France}
%, dans un entretien (2012),
%réalisé par Laurent Salters et Vincent Gaullier, basé sur la revue
%\textit{``Look at science''}\,>>}

Somme toute, on peut penser que Tomatis a été très à l'avant-garde
dans ses recherches.

D'autres études récentes prouvent l'effet anxiolytique lors de
l'application de cette
méthode\footnote{\emph{Tomatis Research and Publication} www.tomatisassociation.org}  avec Flehming
I., 1996 \footnote{Dr. med. Inge Flehming,
	neurologue, neuropédiatre \emph{``Grundsatz-Gutachten zur Behandlungsmethode
		nach Prof. Tomatis''}. Voir
\href{http://www.analytische-hoertherapie.de/uploads/tx\_templavoila/Grundsatzgu
tachten\_zur\_Behandlungsmethode\_nach\_Prof.\_Tomatis.pdf}{le site
web.}} et Du Plessis W. F. and Van Jaarsveld P. E., 1988.
 \footnote{Du Plessis W. F. and Van Jaarsveld P. E. ,1988, \textit{``Troubles
psychologiques''} (Université de Potchefstroon
- Afrique du Sud).
	\textit{``Audio-psycho-phonology : A comparative outcome study on anxious
primary school pupils'''},  Afr. Tydskr. Sielk. 19818 (4) 144--151. Du Plessis, W.F., Burger, S. (2001) [\ldots]
	\emph{A pilot study involving the Tomatis method.}, Sud Africa J.
Psychol.}
