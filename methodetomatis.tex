




\subsection{Synthèse de la méthode} 

\emph\textbf{{L'audio-psycho-phonologie}}
est une thérapie de l'écoute avec un
 outil électronique appelé
\label{outil_oreille_electro}
Oreille Electronique et l'utilisation d'une technique, la 
bascule, qui permet de créer une alternance entre deux conditions perceptives 
du même message sonore avec un passage soudain et imprévu de fréquences graves à des 
fréquences aiguës. Cette façon de travailler permet au
cerveau d'améliorer naturellement \emph{l'interprétation du message
sensoriel}.
Cette méthode répond ainsi à des objectifs éducatifs et rééducatifs
correspondant au concept de l'interdisciplinarité, telle celle représentée en 
psycho-neuro-immunologie,
(PNI).\footnote{La PNI étudie 
l'impact des événements psychiques sur le système immunitaire. Elle repose sur 
la mise en évidence d'interrelations entre le système
nerveux central, le système neuroendocrinien et le système immunitaire.
C'est une approche interdisciplinaire incorporant des données de la
psychologie, de la neuroscience, de la neurologie, dont l'endocrinologie
et l'immunologie. (entre autres) Source : Wikipédia, février 17.}
Sa conception est une conception intégrative
de l'homme, mettant en interaction toutes ses dimensions corporelles
et psychologiques dont 
les émotions et les cognitions.


Il a énoncé les lois qui constituent ``l'effet Tomatis'' : 
\begin{itemize}
	\item La voix ne contient que ce que l'oreille entend.
	\item Si l'on modifie l'audition, la voix est immédiatement et 
inconsciemment
		modifiée.
	\item Il est possible de transformer la phonation par une stimulation 
auditive
		entretenue pendant un certain temps (loi de rémanence).
\end{itemize}

Cet effet a une action simultanée sur trois fonctions essentielles de
l'oreille dont


\textbf{ l'audition} :  lorsque l'on s'entend, on peut mieux se
connaître et se structurer.


Il a  aussi une action sur le le\textbf{ SNC (système 
nerveux central) }qui touche l'équilibre et la coordination  par 
l'intermédiaire
		du vestibule, équilibrant certains troubles
                neurophysiologiques.

                
                Et enfin, il a une fonction de \textbf{stimulation} :
               le cerveau est dynamisé par des fréquences spécifiques
          et par là-même le corps tout entier.




\emph{\textbf{  Conception différente de la physiologie auditive}}

Tomatis  s'oppose sur plusieurs points à G. Békésy\footnote{Georg Békésy, prix 
Nobel de physiologie 1961, l'oreille ne sert qu'à transmettre les sons de manière passive
comme peut le faire un micro et le rôle des osselets 
se limite à la transmission du
son. } au sujet de la
physiologie auditive, qu'il considère comme active et non passive. \footnote{Cf. Annexe sur l'anatomie de l'oreille et sa physiologie}
Son originalité réside ainsi dans sa conception de la transmission du son
au niveau de l'oreille interne. 



\begin{itemize}
	\item l'oreille moyenne et son rôle de transmetteur; le tympan, grâce aux muscles de l'étrier et du marteau, 
fait
		un\textbf{ travail de visée} en ciblant les sons à
                écouter. Il 
se tend
		pour se mettre en résonance avec les sons à percevoir
                et fait aussi un autre travail qui est celui de \textbf{sélectionner des 
sons
		pour se protéger} : la tension tympanique se détend pour amortir 
l'intensité
		sonore qui inonde l'oreille interne. 

	\item l'analyse fréquentielle au niveau de la cochlée, 
\end{itemize}








\begin{figure}
	\centering
	\includegraphics[width=0.7\linewidth]{images/Cochleederoule_bas.jpg}
	\caption[Modèle de Békésy]{Modèle de Békésy}
	\label{fig:cochleederoulebas}
\end{figure}


 \begin{figure}
	\centering
	\includegraphics[width=0.7\linewidth]{images/Cochleederoule_haut.jpg}
	\caption[Cochlée selon Tomatis]{Cochlée selon Tomatis}
	\label{fig:cochleederoulehaut}
      \end{figure}

      
D'après son hypothèse,\footnote{Conférence au IIème Congrès International d'Audio-Psycho-Phonologie
Paris 1972:  \emph{Nouvelles théories sur la physiologie auditive}.} les sons arrivent bien par le canal auditif
jusqu'au tympan. L'onde acoustique excite la membrane tympanique et
par voie de conséquence, l'os de la caisse du tympan. 
A l'instar d'une
peau de tambour qui fait chanter le bois auquel elle est attachée,
c'est toute la boîte crânienne qui inondée de sons et en particulier
l'oreille interne. Celle-ci, de par sa grande densité, capte les sons
et résonne comme du cristal\footnote{La transmission du son par l'os est de 
5000 $m/s$.}.
Les fréquences qui forment les sons vont ainsi exciter les cellules
ciliées qui tapissent la cochlée, tel un piano enroulé.
{\textbf{L'analyse multifréquentielle ne se pose plus} 
avec sa théorie: chaque fréquence se dirige instantanément et
naturellement vers la cellule ciliée qui lui correspond. 
C'est grâce à la forme particulière du limaçon qu'il y a un tri fréquentiel 
instantané.
Le son de fréquence identique s'installe toujours au même endroit, sur une 
ligne isofréquentielle, qui est une tranche perpendiculaire à l'axe.
Le rôle des tourbillons est de s'adapter aux bruits
et non de transmettre les sons.
Lorsque l'intensité des sons aug\-men\-te,
l'ex\-ci\-ta\-tion des cellules ciliées provoque des perturbations liquidiennes
dans l'oreille interne, c'est-à-dire des tourbillons. Ceux-ci se propagent
et sont amortis par l'étrier. Si les sons atteignent une intensité
dangereuse pour les cellules ciliées, l'étrier réagit fortement et
entraîne une réaction du marteau qui modifie la tension du tympan.
A son tour, le tympan, relâché, amortit le volume sonore transmis
à l'oreille interne, comme la paupière qui se ferme quand la lumière
est trop intense.


\begin{quotation}
	Le tympan se met dans un certain état de tension pour jouer le
	rôle d'un diapason qui fait vibrer toute la boîte crânienne
	par l'intermédiaire du \emph{sulcus tympani}. 
	\emph{C'est toute la boîte crânienne qui vibre et qui transmet le son à 
la vésicule labyrinthique et non à la chaîne ossiculaire que l'on a l'habitude 
de considérer comme le véhicule du son.} La chaîne ossiculaire est un ensemble 
qui
	joue le rôle d'adaptateur, de régulateur et non de transmetteur. La
	conduction du son par l'air puis par l'os doit donc
	être étudiée d'une façon complémentaire afin que l'on
	puisse déterminer par la suite la posture d'écoute du sujet%
	\footnote{Entretien réalisé par B. Auriol avec Tomatis, Anvers 
1973.}.\pdfmargincomment{quel livre?}

\end{quotation}


 Notons que le  rôle important et
particulier 
de la \textbf{cochlée} sur notre audition fait encore à l'heure
actuelle l'objet de recherches intenses
par l'équipe de Christine Petit qui relève
son aspect encore très mystérieux:
\nomenclature{cochlée}{Anatomie : organe de l'audition, appareil sensoriel, 
en forme de spirale, la cochlée, incluse dans l'os du rocher, 
forme le limaçon membraneux, se situe dans l'oreille interne et 
permet de déceler des sons extrêmement faibles, de discréminer des fréquences
et de masquer des sons faibles par des sons forts.}
<<\,C'est une sorte de minuscule appareil électroacoustique capable
de discréminer des sons extrêmement faibles, capable de \emph{masquer
les sons faibles par des sons forts}, pouvant \emph{distordre les
sons,} et en conséquent, \emph{capable d'élaborer un traitement extrêmement
sophistiqué des sons}%
% OGA: référence stp. Salters et Gaullier ont publié? émission?
\footnote{Christine Petit, titulaire de la chaire Génétique et
physiologie cellulaire au Collège de France, entretien en novembre 2012, 
réalisé par Laurent Salters et Vincent Gaullier, 
Look at science : le système sensoriel auditif confirme 
lors d'un entretien réalisé en 2012 le rôle indéniable de la cochlée.\,>>}.


 A partir des publications, études et recherches récentes \footnote{\emph{Tomatis Research and Publication} www.tomatisassociation.org}    faites sur la
 méthode Tomatis, nous avons notamment
 relevé celles du Dr. med. Inge Flehming, neurologue et 
pédiatre,\footnote{Dr. med. Inge Flehming,
	neurologue, neuropédiatre, texte publié en allemand
	en 1996, \emph{``Grundsatz-Gutachten zur Behandlungsmethode
		nach Prof. Tomatis''}. Voir 
\href{http://www.analytische-hoertherapie.de/uploads/tx\_templavoila/Grundsatzgu
tachten\_zur\_Behandlungsmethode\_nach\_Prof.\_Tomatis.pdf}{le site web.}}
 celles du Docteur Du Plessis (études sur l'anxiété en
 milieu scolaire et universitaire) qui  attestent de la pertinence de ce
 concept et de cette forme de pédagogie. \footnote{Troubles 
psychologiques : Etude du Plessis (Université de Potchefstroon
- Afrique du Sud).}  \footnote{Du même auteur, une autre étude démontre qu'après 14,3
mois le niveau d'an\-xié\-té avait continué à baisser fortement
pour le groupe Tomatis alors qu'aucun
changement n'apparaissait pour le groupe contrôle}%
\footnote{Du Plessis W. F. and Van Jaarsveld, P. E. (1988),
	``Audio-psycho-phonology : A comparative outcome study on anxious 
primary school pupils'',  Afr. Tydskr. Sielk. 19818 (4) 144--151. Du Plessis, W.F., Burger, S. (2001) [\ldots]
	\emph{A pilot study involving the Tomatis method.}, Sud Africa J. 
Psychol.}
 
Il existe également une étude pilote sur l'effet de 
--- la bascule \label{bascule} électronique qui a pour objectif de stimuler le cerveau en lui 
permettant de capter plus facilement les sons---du Dr. Carlos Escera
de l'Université de Barcelone en 2014, menée en collaboration avec le CNRS  a fait 
l'objet d'une validation
par un comité de lecture scientifique et a prouvé la stimulation
apportée au cerveau pour la captation des sons.
%
\footnote{%
\href{http://tomatisassociation.org/scientific-validation-of-the-tomatis-effect-
eeg-recordings-of-sound-from-brainstem-to-cerebral-cortex-encoding-university-of
-barcelona-2014/}{tomatisassociation.org}.}  \label{bascule}\footnote{La bascule permet 
de créer une alternance entre deux 
conditions perceptives du même message sonore: passage soudain et imprévu de 
fréquences graves à des fréquences aigües.}










