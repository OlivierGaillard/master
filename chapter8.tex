\chapter{Etude avec utilisation du test Tomatis en clinique psychiatrique.}
Nous savons que la musicothérapie est de plus en plus intégrée dans
les milieux psychiatriques. J'ai choisi ce terrain d'étude car 
je travaille dans ce domaine  à temps partiel comme musicothérapeute à la Privatklinik
de Meiringen ,spécialisée en
addictologie dans le canton de Berne. La Privatklinik
de Meiringen dispose d'une capacité de 195 lits, 33 médecins et
psychologues, secondés par 177 soignants qui assurent le suivi du
patient. Regula Lehman, musicothérapeute à temps complet, et nous-même avons
collaboré ensemble pour l'organisation de la mise en place de l'étude. Celle-ci porte sur un même type de
population dans le contexte e cadre  précis d'une prise en charge globale
par les médecins, les psychologues, les physiothérapeutes,
ergothérapeutes et les divers ateliers de créativité proposés dans
cette clinique. Nous avons principalement travaillé avec des
pathologies comme celles du burnout, de la dépendance et de la
dépression, ceci dans une tranche d'âge de 20 à 60 ans, masculin et
féminin quasi égale. Au préalable, nous avons fait circuler une
feuille d'information pour expliquer notre démarche d'évaluation sur
l'hypothèse de la transformation de l'écoute du patient lors de son
séjour en thérapie. \emph{Information für Mitwirkende an der klinischen
Studie\  "Evaluirung des aktiven Hörvermögens" }. \underline{Feuille à reproduire à la fin }. Regula Lehman a pu
préparer  les patients avec une explication préalable. Ensuite,
en ma présence, ceux-ci ont signé officiellement à chaque fois leur
accord pour cette participation  \emph{"Eine schriftliche Einbewilligung zum
Test"} avant de passer ces tests dont je me suis occupée exclusivement.
A Regula Lehmann  \footnote{Regula
  Lehmann, musicothérapeute  à 90\%  à la clinique de Meiringen} incombait la
prise en charge des séances en musicothérapie. Nous n'avons pas pu
toujours conserver cette configuration et j'ai dû suivre parfois
ponctuellement ou pendant le temps imparti certains patients en
musicothérapie. Par contre, le travail des tests était de mon ressort.  Nous avons organisé deux groupes, un témoin, sans
musicothérapie et un autre avec. Nous avions projeté d'en avoir
10 à chaque fois, mais ce ne fut pas simple de réunir ce
chiffre. Nous étions dépendants de leur entrée en clinique et de leur
sortie, après 4 semaines de thérapies, ce qui correspond à la durée
d'un séjour moyen dans cet établissement.


\begin{itemize}
	\item 10 patients testés, groupe A en musicothérapie : un
          premier test avant leur prise en charge en musicothérapie;
          puis un deuxième test \textdegree{} : après 4 semaines de
          clinique.
	\item 10 patients testés, groupe B de contrôle qui est un groupe sans musicothérapie,
	toujours dans le même contexte, c.à.dire la clinique, le suivi et les mêmes protocoles que l'autre groupe. Un premier test avant
	le début des autres thérapies puis un deuxième test, après 4 semaines. 
	Les tests ont été faits en avril, mai, juin, juillet, septembre et octobre 2017.
	Nous avons réalisé en tout 40 tests d'écoute Tomatis. Pour cette étude, nous avons intentionnellement exclu la thérapie avec les musiques traitées et appliquées avec Tomatis, en nous restreignant  intentionnellement à ce lieu où l'application de cette forme de thérapie n'existe pas.
	Le test Tomatis devait avoir une durée  moyenne de 50 à 60  minutes pour chaque patient, ce qui fut globlalement le cas.
\end{itemize}

\paragraph{Le WHOQO-Bref}

Nous avons utilisé et fait en parallèle le test WHOQO-Bref avant et
après pour avoir une variable supplémentaire pour confirmer en
parallèle supposée de l'action de la musicothérapie.  C'est une
version test de 1997 issue Programme sur la santé mentale,
Organisation mondiale de la santé, Genève. Il y a 26 questions, que le
patient a rempli lui-même en présence du thérapeute, avant le test
d'écoute. La durée pour les remplir a varié de 8 à10 minutes en
moyenne.  Il a eu 26 tests WHOQO-Bref.

Le manque de temps a été le principal facteur  réducteur
de tests valables, les départs imprévus des patients, et/ou leur
absence momentanée ( visite du psychologue, maladie, etc.). rajoutés au peu de temps de travail( 10\% )ainsi qu'à la
contingence difficile due à la distance séparant Genève du lieu de travail ( Meiringen)
: comment planifier un départ imprévu d'un patient !? il a fallu parfois
faire trois heures de route pour effectuer les tests finaux d'un ou deux
patients.
  
Par conséquent,  de nombreux tests sont restés 
incomplets et n'ont pu
être validés car ils ne remplissaient pas toutes les conditions requises.  En définitive, sur 40 tests d'écoute Tomatis et 26 tests
WHOQO-Bref, nous avons choisi de ressortir l'étude pour le groupe A de
4 patients effectifs en musicothérapie, tests complets, et le groupe
témoin B de 5 patients sur 9 effectifs, sans musicothérapie et tests
complets.  Le matériel utilisé : une table, deux chaises, l'appareil
test Hearing et les écouteurs aériens et osseux, un crayon, deux
feutres ( rouge et bleu), une feuille avec la grille de fréquences à
remplir.
 
Nous sommes clairement en présence d'une ébauche d'études, avec des pistes
suggérées. Ce travail ne peut être en aucun cas considéré comme
quantitatif.  Nous avons ainsi pris l'option de nous tenir à une
observation, celle de la transformation de l'écoute. Nous n'approfondirons pas l'évolution des diagnostics des différentes
pathologies des patients.  (dépression, Burn out et dépendance.)
  
\paragraph{Les résultats}
  
  De manière très générale, les résultats obtenus ne
  sont pas significatifs.  La prise en charge en musicothérapie a eu lieu
  une fois par semaine pendant une heure, ce qui semble trop court pour observer un changement important. Nous pourrions émettre la supposition suivante :  est-ce qu'un un travail journalier, régulier aurait été indiqué pour des résultats plus rapidement visibles avec le test?
  Est-ce qu'une immersion plus intensive en musicothérapie transformerait l'écoute des patients ? 
   En comparaison avec des
  modifications importantes de courbes des tests observées généralement  lors d' une écoute
  régulière de deux heures par jour de musique pendant 15 jours, --en référence à l'entrainement des muscles de l'oreille chez Tomatis, qui, nous le rappelons, est une pédagogie de l'écoute--, il aurait été intéressant de pouvoir faire cette étude comparative dans cette clinique. Ainsi, nous aurions pu éventuellement mettre en avant  l'absolue nécessité de créer et d'instaurer systématiquement la musicothérapie dans de nombreuses institutions mais aussi  de la développer beaucoup plus intensément  si elle est déjà existante.
  
  A fortiori, relevons le cas fort intéressant  d'une patiente du groupe B( sans
  musicothérapie) : 
surprise d'apprendre par le premier test que la musique pouvait l'aider
  dans sa thérapie, celle-ci s'est mise à écouter assidûment du Mozart pendant la période de son séjour, entre le 1° test et le second test.  Les résultats
  graphiques obtenus lors de sa sortie sont clairement significatifs
  et sont en concordance avec le WHOQ-Bref.  Nous reviendrons sur son
  cas après avoir exposé nos observations d'évaluation.

Les critères et paramètres ont déjà été mentionnés au chapitre 5. Nous n'y reviendrons et n'utiliserons pas tout cet éventail d'informations. Nous rappelons que 
  
\paragraph{le test d'écoute Tomatis est une traduction graphique de l'écoute, elle permet d'objectiver la qualité de l'écoute,} avec 
  les paramètres suivants : % faire une phrase ici
  \begin{enumerate}
  \item les seuils d'écoute
  \item le son : dB, 
  \item le volume de $-20$ à 90 % finesse typo: le signe moins des mathématiques.
  \item les fréquence, de 125 à 8000.   \label{chapitre 6.2t} % mets une étiquette plus
    % ``parlante'' stp, genre ``parametres_test_ecoute''.
  \item la courbe, par l'observation des courbes d'écoute relevées
    en comparaison avec la courbe dite idéale : équilibre,
    déséquilibre, harmonie, disharmonie.  
  \item équilibre, déséquilibre graphique entre les deux oreilles et entre les deux courbes mesurées par oreille: observation des croisements, des parallèles, des
      écarts importants entre les courbes aériennes et osseuses\footnote{Remarque :
      Un carré sur le graphique représente une différence de 5dB en
      volume.}.
   \item une modification perçue ou non comme évolutive lors des transformations graphiques de courbes
    \item  Informations croisées avec les informations récoltées   par les 3 zones du test d'écoute: ........
     \item une constatation de la posture d'écoute et de la qualité de
      la voix. La voix se caractérise par son volume, son timbre, sa mélodie et son langage. 
  
   Descriptif général de la voix d'un patient dépressif :
      \begin{enumerate}
        \item le volume : basse intensité
        \item la mélodie : monotone, sans modulation
        \item le timbre : mauvaise qualité due à une pertes des harmoniques
        \item le langage : difficulté d'élocution
          \end{enumerate}
   
     Différents tests de et par la voix ont été faits pour tenter de déterminer un état dépressif ( Test et échelle d'Hamilton)Les chercheurs de l'université de Maryland en 2004, en émettant l'hypothèse de la modification de l'articulation vocale lors d'état dépressif( on sait que la dépression provoque des changements  neuro-physiologiques) ont révélé lors du 168ème congrès de la Société américaine d'acoustique que les caractéristiques vocales se trouvaient modifiées lors de sentiments dépressifs. @online{https://www.lci.fr/sante/et-si-on-diagnostiquait-la-depression-avec-un-test-vocal-sur-smartphone-1562728.html,
     
     }
     
   
   
   
   
  \end{enumerate}


\section{Les graphiques des tests Tomatis réalisés en clinique }
.......à rajouter...



\section{Les tests d'écoute Tomatis}
\paragraph{Hypothèse}

Est-ce possible d'évaluer un travail musicothérapeutique au moyen
d'un test d'écoute?

Est-ce que le processus d'écoute en musicothérapie améliore la capacité
d'écoute ?

Est-ce que les test auditifs avant et après la musicothérapie permettent
de visualiser l'action de la musicothérapie?

\paragraph{Y-a-t-il une modification de l'écoute du patient après une prise
en charge en musicothérapie ?}

\paragraph{Est-ce que les résultats ( = un changement dans l'écoute) d'une prise
en charge musicothérapeutique peuvent être lisibles et visibles dans
un test d'écoute Tomatis ?}

Est-ce que ces résultats sont significatifs? 

\paragraph{Est-ce que l'écoute du patient s'est modifié ? si on a pu observer
une modification, dans quel sens va -t-elle ?}

Est-ce ce test valable ? est-ce que le contexte est suffisant pour
ressortir des résultats ?





