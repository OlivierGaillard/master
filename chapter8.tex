\chapter{Etude avec utilisation du test Tomatis en clinique psychiatrique.}

\section{Les tests d'écoute Tomatis : }

\paragraph{Hypothèse :}

\paragraph{Est-ce possible d'évaluer un travail musicothérapeutique au moyen
d'un test d'écoute?}

Est-ce que le processus d'écoute en musicothérapie améliore la capacité
d'écoute ?

Est-ce que les test auditifs avant et après la musicothérapie permettent
de visualiser l'action de la musicothérapie?

\paragraph{Y-a-t-il une modification de l'écoute du patient après une prise
en charge en musicothérapie ?}

\paragraph{Est-ce que les résultats( = un changement dans l'écoute) d'une prise
en charge musicothérapeutique peuvent être lisibles et visibles dans
un test d'écoute Tomatis ?}

Est-ce que ces résultats sont significatifs? 

\paragraph{Est-ce que l'écoute du patient s'est modifié ? si on a pu observer
une modification, dans quel sens va -t-elle ?}

Est-ce ce test valable ? est-ce que le contexte est suffisant pour
ressortir des résultats ?

\paragraph{Comment ?Nous allons vérifier, tester et comparer.}

\subparagraph{Un test d'écoute est une traduction graphique de l'écoute avec }

les paramètres suivants : 
\begin{itemize}
\item le son : dB, le volume de -20 à 90 et les Herz, fréquences de 125
à 8000. cf.ch.6.2.
\end{itemize}

\subparagraph{Par observation des courbes d'écoute relevées avec modèle de la courbe
idéale : équilibre, déséquilibre, harmonie, dysharmonie.}
\begin{itemize}
\item une interprétation psychologique va en être déduite, une constatation
de la posture d'écoute
\item une modification et une évolution ou transformation des courbes
\end{itemize}

\paragraph{Nous avons utilisé et fait en parallèle le test WHOQO-Bref avant
et après pour avoir une variable supplémentaire pour confirmer en
parallèle supposée de l'action de la musicothérapie.}

C'est une version test de 1997 issue Programme sur la santé mentale,
Organisation mondiale de la santé, Genève. Il y a 26 questions, que
le patient remplit lui-même en présence du thérapeute. p

Nous spécifions qu' il n'y a pas eu de traitement Tomatis et que le
suivi a été fait en musicothérapie sans exclure celui fait avec les
médecins, les psychologues, 

et les divers ateliers de créativité proposés dans cette clinique.

Nous avons pu organiser les tests d'écoute pour deux groupes : un
groupe en musicothérapie et un groupe de contrôle.
\begin{itemize}
\item 10 patients testés, groupe en musicothérapie : avant leur prise en
charge en musicothérapie; 2\textdegree{} test : après 4 semaines de
clinique. 
\item 9 patients testés, groupe de contrôle qui est un groupe sans musicothérapie
avec le même contexte, c.à.dire la clinique. 1\textdegree Test avant
le début des autres thérapies puis 2\textdegree{} test, après 4 semaines
qui équivalent à un séjour moyen en clinique, avant leur sortie.
\end{itemize}
Nous avons effectué les deux tests pour les deux groupes, avant et
après et fait remplir le questionnaire WHOQO-Bref.

A l'autre musicothérapeute, Regula Lehmann, travaillant régulièrement
à la clinique \footnote{Regula Lehmann, musicothérapeute à la clinique de Meiringen}
incombait la prise en charge complète de toutes séances en musicothérapie.

Nous mentionnons que ce sont des patients diagnostiqués avec une pathologie
de dépression, de burn-out et un cas de dépendance. 

Nous précisons que nous nous tiendrons à un comparatif de l'évolution
ou de la transformation de l'écoute, mais non de celles selon les
pathologies.

( à compléter.....)!

