\chapter{Etude avec utilisation du test Tomatis en clinique psychiatrique.}
Je travaille à temps partiel comme musicothérapeute à la  Privatklinik de Meiringen dans le canton de Berne, qui est spécialisée en addictologie. Elle dispose d'une capacité de 195 lits, 33 médecins et psychologues, secondés par 177 soignants qui assurent le suivi du patient.Regula Lehman est la musicothérapeute principale et nous avons collaboré ensemble. L'étude qui a été faite porte sur un même type de population dans le contexte très précis d'une prise en charge globale par  les médecins, les psychologues, les physiothérapeutes, ergothérapeutes et les divers ateliers de créativité proposés dans cette clinique. Nous avons principalement travaillé avec des pathologies comme celles du burnout, de la dépendance et de la  dépression, ceci dans une tranche d'âge de 20 à 60 ans, masculin et féminin quasi égale. Au préalable, nous avons fait circuler une feuille d'information pour expliquer notre démarche d'évaluation sur  l'hypothèse de la transformation de l'écoute du patient lors de son séjour en thérapie. Information für Mitwirkende an der klinischen Studie\"Evaluirung des aktiven Hörvermögens" Regula Lehman a pu préparer parfois  les patients avec un discussion explicative. Ensuite, en ma présence, ceux-ci ont signé officiellement  à chaque fois leur accord pour cette participation.( eine schriftliche Einwilligung zum Test) avant de passer ces tests dont je me suis occupée exclusivement.
A Regula Lehmann, qui travaille à 90\%
à la Clinique \footnote{Regula Lehmann, musicothérapeute à la clinique de Meiringen}
incombait la prise en charge des séances en musicothérapie. Nous n'avons pas pu toujours conserver cette configuration et j'ai dû suivre parfois ponctuellement ou complètement certains patients en musicothérapie. Par contre, le travail des tests m'est revenu complètement. 
Nous avons  organisé  deux groupes, un groupe témoin sans musicothérapie et un autre groupe avec. Nous avions projeté d'en avoir 10 à chaque fois, mais ce ne fut pas si simple de réunir ce chiffre. Nous étions dépendants de leur entrée en clinique et de leur sortie, après 4 semaines de thérapies, ce qui correspond à la durée d'un séjour moyen dans cet établissement.
\begin{itemize}
	\item 10 patients testés, groupe A en musicothérapie :  un premier test avant leur prise en
	charge en musicothérapie; puis un deuxième test \textdegree{}  : après 4 semaines de
	clinique. 
	\item 10 patients testés, groupe B de contrôle qui est un groupe sans musicothérapie,
	toujours dans le même contexte, c.à.dire la clinique, le suivi et les mêmes protocoles que l'autre groupe. Un premier test avant
	le début des autres thérapies puis un deuxième test, après 4 semaines. 
	Les tests ont été faits en avril, mai, juin, juillet, septembre et octobre 2017.
	Nous avons réalisé en tout 40 tests d'écoute Tomatis. Pour cette étude, nous avons intentionnellement exclu la thérapie avec les musiques traitées et appliquées avec Tomatis, en nous restreignant  intentionnellement à ce lieu où l'application de cette forme de thérapie n'existe pas.
	Le test Tomatis devait avoir une durée  moyenne de 50 à 60  minutes pour chaque patient.
\end{itemize}
 \paragraph{Nous avons utilisé et fait en parallèle le test WHOQO-Bref avant  
	et après pour avoir une variable supplémentaire pour confirmer en
	parallèle supposée de l'action de la musicothérapie.}
C'est une version test de 1997 issue Programme sur la santé mentale,
Organisation mondiale de la santé, Genève. Il y a 26 questions, que
le patient a rempli lui-même en présence du thérapeute, avant le test d'écoute. La durée pour les remplir a varié de 8 à10 minutes en moyenne.
Il a eu 26 tests WHOQO-Bref.

Le manque de temps a été le principal facteur qui a réduit ce nombre de tests, ainsi que les départs imprévus des patients, et/ou leur absence momentanée ( visite du psychologue, maladie, etc.)


 
  Il y a eu aussi une limitation dans le temps, liée à mon 10\% et  une contingence difficile due à la distance qui me sépare de mon domicile( Genève) de celui  du lieu de travail et d'expérimentation ( Meiringen) : comment planifier un départ imprévu d'un patient !? j'ai dû parfois faire  trois heures de route pour effectuer  les tests finaux d'un seul patient. 
  
   Par conséquent,  je me suis retrouvée avec de nombreux tests incomplets, soit avec les deux tests Tomatis mais sans WHOQO-Bref, soit le manque du deuxième test Tomatis. De nombreux tests n'ont pu être validés car ils ne remplissaient pas toutes les conditions requises.
    En définitive, sur 40 tests d'écoute Tomatis et 26 tests WOQO-Bref, nous avons choisi de ressortir  l'étude  pour  un groupe A de 4 patients effectifs en musicothérapie, tests complets, et un groupe témoin B de 5 patients sur 9 effectifs, sans musicothérapie et tests complets.  
   Le matériel utilisé : une table, deux chaises, l'appareil test Hearing et les écouteurs aériens et osseux, un crayon, deux feutres ( rouge et bleu), une feuille avec la grille de fréquences à remplir. 
 
 Nous sommes en présence d'une ébauche d'études, de pistes suggérées. Ce travail ne peut être en aucun cas considéré comme quantitatif.
 Nous avons ainsi pris l'option de nous tenir 
 à une observation, celle  de la transformation de l'écoute exclusive de l'oreille droite, tout en sachant que l'oreille gauche joue aussi son rôle comme précédemment mentionné \{indexchap.5.2} ).  De même , nous n'approfondirons pas l'évolution des diagnostics des différentes pathologies des patients. 
  (dépression,  Burn out et dépendance.)
  
\paragraph{Les résultats}  
  Les résultats : De manière très générale, les résultats obtenus ne sont pas significatifs. 
  La prise en charge en musicothérapie a lieu une fois par semaine pendant une heure, ce qui est trop court. Nous pouvons émettre cette supposition en comparaison avec des modifications importantes de courbes observées lors d' une écoute régulière de deux heures par jour de musique pendant 15 jours. Je m'en réfère à ma propre expérience comme à celles des résultats obtenus dans les études Tomatis sus-mentionnées.
  Il est aussi intéressant de relever le cas d'une patiente du groupe B( sans musicothérapie). Etonnée d'apprendre que la musique pouvait l'aider dans sa thérapie, celle-ci s'est mise à écouter assidument de la musique classique (Mozart) pendant son séjour. Les résultats graphiques obtenus lors de sa sortie sont clairement significatifs et sont en concordance avec le WHOQ-Bref.
  Nous reviendrons sur son cas après avoir exposé nos critères d'évaluation.
  \paragraph{Comment ?Nous allons vérifier, tester et comparer.}
  
  \subparagraph{Un test d'écoute est une traduction graphique de l'écoute avec }
  
  les paramètres suivants : 
  \begin{l}
  	les seuils d'écoute \item le son : dB, le volume de -20 à 90 et les Herz, fréquences de 125 à 8000.   \label\label{chapitre 6.2t}
\item{la courbe} : par l'observation des courbes d'écoute relevées en comparaison avec la courbe dite 
    idéale : équilibre, déséquilibre, harmonie, disharmonie.} 
Croisement, parallélisme, écart important des courbes aériennes et osseuses.
Remarque :  Un carré sur le graphique représente une différence de 5dB en volume.

\begin{itemize}
\item  une constatation
de la posture d'écoute 
et de la qualité de la voix.

Descriptif général de la voix d'un patient dépressif : 
le volume : basse intensité
la mélodie : monotone, sans modulation
le timbre : mauvaise qualité due à une pertes des harmoniques
le langage : difficulté d'élocution


\item une modification et une évolution ou transformation des courbes
\end{itemize}

\section{Les tests d'écoute Tomatis : }
{Hypothèse :}

\paragraph{Est-ce possible d'évaluer un travail musicothérapeutique au moyen
d'un test d'écoute?}

Est-ce que le processus d'écoute en musicothérapie améliore la capacité
d'écoute ?

Est-ce que les test auditifs avant et après la musicothérapie permettent
de visualiser l'action de la musicothérapie?

\paragraph{Y-a-t-il une modification de l'écoute du patient après une prise
en charge en musicothérapie ?}

\paragraph{Est-ce que les résultats( = un changement dans l'écoute) d'une prise
en charge musicothérapeutique peuvent être lisibles et visibles dans
un test d'écoute Tomatis ?}

Est-ce que ces résultats sont significatifs? 

\paragraph{Est-ce que l'écoute du patient s'est modifié ? si on a pu observer
une modification, dans quel sens va -t-elle ?}

Est-ce ce test valable ? est-ce que le contexte est suffisant pour
ressortir des résultats ?





\subparagraph{




