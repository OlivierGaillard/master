\chapter{Etude avec utilisation du test Tomatis en clinique psychiatrique.}
Je travaille à temps partiel comme musicothérapeute à la Clinique psychiatrique de Meiringen dans le canton de Berne, en tant que remplaçante fixe, depuis janvier 2013. L'étude qui a été faite porte sur l'évolution de l'écoute du patient dans le contexte très précis de sa prise en charge globale avec les médecins, les psychologues, les physiothérapeutes, ergothérapeutes et les divers ateliers de créativité proposés dans cette clinique. La Privatklinik de Meiringen est spécialisée en addictologie. Elle dispose d'une capacité de 195 lits, 33 médecins et psychologues, secondés par 177 soignants qui assurent le suivi du patient. Lors des tests, nous avons principalement travaillé avec des pathologies comme celles du burnout, de la dépendance et de la  dépression, ceci dans une tranche d'âge de 20 à 60 ans, masculin et féminin à part presque égale.

Nous avons  organisé  les tests d'écoute pour deux groupes : un
groupe en musicothérapie et un groupe de contrôle. L'objectif était d'en avoir 10 par groupe, mais la réalité nous a rattrapée; ce ne fut pas si simple. Il fallait faire en sorte que les patients puissent faire ce test dès leur entrée en clinique et un deuxième test lors de leur sortie, après 4 semaines de thérapies et de suivi, ce qui correspond à la durée d'un séjour moyen dans cet établissement.
\begin{itemize}
	\item 10 patients testés, groupe A en musicothérapie :  un premier test avant leur prise en
	charge en musicothérapie; puis un deuxième test \textdegree{}  : après 4 semaines de
	clinique. 
	\item 10 patients testés, groupe B de contrôle qui est un groupe sans musicothérapie,
	toujours dans le même contexte, c.à.dire la clinique et le suivi et les mêmes protocoles que l'autre groupe. 1\textdegree Test avant
	le début des autres thérapies  puis 2\textdegree{} test, après 4 semaines.
\end{itemize}
 \paragraph{Nous avons utilisé et fait en parallèle le test WHOQO-Bref avant
	et après pour avoir une variable supplémentaire pour confirmer en
	parallèle supposée de l'action de la musicothérapie.}
C'est une version test de 1997 issue Programme sur la santé mentale,
Organisation mondiale de la santé, Genève. Il y a 26 questions, que
le patient remplit lui-même en présence du thérapeute. 


 Pour cette étude, nous avons intentionnellement exclu les musiques traitées et appliquées dans la méthode Tomatis, en nous restreignant  intentionnellement à ce lieu où l'application de cette méthode n'existe pas. Il y a eu aussi une limitation dans le temps, liée à mon 10\% et  une contingence difficile due à la distance qui me sépare de mon domicile( Genève) de celui  du lieu de travail et d'expérimentation ( Meiringen) : comment planifier un départ imprévu d'un patient ? j'ai dû parfois tout laisser en plan et faire  trois heures de route pour effectuer  les tests finaux lors d'un séjour plus court ou d'une sortie imprévisible d'un patient. Ainsi,  je me suis retrouvée avec de nombreux tests incomplets en juin 2017, soit avec les deux tests Tomatis mais sans WHOQO, soit pas de tests possible en fin de séjour, etc. 	J'ai refait une série de nouveaux tests en septembre et octobre pour obtenir un échantillonnage un peu plus large. Par conséquent, en définitive, sur 44 tests d'écoute Tomatis et 25 tests WOQO, nous avons choisi de ressortir  l'étude pour  un groupe de 4 patients en musicothérapie et un groupe témoin de 5 patients, sans musicothérapie. Nous avons donc fait de nombreux tests qui n'ont pu être validés car ils ne remplissaient pas toutes les conditions d'observation prévue.
 A Regula Lehmann, qui travaille à 90\%
 A la Clinique \footnote{Regula Lehmann, musicothérapeute à la clinique de Meiringen}
 incombait la prise en charge des séances en musicothérapie. Nous n'avons pas pu toujours conserver cette configuration et j'ai suivi parfois ponctuellement ou complètement certains patients en musicothérapie. Par contre,   je me suis occupée exclusivement  de faire passer les tests.
 
 Nous nous en tiendrons à une observation de la transformation de l'écoute, mais nous n'approfondirons pas l'évolution des diagnostiques des différentes pathologies des patients 
  (dépression,  burn-out et dépendance.)

\section{Les tests d'écoute Tomatis : }
{Hypothèse :}

\paragraph{Est-ce possible d'évaluer un travail musicothérapeutique au moyen
d'un test d'écoute?}

Est-ce que le processus d'écoute en musicothérapie améliore la capacité
d'écoute ?

Est-ce que les test auditifs avant et après la musicothérapie permettent
de visualiser l'action de la musicothérapie?

\paragraph{Y-a-t-il une modification de l'écoute du patient après une prise
en charge en musicothérapie ?}

\paragraph{Est-ce que les résultats( = un changement dans l'écoute) d'une prise
en charge musicothérapeutique peuvent être lisibles et visibles dans
un test d'écoute Tomatis ?}

Est-ce que ces résultats sont significatifs? 

\paragraph{Est-ce que l'écoute du patient s'est modifié ? si on a pu observer
une modification, dans quel sens va -t-elle ?}

Est-ce ce test valable ? est-ce que le contexte est suffisant pour
ressortir des résultats ?

\paragraph{Comment ?Nous allons vérifier, tester et comparer.}

\subparagraph{Un test d'écoute est une traduction graphique de l'écoute avec }

les paramètres suivants : 
\begin{itemize}
\item le son : dB, le volume de -20 à 90 et les Herz, fréquences de 125
à 8000. cf.ch.6.2.
\end{itemize}

\subparagraph{Par observation des courbes d'écoute relevées avec modèle de la courbe
idéale : équilibre, déséquilibre, harmonie, dysharmonie.}
\begin{itemize}
\item une interprétation psychologique va en être déduite, une constatation
de la posture d'écoute
\item une modification et une évolution ou transformation des courbes
\end{itemize}




