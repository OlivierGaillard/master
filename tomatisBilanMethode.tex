
\section{Le test d'écoute de Tomatis}

Il convient à présent de se pencher de manière plus spécifique sur
A.Tomatis, puisque ce travail repose notamment sur son test d'écoute.
Dans son ouvrage ``Éducation et
    Dyslexie''\autocite{tomatis:education}, la représentation graphique du 
 ``\emph{Hearing Test}'' distingue l'écoute générale de
 l'auto-écoute.\footnote{cf.ch.3, L'auto-écoute consiste en un processus
   circulaire entre sa propre  émission vocale et son écoute, inhérent
   à l'apprentissage.( ``L'oreille et le langage'' )(1963),Ed.Seuil,
   p.72}
 Apparaissent ainsi les modifications respectives
 de la courbe aérienne et de la courbe osseuse, entraînant une nouvelle vision
 du concept d'écoute.\footnote{<<\,Considérations sur le test d'écoute\,>>. Propos
  	recueillis au cours du \textsc{iii}\ieme\ congrès international
  	d'audio-psycho-phonologie (Anvers 1973) lors d'un entretien
        avec B. Auriol. \autocite{auriol_stress}}
Tomatis a défini la «courbe d'écoute idéale», celle qui correspond à l'oreille absolue
des chanteurs et des musiciens, en particulier du ténor italien Enrico
Caruso (1873--1921) dont l'analyse vocale a été effectuée à partir de
disques 78 tours. La courbe de ce dernier a pu être considérée comme
optimale et de référence, caractérisée d'une part par des fréquences allant de 500 et 2000
Hz, par une pente d\textquoteright environ 6 à 18 db/octave,
et d'autre part par un dôme entre 2000 et 4000 Hz.
Le bon fonctionnement de l'oreille a été confirmé par la courbe
de Wegel.\footnote{
		Voir l'annexe \ref{acoustique} p. \pageref{acoustique}
		 pour cette partie technique.}
 Le travail d'acquisition de ce tracé correspond à l'\textsl{harmonisation}
découlant d'une bonne régulation des deux muscles de l'oreille moyenne
sur la pression interne du
labyrinthe.
Ainsi, l'évaluation finale de ce processus mettra en évidence la différence
d'avec la courbe idéale.
Lorsque la forme des 
courbes est continue et parallèle et ne comportant pas d'irrégularités ou
de \textbf{distorsions}, on parle d'harmonie.
Cette dernière influence à son tour 
la régulation des émotions, comme on le verra par la suite.



La \textbf{distorsion d'écoute} est consécutive à une interprétation
erronée des informations transmises entraînant un dysfonctionnement
des deux muscles destinés à favoriser l'arrivée
harmonieuse du son dans l'oreille interne.
En cas d'altération du message sensoriel,
le cerveau déclenche un mécanisme de\textbf{ protection} sous forme
d'inhibition de l'écoute, avec le relâchement des deux muscles en
question.


Cette faculté d'écoute déjà prête à la naissance peut subir des
altérations avec l'âge et
le temps et affaiblir la protection contre les agressions, raison pour
laquelle la phylogénèse \footnote{ phylogenèse: étym. >grec .phulon=
  race; en biologie, le mode de formation des espèces, le développement
  des espèces en cours de l'évolution; tout ce qui (ontogenèse: étym. >grec. ôn, ontos= l'être,
ce qui est)}  a intégré la distorsion comme défense
efficace.
Par ailleurs, comme certains auteurs tel B. Auriol peuvent nous faire
remarquer que 
les différents maux (l'otite, l'eczéma, l'hyper
ou hypo sécrétion de sébum) peuvent être compris comme des problèmes physiques liés à l'interaction des sons refusés
inconsciemment\autocite [19--20] {auriol:cle}.
Le pouvoir protecteur du cerveau consiste en un  ``d'étouffement'' de
certaines fréquences,  en engageant les zones corticales prédestinées
tant à 
l'intégration sonore qu' à l'écoute sélective,  avec l'aide synergique de la
modification (modulation) des impulsions électriques et l'augmentation de
l'irrigation sanguine\autocite [14] {auriol:cle}; 
cet ``étouffement'' correspond à la\textbf{distorsion}
\autocite{tomatis:education}.

  

%\enquote{\emph{L'oreille a un
%psychisme\autocite[{tomatis:loreille}.}} 



  




 




  

