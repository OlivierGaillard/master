
\section{Le test d'écoute de Tomatis}


Selon son ouvrage,\footnote{\emph{Éducation et
    Dyslexie}\autocite{tomatis:education}}la représentation graphique tirée du 
 ``\emph{Hearing Test}'' distingue l'écoute générale de l'auto-écoute avec l'observation des modifications et des évolutions des courbes
  aériennes et osseuses.\footnote{<<\,Considérations sur le test d'écoute\,>>. Propos
  	recueillis au cours du \textsc{iii}\ieme\ congrès international
  	d'audio-psycho-phonologie (Anvers 1973) lors d'un entretien
        avec B. Auriol. \autocite{auriol_stress}}





Tomatis a défini la «courbe d'écoute idéale», courbe qui correspond à l'oreille absolue
des chanteurs et des musiciens,  avec  le ténor italien Enrico
Caruso (1873--1921) dont il a analysa la voix à partir des
enregistrements sur disque. Caruso représentait la courbe auditive
optimale dont il décida de se référer. C'est une courbe ascendante entre 500 et 2000
Hz qui correspond à une pente d\textquoteright environ 6 à 18 db/octave,
puis un dôme entre 2000 et 4000 Hz et ensuite une légère descente. 

      Sur le plan de la physique pure, elle indique les réponses de l'oreille
lorsque celle-ci fonctionne bien. Elle répond en fait à la courbe
de Wegel dite ``courbe en citron", inversée.\footnote{
		Voir l'annexe \ref{acoustique} p. \pageref{acoustique}
		 pour cette partie technique.}.

               

L'acquisition de cette courbe idéale correspond à l'\textsl{harmonisation}
du jeu de deux muscles de l'oreille moyenne. Celui-ci 
permet de régler en permanence la pression interne au niveau du
labyrinthe.

Sur le plan du test d'écoute, on évaluera et visualisera la différence
graphique selon la courbe dite 
idéale. Lorsque la forme de
courbes est continue et parallèle et qu'il n'y a pas de distorsions, on parle d'harmonie. L'harmonie
est la représentation de la régulation des émotions, l'équilibre entre
une écoute
intérieure et extérieure.

Ces paramètres sont importants et nous reviendrons plus en détail sur
la passation du test d'écoute.
%\enquote{\emph{L'oreille a un
%psychisme\autocite[{tomatis:loreille}.}} 

Lorsque l'interprétation des informations transmises à l'oreille est
erronée, il y a une 
\textbf{distorsion d'écoute}, liée au dysfonctionnement
de ces deux muscles dont le rôle est de permettre l'arrivée
harmonieuse du son dans l'oreille interne, puis au cerveau. Car, lorsque
le message sensoriel est altéré, le cerveau se protège en déclenchant
des mécanismes d'inhibition de l'écoute, traduit souvent par un relâchement de
ces plus petits muscles du corps humain. Ce potentiel acquis à
la naissance peut s'altérer avec les difficultés inhérentes à
la vie et la protection recherchée (inconsciemment) contre certaines agressions et le subterfuge le plus
efficace qu'a élaboré le cerveau est
d'introduire des distorsions, comme citées plus haut. De même, constate B. Auriol
les différents maux comme l'otite, l'eczéma, l'hyper
ou hypo sécrétio de sébum sont liés à l'interaction de sons refusés
inconsciemment.  \autocite  [19--20] {auriol:cle}




Car,même si le relevé des seuils donne des résultats objectifs ---
quoique la notion d'objectivité comme dit précédemment, est très
complexe avec le son ---il peut paraître paradoxal de pouvoir détecter
le potentiel d'écoute pour chaque patient en particulier.  En fait,
selon les désirs du patient de se servir des sons qu'il a à sa
disposition, celui-ci choisira d'en entendre seulement une partie. Il
est en vérité \textit{capable physiquement} de les entendre mais ne
les veut pas psychologiquement. Le cerveau a le pouvoir d'assourdir
certaines fréquences, de les masquer jusqu'à les faire disparaître peu
à peu de son champ d'écoute. Les zones corticales, qui se chargent de
l'intégration sonore et de l'écoute sélective, sont sollicitées par
des impulsions électriques mais aussi par une forte irrigation
sanguine et jouent ce rôle de protection.  \autocite [14] {auriol:cle}
Car par réflexe d'auto-défense, de ``survie'', les sons sont en
quelque sorte annihilés alors que l'oreille peut les collecter. Le
cerveau crée ainsi des\textbf{ distorsions d'écoute}
\autocite{tomatis:education}.

  



  




 




  

