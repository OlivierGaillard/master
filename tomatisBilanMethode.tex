
\section{Le test d'écoute de Tomatis}


Dans son ouvrage, ``Éducation et
    Dyslexie'',\autocite{tomatis:education} la représentation graphique du 
 ``\emph{Hearing Test}'' distingue l'écoute générale de l'auto-écoute.
 Apparaissent ainsi les modifications respectives
 de la courbe aérienne et de la courbe osseuse, entraînant une nouvelle vision
 du concept d'écoute.\footnote{<<\,Considérations sur le test d'écoute\,>>. Propos
  	recueillis au cours du \textsc{iii}\ieme\ congrès international
  	d'audio-psycho-phonologie (Anvers 1973) lors d'un entretien
        avec B. Auriol. \autocite{auriol_stress}}





Tomatis a défini la «courbe d'écoute idéale», celle qui correspond à l'oreille absolue
des chanteurs et des musiciens, en particulier du ténor italien Enrico
Caruso (1873--1921) dont l'analyse vocale a été effectuée à partir de
disques 78 tours. La courbe de ce dernier a pu être considérée comme
optimale et de référence, caractérisée d'une part par des fréquences allant de 500 et 2000
Hz, par une pente d\textquoteright environ 6 à 18 db/octave,
et d'autre part par un dôme entre 2000 et 4000 Hz.
Le bon fonctionnement de l'oreille a été confirmé par la courbe
de Wegel.\footnote{
		Voir l'annexe \ref{acoustique} p. \pageref{acoustique}
		 pour cette partie technique.}
               

Le travail d'acquisition de ce tracé correspond à l'\textsl{harmonisation}
découlant du jeu de régulation des deux muscles de l'oreille moyenne
sur la pression interne du
labyrinthe.

Ainsi, l'évaluation finale de ce processus mettra en évidence la différence
d'avec la courbe idéale.

Lorsque la forme de
courbes est continue et parallèle et qu'il n'y a pas d'irrégularités ou
de \textbf{distorsions},
on parle en fait d'harmonie qui influence à son tour 
la régulation des émotions, comme on le verra par la suite.



%\enquote{\emph{L'oreille a un
%psychisme\autocite[{tomatis:loreille}.}} 

La \textbf{distorsion d'écoute} est consécutive à une interprétation
erronée des informations transmises, (liée au dysfonctionnement
de ces deux muscles dont le rôle est de permettre l'arrivée
harmonieuse du son dans l'oreille interne). En cas d'altération du message sensoriel,
le cerveau déclenche un mécanisme de\textbf{ protection} sous la forme
d'inhibition de l'écoute, avec le relâchement des deux muscles en question.
Cette faculté d'écoute déjà prête à la naissance peut s'altérer avec
le temps et affaiblir la protection contre les agressions, raison pour
laquelle la phylogénèse a introduit la distorsion comme défense
efficace. Comme B. Auriol nous le fait remarquer dans 
les différents maux dont l'otite, l'eczéma, l'hyper
ou hypo sécrétion de sébum, ce sont des problèmes physiques liés à l'interaction de sons refusés
inconsciemment.  \autocite  [19--20] {auriol:cle}

Le cerveau a le pouvoir protecteur ``d'étouffer''
certaines fréquences en engageant les zones corticales prédestinée à 
l'intégration sonore et de l'écoute sélective avec l'aide synergique de la
modulation des impulsions électriques et l'augmentation de
l'irrigation sanguine\autocite [14] {auriol:cle}.
Le produit de cet ``étouffement'' est la
\textbf{distorsion}.
\autocite{tomatis:education}.

  



  




 




  

