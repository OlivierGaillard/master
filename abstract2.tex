% !TEX root = ./master.tex


\begin{abstract}
\centerline{\textbf{Abstract}}
\begin{french}	
\textbf{L'axe principal}: observation par comparaison de la faculté d'écoute des patients lors 
de l'aboutissement d'une musicothérapie.
\textbf{Les outils:} les test d'écoute Tomatis et questionnaire WHOQOL sur la qualité de vie sont réalisés 
pour synthétiser les différences  pré/post traitement.
 %Des graphiques de courbes d'écoute synthétisent des différences pré/post traitement.
 13 patients avec difficulté de régulation des émotions sont séparés en 2 
 groupes: 6 patients avec musicothérapie et  7 de contrôle.
 % Le WHOQOL est rempli avant et 
 %après séjour afin d'évaluer leur état psychique.
\textbf{Résultats:} une modification positive et significative d'écoute est observée post traitement %par le 
%test 
%d'écoute 
%Tomatis
 pour le groupe de 
musicothérapie, confirmée par le  WHOQOL, contrairement au groupe contrôle. % avec des valeurs plus 
%élevées. 
%Pour le groupe contrôle, la transformation a été faible et le questionnaire majoritairement 
%négatif.
\textbf{Conclusions:} la musicothérapie a eu un impact positif sur la transformation de l'écoute, corrélée 
à l'état psychique mis en évidence par le WHOQOL.
\textbf{Remarque:} la petite taille de notre étude et le total de questionnaires
inférieur à celui attendu empêchent nos résultats d'être 
statistiquement significatifs, rendant notre travail plus qualitatif que quantitatif.
%Ayant tout à fait conscience des compétences scientifiques qu'une telle étude aurait exigé, ce travail 
%oscillera 
\textbf{Mots-clés: musicothérapie; écoute; son; oreille; test}
\end{french}

\begin{english}
\textbf {The main axis} is about the observation by comparison of the auditory perception faculty of 
patients during the outcome of music therapy.
\textbf{Tools:} Tomatis listening test and WHOQOL quality of life questionnaire.
Listening curve graphs summarize the difference before and after treatment.
13 patients with the same type of pathology (difficulty in regulating
emotions) are divided into 2 different groups: one experimental group of 6 patients following music 
therapy, and one control group of 7 patients. Patients fill out the WHOQOL questionnaire before and after 
treatmenas to determine the impact or the lack of impact of the therapy on their listening.
\textbf{Results:} analysis of observation shows a positive and significant modification for the music 
therapy group regarding the WHOQOL test. The post-therapy questionnaire indeed shows an increase in 
values compared to the pre-therapy one.
For the control group, the transformation was weak and the questionnaire was mostly negative.
\textbf{Conclusions:} Music therapy had a positive impact on the transformation of listening, correlated 
to the psychic state, as the WHOQOL questionnaires results clearly showed.
\textbf{Note:} the small study added to the final number of WHOQOL lower than expected prevent our 
results from being statistically 
significative.
% the final number of listening tests and WHOQOL test was found lower than expected. 
%Added to the fact this was a small study. This therefore prevents our results from being statistically 
%significative. 
%Being fully aware of the scientific skills that such a study would’ve required, 
This work will oscillate more 
towards the qualitative aspect of the study than the quantitative.
\textbf{Keywords: music therapy; listen; sound; hear; test}
\end{english}
\end{abstract}












%\begin{Remerciements}
