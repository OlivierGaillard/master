\section{Technique de passation du test Tomatis}
L'appareil de Tomatis, basé sur la reconnaissance des sons purs et
permettant d'objectiver la qualité de l'écoute
 a été créé dans les années 50, comportant un générateur de fréquences
 (avec
 des sons
  purs de \SIrange{125}{8000}{\Hz}, d'octave en octave, en passant par les valeurs
\SIlist{1500;3000;6000}{\Hz}, et dont l'intensité peut varier de 5 en \SI{5}{\dB}, de \SIrange{10}{100}{\dB}.)
Ces derniers sont propagés par une
  transmission aérienne avec un casque, et par une propagation osseuse
  avec un vibrateur.

  L'identification de ces sons est 
  signalée par la levée de la main homolatérale (droite, gauche ou
  bilatérale).
Un volume initial très faible est suivi d'une intensité
progressive jusqu'à la manifestation d'une réponse gestuelle.
 
Nous allons développer à l'aide de la représentation
graphique ci-dessous,  les paramètres du\textbf{ seuil}, de la
\textbf{spatialisation}, de la \textbf{sélectivité} et de l'\textbf{audiolatérométrie}.


\begin{figure}
	\centering
	\includegraphics[width=0.7\linewidth]{images/courbeideale.jpg}
	\caption{Diagrammes des courbes relatives à l'oreille droite et
          gauche; tracé bleu: c. aérienne; tracé rouge: c.
          osseuse, (Copyrights Tomatis Développement S.A.  2014) }
	\label{fig:courbeideale}
\end{figure}




\subsection{Identification des seuils auditifs individuels}

Cette \textbf{détection}, destinée à relever les deux profils d'écoute
en vue d'une application thérapeutique, 
s'effectue, d'une part, à l'aide d'une
conduction aérienne par \textbf{écouteurs}, où l'oreille interne
informe le nerf auditif,  et d'autre part, à l'aide
d'une conduction osseuse par\textbf{ vibrateur}, excitant le crâne au
niveau de l'
\textit{os mastoïde} transmettant à son tour à  la voie nerveuse
auditive.

\subsection{Représentation graphique}

Parmi les quelques éléments différentiels
apparaissant par la suite dans les observations cliniques, il est utile de retenir
que le seuil d'écoute est représenté par un point, résultant entre la
fréquence (abcisse) --spectre couvrant 20
fréquences (de 125 à 8000 Hz)--   et le volume
(ordonnée) dont chaque carré représente une différence de \SI{5}{\dB} en
volume, partant de dB de $-20$ à 90 dB.


Les points reliés dessinent deux courbes caractéristiques, (aérienne
et osseuse), permettant de relever les paramètres d'harmonie ou
          d'équilibre, ceci 
 	en comparaison avec la courbe idéale : on parlera
        d'équilibre ou de
 	déséquilibre, d'harmonie ou de dysharmonie.
        
        \begin{enumerate}
 
  \item   les seuils d'écoute sont reconnaissables par des points au niveau de 
          chaque fréquence émise et selon le volume entendu par le
          patient. Les points reliés créent les deux courbes.
 	\item le son : son pur en 20 fréquences différentes, de 125 à 8000 Hz.   
 	\item le volume: dB de $-20$ à 90; un carré sur le graphique représente une différence de \SI{5}{\dB} en
 		volume 
 	\item la courbe: est le résultat des points reliés des seuils
          d'écoute; ils 
          dessinent deux courbes caractéristiques, l'une aérienne et l'autre osseuse.
          L'observation des courbes d'écoute relevées vont permettre
          de les classer selon les paramètres d'harmonie ou
          d'équilibre, ceci 
 	en comparaison avec la courbe dite idéale : on parlera
        d'équilibre ou de
 	déséquilibre, d'harmonie ou de dysharmonie
        
      \item L'équilibre/déséquilibre graphique s'observe 
        -entre les deux oreilles, l'oreille droite et l'oreille gauche
        et 
        -entre les deux courbes aériennes et osseuses,
        
        dont les 
        croisements, les pics ou les échancrures notifient 
        l'écart en 
        qualifiant l'écoute d'harmonieuse ou de
        déséquilibrée. 
      \end{enumerate}
      
 En  conséquence,  s'il y a une modification
          graphique des courbes, elle 
          permet d'évaluer la transformation de l'écoute pré-et
          post -thérapie.
          

 
 







\paragraph{Remarque}


Tomatis a volontairement décalé les étalonnages des deux courbes (aérienne
	et osseuse) pour pouvoir distinguer les différentes réponses et interpréter
	les distorsions. Lorsque l'écoute est parfaite, les
	courbes aérienne et osseuse se confondent mais pour l'analyse des
	résultats, on a déterminé des courbes parallèles, la courbe aérienne
	devant être au dessus de la courbe osseuse.


\begin{figure}
	\centering
	\includegraphics[width=0.7\linewidth]{images/tomatisListeningTest.jpg}
	\caption[Graphique du test d'écoute]{Graphique du test
          d'écoute d. + g. incluant, en bas, le test de sélectivité}
	\label{fig:tomatislisteningtest}
\end{figure}


\paragraph{Dans cette étude, l'allure générale des courbes, le relevé quantitatif des
  croisements et des seuils auditifs seront notre priorité pour
  extraire nos résultats à partir des tests d'écoute:} 

Nous énumèrerons brièvement les
différentes techniques d'observation telles la
\textbf{spatialisation, la
sélectivité et l'audiolatérométrie}, mentionnées car importantes.
 Il serait intéressant d'englober tous ces paramètres mais
  l'objectif de notre étude serait largement dépassé.
Vu sous l'unique angle de l'observation de la transformation de l'écoute, nous
donnerons ici la priorité à la comparaison graphique  
de la courbe aérienne et osseuse appuyée par les résultats
quantitatifs des \textbf{seuils auditifs}.\footnote{ Nous nous sommes référés à  
  l'étude effectuée par le CNRS de Montpellier citée plus haut.}
Le nombre de \textbf{croisements }seront
également quantifiés.
Nous nous permettrons d'élargir, en
complément d'information, l'exemple d'un test d'écoute d'un patient en détaillant l'impact
des\textbf{ trois zones} et leur corrélation psychique.



\subsection{La spatialisation}

  


En relevant les seuils on assiste à la capacité 
d'\textbf{identification} et de \textbf{localisation} de la
\textbf{source sonore} comportant parfois des confusions et/ou des inversions
latérales.

La \textbf{spatialisation}
indique le degré d'élaboration de la latéralité auditive,
et elle fournit des repères sur la façon dont le cortex intègre les informations
par les faisceaux homo et hétéro-latéraux fonctionnellement différenciés.(cf.annexes?????)
Les erreurs de spatialisation peuvent refléter la confusion
de l'intégration corticale des informations et traduire une latence/
incertitude de localisation de la provenance du son.
(Tomatis souligne, dans son expérience, que la difficulté du sujet à en déterminer la provenance relève d'une mauvaise coordination, d'un manque de confiance en soi ou
d'une mauvaise organisation des idées.)??????


\subsection{La sélectivité}

  
  La \textbf{sélectivité }s'assimile  à  la CAT, capacité d'analyse tonale,\textquote{faculté que possède une oreille de percevoir
une variation de fréquences à l'intérieur d'un spectre sonore, et
de situer le sens de cette variation}\autocite{tomatis:loreille} dont 
le but est de déceler l'ouverture ou la fermeture de cette
caractéristique auditive. Cette dernière permettant de donner des informations sur la
qualité d'écoute, touchant aux aspects  linguistiques (conscience
phonémique), cognitifs ( fonctions exécutives) et émotionnels action
efférente, présence d'anxiété).
Le langage étant lui-même constitué de milliers de phonèmes, on reconnaît les possibilités auditives du patient si celui-ci  distingue au minimum la différence d'un son ``pur" d'une octave à l'autre.  


\subsection{ L'audiolatérométrie}
  
Grâce à l'\textbf{ audiolatérométrie} on définit  la latéralité droite ou gauche du patient. La dominance
de l'oreille droite comme oreille directrice doit être manifeste car
selon ses travaux, il y a une différenciation fonctionnelle
physiologique due à la longueur des nerfs récurrents. 
Si le cerveau préfère prendre l'oreille droite comme
``directrice'', c'est que le trajet emprunté par l'oreille droite au cerveau est plus
court;
ainsi
les informations circulent plus rapidement jusqu'à l'hémisphère gauche.


      



 
Ainsi, après la passation du test d\textquoteright écoute, nous nous
trouvons en présence de deux grilles contenant chacune deux courbes,
en général, de deux couleurs différentes complétées par l'indication
des inversions ou confusions de sons, par des données sur la sélectivité
et en même temps par des chiffres qui correspondent à l'épreuve d'audiolatérométrie.
Les résultats du test permettront de faire une comparaison avec la
courbe idéale.


\subsection{Les trois zones du test d'écoute }
Sur le le graphique du test, les fréquences observées vont être partagée en
trois, permettant la mise en évidence de différentes zones à l\textquoteright intérieur
de chaque diagramme. Les fréquences se répartissent des 
graves aux aigues, de la façon suivante :
\begin{itemize}
\item Zone 1 : de 125 à 1000 Hz : les graves, la zone vestibulaire
\item Zone 2 : de 1000 à 3000 Hz : les mediums, la zone du langage
\item Zone 3 : de 3000 à 8000 Hz : les aigus, zone cochléaire
\end{itemize}
Ces différentes bandes sonores nous donneront des éléments
d'interprétation.
Nous nous appuyons ici sur les affirmations expérimentales de Tomatis.


\section {Analyse et interprétation du test}


De manière générale, l'interprétation du test insiste sur le relevé graphique
des
courbes et accorde des
significations différentes aux zones spectrales.
\footnote{ Ces données interprétatives sont fiables, confirmées par
  les recherches de Tomatis dans le domaine
  empirique.}

Ce sont des comparaisons graphiques des courbes. 
On considère l'allure générale des courbes, on compare leur dessin
: la forme des courbes, l' équilibre, la symétrie ; et on étudie leurs
rapports entre eux : 

courbe aérienne (CA) - courbe osseuse (CO) - rapport entre CA et CO
pour chaque oreille - rapport entre CA et CO d\textquoteright une
oreille à l'autre. si ce rapport est correct, CA est placée au-dessus
de CO sur la grille.


\subparagraph{Les deux types de courbes véhiculent chacune des informations spécifiques
sur la posture d'écoute du sujet : }
\begin{itemize}
\item La conduction aérienne : traduit la vie sociale, la manière de communiquer
et de s'extérioriser
\item La conduction osseuse : traduit la vie intérieure, mode de fonctionnement
organique, d'une façon générale : liée aux tensions. C'est la courbe
de l\textquoteright auto-écoute, de l\textquoteright auto-contrôle,
de l'écoute intérieure.
\end{itemize}

\subparagraph{Les courbes donnent des informations selon leur ascendance, leur
continuité et leur similarité oreille droite/ oreille gauche.}
\begin{itemize}
\item Continuité de la courbe : Si une courbe est continue, elle définie
comme harmonieuse et ne comporte pas de pics, de scotomes (échancrure)
qui laisseraient
supposer l'existence de nombreuse tensions.
\end{itemize}
Situées en CO, ce sont des tensions internes non exprimées : attitude
calme mais très tendue intérieurement.

Situées en CA, ce sont des tensions réelles et exprimées au quotidien
: soit somatisées, soit verbalisées ou soit manifestées sur le plan
affectif (pleurs).

\subparagraph{Les trois zones du test d'écoute : }
\begin{itemize}
\item Zone 1 : de 125 à 1000 Hz : les graves, la zone vestibulaire, élaboration
du schéma corporel, des repères temporo-spatiaux, adresse motrice,
esprit pratique.
\item Zone 2 : de 1000 à 3000 Hz : les mediums, la zone du langage, de la
verbalisation, compréhension, mémorisation, de l'intégration des lois/
des règles, esprit analytique.
\item Zone 3 : de 3000 à 8000 Hz : les aigus, zone cochléaire, de l'énergie,
de l'imagination, de l'expression, motivation, esprit synthétique.
\end{itemize}

\subparagraph{Les trois zones de fréquences du test d'écoute correspondent à des
caractéristiques précises ; et, avec l'allure des courbes, on doit
tenir compte de leurs particularités.}

Lorsqu'une zone du test d'écoute est nettement dominante et semble
traduire une caractéristique de la personnalité, on peut situer un
sujet dans un registre particulier correspondant à son tempérament.

\begin{itemize}
	\item courbe accentuée dans la zone fréquentielle des graves : tempérament
	orienté vers le corps,
	
	\item courbe accentuée dans la zone fréquentielle des médiums : tempérament
	paranoïde, attaché à la logique, la règle, le raisonnement 
	
	\item courbe accentuée dans la zone fréquentielle des aigus : tempérament
	schizoïde, reflétant une recherche de créativité. 
\end{itemize}


La simplicité de
passation pour le test d'écoute et 
 l'évaluation des indices de réception évitent toute confusion
à l'aide de réponses gestuelles.



 






