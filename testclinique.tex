% !TEX root = ./master.tex
\chapter[\'Etude clinique]{\'Etude clinique}

%Souvenons-nous brièvement de l'évolution de la signification du terme
%``clinique'' enraciné dès l'Antiquité.
%A l’origine,\textbf{ l’activité clinique} (<gr. klinê = le lit) est celle du médecin qui, au chevet du malade, procède à l’examen des manifestations de la maladie (méthode de l’observation, de l’interrogation et de l’écoute) en vue de poser un diagnostic*, un pronostic* et une prescription* de traitement.
%Michel Foucault (1926-†1984), psychologue français, dans son ouvrage capital de 1963,
%«Naissance de la clinique» nous rend attentifs au fait que l’adjectif «clinique» fut longtemps du monopole médical, avec %l’observation « naturelle » faite au lit du malade, uniquement à l’aide des organes sensoriels.

%Hippocrate (médecin grec, Cos 460-377) était clinicien : il apprenait à ses étudiants l’art d’observer les symptômes, ceux-ci étant les réactions d’une personnalité à une agression pathogène. Généralisation et rationalisation, selon les critères d’Aristote, permettaient l’élaboration de « ces entités nosologiques* », les maladies, qui s’emparaient du corps du malade et poussant les médecins, exorcistes laïques, à les en faire sortir.
%Les publications de Th.Ribot à la Sorbonne rejoignent [avec «La psychologie des sentiments »(1896), «Les maladies de la mémoire »(1883) et «Les maladies de la personnalité » (1885)], les idées de S. Freud, montrant la primauté de la vie affective, où les tendances inconscientes jouent un rôle fondamental, et pouvant s’extérioriser soit par l’arrêt du développement affectif, soit par la dissolution des acquisitions plus récentes.

%L’expression « psychologie clinique » apparaît sous la plume de S. Freud, dans sa lettre à W. Fliess du 30.01.1899 : « Maintenant, la connexion avec la psychologie, telle qu’elle se présente sur les Etudes sur l’hystérie(1895) sort du chaos, j’aperçois les relations avec le conflit, la vie, tout ce que j’aimerais appeler psychologie clinique ».

%La définition « officielle » de la psychologie clinique mobilise et articule la singularité et
%la totalité, de façon à reconnaître une discipline psychologique basée sur l’étude approfondie des cas individuels, l’étude de la conduite humaine individuelle et de ses conditions (hérédité, maturation, conditions psychologiques et pathologiques, histoire de la vie), en somme, l’étude de la personne totale «en situation», c.à.d. l’expérience vécue de ce rapport à l’environnement.

%Au vue de cet aperçu historique, on peut reconnaître que
%dans certains milieux psychiatriques actuels, la musicothérapie trouve plus sa place aussi dans le complètement des tableaux cliniques.

%Dans notre étude, l'axe principal porte sur la
%vérification de l'amélioration de
%la capacité d'écoute suite au travail musicothérapeutique.

% et procéder à la comparaison des modifications de
%l'écoute.



%\textbf{Organigramme}: nous avons obtenu l'autorisation de vous
%montrer l'organigramme de la Clinique de Meiringen lors de la
%soutenance.




%(-OH*:  le radical hydroxile, oxydrile de la molécule éthylique)



%\textbf{Le Groupe Musicothérapeutique (expérimental) GM} comporte 21
%femmes et 15 hommes.

%\textbf{Le  Groupe Contrôle GC} comporte 8 patients, dont 4 femmes et 4 hommes.

%Problème avec la comparaison et la corrélation du questionnaire avec le test d'écoute.
%Fallait-il maintenir le WQ?
%Si seulement test d'écoute
%En voulant maintenir la comparaison autant dans le nombre de WQ que de tests d'écoute
%le critère d'exclusion a été de celui d'obtenir à chaque fois le comparatif pour chaque patient
%Finalement, en tenant compte de tous ces paramètres, les participants sont au nombre de 15, tous les 
%groupes confondus:

%Afin d'obtenir une comparaison similaire tout en se basant sur la quantité de questionnaires remplis,
%de patients a dû se restreindre à
%\textbf{GM = 8 patients (4 hommes/4 femmes)
%GC = 7 patients (4 hommes/3 femmes)}



En synthèse:
 \begin{itemize}

 \item \textbf{Nombre total de personnes}: N=  \textbf{15}:


 \textbf{GC: 7 et GM: 8}
\item\textbf{Genre et âge de la population étudiée:}  19 hommes et 10 femmes, de 25 à 72
  ans dont l'âge moyen est de 48 ans.
 
 \item \textbf{Total de séances} par personne en
   musicothérapie= 4 ;   \textbf{mu}=1/semaine;
 \textbf{t}= 50--60 min, période = 3 -- 4 semaines.
\end{itemize}




 %mu en grec


%Après ces prémisses, l'étude commence véritablement avec l'application du test
%audiométrique suivie du questionnaire qualitatif.
Sur \textbf{25 questionnaires WHO QOL}, il y a \textbf{10 pour GM} remplis
avec 8 pré- et seulement 2
     post- thérapies; et \textbf{15 pour GC} dont 8 pré-
     et 7 post-thérapie.
      Nous avons dans l'ensemble un total de \textbf{9 questionnaires} pour le
     comparatif des 2 groupes réunis.
     
     
 


 \section{Test d'écoute}

  \subsection{Groupe Contrôle: observation des tests d'écoute de 3 patients.}


      \textbf{Groupe Contrôle : 3 patients}
  \paragraph{ A. Patient Br.:}
  \begin{figure}[ht]
\centering
\includegraphics[width=1\linewidth]{images/graphiques/bru_pre.png}
\caption[  \textbf{Groupe Contrôle}: Patient A. 1° test]{Premier test Br.}
%moyenne OG et OD
%\label{groupecontroleimage1}
\end{figure}



 \begin{figure}[th]
\centering
\includegraphics[width=1\linewidth]{images/graphiques/bru_post.png}
\caption[Patient A. :  2° test]{Second test Br.}

%\label{groupecontroleimage1}
\end{figure}

	\begin{enumerate}
 		\item  c.a.: pas de modification, augmentation des
                  seuils: $-$
 		\item  c.o.: redressement des seuils: $+$
 		\item  croisements: $5/4$ : $+$ : ce qui signifie:  5 croisements lors du 1°test// 4 croisements lors du 2° test= nous avons 1 croisement en moins, donc le résultat est considéré comme positif en fin
                  de séjour.
                \end{enumerate}

                \textbf{ Résultat:  -    +    +       :  +}




\paragraph{B. Patient Sch.:}

	\begin{enumerate}

 		\item : c.a.: pas de modification, très légère augmentation des
                  seuils: +/-
 		\item : c.o.: a passé sous c.a., modification des seuils: +
 		\item : croisements: 2/2 :     =
                   \end{enumerate}
 \textbf{ Résultat:  +/-    +    =        :  =}

\begin{figure}[th]
\centering
\includegraphics[width=1\linewidth]{images/graphiques/schaff_pre.png}
\caption[Patient B. : 1° test]{Premier test Sch.}

%\label{groupecontroleimage1}
\end{figure}


         \begin{figure}[th]
\centering
\includegraphics[width=1\linewidth]{images/graphiques/schaff_post.png}
\caption[Patient B. : 2° test]{Second test Sch.}

\label{groupecontroleimage1}
\end{figure}


\paragraph{C. Patient Wal.:}



\begin{figure}[th]
\centering
\includegraphics[width=1\linewidth]{images/graphiques/wal_pre.png}
\caption[Patient C. :1° test]{Premier test Wal.}

%\label{groupecontroleimage1}
\end{figure}

	\begin{enumerate}

 		\item : c.a.: peu de modification: =

 		\item : c.o.: reste dominante, tentative de rapprochement de c.a.: -
 		\item : croisements: 1/3 :  -

                \end{enumerate}

                \textbf{ Résultat:  $= -  -        : -$ }

               \begin{figure}[th]
\centering
\includegraphics[width=1\linewidth]{images/graphiques/wal_post.png}
\caption[Patient C. : 2° test]{Second test Wal.}

\label{groupecontroleimage1}
\end{figure}






\subsection{Groupe Musicothérapie: observation des tests d'écoute de 3 patients}
  \textbf{Groupe de Musicothérapie: 3 patients}

\paragraph{ A. Patient Sw.:}



 \begin{figure}[th]
\centering
\includegraphics[width=1\linewidth]{images/graphiques/sw_pre.png}
\caption[ \textbf{Groupe Musicothérapie}: Patient A. : 1°Test]{Premier test Sw.}

%\label{groupecontroleimage1}
\end{figure}

	\begin{enumerate}

 		\item : c.a.: pas de modification: = %  1,27/1,27

 		\item : c.o.: redressement et rapprochement,
                  relèvement des seuils: -       %  3,07/3,39
 		\item : croisements: 1/3 :  -

                \end{enumerate}

                \textbf{  Conclusion:  = +  -        : ``=''}

                \begin{figure}[th]
\centering
\includegraphics[width=1\linewidth]{images/graphiques/sw_post.png}
\caption[Patient A. : 2° test]{Second test Sw.}

%\label{groupecontroleimage1}
\end{figure}




\paragraph{B. Patient Cav.: }

(pas de WOQOL fin de séjour)
\begin{figure}%[th]
\centering
\includegraphics[width=1\linewidth]{images/graphiques/cav_pre.png}
\caption[Patient B. : 1° test]{Premier test Cav.}

%\label{groupecontroleimage1}
\end{figure}

	\begin{enumerate}

 		\item : c.a.: redressement: +

 		\item : c.o.: redressement et rapprochement, relèvement des seuils: +
 		\item : croisements: 3/1 :  +

                \end{enumerate}

                \textbf{  Conclusion:  + + +       : ``+''}

                \begin{figure}%[th]
\centering
\includegraphics[width=1\linewidth]{images/graphiques/cav_post.png}
\caption[Patient B. : 2° test]{Second test Cav.}

%\label{groupecontroleimage1}
                \end{figure}
         \paragraph{ C. Patient M.:}
	\begin{enumerate}

 		\item : c.a.: redressement: : +   % 6,43/6,03

 		\item : c.o.: redressement et rapprochement,
                  relèvement des seuils:  +     %6,25/5,85:
 		\item : croisements: 3/3 :  =

                \end{enumerate}

                \textbf{  Conclusion:  +  +  =     : ``+''}

                \begin{figure}[th]
\centering
\includegraphics[width=1\linewidth]{images/graphiques/m_pre.png}
\caption[Patient M. : 1° test]{Premier test M.}

%\label{groupecontroleimage1}
\end{figure}


                        \begin{figure}[th]
\centering
\includegraphics[width=1\linewidth]{images/graphiques/m_post.png}
\caption[Patient M. : 2° test]{Second test M.}

%\label{groupecontroleimage1}
\end{figure}

 

\subsection{Test d'Ecoute: résultats du comparatif pré/post-thérapie}

\textbf{Conclusions générales}:

             Nous nous trouvons
           en présence de deux groupes, un groupe de contrôle et un
           groupe de musicothérapie ayant le même type de
           pathologie --difficulté de régulation des émotions-- .


Nous constatons, d'une part, que l'écoute est quantifiable.
           D'autre part, il existe bien
          une \textbf{modification de l'écoute pré -- et post -- traitement}.
Ensuite, il est à observer que
          cette modification est nettement plus marquée
          pour GM, groupe de musicothérapie, qui a un résultat positif.


          \textbf{GM: ``+''}.


Par contre,  pour le groupe de contrôle, GC, le résultat est mitigé, il correspond au signe d'égalité et n'apporte aucune vraie modification.

          \textbf{GC:  ``='' ou +/-}.


        Remarquons que les données quantitatives observables dans ces graphiques semblent aller dans le
sens de  l'étude faite par le
CNRS (Cf. Ch. Introduction) \autocite{affectiveDisorders} réalisée à partir des seuils auditifs, à savoir
les patients souffrant de troubles post-traumatiques souffrent d'un
\textbf{appauvrissement caractéristique de fréquences.}

\section{Questionnaires WHO - QOL : comparatif pré/post-thérapie }


\begin{sidewaysfigure}
\centering
\includegraphics[width=\linewidth]{images/graphiques/questionnaire_wq.png}
\caption[Questionnaire WHO QOL-BREF]{GM/GC - Pré/Post avec la moyenne des scores par domaine}

%\label{groupecontroleimage1}
\end{sidewaysfigure}
Voici à présent le schéma représentant la
moyenne pré -- et post -- traitement, calculée pour chaque patient, des scores
des 4 domaines.
%Les chiffres ont été obtenus à partir des 4
%domaines, avec le pré/post-séjour.
Remarque: si, par comparaison, le chiffre post-séjour est plus élevé
que celui du
pré-séjour, le résultat final obtenu est considéré comme
positif. Par conséquent, nous
observerons soit un score négatif, positif ou égal (sans changement).


Nous avons mis en détail  \textbf{à titre d'exemple } 3 patients du GC et 2 du GM
( les seuls valides pour la comparaison, le patient CAV mentionné comme preuve de l'inexistence du questionnaire rempli en fin de thérapie) afin d'être le plus clair possible
dans notre façon de procéder.
A la fin, nous avons illustré en couleur (Fig. 5.25 ) les
résultats finaux des deux groupes au complet avec 2 schémas
comparatifs.
\subsection{Groupe Contrôle: observation des résultats }
%\paragraph{ GC: Représentation des résultats avec 3 patients du groupe contrôle:}
Les patients présentés ici ont rempli toutes les conditions de l'étude.
\begin{enumerate}
\item : A. Patient Br.:  25/27 - 21/22 - 12/11 - 33/32 =  ''-''

          Résultat: 21,6 contre 23 pré-traitement,  ce qui
        correspond au signe négatif.
      \item : B. Patient Sch.: 30/27 - 20/20 -  10/10 - 35/30 = ''-''

         Résultat: 21,75 contre 23,75 pré-traitement, ce qui
        correspond au signe négatif.

 		\item :  C. Patient Wal. : 24/19 -  17/18 - 6/5 -
                  27/20 =  ''-''

                  Résultat: 15,5 contre 18,5 pré-traitement, ce qui
        correspond au signe \textbf{négatif: ''-''}.
 	\end{enumerate}


       \textbf{ Conclusion}: les résultats sont \textbf{négatifs}.
        Ces exemples confirment
        le ressenti subjectif moyen de l'ensemble des patients
        GC post-traitement.
        \subsection{Groupe Musicothérapie: observation des résultats}
 Il ne s'agira ici que de  deux patients car ce sont les seuls qui ont rempli toutes les conditions de l'étude, 
 y compris celle de remplir le questionnaire en fin de séjour.
       %\paragraph{ GM: Représentation de résultats avec 2 patients du groupe de musicothérapie}

\begin{enumerate}
 		\item : A. Patient Sw. : 26/25 - 19/19 - 8/8 - 29/30 =  ''=''



  Résultat: 20,5 contre 20,5 pré-traitement, ce qui
        correspond au signe égal.



 		\item : B. Patient M. : 17/27 - 13/23 -  9/10 - 24/32 = ``++''

              Résultat: 23 contre 15,75 pré-traitement, correspondant
              au signe \textbf{positif: ""+""}
            \end{enumerate}
 \textbf{ Résultat final}: les résultats sont \textbf{positifs}.


                 Ainsi,  GM s'exprime
                 \textbf{positivement}
                 sur l'ensemble du séjour en clinique, résultat très relatif, ne pouvant être  représentatif du 
                 groupe de 
                 musicothérapie.

                 \subsection{Questionnaires WHO - QOL : résultats du comparatif pré/post-thérapie}



\begin{figure}
\centering
\includegraphics[width=\linewidth]{images/Compcontrole.png}
\caption[Schéma du déroulement]{WHO - QOL:  GC. Comparatif pré/post-traitement, avec 7 patients}

%\label{groupecontroleimage1}
\end{figure}

\begin{figure}
\centering
\includegraphics[width=1.0\linewidth]{images/Compmusico.png}
\caption[Schéma du déroulement]{ WHO -  QOL: GM. Comparatif pré/post-traitement avec 2 patients}

%\label{groupecontroleimage1}
\end{figure}


Nous avons obtenu un comparatif graphique  des résultats des questionnaires
pré/post-traitement du groupe de contrôle, puis du groupe de musicothérapie,
graphiques se trouvant sous les Fig. 5.14 et 5.15.
       En résumé, nous observons que, selon les chiffres obtenus, le ressenti
       subjectif d'amélioration psychique
        des patients suivis en musicothérapie apparait comme
        supérieur.
        De manière générale, l'ensemble des données des deux groupes représentés
        par les graphiques corrobore ce résultat.
        Ces données sont des valeurs indicatives car nous avons conscience que l'échantillonnage ne
        peut pas être représentatif, comme déjà dit plus haut, dû
        notamment à un
        manque de
        questionnaires WHO - QOL, raisons pour lesquelles nous avons
        restreint le nombre d'exemples WQ présentés ici, pour obtenir
        une parité avec les tests d'écoute et obtenir la
        \textbf{corrélation test d'écoute et questionnaire} qui
        suit:

  \section{Corrélation des tests d'Ecoute et WHO - QOL avec résultats }
\textbf{Groupe Contrôle:} 	          \textbf{ test d'écoute: ``=''   et    WQ: ``-'}


\textbf{Groupe Musicothérapie:}     \textbf{test d'écoute: ``+''      et    WQ: ``+''}


\begin{sidewaysfigure}
\centering
\includegraphics[width=\linewidth]{images/graphiques/comparaison_pre_post.png}
\caption[Corrélation résultats pré/post]{Comparatif
  pré/post-traitement, WHO - QOL, test d'écoute, GM, GC.}

\label{comparaison_pre_post}
\end{sidewaysfigure}
\begin{figure}
\centering
\includegraphics[width=\linewidth]{images/graphiques/comparatifWQecoute.png}
\caption[Comparatif résultats pré/post]{Comparatif
  pré/post-traitement, WHO - QOL, test d'écoute, GM, GC. 
   Cav. et K. de GM n'ont pas rempli le WQ final.}

\label{comparaison_pre_post}
\end{figure}

Le résultat final comparatif en corrélation du test d'écoute et du questionnaire WH QOL nous permet
                de relever l'impact positif de la
                musicothérapie sur GM, résultat renforcé
                avec le WQ, même si nous n'avons pas la confirmation de tous les questionnaires finaux.


                Pour GC, l'ensemble des résultats sont neutres pour le
                test d'écoute. En ce qui concerne le
                regard des patients sur eux-même avec le WQ, il est
                même négatif. Avec les patients du Groupe de
              Contrôle, nous remarquons, grâce aux tests, une courbe aérienne
              sans modification mais une courbe osseuse plus
              particulièrement réactive. Contrairement à
              ce que le patient pouvait ressentir ou estimer, nous pouvons supposer qu'il y a indication  et attestation d'une amorce de
              processus intérieur et ce, par un autre biais, celui de
              la transformation de son
              écoute.

 Par ailleurs, indistinctement pour les deux
 groupes, il existe ainsi pour le thérapeute des
 suggestions de différentes pistes de travail dans le but de
 solliciter le patient plus spécifiquement en se référant aux
              différentes zones (Cf. Perspectives, Ch. 6, Fig. 6. 5/ 6. 6), également zones
 d'élaboration psychique. Ce peut être, par exemple,
              l'expression verbale, si la courbe aérienne est restée
              totalement ``muette'' et la zone 2 non
              réactive.

              Pour le groupe de contrôle, visiblement, le travail
                thérapeutique pouvait être plus accentué dans ce
                sens, renforcé à plus forte raison sous la forme musicothérapeutique, pour soutenir le
                patient dans sa transformation et sa mise en résonance
                interpersonnelle.


                Par conséquent,  le test d'écoute a
                apporté un autre regard avec des compléments d'informations au questionnaire
                WQ.

                \textbf{ En conclusion, le test
                d'écoute peut être \textbf{révélateur d'un
                travail réalisé en musicothérapie}}.






















































































%\paragraph{Hypothèse}



%\paragraph{Y-a-t-il une modification de l'écoute du patient après une prise
%en charge en musicothérapie ?}
%Est-ce que le processus d'écoute en musicothérapie améliore la capacité
%d'écoute ? Devient-elle différente après une musicothérapie?

%Est-ce que les test auditifs avant et après la musicothérapie permettent
%de visualiser l'action de la musicothérapie?


%\paragraph{Est-ce que les résultats ($=$ un changement dans l'écoute) d'une prise
%en charge musicothérapeutique peuvent être lisibles et visibles dans
%un test d'écoute?}
%Est-ce possible d'évaluer un travail musicothérapeutique au moyen
%d'un test d'écoute?
%Est-ce que ces résultats sont significatifs?

%\paragraph{Est-ce que l'écoute du patient s'est modifié ? si on a pu observer
%une modification, dans quel sens va -t-elle ?}

%Le contexte:
%est-ce que le contexte est suffisant pour
%ressortir des résultats ?
