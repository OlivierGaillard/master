% !TEX root = ./master.tex
\chapter[\'Etude clinique]{\'Etude clinique}

%Souvenons-nous brièvement de l'évolution de la signification du terme
%``clinique'' enraciné dès l'Antiquité.
%A l’origine,\textbf{ l’activité clinique} (<gr. klinê = le lit) est celle du médecin qui, au chevet du malade, procède à l’examen des manifestations de la maladie (méthode de l’observation, de l’interrogation et de l’écoute) en vue de poser un diagnostic*, un pronostic* et une prescription* de traitement.
%Michel Foucault (1926-†1984), psychologue français, dans son ouvrage capital de 1963,
%«Naissance de la clinique» nous rend attentifs au fait que l’adjectif «clinique» fut longtemps du monopole médical, avec %l’observation « naturelle » faite au lit du malade, uniquement à l’aide des organes sensoriels.

%Hippocrate (médecin grec, Cos 460-377) était clinicien : il apprenait à ses étudiants l’art d’observer les symptômes, ceux-ci étant les réactions d’une personnalité à une agression pathogène. Généralisation et rationalisation, selon les critères d’Aristote, permettaient l’élaboration de « ces entités nosologiques* », les maladies, qui s’emparaient du corps du malade et poussant les médecins, exorcistes laïques, à les en faire sortir.
%Les publications de Th.Ribot à la Sorbonne rejoignent [avec «La psychologie des sentiments »(1896), «Les maladies de la mémoire »(1883) et «Les maladies de la personnalité » (1885)], les idées de S. Freud, montrant la primauté de la vie affective, où les tendances inconscientes jouent un rôle fondamental, et pouvant s’extérioriser soit par l’arrêt du développement affectif, soit par la dissolution des acquisitions plus récentes.

%L’expression « psychologie clinique » apparaît sous la plume de S. Freud, dans sa lettre à W. Fliess du 30.01.1899 : « Maintenant, la connexion avec la psychologie, telle qu’elle se présente sur les Etudes sur l’hystérie(1895) sort du chaos, j’aperçois les relations avec le conflit, la vie, tout ce que j’aimerais appeler psychologie clinique ».

%La définition « officielle » de la psychologie clinique mobilise et articule la singularité et
%la totalité, de façon à reconnaître une discipline psychologique basée sur l’étude approfondie des cas individuels, l’étude de la conduite humaine individuelle et de ses conditions (hérédité, maturation, conditions psychologiques et pathologiques, histoire de la vie), en somme, l’étude de la personne totale «en situation», c.à.d. l’expérience vécue de ce rapport à l’environnement.

%Au vue de cet aperçu historique, on peut reconnaître que
%dans certains milieux psychiatriques actuels, la musicothérapie trouve plus sa place aussi dans le complètement des tableaux cliniques.

%Dans notre étude, l'axe principal porte sur la
%vérification de l'amélioration de
%la capacité d'écoute suite au travail musicothérapeutique.

% et procéder à la comparaison des modifications de
%l'écoute.



%\textbf{Organigramme}: nous avons obtenu l'autorisation de vous
%montrer l'organigramme de la Clinique de Meiringen lors de la
%soutenance.




%(-OH*:  le radical hydroxile, oxydrile de la molécule éthylique)



%\textbf{Le Groupe Musicothérapeutique (expérimental) GM} comporte 21
%femmes et 15 hommes.

%\textbf{Le  Groupe Contrôle GC} comporte 7 patients, dont 4 femmes et 3 hommes.

%Problème avec la comparaison et la corrélation du questionnaire avec le test d'écoute.
%Fallait-il maintenir le WQ?
%Si seulement test d'écoute
%En voulant maintenir la comparaison autant dans le nombre de WQ que de tests d'écoute
%le critère d'exclusion a été de celui d'obtenir à chaque fois le comparatif pour chaque patient
%Finalement, en tenant compte de tous ces paramètres, les participants sont au nombre de 15, tous les 
%groupes confondus:

%Afin d'obtenir une comparaison similaire tout en se basant sur la quantité de questionnaires remplis,
%de patients a dû se restreindre à
%\textbf{GM = 8 patients (4 hommes/4 femmes)
%GC = 7 patients (4 hommes/3 femmes)}


 \begin{itemize}

 \item \textbf{Nombre total de personnes}: N=  \textbf{13}:
%\textbf{Le  Groupe Contrôle GC} comporte 7 patients, dont 4 femmes et 3 hommes.

\textbf{GC: 7 : 4 femmes et 3 hommes} : Total: 14 tests pré/post
 	
\textbf{	GM: 6 : 3 hommes et 3 femmes}: Total: 12 tests pré/post
\item \textbf{Total de séances} par personne en
   musicothérapie= 4 ;   \textbf{mu}=1/semaine;
 \textbf{t}= 50--60 min, période = 3 -- 4 semaines.
  \item\textbf{Durée}: 4 semaines distribuées dans l'intervalle juin -- octobre 2017.
 \item\textbf{Genre et âge de la population étudiée:}  de 25 à 72
 ans dont l'âge moyen est de 48 ans.
 %19 hommes et 10 femmes,
 \item\textbf{Total  des test d'écoute: 26}
 \item\textbf{Total des questionnaires: 9}
\end{itemize}


\begin{figure}
	\centering
	\includegraphics[width=1\linewidth]{images/graphiques/Testecoute.png}
	\caption{Nombre de tests d'écoute avec GM et GC: 14 tests comparables (2x7) avec GC et 12 (2x6) 
	pour GM, on 
	constate 
	un nombre importants de tests  inutilisables en Pré/thérapie.}
	
	%\label{groupecontroleimage1}
\end{figure}



\begin{figure}
	\centering
	\includegraphics[width=1\linewidth]{images/graphiques/TestWQ.png}
	\caption{Nombre de WHOQOL avec GM et GC: 7 questionnaires comparables avec GC et 2 
	avec GM.}
\end{figure}



 %mu en grec


%Après ces prémisses, l'étude commence véritablement avec l'application du test
%audiométrique suivie du questionnaire qualitatif.

    
     
     % Cela nous a 
     %conduit à 
     %la difficulté d'obtenir un nombre égal pour comparer.
     Le nombre de tests d'écoute a été presque obtenu dans sa totalité pour les 2 groupes mais 
     pas 
     pour le questionnaire: un nombre insuffisant  de WQ en fin de 
     séjour dans le groupe de musicothérapie.
     Cette  incidence a réduit considérablement la démonstration de cas proposés.
     %Nous nous sommes donc tenus à notre plan, constatant ces données  manquantes.
     En maintenant nos objectifs, c. à. dire  le comparatif autant  dans le nombre de WQ que dans le  tests 
      d'écoute, 
     %le critère d'exclusion a été de celui d'obtenir à chaque fois le comparatif pour chaque patient
     %Finalement, en tenant compte de tous ces paramètres,
     nous obtiendrons pour  les deux groupes confondus, \textbf{13 patients, 7 pour le groupe de contrôle 
     et 
     	6 pour 
     	le groupe de musicothérapie.}
     Sur \textbf{25 questionnaires WHOQOL}, il y en aura  \textbf{15 pour GC} dont 8 pré-
     et 7 post-thérapie et  \textbf{10 pour GM} remplis
     avec 8  pré- mais seulement 2
     post- thérapies:  nous aurons donc dans l'ensemble un total de \textbf{9 questionnaires} pour le
     comparatif des 2 groupes réunis.
    % Si statistique il y aurait, la survie ne serait pas élevée.
   
   
   
   Il convient ainsi de mentionner, vu la taille réduite des échantillons, qu'il n'est pas
   pertinent de se lancer dans une analyse purement
   quantitative.
   Les données que nous obtiendrons seront des valeurs indicatives.
   %dû entre autres à un très petit nombre 
   %de
   % questionnaires WQ remplis en fin de séjour pour le groupe de musicothérapie. 
   Toutefois, nous avons tenu à décrire exactement le déroulement de cette étude. Nous avons 
   décortiqué 
   tous les questionnaires et étudié tous les tests d'écoute des deux groupes.
   Avec le questionnaire 
   WHOQOL,  il  s'agira d'un analyse qualitative avec des résultats chiffrés, mais ici dans ce contexte
   statistiquement pas évaluable. 
   Avec le test d'écoute, nous obtiendrons des valeurs quantitatives sur la qualité de l'écoute, illustrés 
   par 
   des tableaux.
   Dans son ensemble, l'analyse sera considérée comme quantitative  et qualitative.
   
   
   
   \section{Questionnaires WHOQOL}
  % \section*
   \textbf{Le comparatif pré/post-thérapie:}
   nous avons mis en détail, afin que notre procédure soit claire, un comparatif entre des patients du 
   groupe de contrôle et du groupe de 
   musicothérapie dont  seuls deux du GM étaient valides car remplis en fin de thérapie. Nous aurons la 
   totalité des 
   résultats Fig. 5.1, 5.2, 5.3.
   %le patient Cav est  mentionné pour exemple de preuve de l'inexistence du questionnaire rempli en fin 
   %de 
   %thérapie) .
   %A la fin, nous avons illustré en couleur (Fig. 5.16) les
   %résultats finaux des deux groupes au complet avec 2 schémas
  % comparatifs.
   %\section*
   \paragraph{Groupe Contrôle: observation des résultats}
   %\paragraph{ GC: Représentation des résultats avec 3 patients du groupe contrôle:}
   : voici les exemples de trois patients qui ont rempli toutes les conditions de l'étude.
   \begin{enumerate}
   	\item : A. Patient Br.:  25/27 - 21/22 - 12/11 - 33/32 : $-$
   	
   	Résultat: 21,6 contre 23 pré-traitement,  ce qui
   	correspond au signe négatif.
   	\item : B. Patient Sch.: 30/27 - 20/20 -  10/10 - 35/30 :  $-$
   	
   	Résultat: 21,75 contre 23,75 pré-traitement, ce qui
   	correspond au signe négatif.
   	
   	\item :  C. Patient Wal. : 24/19 -  17/18 - 6/5 -
   	27/20 :   $-$
   	
   	Résultat: 15,5 contre 18,5 pré-traitement, ce qui
   	correspond au signe \textbf{négatif:  $-$}.
   \end{enumerate}
   
   
   \textbf{ Résultats }: les résultats présentés sont  \textbf{négatifs} 
   et confirment le ressenti subjectif moyen de l'ensemble des patients
   GC post-traitement,  comme nous le verrons avec  le résultat final  (Fig. 5.13) de
   cinq patients:  \textbf{négatif:  $-$} et deux patients:  \textbf{positif:  $+$}.
   
% \section*
\paragraph {Groupe Musicothérapie: observation des résultats}
: il ne s'agira ici que de  deux patients car ce sont les seuls qui ont rempli toutes les conditions de 
l'étude, 
 c'est-à-dire y compris celle de remplir le questionnaire en fin de séjour.
 %\paragraph{ GM: Représentation de résultats avec 2 patients du groupe de musicothérapie}
 
 \begin{enumerate}
 	\item :  Patient Sw. : 26/25 - 19/19 - 8/8 - 29/30 :   $=$
 	
 	
 	
 	Résultat: 20,5 contre 20,5 pré-traitement, ce qui
 	correspond au signe égal.
 	
 	
 	
 	\item : Patient M. : 17/27 - 13/23 -  9/10 - 24/32 :  $++$
 	
 	Résultat: 23 contre 15,75 pré-traitement, correspondant
 	au signe \textbf{positif: $+$}
 \end{enumerate}
 \textbf{ Résultat final}: les résultats sont \textbf{positifs}.
 Ainsi,  GM s'exprime
 \textbf{positivement}
 sur l'ensemble du séjour en clinique, résultat très relatif, vu le petit nombre, ne pouvant être  
 représentatif du 
 groupe de 
 musicothérapie.
 Voici en résumé (Fig. 5.3) le schéma représentant la moyenne pré/post traitement, calculée pour chaque 
 patient, des scores des 4 domaines.
 
 
 \begin{sidewaysfigure}
 	\centering
 	\includegraphics[width=\linewidth]{images/graphiques/questionnaire_wq.png}
 	\caption[Questionnaire WHOQOL-BREF]{GM/GC - Pré/Post avec la moyenne des scores par 
 		domaine; GC: réponses positives: 2/7 - 28,6\% ; GM: réponse positive: 1/2 - 50\%}
 	
 	%\label{groupecontroleimage1}
 \end{sidewaysfigure}
 
 %Les chiffres ont été obtenus à partir des 4
 %domaines, avec le pré/post-séjour.
 
 
 
 
 %\section*
 \paragraph{Questionnaires WHOQOL: total des résultats du 
 	comparatif pré- / post- thé\-ra\-pie}
 
 
 
 \begin{figure} [th]
 	\centering
 	\includegraphics[width=\linewidth]{images/Compcontrole.png}
 	\caption[Schéma du déroulement]{WHOQOL:  GC: Comparatif pré (en bleu)/ et post (en jaune) 
 	/traitement, avec 7 patients}
 	
 	%\label{groupecontroleimage1}
 \end{figure}
 
 \begin{figure}[th]
 	\centering
 	\includegraphics[width=1.0\linewidth]{images/Compmusico.png}
 	\caption[Schéma du déroulement]{ WHOQOL: GM: Comparatif pré (en bleu)/ et post (en jaune) avec 2 
 	patients}
 	
 	%\label{groupecontroleimage1}
 \end{figure}
 
 
 Nous avons créé un schéma des résultats des questionnaires
 pré/post-traitement du groupe de contrôle, puis du groupe de musicothérapie,
 qui  se trouve sous les Fig. 5.3/ 5.4/ 5.5.
 En résumé, nous observons selon les chiffres obtenus que  le ressenti
 subjectif d'amélioration psychique
 des patients suivis en musicothérapie apparait comme
 supérieur.
 De manière générale, l'ensemble des données des deux groupes représentés
 corrobore ce résultat.
 Ces données sont des valeurs indicatives car nous avons conscience que le petit échantillonnage ne
 peut être représentatif.

 
 \clearpage

 \section{Test d'écoute}
 % \subsection*
  \paragraph{Comparatif pré/post-thérapie:}
  %Avant d'aborder la synthèse des résultats avec la
 % \textbf{corrélation test d'écoute et questionnaire}, nous avons réuni et rassemblé sous un seul 
 %chapitre tous les autres tests d'écoute.
 pour une comparaison simplifiée et mise en parallèle avec le WQ, nous avons pris les mêmes   
 patients du GC et du GM. Avec GM, nous avons rajouté  le patient Cav et spécifié l'absence de 
 questionnaire final sur la qualité de vie.
 Tous les autres seront visibles et résumés à la figure 6.1.
 \subsection*{Groupe Contrôle: observation}
  % \textbf{Groupe Contrôle: }
  \paragraph{ Patient Br.:}
  Anamnèse: le patient Br., de sexe masculin né en 1964, marié, a suivi des études supérieures. Il a été 
  diagnostiqué souffrant de dépression avec 
  traumatisme.  Naturellement gaucher, il raconte avoir été forcé à écrire de la main droite.
  
  \begin{figure}[ht]
\centering
\includegraphics[width=1\linewidth]{images/graphiques/bru_pre.png}
\caption[  \textbf{Groupe Contrôle}: Patient Br.. 1° test]{Premier test Br.}
%moyenne OG et OD
%\label{groupecontroleimage1}
\end{figure}



 \begin{figure}[th]
\centering
\includegraphics[width=1\linewidth]{images/graphiques/bru_post.png}
\caption[Patient Br. :  2° test]{Second test Br.}

%\label{groupecontroleimage1}
\end{figure}

	\begin{enumerate}
 		\item  c.a.: pas de modification, augmentation des
                  seuils: $-$
 		\item  c.o.: redressement des seuils: $+$
 		\item  croisements: $5/4$ : $+$ : ce qui signifie:  5 croisements lors du 1°test// 4 croisements lors du 
 		2° test= nous avons 1 croisement en moins, donc le résultat est considéré comme légèrement 
 		positif en fin
                  de séjour.
                \end{enumerate}

                \textbf{ Résultat:  $- $  $+ $   $+ $     :   $+$}
                
                Observation et interprétation: Après plusieurs séjours en clinique, il a participé durant cette 
                période à plusieurs groupes: le 
                Psychogroupe, le Stressgroupe et l'atelier du bois. Le TMS, la stimulation  magnétique 
                trans-crânienne 
                (15x, à raison 
                d'une fois par jour) n'a pas eu beaucoup d'effet et a été vécue difficilement.
                Il n'a pas participé à la musicothérapie, donc appartenait au groupe contrôle. 
                Toute la zone concernant les hautes fréquences et  qui 
                correspond à l'énergie, à 
                l'esprit d'entreprise ou de motivation, n'a presque pas été modifié, comme nous pouvons le 
                constater  
                par la lecture du test, en considérant la courbe osseuse qui a pris un chemin légèrement 
                meilleur en se glissant sous l'aérienne;  ce qui correspond à une légère réduction des tensions 
                internes et une meilleure auto-écoute, d'où le résultat considéré comme positif au niveau de la 
                capacité d'écoute.
                
                En comparant avec le questionnaire WQ, le résultat: 21,6 contre 23 pré-traitement, 
                correspond au signe  négatif, mais avec peu de différence.
                En conclusion, le résultat du séjour dans son ensemble est mitigé, légèrement positif et 
                légèrement 
                négatif.
% \clearpage

\paragraph{Patient Sch.:}
Anamnèse: le patient Sch, de sexe masculin, né en 1959, vit séparé, a suivi des études supérieures. Le 
patient joue de la guitare en auto-didacte depuis deux ans. Sportif et très actif, il ne 
comprend plus ce qui se passe dans sa tête. Il devrait porter un appareil auditif qu'il a refusé il y a 6 mois. 
Il a été diagnostiqué dépressif. Il s'agit 
de son premier séjour en clinique et  a participé à une formation de soutien pour comprendre son état 
actuel. Il n'a pas participé au groupe de musicothérapie mais se montre très intéressé par l'étude.
\begin{figure}[th]
\centering
\includegraphics[width=1\linewidth]{images/graphiques/schaff_pre.png}
\caption[Patient Sch. : 1° test]{Premier test Sch.}

%\label{groupecontroleimage1}
\end{figure}


         \begin{figure}[th]
\centering
\includegraphics[width=1\linewidth]{images/graphiques/schaff_post.png}
\caption[Patient Sch.: 2° test]{Second test Sch.}

\end{figure}


\begin{enumerate}
	
	\item : c.a.:  modification : +, très légère augmentation puis chute des
	seuils: -
	\item : c.o.: a passé sur c.a. : -, abaissement des seuils:  -
	\item : croisements: 2/1 :     +
\end{enumerate}
\textbf{ Résultat:  $-$    $-$   $+$         :   $-$ }


Observation et interprétation: comme pour le patient précédent, la 3° zone est en chute libre. Avec le 
test, il est visible qu'il ne peut entendre les fréquences hautes (8000 à 3000 Hz) qu'à fort volume.  Il a  
beaucoup de difficulté à extraire des informations dans un 
lieu normalement bruyant. Le patient avec qui il doit partager sa chambre en clinique est extrêmement 
bruyant 
(ronflement), il ne peut plus dormir et devient très nerveux. Il espère quitter la clinique au plus 
vite.


Remarque: le séjour en clinique n'a pas été très fructueux puisque le patient ne se sent pas 
particulièrement mieux à sa sortie. Le questionnaire rempli l'attestera ainsi que la lecture du second test 
d'écoute: avec ce dernier, nous pouvons observer même un déplacement de la courbe osseuse sur la 
courbe aérienne attestant d'une réactivité et sensibilité même plus importante que lors du début du 
séjour. Le tableau illustrant le  comparatif WQ et test d'écoute (Cf. Fig. 6. 1)  met clairement en évidence 
l'ensemble des 
résultats, ici négatifs pour ce patient: résultat du WQ: 21,75 contre 23,75 pré-traitement.
En conclusion, les résultats concordent entre le test d'écoute et le WQ.
 \clearpage

\paragraph{ Patient Wal.:}

Anamnèse: la patiente Wal., de sexe féminin, née en 79, a suivi une formation supérieure, vit seule,  
souffre de troubles post-traumatique combinés avec la dépendance. Traumatisée très jeune 
sexuellement, elle souffre de solitude;  un séjour d'intégration dans une famille paysanne sera prévu 
ultérieurement.
	\begin{enumerate}

 		\item : c.a.: peu de modification: = ;  seuils: léger redressement: +

 		\item : c.o.:  modification:  + ; reste dominante, tentative de rapprochement de c.a.: +
 		\item : croisements: 1/3 :  -

                \end{enumerate}

                \textbf{ Résultat:  $ + +  -        : +$ }
\begin{figure}[th]
	\centering
	\includegraphics[width=1\linewidth]{images/graphiques/wal_pre.png}
	\caption[Patient Wal. :1° test]{Premier test Wal.}
	
	%\label{groupecontroleimage1}
\end{figure}
               \begin{figure}%[tb]
\centering
\includegraphics[width=1\linewidth]{images/graphiques/wal_post.png}
\caption[Patient Wal. : 2° test]{Second test Wal.}

\label{groupecontroleimage1}
\end{figure}


Observation et interprétation:: si l'on considère la zone 1, (Cf. Moyenne après Wall Fig. 5. 11),  concernant 
le 
physique, on 
peut observer 
encore beaucoup de 
perturbation, visible avec la chute des décibels lors du second test,  et une perturbation dans la zone 2, 
qui concerne l'expression verbale et la communication. En effet, la patiente s'exprimait  à voix basse, 
chuchotait, lors du premier comme du second test.


Remarque: la façon d'écouter s'est modifiée, apportant des signes positifs dans son processus,  ici sans 
aide de musicothérapie. On verra par contre que les résultats du WQ sont négatifs: 15,5 contre 18,5 
pré-traitement, ce qui
correspond au signe négatif.
En conclusion, le séjour, trop bref selon notre hypothèse, a apporté des résultats mitigés (+ et -). 
\clearpage


\subsection*{Groupe Musicothérapie: observation}
 % \textbf{Groupe de Musicothérapie: }

\paragraph{ Patient Sw.:} anamnèse : patient de sexe masculin âgé de 40 ans, études supérieures, vit 
avec sa partenaire, a une petite fille de 4 ans, souffre de troubles dépressifs avec épisode et récidive de 
degré moyen: dépression dite d'épuisement dans le cadre d'une situation psychosociale difficile.
Le patient a subi une opération du coeur à 30 ans. Lors du test d'écoute, le battement  cardiaque était 
fortement amplifié avec les écouteurs. Il souffre d'acouphènes que nous avons localisé à la hauteur de 
6000 
et 8000 Hz. Il n'est ni droitier ni gaucher, donc ambidextre dans sa façon d'écouter, a de la peine à 
extraire 
des conversations dans un milieu moyennement bruyant et souffre de problème de concentration.



 \begin{figure}[th]
\centering
\includegraphics[width=1\linewidth]{images/graphiques/sw_pre.png}
\caption[ \textbf{Groupe Musicothérapie}: Patient Sw. : 1°Test]{Premier test Sw.}

%\label{groupecontroleimage1}
\end{figure}

	\begin{enumerate}

 		\item : c.a.: pas de modification: = %  1,27/1,27

 		\item : c.o.: redressement et rapprochement,
                  relèvement des seuils: -       %  3,07/3,39
 		\item : croisements: 1/3 :  -

                \end{enumerate}

                \textbf{ Résultat:  = +  -        : ``=''}

                \begin{figure}[th]
\centering
\includegraphics[width=1\linewidth]{images/graphiques/sw_post.png}
\caption[Patient Sw.: 2° test]{Second test Sw.}

%\label{groupecontroleimage1}
\end{figure}

 Observation et interprétation:  Séances de musicothérapie: 
 le patient écoute beaucoup de musique classique et adore F. Chopin.
 Nous avons ciblé un travail très actif qui convenait au patient, très curieux, particulièrement avec un 
 instrument comme le psalterium qui touche beaucoup de fréquences; celui-ci l'intriguait beaucoup et 
 des échanges non-verbaux s'en sont suivis longuement; il a aimé faire  l'exercice d'écoute attentive 
 d'un 
 son qui doit finir de résonner avant la réponse de l'autre. 
Lors du test de fin de séjour, (avec un résultat notifié sous le signe = ) il a répondu vite et avec 
concentration. Il se sentait bien, 
projette de changer de métier, débutera une formation différente et abandonne  son travail 
d'informaticien.


Le résultat du WQ a été de 20,5 contre 20,5 pré-traitement, ce qui
correspond aussi au signe égal.
En conclusion, nous pouvons considérer le test équivalent au ressenti recueilli par le questionnaire WQ. 
Il ne correspond ni à un signe négatif ou positif. On peut supposer une bonne influence mais non 
surprenante ou spectaculaire de la musicothérapie, puisque le patient baigne lui-même beaucoup dans 
la musique, il avait sans doute déjà en main des ressources et  
besoin de ce séjour (avec tout le contexte de la clinique, dont la stabilisation médicamenteuse) pour 
retrouver un équilibre et tendre vers une meilleure qualité de vie. 
Nous ne pouvons ni infirmer ou confirmer dans son cas l'impact d'une prise en charge en  
musicothérapie mais il aurait été  peut-être 
intéressant de le suivre de manière plus méthodique. 
%\clearpage

\paragraph{ Patient Cav.: }
Anamnèse: la patiente Cav de sexe féminin, psychotique, souffre de troubles bipolaires. Sa mère est 
hospitalisée depuis 5 semaines en psychiatrie.
Elle  a commencé le violon à 5 ans, en a joué pendant 13 ans et a suivi également une école de jazz en 
chant. Elle se plaint d'un manque important d'énergie et d'attaques de panique et n'a pas rempli le 
questionnaire WOQOL en fin de séjour.
\begin{figure}%[th]
\centering
\includegraphics[width=1\linewidth]{images/graphiques/cav_pre.png}
\caption[Patient Cav. : 1° test]{Premier test Cav.}

%\label{groupecontroleimage1}
\end{figure}

	\begin{enumerate}

 		\item : c.a.: redressement: +

 		\item : c.o.: redressement et rapprochement, relèvement des seuils: +
 		\item : croisements: 3/1 :  +

                \end{enumerate}

                \textbf{ Résultat:  + + +       : ``+''}

                \begin{figure}%[th]
\centering
\includegraphics[width=1\linewidth]{images/graphiques/cav_post.png}
\caption[Patient Cav. : 2° test]{Second test Cav.}

%\label{groupecontroleimage1}
                \end{figure}
            
            Observation et interprétation:  séances de musicothérapie: c'est la première fois que la patiente 
            commence une  musicothérapie.
            Elle s'est montrée extrêmement sensible au 
            moindre bruit.
            Elle a touché et caressé avec ses doigts doucement et longuement le grand tambour. 
            Il apparaît qu'elle utilise beaucoup son oreille droite, son corps pivotant  se tournant 
            constamment du 
            côté droit à chaque léger son produit par la thérapeute, comme par protection et  pour contrôler la 
            situation. 
            D'après le test, il est très difficile pour elle de s'extraire des sons environnants: en effet, on 
            pourrait en 
            déduire que ce stress permanent dans lequel elle vit et dont elle se plaint, est issu de cette 
            incapacité à 
            faire face à l'environnement sonore. Elle souffre aussi de cette séparation corps-esprit qui semble 
            être  
      la seule et ultime protection trouvée pour échapper à une situation  de menace  et d'agressivité 
      permanente vécue dans le milieu familial.
            La patiente parlera beaucoup, montrera un besoin accru de confidence. Lors du second test,  les 
            attaques de panique vécues entre les séances auront fortement 
            diminué, dira-t-elle. Après lecture du test, le profil de la courbe osseuse reste dominant et 
            toujours  
            supérieur à l'aérien dénotant la très grande fragilité de la patiente, 
            sujet très à vif et n'ayant pas la capacité de se protéger d'intrusions ou d'événements extérieurs; 
            après le second 
            test, on observera  qu'elle reste encore très 
            perturbée et hypersensible: elle sera placée en sécurité 
            lors de sa sortie dans un foyer protecteur. 
            
            
            Remarque: la possibilité de se retrouver dans un milieu clos et protégé lors des séances de 
            musicothérapie ont été bénéfiques:  l'alliance thérapeutique construite par les dialogues et les 
            confidences,  le respect et l'écoute de la patiente dans sa douleur, dans son expression verbale, 
            ainsi 
            que le fait  de pouvoir 
            évoluer librement au milieu de cet espace, tout en effleurant 
            les instruments.
             %car les menaces exercées par  son oncle 
           % existent 
           % toujours. 
            %Remarque: la thérapeute ne fait aucun commentaire sur le test lors de la sortie des patients.
         \paragraph{ Patient M.:}
       
       Nous avons, comme pour les autres patients, le résultat comparatif de la synthèse pré/ post thérapie 
       de la faculté d'écoute des deux oreilles. Lors des séances commentées plus en détail, (Cf. Ch. 3. 3.: 
       séance de M. commentée plus haut), nous avons analysé l'oreille droite plus en profondeur.
	\begin{enumerate}

 		\item : c.a.: redressement: : +   % 6,43/6,03

 		\item : c.o.: redressement et rapprochement,
                  relèvement des seuils:  +     %6,25/5,85:
 		\item : croisements: 3/3 :  =

                \end{enumerate}

                \textbf{ Résultat:  +  +  =     : ``+''}
                
                
    Le résultat du WQ est de  23 contre 15,75 pré-traitement, correspondant
    au signe très positif. En conclusion, l'impact de la musicothérapie a été très bénéfique dans le cas de 
    ce patient.           

                \begin{figure}[th]
\centering
\includegraphics[width=1\linewidth]{images/graphiques/m_pre.png}

\caption[Patient M. : 1° test]{Premier test M.}

%\label{groupecontroleimage1}
\end{figure}


                        \begin{figure}[th]
\centering
\includegraphics[width=1\linewidth]{images/graphiques/m_post.png}
\caption[Patient M. : 2° test]{Second test M.}

%\label{groupecontroleimage1}
\end{figure}

  \clearpage

%\section*
\paragraph{Test d'Ecoute: Total des résultats du comparatif pré/post-thérapie}
%\textbf{Conclusions générales}:
 : nous nous trouvons
           en présence de deux groupes, un groupe de contrôle et un
           groupe de musicothérapie ayant le même type de
           pathologie -- difficulté de régulation des émotions -- .
Nous constatons, d'une part, que l'écoute est quantifiable.
           D'autre part, il existe bien
          une \textbf{modification de l'écoute pré / post -- traitement}.
Ensuite, il est à observer que
          cette modification est nettement plus marquée
          pour GM, groupe de musicothérapie, qui a un résultat positif.
          


          \textbf{GM: ``+''}.
          En d'autres termes: 4 résultats positifs sur 6 : 4/6 - 66,7\%
/1 résultat neutre sur 6 : 1/6 - 16,7\%
/1 résultat négatif sur 6: 1/6 - 16,7\%

Par contre,  pour le groupe de contrôle, GC, le résultat est mitigé, il correspond au signe d'égalité et n'apporte aucune vraie modification.

          \textbf{GC:  ``='' }.
En d'autres termes:  2 résultats positifs sur 7: 2/7 - 28,6\%
/3 résultats neutres sur 7: 3/7 - 42,9\%
/2 résultats négatifs sur 7:2/7 - 28,6\%







Remarquons que les données quantitatives observables dans
ces graphiques sem\-blent aller dans le
sens de  l'étude faite par le
CNRS  réalisée à partir des seuils auditifs, à savoir
les patients souffrant de troubles post-traumatiques souffrent d'un
\textbf{appauvrissement caractéristique de fréquences} \autocite{affectiveDisorders}.
Il suffit de regarder attentivement les graphiques suivants pour le constater.
 
 
 





















































































%\paragraph{Hypothèse}



%\paragraph{Y-a-t-il une modification de l'écoute du patient après une prise
%en charge en musicothérapie ?}
%Est-ce que le processus d'écoute en musicothérapie améliore la capacité
%d'écoute ? Devient-elle différente après une musicothérapie?

%Est-ce que les test auditifs avant et après la musicothérapie permettent
%de visualiser l'action de la musicothérapie?


%\paragraph{Est-ce que les résultats ($=$ un changement dans l'écoute) d'une prise
%en charge musicothérapeutique peuvent être lisibles et visibles dans
%un test d'écoute?}
%Est-ce possible d'évaluer un travail musicothérapeutique au moyen
%d'un test d'écoute?
%Est-ce que ces résultats sont significatifs?

%\paragraph{Est-ce que l'écoute du patient s'est modifié ? si on a pu observer
%une modification, dans quel sens va -t-elle ?}

%Le contexte:
%est-ce que le contexte est suffisant pour
%ressortir des résultats ?
