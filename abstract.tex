% !TEX root = ./master.tex

\begin{abstract}
	\centerline{\textbf{Abstract}}
	\begin{french}	
L'axe principal porte sur la comparaison de la faculté d'écoute des patients en conclusion d'un suivi 
individuel en musicothérapie après trois semaines en cadre clinique.
Le test d'écoute Tomatis\textsuperscript \textregistered  et le questionnaire WHOQOL sur la qualité de vie 
sont les moyens 
utilisés 
pour synthétiser les différences.
 %Des graphiques de courbes d'écoute synthétisent des différences pré/post traitement.
 Treize  patients, avec troubles de l'humeur, sont assignés à deux groupes de recherche: 
 six au groupe
 musicothérapie (GM) et sept au groupe contrôle (GC).
 % Le WHOQOL est rempli avant et 
 %après séjour afin d'évaluer leur état psychique.
L'analyse comparative des résultats révèle une modification positive et significative 
de qualité d'écoute après musicothérapie  %par le 
%test 
%d'écoute 
%Tomatis
 pour GM contrairement à GC, résultats correspondants à ceux recueillis par  %au ressenti subjectif 
 le questionnaire WHOQOL. 
  % où la transformation est
%faible et le questionnaire majoritairement négatif. % avec des valeurs plus 
%élevées. 
%Pour le groupe contrôle, la transformation a été faible et le questionnaire majoritairement 
%négatif.
En conclusion, la musicothérapie a %nfluencé positivement la façon d'écouter %
eu un impact positif autant sur la qualité d'écoute que sur la qualité de vie  du patient.
%corrélée à l'état psychique mis en évidence par le WHOQOL.
La totalité des tests d'écoute et des questionnaires demeure toutefois 
inférieure à ce qui avait été initialement planifié, et, en dépit d'une plus ample dimension statistique, ce 
travail avantagera plus les aspects qualitatifs que quantitatifs. %d'être 
%statistiquement significatifs, rendant notre travail plus qualitatif que quantitatif.
%Ayant tout à fait conscience des compétences scientifiques qu'une telle étude aurait exigé, ce travail 
%oscillera 

\textbf{Mots-clés: musicothérapie; écoute; son; oreille; test}
\end{french}

\begin{english}	
The main focus is on the comparison of the listening skills of patients at the conclusion of an individual 
follow-up in music therapy after three weeks in a clinical setting. The Tomatis® listening test and the 
WHOQOL questionnaire on quality of life are the means used to synthesize the differences. Thirteen 
patients with mood disorders were assigned to two research groups: six to the music therapy group (GM) 
and seven to the control group (GC). Comparative analysis of the results reveals a positive and 
significant change in listening quality after music therapy for GM, as opposed to GC, with the following 
results 
corresponding to those re-collected by the WHOQOL questionnaire. In conclusion, music therapy has 
had a positive impact on both the quality of listening and the quality of life of the patient. However, the 
totality of listening tests and questionnaires remains lower than initially planned and, despite a larger 
statistical dimension, this work will be more qualitative than quantitative. 
% the final number of listening tests and WHOQOL test was found lower than expected. 
%Added to the fact this was a small study. This therefore prevents our results from being statistically 
%significative. 
%Being fully aware of the scientific skills that such a study would’ve required, 

\textbf{Keywords: music therapy; listen; sound; hear; test}
\end{english}
\end{abstract}













%\begin{Remerciements}
