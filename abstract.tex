% !TEX root = ./master.tex

\begin{abstract}
\textbf{	Abstract}

Ecouter, s'écouter, communiquer.
 La musicothérapie permet de développer la communication en
 travaillant sur l'écoute. La musique influe notre corps
 tout entier ainsi que notre écoute, au point même de la
modifier.
Au moyen d'un test avec
un appareil spécifique, nous approfondirons cette
``évidence" dans le sens premier du terme : la visualisation de l'écoute et de sa
transformation.
Notre hypothèse est que,
lors d'un traitement en musicothérapie, nous pourrons constater
l'évolution psychique du
patient en analysant son écoute.
Des graphiques nous
permettront de synthétiser les différences pré-- et post-- traitement en
mettant en exergue les différentes écoutes, tels des ``clichés photographiques",  en
relation avec l'état psychique du patient et de son processus suivi en musicothérapie.







\textbf{Mots-clés: musicothérapie; écoute; son; oreille; test}
\end{abstract}

\begin{abstract}
\textbf{	Abstract}


 Listen, listen ourselves, communicate. Music therapy allow us to develop communication
 by working on the listening. Music influences our whole body as well as our listening, to
 the point of even modifying it. By means of a test with a specific device, we will
 deepen this "evidence" in the first sense of the term: the visualization of
 listening and its transformation.
 Our hypothesis is that, during a music therapy treatment, we will be able to observe the
 patient's psychic evolution by analyzing  their listening.
 Graphs will allow us to synthesize the differences pre and post treatment by highlightting
 the different listening sessions, those acting as "stills", and allowing us to see the psychic
 state of the patient and his evolution followed in music therapy.

 \textbf{Keywords: music therapy; listen; sound; hear; test}
\end{abstract}
