\begin{abstract}
Se visualiser, se voir, se découvrir à travers le reflet de son écoute.
Situer des sons dans l'espace pour se positionner soi-même. Se situer
dans l'espace sonore.

Ecouter et s'écouter.



La musique influe notre corps tout entier. Elle peut influer également notre écoute et même la modifier. Il existe donc un lien entre musique et écoute et nous pouvons le prouver au moyen d'un test et d'un appareil test spécifique que nous estimons adéquat pour se faire.
Cette étude a pour objet de faire l'hypothèse d'une possible visualisation
de la transformation de l'écoute du patient 
lors d'un traitement en musicothérapie. Nous utiliserons la comparaison graphique de ce test qui nous permettra de synthétiser les différences d'écoutes, sortes de "clichés photographiques" de l'écoute avant et après traitement. Nous aurons deux groupes, un groupe test et un groupe témoin.






Mots-clés : musique- musicothérapie-écoute-son-oreille-test
\end{abstract}
