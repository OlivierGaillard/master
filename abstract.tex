\begin{abstract}
Se visualiser, se voir, se découvrir à travers le reflet de son écoute.
Situer des sons dans l'espace pour se positionner soi-même. Se situer
dans l'espace sonore.

Ecouter et s'écouter.



La musique influe notre corps tout entier. Elle influe notre écoute au point même de la modifier. On le sait, il existe un lien entre musique et écoute. Mais au moyen d'un test avec un appareil spécifique, nous allons pouvoir le voir; dès lors, cela deviendra une "évidence" dans le sens premier du terme.
 Cette étude a pour objet de faire l'hypothèse d'une visualisation
de la transformation de l'écoute du patient 
lors d'un traitement en musicothérapie. Nous l'observerons par des graphiques qui, selon une référence d'écoute précise, nous permettront de synthétiser les différences d'avant et après traitement. 
Ils seront des révélateurs de l'écoute, tels des"clichés photographiques". 






Mots-clés : musique- musicothérapie-écoute-son-oreille-test
\end{abstract}
