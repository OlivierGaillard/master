% !TEX root = ./master.tex
\section*{Abstract}


\textbf{L'axe principal} porte sur l'observation par comparaison de la faculté d'écoute des patients lors 
de l'aboutissement d'une musicothérapie.
\textbf{Les outils:} le test d'écoute Tomatis et le questionnaire WHOQOL sur la qualité de vie.
 Des graphiques de courbes d'écoute synthétisent des différences pré/post traitement.
 13 patients souffrant de difficulté de régulation des émotions sont séparés en 2 
 groupes différents: l'expérimental de 6 patients suivant la musicothérapie, et l'autre de
 contrôle avec 7 patients. Le questionnaire WHOQOL est rempli avant et 
 après séjour afin de déterminer l'impact ou non de la musicothérapie sur leur écoute.
\textbf{Résultats:} on constate une modification positive et significative observée par le test d'écoute 
Tomatis pour le groupe ayant suivi la 
musicothérapie et confirmée par le questionnaire WHOQOL. Ce dernier a en effet des valeurs plus 
élevées 
post traitement. 
Pour le groupe contrôle, la transformation a été faible et le questionnaire s'est révélé majoritairement 
négatif.
\textbf{Conclusions:} La musicothérapie a eu un impact positif sur la transformation de l'écoute, corrélée 
à l'état psychique mis en évidence par le WHOQOL.
\textbf{Remarque:} le nombre final de questionnaires a été 
inférieur à celui attendu. Ajouté à la petite taille de notre étude, cela empêche nos résultats d'être 
statistiquement significatifs.
Ayant tout à fait conscience des compétences scientifiques qu'une telle étude aurait exigé, ce travail 
oscillera plus vers le qualitatif que le quantitatif.
\textbf{Mots-clés: musicothérapie; écoute; son; oreille; test}

\section*{Abstract}

\textbf {The main axis} is about the observation by comparison of the auditory perception faculty of 
patients during the outcome of music therapy.
\textbf{Tools:} Tomatis listening test and WHOQOL quality of life questionnaire.
Listening curve graphs summarize the difference before and after treatment.
13 patients with the same type of pathology (difficulty in regulating
emotions) are divided into 2 different groups: one experimental group of 6 patients following music 
therapy, and one control group of 7 patients. Patients fill out the WHOQOL questionnaire before and after 
treatmenas to determine the impact or the lack of impact of the therapy on their listening.
\textbf{Results:} analysis of observation shows a positive and significant modification for the music 
therapy group regarding the WHOQOL test. The post-therapy questionnaire indeed shows an increase in 
values compared to the pre-therapy one.
For the control group, the transformation was weak and the questionnaire was mostly negative.
\textbf{Conclusions:} Music therapy had a positive impact on the transformation of listening, correlated 
to the psychic state, as the WHOQOL questionnaires results clearly showed.
\textbf{Note:} the final number of listening tests and WHOQOL test was found lower than expected. 
Added to the fact this was a small study. This therefore prevents our results from being statistically 
significative. 
Being fully aware of the scientific skills that such a study would’ve required, this work will oscillate more 
towards the qualitative aspect of the study than the quantitative.

\textbf{Keywords: music therapy; listen; sound; hear; test}













%\begin{Remerciements}
