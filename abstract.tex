\begin{abstract}
Se visualiser, se voir, se découvrir à travers le reflet de son écoute.
Situer des sons dans l'espace pour se positionner soi-même. Se situer
dans l'espace sonore.

Ecouter et s'écouter.

Un test d'écoute pourrait avoir un sens pour le patient, qui visualiserait
son écoute. Le fait de tester des sons précis permettrait de déclencher
un travail, un début de cheminement intérieur.

Le test deviendrait-il aussi un outil intéressant pour le thérapeute
qui saisirait le patient de manière différente ? Se situer dans l'espace
sonore implique tout son corps, demande un effort, une prise de conscience,
une prise de distance avec soi-même et pourrait être considéré comme
un point de départ dans un traitement en musicothérapie. 

Cette étude a pour objet de faire l'hypothèse d'une possible visualisation
de la modification, de la transformation de l'écoute d'une personne
lorsque cette personne suit un traitement en musicothérapie. Nous
citerons brièvement les travaux les plus récents prouvant l'impact
de la musique sur le cerveau grâce à l'IRMfct mais nous nous étendrons
pas outre-mesure sur ce sujet. Nous nous limiterons intentionnellement
à ne faire qu' une sorte de ``photographie'' d'une écoute en début
de traitement en musicothérapie et à la comparer avec une seconde,
faite à la fin de celui-ci.

La musicothérapie fait partie de ces thérapies dites subtiles. Elle
est très difficilement quantifiable comme le fait par exemple, la
psychologie cognitivo-comportementaliste, avec des tests. Il n'y a
jamais, à proprement parlé, d'avant et d'après mais il y a transformation.
Et les transformations échappent toujours aux quantifications. Peut-être
ici pourrons-nous apporter un outil plus objectif par un test particulier
d'écoute : la démonstration d'un travail d'écoute, d'une perception
différente, d'une sensibilité nouvelle du patient. Apprendre à écouter,
c'est un travail et des résultats pourraient être visibles.

Mots-clés : la musicothérapie- Tomatis-l'écoute-l'oreille-le test-la
perception- la transformation-l'apprentissage de l'écoute
\end{abstract}
