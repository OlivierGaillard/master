% !TEX root = ./master.tex

\begin{abstract}
	\centerline{\textbf{Abstract}}
	\begin{french}	
L'axe principal porte sur la comparaison de la faculté d'écoute des patients avant et après 
un suivi individuel en musicothérapie pendant 3 semaines en cadre clinique.
Le test d'écoute Tomatis\textsuperscript \textregistered  et le questionnaire WHOQOL sur la qualité de vie 
sont les moyens 
utilisés 
pour synthétiser les différences.
 %Des graphiques de courbes d'écoute synthétisent des différences pré/post traitement.
 Treize  patients, avec difficulté de régulation des émotions, sont assignés à deux groupes de recherche: 
 six au groupe
 musicothérapie (GM) et sept au groupe contrôle (GC).
 % Le WHOQOL est rempli avant et 
 %après séjour afin d'évaluer leur état psychique.
L'analyse comparative des résultats révèle une modification positive et significative 
de qualité d'écoute après musicothérapie  %par le 
%test 
%d'écoute 
%Tomatis
 pour GM contrairement à GC, résultats correspondants au ressenti subjectif du questionnaire WHOQOL. 
  % où la transformation est
%faible et le questionnaire majoritairement négatif. % avec des valeurs plus 
%élevées. 
%Pour le groupe contrôle, la transformation a été faible et le questionnaire majoritairement 
%négatif.
En conclusion, la musicothérapie a %nfluencé positivement la façon d'écouter %
eu un impact positif sur la qualité de vie  du patient et sur sa qualité d'écoute.
%corrélée à l'état psychique mis en évidence par le WHOQOL.
La totalité des tests d'écoute et des questionnaires demeure 
inférieur à ce qui avait été initialement planifié, et, en dépit d'une plus ample dimension statistique, ce 
travail avantagera plus les aspects qualitatifs que quantitatifs. %d'être 
%statistiquement significatifs, rendant notre travail plus qualitatif que quantitatif.
%Ayant tout à fait conscience des compétences scientifiques qu'une telle étude aurait exigé, ce travail 
%oscillera 

\textbf{Mots-clés: musicothérapie; écoute; son; oreille; test}
\end{french}

\begin{english}	
The main axis is on comparing the listening skills of 
patients before and after individuel music therapy during 3 weeks.
The  Tomatis\textsuperscript \textregistered  listening test and the  WHOQOL quality of life questionnaire 
are used to synthesize their 
differences.
Thirteen patients with difficulty regulating
emotions,  are assigned to two research groups: six to the  music 
therapy group (GM), and seven to the control group (GC).
The comparative analysis of the results reveals a positive and significant change in pre/ post treatment 
listening capacity for the music 
therapy group, confirmed by the WHOQOL, contrary to the control group. 
In conclusion, music therapy had a positive impact on the transformation of listening, correlated 
to the mental  state highlighted by the WHOQOL.
The totality of listening tests and questionnaires remains lower than what was initially planned, and, 
despite a larger statistical dimension, this work will focus more qualitative than quantitative aspects.
% the final number of listening tests and WHOQOL test was found lower than expected. 
%Added to the fact this was a small study. This therefore prevents our results from being statistically 
%significative. 
%Being fully aware of the scientific skills that such a study would’ve required, 

\textbf{Keywords: music therapy; listen; sound; hear; test}
\end{english}
\end{abstract}













%\begin{Remerciements}
