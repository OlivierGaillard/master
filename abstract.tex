% !TEX root = ./master.tex

\begin{abstract}
\textbf{	Abstract}

\textbf{L'axe principal} porte sur l'observation par comparaison de la faculté de perception auditive des patients lors de l'aboutissement d'une musicothérapie.
\textbf{Les outils:} le test d'écoute Tomatis et le questionnaire WHOQOL sur la qualité de vie.
 Des graphiques de courbes d'écoute nous ont permis de synthétiser des différences pré/post traitement avec 15 patients répartis en 2 groupes de même type de pathologie (difficulté de régulation des émotions), l'un expérimental avec musicothérapie au nombre de 8 et l'autre, le groupe témoin au nombre de 7. Les patients ont répondu aux questionnaires (le WHOQOL), réalisés en parallèle, pour obtenir confirmation ou infirmation d'une modification de leur écoute en relation avec leur état psychique.
\textbf{Résultats:} l'analyse de l'observation portant sur la transformation de l'écoute a montré une positive et importante modification par comparaison pré/post traitement pour le groupe de musicothérapie en relation avec le questionnaire WHOQOL. Pour le groupe contrôle, la transformation était quasi inexistante et le questionnaire s'est révélé négatif.
\textbf{Conclusions:} La musicothérapie a eu un impact positif sur la transformation de l'écoute, corrélée à l'état psychique, constatant une différence notable entre les deux groupes.
Malgré les difficultés rencontrées lors de la récolte des valeurs, il a été possible de recueillir des considérations fort modestes allant dans le sens des questions de recherche et d'étayer des résultats. En ayant tout à fait conscience des qualités scientifiques qu'un tel travail exigerait, ce travail oscillera plus vers le qualitatif que le quantitatif.


\textbf{Mots-clés: musicothérapie; écoute; son; oreille; test}




Listen, listen to ourselves, communicate. Music therapy allows us to develop communication
by working on the listening. Music influences our whole body as well as our listening, to
the point of even modifying it. By means of a test with a specific device, we will
deepen this "evidence" in the first sense of the term: the visualization of
listening and its transformation.
Our hypothesis is that, during a music therapy treatment, we will be able to observe the
patient's psychic evolution by analyzing his/her listening.
Graphs will allow us to synthesize the differences pre and post treatment by highlightting
the different listening sessions, those acting as "stills", and allowing us to see the psychic
state of the patient and his evolution followed in music therapy.

\textbf{Keywords: music therapy; listen; sound; hear; test}
\end{abstract}













%\begin{Remerciements}
