\begin{abstract}
	

Ecouter, s'écouter, communiquer.

 La musicothérapie permet de développer la communication en travaillant sur l'écoute.

Par la musique, nous  influons notre corps tout entier ainsi que notre écoute, au point 
même de la 
modifier.
Au moyen d'un test avec 
un appareil spécifique, nous approfondirons cette
``évidence" dans le sens premier du terme : la visualisation de l'écoute et de sa 
transformation. 

Voici notre hypothèse: 
lors d'un traitement en musicothérapie, nous pourrons constater l'évolution psychique du 
patient en analysant son écoute.

 
 
 \begin{itemize}
 	\item 	Question: est-ce visible?
 
 \end{itemize}
 	
 \begin{itemize}
 	\item Question: y-a-t-il transformation?
 \end{itemize}
 \begin{itemize}
 	\item Question: dans l'affirmative, si cette transformation est visible, est-elle en 
 	concordance 
 	avec celle de l'état psychique du patient  et  en relation directe sur le plan de la 
 	communication?
 \end{itemize}
 

Des graphiques nous 
permettront de synthétiser les différences d'avant et après traitement.
Ils seront des révélateurs des différentes écoutes, tels des ``clichés photographiques",  en 
relation avec l'état psychique du patient et de son processus suivi en musicothérapie.







Mots-clés: musicothérapie-écoute-son-oreille-test
\end{abstract}
