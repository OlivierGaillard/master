\begin{abstract}
Se visualiser, se voir, se découvrir à travers le reflet de son écoute.
Situer des sons dans l'espace pour se positionner soi-même. Se situer
dans l'espace sonore.

Ecouter et s'écouter.

Un test d'écoute pourrait avoir un sens pour un patient pour plusieurs raisons. Dans notre monde d'uniformisation, il est intéressant  de pouvoir constater l'aspect ( critère) très personnel de notre écoute sous le couvert d'un test commun. Notre écoute est  individuelle et non standard ou conforme à tout un chacun, comme on est  trop souvent porté à le croire. 
D'autre part, il y a l'aspect de visibilité et de lisibilité : avoir  la possibilité de visualiser sa propre écoute. Ce processus de visualisation tente de matérialiser l'abstraction innée dues aux propriétés du son, l'aspect intemporel et éphémère du son entre l'écoute et la vue. Se situer dans l'espace sonore implique tout son corps, demande un effort (- tendre l'oreille-)implique des fonctions cognitivo-proprioceptives, et peut induire par conséquent  une prise de conscience et une prise de distance avec soi-même. Cette dernière pourra tout au  moins être initiatrice d' un début de travail sur soi.  
Le fait de se décider pour tel type de thérapie, d'accepter de faire un test, d'obtempérer aux consignes d'un consultant, de faire un choix de sons précis, de dialoguer sur les résultats avec le thérapeute, relève déjà d'une volonté de changement intérieur, en  éveille tout au moins cette idée de déclenchement d'un travail, d'un début de cheminement intérieur.

Et vice et versa, le test deviendrait-il aussi un outil intéressant pour le thérapeute ?
Cette façon de faire une anamnèse du patient en s'appuyant aussi sur un test, pourrait être complémentaire pour saisir le patient sur des aspects non conventionnels ou peu courants, peu usités, avec des éléments très appropriés dans ce contexte puisqu'ils appartiennent au domaine du son et de la musique. 
Nous pourrions faire l'hypothèse de l'inclure dans un  bilan en musicothérapie.


Cette étude a pour objet de faire l'hypothèse d'une possible visualisation
de la modification ou de la transformation de l'écoute du patient 
lors d'un traitement en musicothérapie. Nous utiliserons la comparaison graphique qui nous permettra de synthétiser les différences des "photographies " de l'écoute avant et après traitement et d'en tirer des conclusions. Nous nous limiterons intentionnellement aux `photographies'' avec deux groupes, dont un groupe témoin.
 Nous citerons brièvement les travaux les plus récents prouvant l'impact
de la musique sur le cerveau grâce à l'IRMfct mais nous nous étendrons
pas outre-mesure sur ce sujet. 
La musicothérapie fait partie de ces thérapies dites subtiles. Elle
est très difficilement quantifiable comme le fait par exemple, la
psychologie cognitivo-comportementaliste, avec des tests. Il n'y a
jamais, à proprement parlé, d'avant et d'après mais il y a transformation.
Et les transformations échappent toujours aux quantifications. Peut-être
ici pourrons-nous apporter un outil plus objectif par un test particulier
d'écoute : la démonstration d'un travail d'écoute, d'une perception
différente, d'une sensibilité nouvelle du patient. Apprendre à écouter,
c'est un travail et des résultats pourraient être visibles.

Mots-clés : la musicothérapie- Tomatis-l'écoute-l'oreille-le test-
\end{abstract}
