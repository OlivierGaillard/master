\chapter{Discussion}

Notre travail souligne l'importance de l'écoute en
musicothérapie. En effet, grâce à
l'oreille et à ses nombreuses connexions entre les différentes parties
cérébrales et grâce à l'intégration du système
nerveux avec ses aspects kinésiques et posturaux et dans
toute production vocale et langagière, l'ensemble de ces fonctions en
dépend. Le constat initial d'une carence d'outils d'évaluation
objective %des résultats issus de la musicothérapie
nous reconduit aux questions posées au début de ce travail, soit:

\begin{itemize}
     \item Si l'écoute est quantifiable  par un test, peut-on constater sa
transformation?
\item Si cette transformation existe, serait-elle reliable avec
une prise en charge musicothérapeutique?
\item Cette transformation de l'écoute aurait-elle un impact sur l'état
psychologique du patient? %source principale de notre intérêt.
\end{itemize}
L'application du test de Tomatis nous révèle les possibilités
de transformation individuelle et catégorielle de l'écoute, malgré la
complexité des aspects audiologiques et psychologiques.

%bibliographie + réf. chapitre partie théorique!!!!
S. Aubert -Khalfa et son équipe multidisciplinaire (2010) avaient déjà
exploité le test Tomatis -- sensible à la différence des
seuils auditifs -- entre une population
dépressive avec stress post-traumatique et une population normale.
Cette étude figure comme un cheminement précurseur de notre
travail, compte tenu de certaines différences dans la
 procédure:  %dans le recueil de points communs des seuils
 % auditifs,
 en effet, nos résultats convergent, tout en disposant d'une population hospitalisée,
dont la moitié a
bénéficié de
musicothérapie.


En plus, suite aux comparaisons
pré/post-thérapies, nous avons observé des transformations
non-significatives dans le groupe contrôle, mais
traduisant 
d'importantes différences en faveur des membres du groupe de musicothérapie.



%Nous avons utilisé un type de test d'écoute dont nous soulignons la
%complexité du test d'écoute utilisé.%, puisant
%à la fois ses racines dans l'audiologie et à la fois dans la psychologie. 
%Par celui-ci, nous voyons  
%qu'il  existe une possible transformation
%individuelle lors d'une
%thérapie tout 
% en constatant une forme de singularité pour chacun et une
%certaine similarité selon certaines pathologies. --Comme nous l'avons
%constaté avec l'étude J.P.Granier... il existe des points communs lors des recueils des seuils
%auditifs--.

%Nos résultats
%principaux sont les suivants:

%le groupe de musicothérapie a démontré une réactivité très positive
%dans son écoute par comparaison pré/post-thérapies tandis que 
%le groupe de contrôle n'a démontré qu'une très faible
%modification dans son écoute. 
%Il existe ainsi une différence importante entre les résultats des
%deux groupes.

Il convient à présent de discuter des aspects énoncés (test et
mesures),  puis des
limites et perspectives.
%et les résultats obtenus.
 
\section{Un test pour mesurer les trans\-for\-ma\-tions de l'écoute}

Nous utilisons un outil qui est le son. Nous accompagnons
le patient d'un point A pour aller au point B et 
nous avons pu observer des résultats
d'une transformation de la
perception basée sur un test de reconnaissance de sons permettant sa
visualisation lors de séances musicothérapeutiques.
%Par comparaison, nous pouvons faire ce constat, positif ou
%négatif, ce qui est
%difficile dans la majorité des cas, nous fiant plus à notre intuition
%qu'à des données.
%Il y a une modification de l'écoute du patient lors de séances en
%musicothérapie et celle-ci est visible et lisible au moyen du test
%d'écoute employé.
\textit{``Le tracé sonore''} du patient ouvre ainsi à une compréhension
différente en vue d'une modulation musicothérapeutique mieux adaptée.
%afin de l'accompagner dans son cheminement thérapeutique.

\subsection{L'anamnèse et le bilan en musicothérapie}

  Avant d'amorcer une thérapie, le test préalable fournit des
  renseignements
  complémentaires à l'anamnèse, et l'inclusion dans un bilan
  est  une forme d'intervention
   musicothérapeutique intégrant plusieurs fonctions simultanées
   centrées sur le son dont voici l'énumération:

   \textbf{Un appareil-test} : la  prise en charge 
   technique et cadrée est perçue la plupart des fois comme rassurante,
   même dans les cas les plus perturbés.


  %ent soit abordé par une\textbf{
 % procédure de test} à l'aide d'un appareil peut être fréquemment
  % perçu comme rassurant, dans leur confusion la plus totale selon
  % certains troubles, puisque 
   
 %jouant souvent 
  %un rôle de fiabilité, de soutien.
   %et la création d'une alliance thérapeutique.
   %la présence d'un thérapeute et son alliance.
%   consiste en plusieurs rôles intrinsèques dont le travail sur le son et
% l'alliance thérapeutique.
 
%Devant le flou persistant souvent sur le concept de la
%musicothérapie, il représente une forme
%d'introduction simple et claire.

   
\textbf{L'alliance thérapeutique} est indispensable à toute
thérapie, et dans notre cas, elle peut se construire à travers le
   dialogue alimenté aussi par l'explication des modalités du test
   sous forme d'entrée en matière.
  Suivent les considérations sur les résultats.

  
\textbf{La communication}:
  Le test d'écoute 
    provoque un effet de surprise entraînant le dialogue, %, à son issue et à l'insu du patient. 
   et les réponses du patient sont essentiellement
  gestuelles, donc appartenant à une forme de communication
  complémentaire; 
  en fin de passation, 
 le patient pose fréquemment des questions en livrant des détails plus
 insolites, parfois même totalement inattendus.
Les informations anamnestiques complétées par les aspects sonores
 comportent alors  une \textit{double dimension}, `` muette'' et verbale.
 

\textbf{Le travail sur le son} induit  immédiatement le patient
 dans un \emph{rôle actif} par la reconnaissance de sons dans
 l'espace en fonction de la fréquence et du volume, en fixant son attention sur les
 sons émis et perçus par les écouteurs, accomagnés par des réponses gestuelles.
 Ce rôle l'interpelle directement, le conduit à
    s'interroger sur sa façon de 
    se situer dans le sonore.
    %Parfois même, selon les cas, cela l'incitera à
    %créer et trouver par lui-même et de manière
 %inconsciente  
 %son propre espace; ce  qui permettra ainsi d'entamer, pour certains dumoins, leur propre
 %processus thérapeutique.
\footnote{ Nous pouvons rejoindre les propos de Mme S. Lutz
 Hochreutener confirmant la participation du patient à la réussite de sa thérapie.}
 
  
 \textbf{La complémentarité du test } va permettre au thérapeute
 de créer d'autres nouveaux chemins, comme l'utilisation de la voix,
 l'application de rythmes ou l'improvisation, (Cf. graphique des liens),
 le tout visant 
 à une optimalisation de l'écoute. 
 L'évaluation de la transformation de l'écoute peut être effectuée à
 tout moment y compris au 
 cours du processus thérapeutique, en guise de jauge lors d'un voyage pour estimer le chemin
parcouru et s'interroger sur la voie à prendre.
 
%  \item Remarque : le concept de Piaget de l'\textit{assimilation} majorante (parallélisme avec une amélioration cognitive dite majorante)
%une amélioration perceptuelle impliquant tout le procédé va toucher tout le
%système, développement de l'intelligence, pas que dans la logique, mais
%par \textbf{l'art, 
%processus d'accomodation}.



Ce qui  ressort distinctement sont \textbf{ la visibilité et la visualisation.}
\begin{itemize}
 
\item La visibilité du \textit{tracé sonore} permet une prise de conscience
  extérieure traduisant une forme  d'écoute individuelle.% \textit{``comportementalisme de l'audition''} .


\item Par  visualisation, on entend le fait de voir son chemin d'écoute
matérialisé  graphiquement et traduisant la perception
neuro-physiologique du son (l'abstraction innée dues aux propriétés du son), l'aspect éphémère du son entre l'écoute et la vue.
Il permet ainsi de se situer dans l'espace sonore et, en un même
temps, induit une \textit{prise de conscience }et de distance avec
soi-même.  Détecter un son précis, l'entendre, reconnaitre sa présence et même à très
 faible intensité, le situer dans l'espace, à droite, à gauche, au
 milieu ou ailleurs, n'est pas un acte anodin. En fait,  il apporte
 une double information: \textbf{une information
   physiologique et psychologique} sur le patient.
 C'est pour cette raison que ce test a été nommé par Tomatis
 audio-psycho-phonologique (oreille-psyché-voix).\footnote{Avec le professeur Tomatis: formation suivie dès 1995, Boulevard de Courcelles, Centre de l'écoute 
Tomatis à Paris; puis en 2009/11/13/15 avec V. Gas, V. Drouot et J.P. Granier, formateurs et consultants. Source: site internet officiel: \cite{tomatis.com}.}
Tout le corps est impliqué (--- tendre l'oreille---) avec toutes les fonctions cognitivo-proprioceptives.
Par conséquent, et de plus, le conflit ou source de souffrance --raison pour laquelle le
patient se retrouve en thérapie--  n’est pas représenté
comme un nœud dans l’inconscient, ainsi que nous le fait remarquer Lagache,\footnote{Dès 1938, celui-ci parle d’une « psychologie nouvelle» devenue fonctionnelle, non plus seulement structurale, décrivant l’homme en
   situation.}
 mais comme la
dynamique et la condition nécessaire de la \textit{prise de
  conscience}.
\end{itemize}

	
	Le rôle du patient est différent dans son
        essence même, non pas dans le sens de\textit{ "patere"
        }souffrir et subir, mais valorisé dans celui du rôle actif
        qu'il peut jouer : celui-ci peut se rendre compte de sa
        capacité à influencer sa façon d'écouter, qu'il a une présence
        signifiante dans sa thérapie.  Il n'est pas passif. Bien sûr,
        cela peut paraître une évidence car on sait l'impact de la
        musique sur le corps tout entier. Mais ici on rejoint  le
        concept de musique intégrative déjà cité \autocite[Cf.]
        {vrait_musicotherapie_2018}. Le patient peut être nourri par
        la musique mais n'est pas ``passif'' et seulement ``l'objet'' qui bénéficie du traitement musical. Son  écoute lui appartient en propre, elle est personnelle et modifiable. S'il y a modification, il peut y avoir un changement; et le mot "changement" prend alors une connotation différente,  le mouvement est y  sous-entendu,  une démarche peut en découler, voire une évolution. 



        % \textbf{Réflexion sur le test et son aspect psychologique:   }


 %Le patient obtempère aux consignes du thérapeute, fait un choix parmi des sons précis, dialogue sur les résultats, tout ceci relève déjà d'une volonté de changement intérieur, en  éveille tout au moins cette idée de déclenchement d'un travail, initie un début de cheminement.
%Être accompagné dans la lecture de son évolution par le biais de
%son écoute  permet de constater un résultat qui  peut le conforter ou non dans son sentiment intérieur.
        
        \paragraph{Un test comme élément déclencheur}

        

L'écoute n'est donc pas une notion si abstraite. Elle est comme de la terre façonnable.
 Grâce à l'impact de la musique et de la musicothérapie se cristallisent
 une prise de conscience et une modification
 de  l'état psychique en symbiose avec la transformation de
 l'écoute. Par conséquent, elle peut amener à la \textbf{restructuration }de la personne et à
 sa \textbf{redynamisation}.
 Elle ouvre des portes comme peut le faire, et nous
 en avons été témoin lors de ce travail avec une patiente, une simple passation de test d'écoute. 
  Prendre conscience que notre propre écoute nous appartient
  personnellement, que l'on peut agir et y jouer un rôle jusqu'à sa
  transformation, en soi, c'est devenir \textbf{ acteur de sa
    métamorphose} toute entière.
  Cet unique élément a été dans cette situation clinique le seul
  catalysateur de la transformation.





  L'écoute est-elle universelle ou diférente selon l'individu?
 Chaque être humain semble constitué comme son prochain;
 nos oreilles ont  une anatomie semblable. Et pourtant, si l'on
 observe de près nos empreintes digitales, on s'aperçoit de l'unicité
 de chaque être. De même, chaque oreille est particulière, singulière
 quoique physiquement identique. Elle s'ouvre différemment à l'écoute par l'imprégnation
 aux expériences de la vie, par le temps qui infuse des changements dans
 notre être intérieur et extérieur. Nous sommes vivants dans nos
 mouvements psychiques et physiques. Des traces en sont 
 visibles dans l' écoute constamment  façonnée par les
 ressentis, par l'environnement, et, plus brièvement décrit, par cette perméabilité au
 changement, la rendant unique et évolutive.
Visualiser ces
changements via des tests et catégoriser certains affects de
l'écoute du patient au fil de la thérapie est un des moyens qui nous permet d'évaluer avec
plus d'objectivité
son évolution mais aussi
de constater la présence ou l'absence de corrélation entre le
traitement et le psychisme.


Le but utime pour le patient est de retrouver un état d'équilibre
dynamique, une forme d'homéostasie et cette harmonie peut être 
 représentée par un test d'écoute.







  \section{Limites et perspectives: }

\subsection{Limites}


  Il faut considérer la petite taille de l'échantillon ne permettant
  pas d'analyse plus poussée ni de généralisation des résultats; les
  questionnaires sont sous forme d'auto-évaluation, et donc forcément
  c'est une mesure subjective;  une hétéro-évaluation par un proche
  aurait pu être pertinente.

  
S'il n'y a pas de changement visible dans le test, quelles conclusions
peut-on en tirer ? le changement va-t-il toujours de pair avec le
patient? synchronisé ou différencié dans le temps?
Avec le test, le patient obtempère aux consignes du thérapeute, fait
un choix parmi des sons précis, dialogue sur les résultats, tout ceci
relève déjà d'une volonté de changement intérieur, en  éveille tout au
moins cette idée, un déclenchement de travail initié tel un début de cheminement.
Être accompagné dans la lecture de son évolution par le biais de
son écoute
est une façon différente d'entrevoir sa problématique. Elle permet de
constater un résultat qui  peut le conforter ou non dans son sentiment
intérieur.

Il en est de même pour la simultanéité des
résultats et le décalage intérieur vécu. Il y a parfois une
synchronicité dans l'action et les résultats. Mais elle n'est pas systématique,
car on le sait en thérapie, le temps a son rôle, il est
nécessaire et les étapes ne peuvent être brûlées. Le patient
suit un long
cheminement. %avec différentes étapes --acceptation, refus, que
%ce soit au sujet de  son
%identité, de sa transformation--.
De même, quitter le ``confort'' de ses habitudes, même vu sous
l'angle de l'écoute, peut être très perturbant, dérangeant.

Ainsi, comme pour toutes thérapies, il y a un temps d'intégration. Si
le patient est toujours en
attente des résultats,comme un score à atteindre, impatient d'un résultat visible,
avec la volonté ferme d'acquérir la courbe dite ``idéale'', n'écoutant pas son
sentiment intérieur car trop sensible aux notions de performance,
le test, de manière générale, peut être complètement
contradictoire et totalement inapproprié.
Ou, au contraire, selon les cas, ce peut être une stimulation, un appui.
Remarquons que ces paradoxes sont
intrinsèques à tous les tests.
% Changer de rail, être
%sur un quai de gare
%en attente d'un autre train, peut être vécu sous l'emprise de la
%panique ou la joie de l'inconnu.

%suggérées par une modification d'écoute.

%Quitter le confort même si elles sont jugées négatives par lui-même et par les autres --- . Tout ceci prend du temps et ne peut pas 



%Par cette démarche, nos sommes conscients qu'il y a un risque de
%catégoriser le patient par les  transmis
%aux assurances-maladies.
\subsection{Perspectives}
Dans les perspectives futures, fort de ce travail, nous serons plus à
même d'élaborer une étude à plus grande échelle avec précision et données.\footnote{Travaillant
actuellement à 40 pour
cent, les conditions sont différentes que celles vécues pour le
travail présenté à ce jour (10 pour cent).}
Nous  avons répertorié 
de nombreux détails nécessaires et complémentaires: 
\begin{itemize}

 \item la passation d'une échelle de
     dépression afin d'objectiver le degré de sévérité des
     dépressions
\item  la nécessité de conduire l'étude sur du long terme, sur
  plusieurs mois avec de nombreuses mesures pour tirer des conclusions
 \item   la nécessité
     d'avoir un échantillonnage beaucoup plus nombreux de 
     participants
   \item  la différenciation des différentes pathologies
     \item l'utilisation d'autres appareils de tests pour faire des
       comparaisons
        \item travailler davantage sur les zones d'intersection entre le
          psychologique et le physique: alliage des 3 zones relevées
          dans le test dans un but musicothérapeutique.(Cf.Fig.6.19.)
% Nous avons conscience que nous présentons une ébauche d'étude, avec des pistes
%  suggérées. 
     

 
 

%Différents paramètres inhérents à ce type d'étude sont à
%considérer. Que ce soit le manque de temps, les départs imprévus des patients, et/ou leur
%absence momentanée (visite du psychologue, maladie, etc.). rajoutés à la
%contingence difficile due à la distance séparant le lieu de domicile à
%celui du lieu d'étude, toutes ces contingences ont été les principaux facteurs réducteurs de
%tests valables.
%: comment planifier un départ imprévu d'un patient !? il a fallu parfois
%faire beaucoup de route pour effectuer les tests finaux d'un ou deux
%patients.
  
%Par conséquent,  de nombreux tests sont restés 
%incomplets et n'ont pu
%être validés car ils ne remplissaient pas toutes les conditions requises. 
 
  \item la fréquence de prise en charge en musicothérapie:
  une fois par semaine pendant une heure semble a posteriori
  trop court pour observer un changement réellement important. Puisque nous avons pu
  constater un certain bénéfice pour le groupe de musicothérapie, il
  est probable qu'un travail journalier ou 
 une immersion plus intensive en
  musicothérapie aurait amené des résultats plus marquants.

  % En comparaison avec des
  %modifications importantes de courbes des tests observées généralement  lors d' une écoute
  %régulière de deux heures par jour de musique pendant 15 jours --- en
  %référence à l'entrainement des muscles de l'oreille chez Tomatis,
  %qui, nous le rappelons, est une pédagogie de l'écoute --- il aurait
  %été intéressant de pouvoir faire cette étude comparative dans cette
 % clinique. Ainsi, nous aurions pu éventuellement mettre en avant
 % l'absolue nécessité de
  %Créer et d'instaurer systématiquement la
  %%musicothérapie dans de nombreuses institutions, 
  %la développer beaucoup plus intensément, plusieurs fois par semaine.
    % Le patient reste au centre de nos préoccupations.
%Serait-ce une façon de démontrer l'utilité de la musicothérapie
%pour une plus large acceptation de 
%cette thérapie dans plus de milieux hospitaliers ou autres ?
%, avec les
%nouveautés en neuroscience et en neuropsychologie sur les musies, les
%fonctions musicales du cerveau en les reliant avec les Muses.

\end{itemize}


 Perspectives musicales futures: le 1° Symposium
  NeuroTechSymphony, une première en Europe, a eu lieu au CHUV le 18 et 19 septembre 2019. Il  nous a
  donné un aperçu de l'ampleur de la grande avancée technologique
  et de l'émergeance entre  l'interface de la musique, la technologie, la
  création de jeux interactifs spécifiques avec leur fort impact sur la
  réhabilitation. Nous avons aussi suivi avec intérêt l'étude en cours avec utilisation de
  biomarkers en neuromusicology par le Prof. Artur Jaschke sur des
  bébés prématurés prouvant l'effet de la musique ``en live'' sur leur
  oxygénation, et donc sur leur diminution de stress.

  
 Nous aimerions rajouter encore quelques précisions sur un possible alliage de
\textbf{la musicothérapie et de la méthode Tomatis.} Nous sommes tout à fait conscients des
 grandes divergences d'opinions entre les adeptes d'une musicothérapie
 traditionnelle et cette méthode.
Ce sont des concepts très différents. Bien que la notion d'écoute les réunit, bien que leur medium soit la musique et plus particulièrement le son, d'un côté il s'agit d'une thérapie et de l'autre, il s'agit d'une pédagogie, d'un entrainement de la musculature de l'oreille. 
Tomatis se focalise et opère essentiellement sur le capteur auditif
(vestibulo-cochléaire) pour amener, par ce processus, le patient à une
certaine  amélioration par rapport à sa vie actuelle et à des attentes
précises. % cette amélioration peut se réaliser aussi au niveau du
%langage et ce, par l'intermédiaire de la musique et du chant.
Avec le travail dit\textbf{ passif}, le  but est l'\emph{ouverture} de l'oreille
aux sons et sa sensibilisation car l'objectif est de réintégrer
des fréquences perdues ou annihilées inconsciemment. 
%Cette technique de travail amène un résultat
%physiologique, dérange les habitudes d'écoute pour faire agir
%et ré-agir le patient, en étant parfois pertubatrice jusqu'à provoquer
%un phénomène de rejet.
 Avec le travail \textbf{actif}, on corrige la voix grâce à des écouteurs spécifiques 
car la correction de la voix y est instantanée et instaure les bons
réflexes de la boucle audio-vocale. Puisque ce processus incite à
assimiler et à analyser l'information sonore pour l'ajuster et
l'émettre en retour,%Le patient
%analyse cette boucle permanente entre l'écoute et l'émission vocale
%afin de créer des réflexes sur lesquels il peut ``s'asseoir''
\footnote{Jean-Pierre Granier, Tomatis 
Développement,\emph{Conférence Paris lors de la Convention du 13 mai 2012}, 13.5.2012.}
nous pourrions émettre la supposition suivante: lorsque cette boucle
phono-auditive est bien élaborée et installée, interviendrait très judicieusement le
rôle éminemment important de la musicothérapie. L'oreille est
prête, entraînée et apte à se transformer encore plus en profondeur, notamment sur le
plan psychique. % avec tous les
%riches moyens proposés par la musicothérapie.
%C'est une préparation de terrain, un travail physique de
`%fond, un entraînement de l'oreille pour que celle-ci soit totalement
% opérationnelle, prête à approfondir sur de nombreux plans
 La première phase peut être considérée comme une étape préparatoire afin que la
 \textbf{musicothérapie} puisse jouer ensuite pleinement son rôle dont
 %ne serait-ce que d'accepter d'entendre sa
%propre voix.
%Le soutien du thérapeute est nécessaire pour permettre
%au patient de franchir cette étape.
la poursuite de la réintégration progressive de la voix dans
le corps, la découverte ou redécouverte de ses ressources,
la régénération énergétique, la capacité d'être auteur de sa propre
restructuration et chef d'orchestre de sa
vie.

\enquote{\emph{L'émission vocale confirme et reconfirme à chaque
fois le sujet dans son intégrité et son identité.}}%
\autocite{tomatis:loreille}\footnote{Tomatis en fait une description précise dans la troisième partie de
son livre, pp. 185--301.}
%Relevons le paradoxe de l'empreinte vocale obtenue sur l'écran d'un
%même sujet avec le même mot en s'efforçant de le faire de manière
%identique; ce qui prouve que chaque phonation est unique. Est-ce alors
%possible de l'analyser?cf.p.104 "la thérapie par les sons"



  



\section{Conclusions}

\label{Conclusions}

\paragraph{L'objectif de ce travail} vise à l'ouverture vers d'autres
domaines d'exploration. 
  Il s'agit d'un mixe: quantitatif et qualitatif. Nous avons ainsi pris l'option de nous tenir à une
  observation, celle de la transformation de l'écoute, raison pour
  laquelle nous avons utilisé l'appareil test, nous limitant intentionnellement
     à ce concept.
  Est-ce indispensable à tout musicothérapeute d'avoir un
        appareil test d'écoute ? certainement pas mais il s'agit d'une technique
        utile et intéressante. 
 La matière sonore est et reste la matière première de la  musicothérapie. 
 Par son biais, elle  apporte de multiples éléments d'évaluation du
 sujet. 
 Certains  intégrent plus que d'autres dans leur pratique des techniques relevant du domaine de la psychothérapie--de l' analytique, 
  du comportementalisme, du cognitivisme, de la  systémique ainsi que
  celles dites humanistes.  Le concept de "médium malléable" a été
  développé par  R.Rousillon\footnote{R.Rousillon,\textit{Paradoxes et situations limites,  
  		de la psychanalyse} Paris, Puf. 1991} 
  et qu'il est possible de transposer dans la matière \textit{musique} 
  pour "favoriser et accompagner le processus 
  de symbolisation"\footnote{F.X.Vrait, \textit{La musicothérapie},Ch.3, p. 112}.

 Le musicothérapeute est un être extrêment sensible avec de multiples
 ``antennes'' : l'intuition reste primordiale tout autant que  
 l'écoute. L'oreille se dresse pour une écoute empathique, pour ``\textit{rester en contact émotionnel  
 avec le patient}'' Eckert (2007) par le son qui va au plus profond de
 l'être.

 
\begin{quotation}
	`\textit{`L'oreille est l'organe le plus sensible des sens 
et l'instrument de diagnostic  le plus important du
musicothérapeute}.''\autocite{seminar_zuerich}
 	
\end{quotation}

Vivant dans un monde très visuel, les preuves sous cette forme sont
validées pour soutenir l'argumentation du bien-fondé d'une thérapie.\footnote{
	\pdfcomment{Faire une phrase}les critères de l'EBM (evidence based medecine, médecine basée sur des 
        preuves) F. X. Vrait, ch. II, pp. 105--106 }
On veut voir pour croire. Est-ce 
notre esprit formaté cartésien depuis quelques centaines d'années qui nous 
empêche de penser différemment? 
%Actuellement, c'est une nécessité due à notre époque pour
Comment crédibiliser l'impact 
du son sur notre être?
La musicothérapie fait partie de ces thérapies dites subtiles. Elle
est très difficilement quantifiable. La
psychologie cognitivo-comportementaliste peut le faire avec des tests. Mais avec la musicothérapie ou d'autres formes de thérapie, il n'y a
jamais, à proprement parlé, d'avant et d'après mais il y a transformation.
Et les transformations échappent toujours aux quantifications. Peut-être
ici pourrons-nous apporter un outil plus objectif: la démonstration d'un travail d'écoute, d'une perception
différente, d'une sensibilité nouvelle du patient.

 

% % Tomatis,utilisation de l'intuition de Tomatis comme Rogers(
% cf. ch. l'approche de C.Rogers dans Nicolas Duruz, Traité  des
%  psychothérapies comparées.

%Apprendre à écouter,
%c'est un travail et des résultats pourraient être visibles.
Comme l'exprime à juste titre André Malraux : \enquote{\emph{Le monde de
	l'art n'est pas celui de l'immortalité, c'est celui de la métamorphose.}}
De même, la musique est un art produit par l'homme et qui a un impact
sur lui-même. Les deux interagissent, s'interpénètrent et s'auto-transforment
au cours des siècles.
 Ce que nous pouvons constater lors de l'aboutissement
d'une thérapie n'est pas de trouver une autre personne mais une transformation
de la perception de celle-ci par rapport au monde qui l'entoure. 
Selon
ce que nous vivons, nous nous transformons et continuons à être
soi. Nous ``sommes soi" mais autrement. Nous ne perdons
pas notre identité.


\label{jeSuisLaMusique:viret}
\begin{quotation}
\emph{<<\,\emph{Par le Son, le Silence du Non-Être vient à l'Être}. [\dots] 
\textsl{Je suis}
	\emph{la musique que je fais ou écoute}. [\dots]\,>>
[\ldots] \emph{la musique a la capacité d'harmoniser
les composantes d'une entité psychophysique pour qu'il soit ``bien
dans sa peau'' et ``bien dans son âme.}''}\, \autocite[ch. 1, p.8]{viret:b}
\end{quotation}


%Peut-être pouvons-nous nous imaginer que la technologie future apportera d'autres outils
%directement accessibles pendant les séances, afin
% de, si nécessaire, visualiser directement l'effet en temps réel de la musique sur le
% cerveau:  imager couramment et facilement la manière d'ouïr de chaque
 %patient peut se révéler important, complémentaire et intéressant pour
 %l'anamnèse et la prise en charge. Ce serait l'image de la
 %physiologie de l'audition personnalisée à chaque séance!

 




Être au diapason, en harmonie avec soi et les autres
nécessite une écoute afin de nous accorder ou réaccorder à l'univers.
Car en définitive, comme le dit si poétiquement David Elbaz, nous sommes tous les
descendants de la cristallisation de la musique primordiale de
l'univers. \autocite{delbaz_recherche_2016} \footnote{David Elbaz, astrophysicien, chef de laboratoire au CEA et Alain
Destexhe, chercheur en neurosciences intégratives et computationnelles
à l'Institut  NeuroPsi de Paris Saclay} 

%simplement pour se
%sentir bien sa peau, bien dans son âme.
%Voilà l'hypothèse énoncée et ce que nous nous pouvons conclure, en effet.

%Perspectives et rétrospectives
%politiciens musicothérapeutes
%jamais oublier la cosmogonie grecque, le mythe, Orphée,
%progrès qui  ont été fait 