
\chapter{Conclusions, réflexions}

 
\section{Un test pour mesurer les trans\-for\-ma\-tions de l'écoute}

L'écoute est-elle universelle ou diférente selon l'individu?
 Chaque être humain est constitué anatomiquement comme son prochain;
 nos oreilles ont  une anatomie similaire. Néanmoins, comme le sont 
les empreintes digitales, l'écoute et l'approche des sons est unique
et personnelle à chacun. En partant de ce concept, chaque oreille
fonctionne différemment et va donc entendre différemment. S'en suit le concept de perméabilité au
changement. L'environnement, le temps, les ressentis apportent leur
part au façonnement constant de l'oreille et ainsi de l'écoute. Ces
changements sont quantifiables, mesurables et se retrouvent liés aux
expériences de la vie. Il nous est possible de visualiser ces
changements via des tests et de catégoriser certains affects de
l'écoute du patient au fil de la thérapie par le biais de
ceux-ci. Cette quantification nous permet d'évaluer objectivement
l'évolution de l'écoute du patient au cours de la thérapie  mais aussi
de constater la présence ou l'absence de corrélation entre le
traitement et l'évolution psychique du patient.

Y-a-t-il une modification de l'écoute du patient lors de séances en musicothérapie?
Est-ce visible et lisible au moyen du test d'écoute employé?
Peut-on définir et retracer un chemin sonore psychologique du patient au moyen d'un test d'écoute?


Selon les hypothèses évoquées, nous avons été amenés aux réflexions suivantes.

Nous utilisons un outil qui est le son. Nous accompagnons
le patient d'un point A pour aller au point B : que s'est -il passé
dans son écoute?
En comparant les données, nous pouvons observer des résultats visibles et tangibles 
d'une transformation de la
perception basée sur un test de reconnaissance de sons;  qui permet de visualiser une transformation psychologique
de l'écoute. 


Il y a des résultats : nous pouvons faire un constat.
Ce sont des données pouvant servir à mieux comprendre le patient
et à l'accompagner dans son cheminement thérapeutique.


\subsection{L'anamnèse et le bilan en musicothérapie}

  Avant d'amorcer une thérapie, le test effectué fournit des renseignements, toujours basés sur le son, qui seront complémentaires à l'anamnèse. Nous pouvons donc l'inclure dans un  bilan en musicothérapie.

  \
\subsubsection{La communication}

Le test d'écoute instaure un dialogue.
  
\begin{itemize}
  \item Durant le test d'écoute, les réponses du patient sont
    non-verbales, elles sont 
 uniquement gestuelles. Lorsque c'est terminé,
 le patient pose des questions au sujet des résultats obtenus. Selon
 l'échange entre le thérapeute, il est amené de façon
 indirecte à livrer et à évoquer des détails auxquels il
 n'aurait pas forcément pensé et prêté attention. L'anamnèse prend donc ici une autre
dimension puisqu'elle est complétée par le biais du son et l'écoute.

	 La relation avec le patient est indispensable à créer pour toutes thérapies. Le test peut créer dès le premier entretien une alliance thérapeutique.
	Il représente une procédure claire et  peut être
        considéré comme un outil utile d'une part pour instaurer un dialogue avec
        le patient et d'autre part servir d'indicateur sur le plan du travail à
        envisager.
  \end{itemize}      
\paragraph{Le test d'écoute est une forme d'intervention
  musicothérapeutique}

Le test d'écoute est en soi une forme d'intervention
 musicothérapeutique puisqu'il peut
 comporter plusieurs rôles intrinsèques dont le travail sur le son et
 l'alliance thérapeutique.
 


 \begin{itemize}

                           
\item Prise en charge cadrée: il peut être perçu comme  rasssurant  par
  nombre de patients d'être pris en charge non seulement par le
  thérapeute mais aussi par l'appui d'un appareil-test, jouant ainsi un rôle de soutien.
 
   
  \item Travail sur le son:  avec l'implication directe du
     patient dans le rôle de la 
     reconnaissance de sons: le travail sur le son implique immédiatement le patient
 dans un rôle actif: il fait faire un travail de reconnaissance de sons dans
 l'espace (repérage de fréquences selon le volume). Le patient
 participe activement en fixant son attention sur les
 sons émis et transmis par l'appareil aux écouteurs en les repérant
  puis en y
  répondant par la gestuelle.
  
  \item Rôle du patient: cette implication directe interpelle le patient, le conduit à
    s'interroger sur sa façon de 
    se situer selon les sons, et  peut-être aussi selon les cas, à
    créer et trouver par lui-même et de manière
 inconsciente  
 son propre espace; ce  qui permettra par là même déjà d'entamer, pour certains dumoins, leur propre
 processus thérapeutique.
 Par conséquent, nous appuyons et soulignons  les propos de Sandra Lutz
 Hochreutener qui confirme la participation du patient à la réussite de sa thérapie à la hauteur de 
 40 pour cent. Le thérapeute ne participe qu'à 30, la
 méthode et l'effet placebo sont à 15 pour cent chacun.
 
 \item Une alliance thérapeutique: une alliance thérapeutique  peut
   s'instaurer et  découler lors du
   dialogue amené par l'explication du test.
 
\item Pour le thérapeute: en cours et en fin de la thérapie, il peut
  être utile d'évaluer la transformation de l'écoute, tout
comme se servir d'une jauge  lors d'un voyage pour estimer le chemin
parcouru, s'interroger.

  
\end{itemize}

\paragraph{La visibilité}

Deux aspects ressortent distinctement: la visibilité et la visualisation.
\begin{itemize}
 
\item La visibilité: être visible par le son est une façon de se concrétiser extérieurement en en prenant conscience: une forme de ``comportementalisme de l'audition'' .


\item La visualisation: se visualiser par son chemin d'écoute.  
Ce processus de visualisation tente de matérialiser l'abstraction innée dues aux propriét:és du son, l'aspect intemporel et éphémère du son entre l'écoute et la vue.
Il permet de se situer dans l'espace sonore et, en un même temps, induit une prise de conscience et de distance avec soi-même. Tout le corps, physique et psychique est impliqué (--- tendre l'oreille---) avec toutes les fonctions cognitivo-proprioceptives.

\end{itemize}

Le patient obtempère aux consignes du thérapeute, fait un choix parmi des sons précis, dialogue sur les résultats, tout ceci relève déjà d'une volonté de changement intérieur, en  éveille tout au moins cette idée de déclenchement d'un travail, initie un début de cheminement intérieur.
Être accompagné dans la lecture de son évolution sonore permet de constater un résultat qui  peut le conforter ou non dans son sentiment intérieur.
	
	D'autre part, le rôle du patient est différent dans son essence même, non pas dans le sens de "patere" souffrir et subir, mais valorisé dans celui du rôle actif qu'il peut jouer : celui-ci peut se rendre compte de sa capacité à influencer sa façon d'écouter, qu'il a une présence signifiante dans sa thérapie.  Il n'est pas passif. Bien sûr, cela peut paraître une évidence car on sait l'impact de la musique sur le corps tout entier. Mais ici on rejoint  le concept de musique intégrative déjà cité\autocite[Cf.]{vrait_musicotherapie_2018}. Le patient peut être nourri par la musique mais n'est pas "passif" et seulement l'"objet " qui bénéficie du traitement musical. Son  écoute lui appartient en propre, elle est personnelle et modifiable. S'il y a modification, il peut y avoir un changement; et le mot "changement" prend alors une connotation différente,  le mouvement est y  sous-entendu,  une démarche peut en découler, voire une évolution. 




        
\paragraph{Un test comme élément déclencheur}

L'écoute n'est pas une notion si abstraite. Elle est comme de la terre façonnable.
 Un point de
 repère important dans l'élaboration de notre travail fut la prise de
 conscience par une patiente de l'impact de la musique sur elle.
 Elle faisait partie du groupe testé sans
  musicothérapie. Mais, réalisant que l'écoute pouvait être le reflet
  de son état psychique et  que la
  musique pouvait avoir un effet bénéfique sur elle, elle s'est
  brutalement mise à écouter
  assidûment des oeuvres classiques telles du 
  Mozart. Et, entre les 1° tests à son arrivée et les
  seconds lors de sa sortie de clinique, les résultats
  obtenus ont été clairement significatifs.
  Elle avait pris conscience que son écoute lui appartenait
  personnellement, qu'elle pouvait agir, y jouer un rôle sur sa
  transformation. En soi, elle est devenue actrice de sa
  métamorphose toute entière. Le seul élément déclencheur avait été un
  test d'écoute.


  \section{Les instruments et leur fréquence }
 De manière très pragmatique, le test peut être
considéré comme un indicateur des zones de fréquences à
privilégier dans le travail avec le patient. Chaque instrument a une tessiture
différente avec une plage
définie de fréquences. Selon l'analyse de l'écoute du patient, le choix
d'instrument à privilégier sera plus rapide et plus sûr.

      \begin{figure}
	\centering
	\includegraphics[width=1\linewidth]{images/instrumentfréq.pdf}
	\caption[Les instruments et leurs fréquences]{Ilustration des instruments avec leurs fréquences}
       
	\label{instrumentfreq}
\end{figure}
  



\section{Graphique: Zones 1-2-3 du Test d'écoute // Musicothérapie}


        \begin{figure}
	\centering
	\includegraphics[width=1\linewidth]{images/testtechnmethbut}
	\caption[Zones du test avec la musicothérapie]{Schéma des
          zones avec leur application en musicothérapie}
       
	\label{testbutetfonction}
\end{figure}



S'il n'y a pas de changement visible dans le test, quelles conclusions
peut-on en tirer ? le changement va-t-il toujours de pair avec le
patient? synchronisé ou différencié dans le temps?
Comme pour toutes thérapies, le patient doit avoir le temps de l'intégrer. Quand
il aura été amené à une certaine prise de distance par rapport à
lui-même et à son environnement, il passera par différentes phases qui peuvent être celles de l'acceptation, que ce soit celle de son identité, de sa transformation, ou d'un changement dans ses habitudes --- de quitter le confort de ceux-ci même si elles sont jugées négatives par lui-même et par les autres --- . Tout ceci prend du temps et ne peut pas 
toujours  aller en simultanéité avec des résultats immédiats, avant/ après.

\subsection{Critiques: }

Par cette démarche, nos sommes conscients qu'il y a un risque de catégoriser le patient. Les
rapports doivent rester confidentiels et en aucun cas transmis
aux assurances-maladies.
Du côté du patient, s'il est toujours en
attente de résultats, cette façon de faire peut le figer dans son parcours  Ou, au contraire, ce peut être une aide 
dans son travail, son évolution. Ces deux possibilités sont
intrinsèques à tous les tests.

Les séances de musicothérapie se sont déroulées mais  n'ont pas été
décortiquées et analysées. Ce serait passionnant de le faire mais ce n'était pas l'objectif de ce travail.		
        
 

Différents paramètres inhérents à ce type d'étude sont à
considérer. Que ce soit le manque de temps, les départs imprévus des patients, et/ou leur
absence momentanée (visite du psychologue, maladie, etc.). rajoutés à la
contingence difficile due à la distance séparant le lieu de domicile à
celui du lieu d'étude, toutes ces contingences ont été les principaux facteurs réducteurs de
tests valables.
: comment planifier un départ imprévu d'un patient !? il a fallu parfois
faire beaucoup de route pour effectuer les tests finaux d'un ou deux
patients.
  
Par conséquent,  de nombreux tests sont restés 
incomplets et n'ont pu
être validés car ils ne remplissaient pas toutes les conditions requises. 
 
 La prise en charge en musicothérapie a eu lieu
  une fois par semaine pendant une heure, ce qui semble trop court pour observer un changement important. Nous pourrions émettre la supposition suivante :  est-ce qu'un un travail journalier, régulier aurait été indiqué pour des résultats plus rapidement visibles avec le test?
  Est-ce qu'une immersion plus intensive en musicothérapie transformerait l'écoute des patients ? 
   En comparaison avec des
  modifications importantes de courbes des tests observées généralement  lors d' une écoute
  régulière de deux heures par jour de musique pendant 15 jours --- en référence à l'entrainement des muscles de l'oreille chez Tomatis, qui, nous le rappelons, est une pédagogie de l'écoute --- il aurait été intéressant de pouvoir faire cette étude comparative dans cette clinique. Ainsi, nous aurions pu éventuellement mettre en avant  l'absolue nécessité de créer et d'instaurer systématiquement la musicothérapie dans de nombreuses institutions mais aussi  de la développer beaucoup plus intensément  si elle est déjà existante.
  Nous sommes clairement en présence d'une ébauche d'études, avec des pistes
  suggérées. 
  Cette étude est un mixe: quantitatif et qualitatif. Nous avons ainsi pris l'option de nous tenir à une
  observation, celle de la transformation de l'écoute.
  



  

\begin{itemize}
\item Est-ce que ce test pourrait être un outil pour les thérapeutes et
les patients ? 
\item Avoir un support réel, visible car graphique pourrait-il être d'une
quelconque utilité pour le thérapeute ?
\item Est-il possible, à partir de deux tests d'écoute, de tirer des hypothèses
sur l'impact du son, de la musicothérapie, du soin par le son, sur
un patient ?
\item Le patient reste au centre de nos préoccupations.
\item Serait-ce une façon de démontrer l'utilité de la musicothérapie
pour une plus large acceptation de 
cette thérapie dans plus de milieux hospitaliers ou autres ?


\end{itemize}

\section{La musicothérapie et la méthode Tomatis}

La musicothérapie et la méthode Tomatis sont des concepts très différents. Bien que la notion d'écoute les réunit, bien que leur medium soit la musique et plus particulièrement le son, d'un côté il s'agit d'une thérapie et de l'autre, il s'agit d'une pédagogie, d'un entrainement de la musculature de l'oreille. 
Tomatis se focalise et opère essentiellement sur le capteur auditif (vestibulo-cochléaire) pour amener, par ce processus, le patient à une certaine  amélioration par rapport à sa vie actuelle, à des souhaits ou à des attentes précises; celle-ci peut se réaliser au niveau du langage et ce, par l'intermédiaire de la musique et du chant. Nous pouvons de notre côté  émettre l'hypothèse que si le contrôle auditif est de bonne qualité ainsi que l'émission vocale, c'est-à-dire que la boucle phono-auditive est élaborée sans problème, l'oreille est prête, même peut-être plus prête et apte à travailler beaucoup plus en profondeur avec tous les riches moyens que la musicothérapie propose.
Préparer le terrain, faire un travail physique de fond, une
préparation de l'oreille pour que celle-ci soit totalement
opérationnelle et prête à aborder si nécessaire, un travail en
musicothérapie sur le plan physique ou psychique. simplement pour se
sentir bien sa peau, bien dans son âme.
Voilà l'hypothèse énoncée et ce que nous nous pouvons conclure, en effet.


        Est-ce utile à tout musicothérapeute d'avoir un appareil test d'écoute ? certainement pas. C'était un moyen de faire cette étude.
     Nous nous sommes  limités ici intentionnellement
     à ce concept de test.
 La matière sonore est la matière première de la  musicothérapie. 
 Par son biais, elle  apporte de multiples éléments d'évaluation du
 sujet. 
 Certains  intégrent plus que d'autres dans leur pratique des techniques relevant du domaine de la psychothérapie--de l' analytique, 
  du comportementalisme, du cognitivisme, de la  systémique ainsi que
  celles dites humanistes.  Le concept de "médium malléable" a été
  développé par  R.Rousillon\footnote{R.Rousillon,\textit{Paradoxes et situations limites,  
  		de la psychanalyse} Paris, Puf. 1991} 
  et qu'il est possible de transposer dans la matière \textit{musique} 
  pour "favoriser et accompagner le processus 
  de symbolisation"\footnote{F.X.Vrait, \textit{La musicothérapie},Ch.3, p. 112}.

 Le musicothérapeute est un être extrêment sensible avec de multiples
 ``antennes'' : l'intuition reste primordiale tout autant que  
 l'écoute. L'oreille se dresse pour une écoute empathique, pour ``rester en contact émotionnel  
 avec le patient'' , Eckert (2007) par le son qui va au plus profond de
 l'être.

 
\begin{quotation}
	L'oreille est l'organe le plus sensible des sens 
et l'instrument de diagnostic  le plus important du
musicothérapeute.\autocite{seminar_zuerich}
 	
\end{quotation}
La musicothérapie fait partie de ces thérapies dites subtiles. Elle
est très difficilement quantifiable. La
psychologie cognitivo-comportementaliste peut le quantifier avec des tests et semble avoir gagné depuis en crédibilité. Mais avec la musicothérapie ou d'autres formes de thérapie, il n'y a
jamais, à proprement parlé, d'avant et d'après mais il y a transformation.
Et les transformations échappent toujours aux quantifications. Peut-être
ici pourrons-nous apporter un outil plus objectif par un test particulier
d'écoute : la démonstration d'un travail d'écoute, d'une perception
différente, d'une sensibilité nouvelle du patient. 


% OGA: source citation Malraux

\paragraph{Apprendre à écouter}

Apprendre à écouter,
c'est un travail et des résultats pourraient être visibles.
Comme l'exprime à juste titre André Malraux : \enquote{\emph{Le monde de
	l'art n'est pas celui de l'immortalité, c'est celui de la métamorphose.}}
De même, la musique est un art produit par l'homme et qui a un impact
sur lui-même. Les deux interagissent, s'interpénètrent et s'auto-transforment
au cours des siècles.
 Ce que nous pouvons constater lors de l'aboutissement
d'une thérapie n'est pas de trouver une autre personne mais une transformation
de la perception de celle-ci par rapport au monde qui l'entoure. 
Selon
ce que nous vivons, nous nous transformons et continuons à être
soi. Nous ``sommes soi" mais autrement. Nous ne perdons
pas notre identité.


\label{jeSuisLaMusique:viret}
\begin{quotation}
\emph{<<\,\emph{Par le Son, le Silence du Non-Être vient à l'Être}. [\dots] 
\textsl{Je suis}
	\emph{la musique que je fais ou écoute}. [\dots]\,>>
[\ldots] \emph{la musique a la capacité d'harmoniser
les composantes d'une entité psychophysique pour qu'il soit ``bien
dans sa peau'' et ``bien dans son âme.}''}\, \autocite[ch. 1,  p. 8]{viret:b}
\end{quotation}




