
\chapter{Réflexions}

ll y a l'aspect de visibilité et de lisibilité : avoir  la possibilité de visualiser sa propre écoute. Ce processus de visualisation tente de matérialiser l'abstraction innée dues aux propriétés du son, l'aspect intemporel et éphémère du son entre l'écoute et la vue. Se situer dans l'espace sonore implique tout son corps, demande un effort (--- tendre l'oreille---) implique des fonctions cognitivo-proprioceptives, et peut induire par conséquent  une prise de conscience et une prise de distance avec soi-même. Cette dernière pourra tout au  moins être initiatrice d' un début de travail sur soi.  

Le fait de se décider pour tel type de thérapie, d'accepter de faire un test, d'obtempérer aux consignes d'un consultant, de faire un choix de sons précis, de dialoguer sur les résultats avec le thérapeute, relève déjà d'une volonté de changement intérieur, en  éveille tout au moins cette idée de déclenchement d'un travail, d'un début de cheminement intérieur.

Et vice et versa, le test deviendrait-il aussi un outil intéressant pour le thérapeute ?
Cette façon de faire une anamnèse du patient en s'appuyant aussi sur un test, pourrait être complémentaire pour saisir le patient sur des aspects non conventionnels ou peu courants, peu usités, avec des éléments très appropriés dans ce contexte puisqu'ils appartiennent au domaine du son et de la musique. 
Nous pouvons  l'inclure dans un  bilan en musicothérapie. Apprendre à écouter, c'est un travail et des résultats peuvent être
visibles. Nous utilisons un outil qui est le son. Nous accompagnons
le patient d'un point A pour aller au point B : que s'est -il passé
dans son écoute? Nous pouvons apporter des résultats visibles et tangibles 
d'une forme d'apprentissage de l'écoute, d'une transformation de la
perception basé sur un test de reconnaissance de sons qui permet de visualiser une transformation psychologique
de l'écoute. 
\begin{itemize}
\item Il y a des résultats : nous pouvons constater soit un changement,
un statisme, un apprentissage,ou un refus d'apprendre et de se transformer.
Ce sont des données qui peuvent servir à mieux comprendre le patient
et à l'accompagner dans son cheminement.
\begin{itemize}
	\item \textbf{La communication : }
	La relation avec le patient est indispensable à créer pour toutes thérapies.
	Le test représente une procédure claire qui peut être considéré comme un cadre médical perçu rassurant par nombre de patients, jouant son rôle de soutien dans la prise en charge.
	
	D'autre part, le rôle du patient est différent dans son essence même, non pas dans le sens de "patere" souffrir et subir, mais valorisé dans celui du rôle actif qu'il peut jouer : celui-ci peut se rendre compte de sa capacité à influencer sa façon d'écouter, qu'il a une présence signifiante dans sa thérapie.  Il n'est pas passif. Bien sûr, cela peut paraître une évidence car on sait l'impact de la musique sur le corps tout entier. Mais ici on rejoint  le concept de musique intégrative déjà cité\autocite[Cf.]{vrait_musicotherapie_2018}. Le patient peut être nourri par la musique mais n'est pas "passif" et seulement l'"objet " qui bénéficie du traitement musical. Son  écoute lui appartient en propre, elle est personnelle et modifiable. S'il y a modification, il peut y avoir un changement; et le mot "changement" prend alors une connotation différente,  le mouvement est y  sous-entendu,  une démarche peut en découler, voire une évolution. 
\end{itemize}


Est-ce utile à tout musicothérapeute d'avoir un appareil test d'écoute ? certainement pas. C'était un moyen de faire cette étude. Elle démontre par ailleurs  l'intérêt qu'il faut donner à la phase dite "active" chez Tomatis. La musicothérapie elle, est toujours active!

\item S'il n'y a pas de changement visible dans le test, quelles conclusions
peut-on en tirer ? le changement va-t-il toujours de pair avec le
patient? synchronisé ou différencié dans le temps?
Le patient doit avoir le temps d'intégrer une forme de thérapie. Quand il aura été amené à une certaine prise de distance par rapport à lui-même et à son environnement, Il passera par différentes phases qui peuvent être celles de l'acceptation, que ce soit celle de son identité, de sa transformation, ou d'un changement dans ses habitudes --de quitter le confort de ceux-ci même si elles sont jugées négatives par lui-même et par les autres-- etc. Tout ceci prend du temps et ne peut pas 
toujours  aller de paire avec des résultats immédiats, avant/ après.


\item Est-ce normatif? 

Par cette démarche, il y a le risque de catégoriser le patient d'une part 
 et d'autre part, le figer dans son parcours. Il peut n'être qu'en attente de résultats. Ou, au contraire, ce peut être une aide 
dans son travail, son évolution. Ces deux possibilités sont
intrinsèques à tous les tests.

Nous sommes confrontés de plus en plus à donner des rapports aux caisse-maladies.
S'il y a une constatation de changement, de progression, le résultat
n'enfermera pas le patient dans une catégorie psychologique, qui,
transmise à celles-ci, pourrait lui être négative pour la poursuite
de son cheminement professionnel, via la vie active. 
\item Est-ce que ce test pourrait être un outil pour les thérapeutes et
les patients ? 
\item Avoir un support réel, visible car graphique pourrait-il être d'une
quelconque utilité pour le patient et pour le thérapeute ?
\item Est-il possible, à partir de deux tests d'écoute, de tirer des hypothèses
sur l'impact du son, de la musicothérapie, du soin par le son, sur
un patient ?
\item Le patient reste au centre de nos préoccupations.
\item Serait-ce un moyen, une façon de démontrer par ce moyen simple (autre
que l'Irmfct) que représente le test d'écoute l'utilité de la musicothérapie
? et ainsi de permettre une plus large acceptation et diffusion de
ce type de thérapie dans plus de milieux hospitaliers ou autres ?


\end{itemize}

La musicothérapie fait partie de ces thérapies dites subtiles. Elle
est très difficilement quantifiable. La
psychologie cognitivo-comportementaliste peut le quantifier avec des tests et semble avoir gagné depuis en crédibilité. Mais avec la musicothérapie ou d'autres formes de thérapie, il n'y a
jamais, à proprement parlé, d'avant et d'après mais il y a transformation.
Et les transformations échappent toujours aux quantifications. Peut-être
ici pourrons-nous apporter un outil plus objectif par un test particulier
d'écoute : la démonstration d'un travail d'écoute, d'une perception
différente, d'une sensibilité nouvelle du patient. 


% OGA: source citation Malraux
Apprendre à écouter,
c'est un travail et des résultats pourraient être visibles.
Comme l'exprime à juste titre André Malraux : \enquote{\emph{Le monde de
	l'art n'est pas celui de l'immortalité, c'est celui de la métamorphose.}}
De même, la musique est un art produit par l'homme et qui a un impact
sur lui-même. Les deux interagissent, s'interpénètrent et s'auto-transforment
au cours des siècles.
 Ce que nous pouvons constater lors de l'aboutissement
d'une thérapie n'est pas de trouver une autre personne mais une transformation
de la perception de celle-ci par rapport au monde qui l'entoure. 
Selon
ce que nous vivons, nous nous transformons et continuons à être
soi. Nous ``sommes soi" mais autrement. Nous ne perdons
pas notre identité.

\section{La musicothérapie et la méthode Tomatis}

La musicothérapie et la méthode Tomatis sont des concepts très différents. Bien que la notion d'écoute les réunit, bien que leur medium soit la musique et plus particulièrement le son, d'un côté il s'agit d'une thérapie et de l'autre, il s'agit d'une pédagogie, d'un entrainement de la musculature de l'oreille. 
Tomatis se focalise et opère essentiellement sur le capteur auditif (vestibulo-cochléaire) pour amener, par ce processus, le patient à une certaine  amélioration par rapport à sa vie actuelle, à des souhaits ou à des attentes précises; celle-ci peut se réaliser au niveau du langage et ce, par l'intermédiaire de la musique et du chant. Nous pouvons de notre côté  émettre l'hypothèse que si le contrôle auditif est de bonne qualité ainsi que l'émission vocale, c'est-à-dire que la boucle phono-auditive est élaborée sans problème, l'oreille est prête, même peut-être plus prête et apte à travailler beaucoup plus en profondeur avec tous les riches moyens que la musicothérapie propose.


Préparer le terrain, faire un travail physique de fond, une préparation de l'oreille pour que celle-ci soit totalement opérationnelle et prête à aborder si nécessaire, un travail en musicothérapie sur le plan physique ou psychique. simplement pour se sentir 
Voilà l'hypothèse énoncée et ce que nous nous pouvons conclure, en effet.