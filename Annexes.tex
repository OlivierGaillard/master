
\chapter{Annexes}

\section{Privatklinik von Meiringen}

Information für Mitwirkende an der klinischen Studie
« Evaluierung des aktiven Hörvermögens »


Sehr geehrte Damen und Herren,

Herzlichen Dank für Ihr Interesse an dieser Studie !

Wozu dient diese Studie und weshalb werden Sie um eine Teilnahme gebeten ?

Während Ihrem Klinikaufenthalt  in der Privatklinik von Meiringen werden Sie im Kontext 
unseres multidisziplinären Teams verschiedene Therapien besuchen, unter anderem auch die Musiktherapie. Bei der vorliegenden Studie möchten wir untersuchen, wie sich die Musiktherapie auf Ihr Zuhörvermögen auswirkt.
Musiktherapie ist eine gut erforschte Intervention im Bereich des Depressions und Burnouts, da Sie ein relativ neues Berufsfeld ist, gibt es noch viel Forschungspotential.
Das Hörtest konnte sich als ein Instrument erweisen, um die Veränderung des Gehörs des Patienten bei einer Musiktherapiebehandlung zu beweisen. Die Verbindung dieses Ansatzes mit der Musiktherapie ist noch nicht erforscht und daher soll dieser Ansatz wissenschaftlich näher untersucht werden.
Wenn Sie keine Musiktherapie besuchen aber Interesse für diese Studie haben, sind Sie herzlich eingeladen, dieses Test zu tun. Im Rahmen under MAS brauchen wir unbedingt eine Kontrollgruppe.

Wie sieht eine Teilnahme an der Studie aus ?

Die Untersuchung erfolgt sehr einfach in mehreren Schritten.
Zu Verfügung steht ein Apparat, mit dem sich spezifische Hörtests durchführen lassen.
Allgemein Verlauf des Tests :  
Sie hören einen sehr leisen Ton mit Zuhörern zu und werden ihn entweder mit der rechten  oder linken Hand  signalisieren. Das dauert ungefähr 30 Minuten.
Es wird zwei Tests geben : ein vor der Therapie und ein nach der Therapie.
Wir bitten Sie auch, eine kleine Fragebogen zu erfüllen.


Falls Sie Fragen haben, dürfen Sie sich gerne via E-Mail melden : valerie.gaillard\@gmx.ch

Wir bedanken uns herzlich für Ihre Zeit und die Teilnahme an dieser Studie.


Valérie Gaillard

 Le matériel utilisé : une table, deux chaises, l'appareil
test Hearing et les écouteurs aériens et osseux, un crayon, deux
feutres ( rouge et bleu), une feuille avec la grille de fréquences à
remplir.

ZhdK : Upgrade MAS Klinische Musiktherapie 15-17
