\chapter{L'écoute}

\section{Ecouter ou entendre : une différence}

La définition du verbe ``entendre'' et du verbe ``écouter'', 
\footnote{dictionnaire Hachette, édition 2012 } nous paraît opportune % opportune?
en raison de la confusion courante des deux termes :
\begin{description}
\item[Entendre] c'est  percevoir des sons, saisir par l'ouïe
\item[Ecouter] a trois sens: 
\begin{enumerate}
	\item prêter l'oreille pour entendre
	\item prêter attention
	à l'avis de quelqu'un, suivre un avis
	\item \emph{fig}: suivre une impulsion,
	une inspiration.
\end{enumerate}
\end{description}



\emph{Entendre} est une attitude passive par rapport au monde sonore
qui nous entoure. Nous recevons les sons sans les interpréter et cela
ne demande aucun effort. C'est une action involontaire et non
sélective. 

Selon Bernard Auriol,\textit{entendre} 
\begin{quote}
	suppose un son (physique), une oreille
	pour le capter, un système nerveux pour le recevoir.
\end{quote} 
Tandis qu'\textit{écouter}
\begin{quote}
	est un
	processus actif supposant préférences et répulsions pour tel son ou
	% référence de la page manquante
	telle séquence sonore\footnote{\cite{auriol:cle}, p. 999?}.
\end{quote}


Entendre et écouter sont «\,deux
fonctions essentiellement distinctes bien qu'évoluant apparemment sur
des terrains identiques\footnote{\textbf{Quel livre?} Extrait de l'entretien réalisé par
	Bernard Auriol avec Alfred Tomatis, 1973.}.\,» ... avec «\,l'élément conscient, facteur essentiel sur lequel repose toute la
différence entre ces deux activités\,».
\begin{quote}
	<<\,[\ldots] Entendre n'implique pas pour autant la présence d'un champ
	conscient. \emph{Entendre, c'est en quelque sorte subir
		un son} ou un message qui nous est adressé. \emph{Ecouter, c'est désirer appréhender ce son} ou ce message [\ldots]\,>>
	\autocite{tomatis:education}.	
\end{quote}


Ainsi \emph{écouter} est une opération de tout autre nature
puisqu'elle suppose une participation active dans le choix du message
ou dans la sélection d'une voix. Elle  implique la volonté,
permet une forme de décodage: il s'agit d'une capacité. Dans un milieu sonore important, bruyant, comme un café, lorsque nous lisons attentivement, nous faisons abstraction
des bruits environnants; en soi, nous les entendons parfaitement mais nous n'y
prêtons pas attention. Nous parvenons à couper les sons parasites, à nous en abstraire pour
nous concentrer uniquement sur les plus  pertinents, en l'occurrence ici ceux de notre lecture intérieure.
 La racine du mot "Ecouter" a des origines sanskrites et signifie "partager"; nous remarquons alors à juste titre que nous écoutons rarement seul mais  le plus souvent en face de quelqu'un dans le but de dialoguer, d'échanger. Le même phénomène se réalise avec un livre qui transmet et partage des connaissances. L'écoute permet donc la communication, sous-entend le plus souvent la présence d'un être vivant et nécessite de la  concentration. Il faut cette volonté inclue dans celle-ci  pour comprendre et rentrer par exemple en contact avec la voix de  l'écrivain qui chante dans le texte avec celle, intérieure,  du lecteur.

 





\textbf{Ecouter} se base certes sur une stimulation prenant sa source à
l'extérieur mais \textbf{devant être intérieurement et  intentionnellement
	recherchée}.
Nous pouvons encore différencier les différents types d'écoute. D'après Edith Lecourt\footnote{Edith Lecourt, \textsl{Découvrir la musicothérapie}, ch. 10, p. 179, <<\,De l'écoute verbale à l'écoute musicale\,>>, \cite{lecourt:decouvrir}.}, on en distingue plusieurs : l'écoute verbale, musicale, plurivocale et multiple.
 L'analyse musicale qui permet la différenciation d'une voix d'un ensemble polyphonique est appelée \emph{plurivocale}. Celle qui est multiple n'est pas analytique  mais 
 \begin{quote}
 	 [\ldots] \textit{ouvre une disponibilité, met en suspens les grilles verbale et musicale} [\ldots] \emph{pour parcourir le vécu sonoro-affectif}\footnote{\textbf{la référence manque}.}.
 \end{quote}
 Employée en musicothérapie, Edith Lecourt la nomme  la technique de la  \emph{communication sonore} qui peut apporter 
 <<\,des ouvertures sur l'analyse des niveaux plus archaïques de l'organisation mentale.\,>>	

 Nous aborderons plus loin cet aspect au point \ref{musicothEtpsycho} intitulé <<\,La musicothérapie et la psychothérapie\,>>.
 
 
Puisqu'écouter implique la notion de son et d'oreille, nous allons dans un premier temps approfondir  la définition du son et ses caractéristiques physiques puis dans un deuxième temps aborder l'anatomie de l'oreille.

\section{Le son}

Le son possède plusieurs caractéristiques physiques. Il peut être
défini très précisément par un ensemble d'unités physiques chiffrées
: les décibels et les hertz. 
\begin{itemize}
	\item Un décibel est l'unité de mesure de l'intensité du son. Un décibel
	est égal à $1/10$ de bel ; une augmentation de l'intensité égale à 1
	bel équivaut à peu près à un doublement de l'intensité sonore. 
	\item Un hertz est une unité de fréquence\footnote{la fréquence est le nombre de vibrations par unité de temps dans un
		phénomène périodique} (symbole : Hz). Équivalent à $1s-1$. Fréquence d'un phénomène périodique
	dont la période est une seconde. Ses multiples sont, entre autres,
	le kilohertz (kHz), le mégahertz (MHz) et le gigahertz (Ghz). Cette
	unité vient du savant allemand Heinrich Hertz, pionnier de la radioélectricité.
\end{itemize}
Le son peut être défini de deux manières : 
\begin{itemize}
	\item d'une manière objective tout d'abord : c'est le phénomène phy\-si\-que
	d'origine mécanique consistant en une variation de pression (très
	faible), de vitesse vibratoire ou de densité du fluide, qui se propage
	en modifiant progressivement l'état de chaque élément du milieu considéré,
	donnant ainsi naissance à une onde acoustique (la propagation des
	ronds dans l'eau suite à un ébranlement de la surface donne une bonne
	représentation de ce phénomène) ; 
	\item d'une manière subjective également : il s'agit de la sensation procurée
	par cette onde, qui est reçue par l'oreille, puis transmise au cerveau
	et déchiffrée par celui-ci.\index{http://www.futura-sciences.com/sante/dossiers/medecine-bruit-effets-sante-259/page/3}
\end{itemize}
De plus, il y a de nombreux paramètres à prendre en compte : par
exemple : l'impression de force sonore : la sensibilité de l'oreille
est une variable de la fréquence. Il faut 1000 fois moins de pression
acoustique pour avoir une sensation auditive à 4000 hertz qu'à 50
hertz. Notre oreille n'a donc pas la même sensibilité pour toutes
les fréquences audibles. Il en est de même pour la sensation auditive
des basses fréquences et pour la dynamique. 





\subparagraph{L'objectivité de l'écoute}.

Objectivité ou subjectivité de notre écoute?  
Nous avons tous,
selon les manuels d'anatomie, la même
oreille, du moins nous pouvons reconnaître une analogie de structure . Nous devrions donc entendre et écouter la même chose
lors d'une même information diffusée tout comme le fait un enregistreur avec un micro. Pourtant il n'y a pas d'écoute "passive". Chacun n'entend pas de la même manière les mêmes
informations. En somme, tout un chacun entend ce qu'il veut bien
entendre, selon sa propre psyché. 

 \emph{Vouloir voir, c'est viser.}  Vouloir entendre dans le but d'écouter est comparable  à
la visée de l'oeil lorsque l'on veut collecter une information. L'oeil regarde avec la rétine et  vise, sous l'ordre du cerveau, avec la macula. Dans la même idée, par l'écoute, nous avons l'oreille et la cochlée ( partie interne de l'oreille) qui permet l'analyse des sons. A nouveau, il faut une commande du cerveau.
En définitive,\textbf{ l'audition est la capacité perceptive du système auditif et l'écoute, c'est ce qu'on en fait.}
\paragraph*{Le cerveau}
Nous nous apercevons du rôle important que peut jouer notre cerveau à cet égard et de son énorme complexité. 
D'ailleurs, les recherches scientifiques pullulent à ce sujet.L'IRM fonctionnelle ( imagerie par résonance magnétique)  permet d'avoir une vision globale de l'activité du cerveau lorsque celui-ci est soumis à une stimulation sensorielle. Toutes les expériences accumulées depuis des siècles en musicothérapie, tous les résultats cliniques trouvent aujourd'hui leur confirmation par les remarquables avancées des sciences humaines. Selon François-Xavier Vrait%
\footnote{"La musicothérapie" p.96,ch.III}, la musicothérapie 
contemporaine est à un tournant décisif, elle devient " \textit{intégrative"} tout en conservant ses racines séculaires. Grâce aux travaux d'Isabelle Peretz%
\footnote{I. Peretz, "Les agnosies auditives", in X. Seron, M.Jeannerod (éd.) Neuropsychologie humaine, Lièg, Mardaga, 1994, pp. 205--216.}
ainsi qu'au chercheur français, Hervé Platel%
\footnote{Hervé Platel, "Neuropsychology of musical perception : new perspectives", Brain, n°125(2002), pp. 223--224.} on sait actuellement que " le cerveau traite distinctement les aspects perceptifs et émotionnels de la musique",(...) que " les circuits neuronaux qui concernent la mémoire musicale à long terme sont distincts des processus neurofonctionnels sollicités pour la mémoire verbale".


Par ailleurs, en restant toujours sur ce sujet - dans l'écoute et le cerveau-, quoique que le domaine de recherche sur l'autisme soit encore en pleine investigation et que rien n'ait été encore défini, il semblerait que la capacité d'entendre des patients autistes soit excessive, qu'il s'agit d'une hypersensibilité aux sons devenant douloureuse quand  le flux des informations est trop important et que le tri ne peut pas se faire. Le cerveau entend mais ne veut pas ou ne peut plus  écouter, pour en réalité, se protéger. L'autisme reste difficile à élucider. Selon Brigitte Harisson, pp. 22--23, les dernières recherches sur le TSA ( trouble du spectre autistique) affirmeraient que leur cerveau soit différemment connecté et qu'il ne s'agit pas d'une déficience intellectuelle ni d'une maladie mais d'un trouble neuro-développemental, un trouble d'intégration sensorielle%
\footnote{L'autisme expliqué aux non-autistes
Livre de Brigitte Harrisson, Kim Thúy et Lise Saint-Charles
\textsc{"Des pistes inédites à partir des neurosciences, des résultats cliniques et de l'expérience d'une personne autiste".} Cet ouvrage propose une description unique du TSA (trouble du spectre de l'autisme) en ... Google Books
Date de publication originale : 2017, éd. Trécarré, Québec
Auteurs : Kim Thúy, Brigitte Harrisson, Lise Saint-Charles.}.




\begin{quotation}
	\char`\"{}\textbf{C'est le son qui a fabriqué l'oreille et si tu veux connaître
		le son, apprends d'abord à étudier l\textquoteright oreille\char`\"{}.}
	Hermès Trimégiste
\end{quotation}
Abordons brièvement l'anatomie de l'oreille pour mieux comprendre le cheminement du son dans notre corps.

\section{L'oreille}

\subsection{Anatomie de l'oreille}

L'oreille\footnote{Biologie humaine, principes d'anatomie et de physiologie, Elaine N.Marieb, éd. Pearson Education, 8ème édition, chap. 8, pp. 319--321.}  se situe à l'intérieur de l'un des os du crâne, le temporal, et plus précisément la pyramide pétreuse ou rocher. Elle se compose de trois parties : externe, moyenne, interne.
\begin{itemize}
	\item  L'oreille externe : formée du pavillon et du méat acoustique externe
(canal auditif). Les ondes sonores entrent dans le méat et percutent
une membrane de 60 mm2, appelée tympan, et la font vibrer. Cette membrane
sépare l'oreille externe de l'oreille moyenne. 
Selon Alfred Tomatis, médecin audio-psycho-phonologue dont nous reparlerons plus loin au chapitre 3.3/ 4, elle
joue un rôle de filtre des graves et d'amplificateur des aigus.
	\item 
	L'oreille moyenne :
se trouve dans l'os temporal constituée de petites cavités dont une, centrale, qui est la caisse du tympan. Sa limite
médiale est une paroi osseuse percée de deux orifices , la fenêtre
du vestibule et la fenêtre de la cochlée. La trompe auditive ou d'Eustache
est un conduit oblique qui relie l'oreille moyenne à la gorge et sert
à équilibrer la pression de l'air entre l'oreille moyenne et l'extérieur.
Les trois osselets de l'ouïe sont : le marteau, l'enclume et l'étrier
( les plus petits os du corps). Ils transmettent les vibrations du
tympan aux liquides de l'oreille interne. Le marteau et l'étrier se
trouve dans l'os temporal constitué de petites cavités dont une,
centrale, qui est la caisse du tympan. Les trois osselets de l'ouïe
sont : le marteau, l'enclume et l'étrier ( les plus petits os du corps);
le marteau et l'étrier sont commandés chacun par un muscle. D'après
Tomatis, son rôle est double : protéger l'oreille interne des sons
trop forts et celui de cibler les sons à écouter.
\item L'oreille interne
: est l' organe de l\textquoteright audition. Il
est constitué d'une coque osseuse d'une très grande densité( la plus
importante du corps), contenant un corps membraneux qui en épouse
la forme. L'oreille interne est une enfilade de cavités osseuses portant
le nom de labyrinthe osseux. Il comprend trois subdivisions : la cochlée,
le vestibule du labyrinthe et les canaux semi-circulaires. Le labyrinthe
osseux est rempli de périlymphe, un liquide. Et dans ce périlymphe
flotte le le labyrinthe membraneux qui contient lui-même un liquide
plus épais appelé endolymphe. Ils jouent leur rôle dans l'équilibre
statique et dynamique. Le vestibule et les canaux semi-circulaires
sont les organes de l\textquoteright équilibration, la cochlée ou
limaçon est l'organe de l'audition. 
\end{itemize}

\section{La physiologie de l'audition}

Le chemin du son dans l'oreille jusqu'au cerveau : 

Chaque son parvenant à l'oreille entre dans le pavillon et se propage
dans le conduit auditif. Les vibrations de l'onde sonore mettent en
mouvement le tympan lié aux trois petits os (marteau, enclume, étrier).
La transformation (et l\textquoteright amplification) des vibrations
aériennes en vibrations solidiennes se fait par l\textquoteright intermédiaire
des osselets : les vibrations du tympan entraînent successivement
celles du bloc marteau-enclume puis celle de l\textquoteright étrier,
qui les transmet à l\textquoteright oreille interne via la fenêtre
ovale.

Le rapport de levier effectif entre le marteau et l\textquoteright enclume
(de l\textquoteright ordre de 20), d\textquoteright une part, et le
rapport de surfaces entre le tympan (60mm2) et la platine de l\textquoteright étrier
(30 mm2) d\textquoteright autre part font du système tympano-ossiculaire
un véritable amplificateur permettant à l\textquoteright énergie sonore
d\textquoteright être transmise presque intégralement à l\textquoteright oreille
interne.

A partir de 80 dB, un réflexe protecteur (stapédien) est mis en place
afin de réduire la transmission des pressions vers l\textquoteright oreille
interne, par l\textquoteright intermédiaire des osselets et des muscles
qui rattachent le marteau et l\textquoteright étrier aux parois de
la caisse du tympan. Il s'agit ainsi d' un procédé mécanique qui amplifient
les vibrations atteignant la cochlée. 
\begin{quotation}
	La cochlée à son tour ``va transformer ces vibrations en impulsions
	nerveuses véhiculées par le nerf auditif.'' (...)Les cellules ciliées
	tapies dans la membrane cochléaire ``transforment ces vibrations
	en messages électriques, circulant dans le nerf auditif. (...) Et
	ces informations vont ``se diriger vers le cortex cérébral, via plusieurs
	relais.(...) ``Comme certaines fibres issues de chaque oreille croisent
	la ligne médiane, chaque aire auditive reçoit des signaux des deux
	oreilles.'' De plus, ``tout au long du trajet, le message subit
	des transformations dues aux caractéristiques de l'activité des neurones.''
	Retenons que `` les cellules ciliées proches de l'étrier sont activées
	par les sons aigus, et celles situées au sommet de la cochlée le sont
	par les sons de basse fréquence''.(...) `` Une scène auditive est
	mêlée d'un ensemble d'ondes acoustiques et son analyse se ferait non
	seulement tout au long du système auditif avec des indices comme la
	fréquence et l'intensité mais aussi au-delà, pour utiliser les informations
	liées aux autres sens ou au contexte.''%
	\footnote{\textbf{Le cerveau mélomane}\cite{Emmanuel Bigand, Le cerveau mélomane,2011}, chap.1, pp.~15--16.}
\end{quotation}



\chapter{Le test d'écoute}

Dans le milieu médical, on le nomme non pas test d'écoute mais audiogramme. Il
sert à mesurer les seuils d'audition des sujets, grâce à l'audiomètre. Cet 
appareil français avait été mis au point en 1933. Les Américains
ont repris ces travaux pendant la dernière guerre pour pouvoir dépister
les dommages subis par ceux qui conduisaient des engins bruyants comme
des avions.

\section{L'audiogramme}

  L'audiogramme est une épreuve d'ordre physiologique. Ce test peut faire partie des examens  pratiqués en otologie\footnote{otologie : branche de la médecine
  	qui étudie l'oreille et ses maladies.} pour poser un diagnostic. 
   C'est un examen à partir duquel se
  dessinent les données dénommées étiologiques\footnote{étiologie : étude des causes
  	d'une maladie} pour détecter un trouble de la fonction auditive. Un pronostic pourra définir le mode de thérapie
médicale, chirurgicale, prothétique ou rééducative.
Il n'y a aucune considération d'ordre psychologique. La procédure technique inclut entre autres des hauteurs de sons à détecter dans un ordre aléatoire, l'emploi du \emph{masking} (un bruit dans l'oreille opposée
au son à reconnaître) et des séries de
mots
à répéter, d'intensités différentes, de moyen,
fort, faible à très faible. (30dB). Suivent évidemment des tests d'auscultations et des gestes propres au corps médical. 

Dans notre recherche, généralement les tests d'écoute, définis comme tels, sont de nature verbale, ils mettent l'accent sur la communication, la capacité d'empathie, ont un aspect psychologique mais sont sans lien direct avec un élément sonore à déceler. 

\subsection{Le test d'écoute en musicothérapie} 

 Il s'agit en général d'un test d'audition d'\oe uvres musicales. Une grille précise est remplie selon les réponses des patients. On les appelle bilans psycho-musicaux. Par ce truchement, un travail différent pourra être fait pour faire une anamnèse plus large ou plus en profondeur du patient. Le son permet de donner un miroir psychologique de la personne. Et inversement, on s'est aperçu du rôle éminemment important que peut jouer la musique dans les traitements psychiatriques. Bien que les bienfaits de la musique soient reconnus depuis l'Antiquité, ce courant s'est installé plus précisément dans ce domaine en Occident vers 1800.
 
 
 Détecter un son précis, une note à reconnaître,  à situer dans l'espace selon son volume,  pourrait-il donner   un indice, \textbf{une information physiologique et psychologique} à la fois sur le patient ? Nous pensons qu'Alfred Tomatis, audio-rhino-laryngologue, dont nous reparlerons plus loin et plus en détails au chapitre ... ,en a mis au point un. 
 Bien sûr, ici on se limite intentionnellement à l'idée du test d'écoute. Il est évident que la matière sonore est la matière première de la  musicothérapie . Par son biais, elle  apporte de multiples éléments d'évaluation du sujet  autres que ceux d'un test. Il est clair qu'il existe de nombreuses approches et techniques qui ont été mises au point  et qui se renouvellent tous les jours dans la pratique. C'est pour cela que nous nous étendrons pas sur d'autres façons de travailler. Par contre, de manière générale, elles relèvent très souvent du domaine de l' analytique, du comportementalisme, du cognitivisme, de la  systémique ou même de psychothérapies dites humanistes. Nous allons aborder tout de suite mais brièvement ce qui a amené certains musicothérapeutes à suivre ce chemin spécifique, ou à  l'inverse, ce qui a amené des psychanalystes ou des professionnels en psychologie à s'intéresser à intégrer le son dans leur pratique.
 
 
 
\section{La musicothérapie et la psychothérapie}
\label{musicothEtpsycho}

	 Les musicothérapeutes sont donc très souvent  issus non seulement du domaine musical, médical mais aussi  de  la psychologie et de la psychiatrie. La musique s'est révélée être  un support d'expérimentation notoire en psychothérapie. Certains, tels Rolando Benenzon, Edith Lecourt,  ont fait fusionner les deux dans leur pratique  en utilisant le son comme élément facilitant l'exploration psychique. Ils ont élaboré des techniques, des façons de procéder, en soulignant l'importance d'un tel support dans la  communication ou l'introspection.
	 
	  \subsection{Rolando Benenzon} Le professeur et docteur Rolando Omar Benenzon structura à partir de 1969 un modèle de musicothérapie en se basant sur Freud, Jung, Winnicott, Watzlawick, influencé par le concept de l'objet sonore notamment avec P.Schaeffer et C.Sachs ainsi que par les grands pédagogues musicaux comme Willems, Dalcroze ou Kodaly. Sa définition de la musicothérapie est celle d'une musico-psychothérapie  \emph{\textsl{qui utilise les expressions corporo-sonoro-non verbales.}}, centrée sur le concept d'identité sonore.

        \subsubsection{Edith Lecourt} est une Docteur ès lettres et sciences humaines, psychanalyste et musicienne à l'Université René Descarte-- Paris V),musicothérapeute.
        Benenzon et Lecourt ont  recherché la place qu'occupe le sonore dans la vie d'un patient, et on peut supposer qu'ils ont sans doute perçu l'idée générale et conductrice de \emph{la méthode projective}, en terme 
	    <<\,d'investigation dynamique et holistique de la
        personnalité\,>>. Les tests projectifs sont devenus à partir
        de 1939 un des instruments très utilisés en psychologie
        clinique. Ils réunissaient trois épreuves : le test
        d'association de mots de Jung (1904), le test des taches
        d'encre de Rorschach (1920) et le TAT (test d'histoires à
        inventer) de Murray (1935)\footnote{\cite{anzieu.chabert:methodes}, \emph{Les méthodes projectives}, ch.~1, p.~13.}.
		

	
Inspirés par ces divers courants, Helen Bonny, Jacqueline Verdeau-Paillès et Fern Nevjinsky ont  mis au point au fil de leur pratique des modèles et tests spécifiques en musicothérapie:


\subsection{Helen Bonny} (USA) était une musicothérapeute,
musicienne et psychothérapeute, qui a mis au point dans les années 70
une technique particulière, le GIM, \og Guided Imagery and Music\fg
l'imagerie guidée et de la musique. Selon GIM
Trainings\footnote{Site visité le 2.1.18 / \href{\#gimsite}{GIM}.}, la
musique associée à la thérapie libère par l'émotion en reliant le
conscient à l'inconscient.(The Evolution of Guided Imagery and Music, by Helen Bonny, Ed. by Lisa Summer (2002), p. 7)
 C'est une forme réceptive de travail
en musicothérapie, avec comme principales influences Carl Rogers, Abraham Maslow et Carl Jung; elle  consiste en une longue anamnèse avec le
patient qui permettra de cibler le programme de musiques appropriées. 
(des \oe uvres de compositeurs tels Beethoven, Brahms, Debussy,
Mozart, Rachmaninov ou Vivaldi. )
Il n'y a  pas de
tests d'écoute, de notes ou de sons spécifiques à proprement parlé à déterminer ou à localiser.

\subsection{Jacqueline Verdeau-Paillès}

 De même, la psychanalyste Jacqueline Verdeau-Paillès a étudié
 et
intégré en 1985 la psychanalyse avec le son.  Le sonore est  introduit
sous forme réceptive avec un test d'audition d'\oe uvres pour réaliser
une relation analytique\footnote{\textsl{Le bilan psycho-musical et la
    personnalité}, \cite{verdeau-pailles:bilan}.}.

% pourquoi une énumération itemize?  je la commente.
%\begin{itemize}
Quelle est la place qu'occupe la musique et le sonore dans la vie d'un patient ? Son test avec un entretien, un test d'audition d'\oe uvres et un texte actif permet d'évaluer la réceptivité et les possibilités de communication par ce médium, ce qui va permettre d'établir un projet thérapeutique.

Les recherches de Benenzon\footnote{Rolando Omar
  Benenzon, \textsl{La musicothérapie, la part oubliée de la
    personnalité}, \cite{benenzon:musicotherapie}.} ont été reprises par
Verdeau-Paillès pour l'élaboration de ce test. Il consiste en la technique du montage en U qui débute avec 5 à 6 morceaux de 3 à 4 minutes chacun en fondus enchaînés, amenant progressivement le patient à la détente; celui-ci a un entretien-questionnaire à la première
séance, et lors de l'audition des musiques choisies par le thérapeute
et/ou par lui-même, il  verbalise son vécu. Le musicothérapeute
va recevoir et analyser ce qui en émerge. 
La musique favorise  <<\,\emph{l'expression et le développement
	de la pensée}\,>> et  va <<\, [\ldots] \emph{permettre la prise de conscience des processus pathologiques développés} [\ldots]\,>>\footnote{Source : ASSOCIATION AMARC,
  Association de musicothérapie, recherches cliniques et
  applications). \textbf{references de l'article AMARC?}}.


 \subsubsection{Fern Nevjinsky et le test de Rorschach}
 
 De son côté, Fern Nevjinsky a développé à partir du  test de Rorschach un test psycho-musical avec des morceaux
de musique en association libre. Il utilise ainsi le test
musical en complémentarité de celui de Rorschach%
\footnote{Fern Nevjinsky, maître de conférences à l'Université de Rouen, musicien, psycho-analyste, \textsl{Adolescence, musique, Rorschach}. Comparaison des modalités de projection et d'expression au test de Rorschach et à un test psycho-musical pour des adolescents de 13 à 16 ans,
  \cite{nevjinsky:adolescence}.}. Il nous dit  que  «\,[\ldots] la
portée diagnostique du test fait avec des sons purs, en se limitant à
l'identification, est insuffisante; mais, si la consigne est libre ---
dire ce que le son signifie --- toutes les perceptions erronées sont
le point de départ d'une expression fantasmatique en relation avec le
passé du sujet, ses souvenirs. [\ldots] 
Il prouve  la valeur privilégiée du son comme éveil
des affects liés à des conflits qui n'apparaissent pas dans
l'entretien ou dans les tests visuels.  [\ldots] A un niveau
psychanalytique, par le biais de la régression, elle peut amener le sujet à abandonner une partie de sa vigilance défensive.\,»\marginnote{pages?}

En définitive, nous revenons donc, avec d'autres façons d'intervenir,
à ce qui a été déjà formulé plus haut dans la technique de
 Verdeau-Paillès, à savoir : la musique est un outil non-anxiogène, déclencheur des expressions qui provoque
l'éveil des affects dans  leur verbalisation. Nous restons néanmoins sur notre faim car si nous pouvons nous convaincre du bienfait de l'utilisation de la musique, nous n'avons toujours pas trouvé un test d'écoute beaucoup plus simplifié, révélateur de l'état d'écoute du patient, une sorte de <<\,\emph{chek-up} d'entendre et d'écouter\,>>
 qui donnerait des indices sur la façon dont le sujet prête l'oreille aux sons autour de lui et s'il existe une évolution, un changement dans son écoute.


\subsubsection{Bernard Auriol}

Bernard Auriol\footnote{Médecin psychiatre, psychothérapeute, 
	né en 1938, a écrit plusieurs ouvrages, dont : \textsl{Le son au subjectif présent} \cite{auriol:son}, \textsl{La clef des sons, Éléments de psychosonique} \cite{auriol:cle}, \textsl{Méditation et
  psychothérapie},  \emph{méthode TOMATIS}\marginnote{compléter?}} 
a étendu ses recherches sur le son, la psychosonie, 
tout en s'inspirant des
travaux d'Alfred Tomatis, avec lequel il s'est également formé et dont nous parlerons plus longuement au chapitre 3 et 4.

Le terme \emph{psychosonique} a été créé en 1991 par Bernard Auriol pour
désigner la discipline qui cherche à évaluer et décrire les effets du
son sur l'être vivant, l'homme, ainsi que les éléments
subjectifs manifestés par l'expression sonore, en particulier la voix.
Il convient de distinguer la psychosonique de la psychoacoustique qui
se situe davantage du côté de la psychophysique que d'une approche
psychodynamique. La psychoacoustique se préoccupe des conditions
acoustiques et neuro-psycho-physiologiques de l'audition, alors que la
psychosonique tente d'étendre le point de vue aux éléments
symboliques, psychodynamiques, inconscients et subjectifs du processus
d'écoute ;  en ce sens, elle est très proche de la musicothérapie.
Bernard Auriol a mis au point divers tests d'écoute inspirés de celui de Tomatis.
  
\section{Alfred Tomatis et le test d'écoute}
  Alfred Tomatis a créé un test d'écoute; c'est un outil qui permet d'objectiver la qualité de l'écoute.
  Le test est  basé sur une chaîne régulière de sons précis dans un même ordre à identifier, des sons purs dont on varie le volume (de très faible à fort). Dans son ouvrage \emph{Éducation et Dyslexie},le professeur Tomatis
  a présenté le test d'écoute comme étant le test le plus important du
  bilan, dénommé audio-psycho-phonologique et devant déterminer les
  possibilités d'écoute du sujet : auto-écoute et écoute de
  l'autre\footnote{``Considérations sur le test d'écoute. Propos
  	recueillis au cours du IIIème congrès international
  	d'audio-psycho-phonologie ( Anvers 1973) à la suite d'un entretien
  	avec le professeur Tomatis.}. 
  Avec un audiogramme classique, le but est de mettre en évidence un trouble de l'audition. Les procédures de passation du test semblent se rapprocher de son test mais en réalité, il n'en est rien. 
  
  Ici, avec ce test de Tomatis, il est possible de détecter si le patient désire ou non se servir des sons
  qu'il a à sa disposition. Il a peut-être la possibilité d'entendre un large spectre de
  sons mais ne souhaite pas, ne veut pas les écouter. Les raisons sont multiples et en général d'ordre psychologique ( traumatismes,
  expériences négatives). Le cerveau aura le
  pouvoir d' assourdir certaines fréquences, de les masquer puis de les faire disparaître peu à peu de
  son champ d'écoute. Par protection, par réflexe de survie, il choisit de les
  annihiler alors que les sons sont là, réels, et que  l'oreille peut physiquement les collecter. Le cerveau crée ce
  que l'on appelle des distorsions d'écoute. \footnote{Professeur
    Tomatis \emph{Education et Dyslexie},  Editions ESF
    Collection Sciences de l'éducation.}. % TODO: ajouter en bibliographie}

\begin{quote}
\emph{``Le test d'écoute sait intégrer ces renseignements dans le
cadre d'un processus psychologique qui va permettre de déceler si
le sujet\textbf{ désire ou non se servir des matériaux }qu'il a à sa disposition
sur le plan perceptif. (...)Il est avant tout un test psychologique
et} les données psychologiques vont permettre d'établir un\emph{ diagnostic}
et d'orienter un mode d'action.''
\end{quote}

Est-ce vraisemblable ?
Il est paradoxal de pouvoir  entendre des sons que l'on
n'entend pas puisque l'on ne veut pas, en somme, les écouter. Parallèlement à cette
adéquation très simple qui est celle de signaler un son dès qu'il est 
entendu, en un même temps, les
réponses donnent des indices, des renseignements sur soi sans possibilité de 
contrôle intellectuel.


Cette forme d'objectivité \footnote{L'objectivité et la
  subjectivité, notions très complexes lorsque l'on parle du
  son.} dans le test - la mise en évidence des seuils d'écoute- et en
même temps cette possibilité -d'analyser par les résultats le potentiel d'écoute de
chaque patient- nous ont intrigués et incités à aller plus loin dans notre recherche sur cette méthode.
% je ne vois pas où commence et finit la citation ici.
 





