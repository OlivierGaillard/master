\begin{thebibliography}{10}
\bibitem{key-1}''Biologie Humaine , Principes d'anatomie et de physiologie'',
Elaine N.Marieb, 8ème édition, Pearson Education.

\bibitem{key-2}d Auriol (1996), \emph{La clé des sons, }Ed.Erès

\bibitem{key-3}Jean-Claude Ameisen (2013), \emph{Sur les épaules
de Darwin, Les battements du temps. }Ed\emph{.}Les liens qui libèrent

\bibitem{key-7}Silvia Bencivelli (2009),\emph{ Pourquoi aime-t-on
la musique?Oreille, émotion,évolution. }Ed.Belin.pour la science.

\bibitem{key-4}Simone Dalla Bella, Movement to Health Laboratory,
Université de Montpellier,lors de la XIIème Rencontre Cerveau-Esprit:
``Les rythmes'', 22 mars 2012, Sion, SUVA.

\bibitem{key-4}Rolando Omar Benenzon (2007)\emph{. La musicothérapie,
La part oubliée de la personnalité. }De Boeck

\bibitem{key-5}Emanuel Bigand (2012) \emph{Conférence, Musicothérapie
et Identité sonore.}Dijon, 6 avril 2012,\emph{ La musique comme outil
de stimulation cognitive. }In Richelle, (2013)\emph{Le cerveau mélomane
. }Belin

\bibitem{key-6}Isabelle Haugmard (2010) \emph{ABC de la thérapie
par les sons. }Michel Granch

\bibitem{Viret}Jacques Viret (2007) \emph{B. A-BA de la musicothérapie,}.
Pardès

\bibitem{key-9}Denis le Bihan \emph{Le cerveau de cristal, ce que
nous révèle la neuro-imagerie} (2012) Odile Jacob, sciences.

\bibitem{key-10}Antonio Damasio (2010) \emph{L'autre moi-même, les
nouvelles cartes du cerveau, de la conscience et des émotions }. Odile
Jacob,sciences.

\bibitem{key-12}Joseph-Marie Verlinde (1998) \emph{L'expérience interdite
.}Saint-Paul

\bibitem{key-13}L'essentiel, Cerveau et psycho (2011) \emph{Le cerveau
mélomane. }Revue de psychologie

\bibitem{key-14-1}Alfred Tomatis (1987) \emph{L'oreille et la voix.
}R. Laffont

\bibitem{key-2-1}Alfred Tomatis\emph{ }(1983)\emph{Vers l'écoute
humaine, }2ème Ed.Tome 2, Collection Science de l'éducation

\bibitem{key-3-1}Alfred Tomatis (1977) \emph{L'oreille et la vie,}
Ed.Robert Laffont

\bibitem{key-4-1}Alfred Tomatis (1972) \emph{De la communication
intra-utérine au langage humain, la libération d'Oedipe,} 5ème Ed.,\emph{
}Collection Science de l'éducation

\bibitem{key-5-1}Alfred Tomatis ((1989) \emph{Neuf mois au paradis,
histoires de la vie prénatale,} Ed.Ergo Press

\bibitem{key-6-1}Alfred Tomatis (1988)\emph{Les troubles scolaires,}
Ed.ErgoPress

\bibitem{key-3}Dominique Sandre, Docteur en médecine, spécialiste
en pédiatrie, conférence du 6 avril 2012, Dijon, \emph{Environnement
sensoriel du bébé dans un contexte hospitalier.}

\bibitem{key-1}Daniel Barenboim (2008), pianiste et chef d'orchestre,
\emph{La musique éveille le temps}. Ed.Fayard

\bibitem{key-2}Jean-Yves Bosseur (2005), \emph{Du Son au Signe, Histoire
de la notation musicale. }Ed. Alternatives

\bibitem{key-1}Emission''Envoyé spécial'', conçue et animée par
Pierre Lane

\bibitem{key-1}Patrick Dumas de la Roque ``L'écoute, c'est la vie'',
éd. Jouvence, trois Fontaines 

\bibitem{key-7}Fern Nevjinsky,\emph{ Adolescence, musique, Rorschach,
}, publication de l'Université de Rouen n\textdegree 215, source internet
décembre 2016.

\bibitem{key-1}E.Gendlin,\emph{ Focusing,} (2010) Ed. Pocket Evolution 

\bibitem{key-2}Felicitas Sigrist, \emph{Burnout und Musiktherapie,
Grundlagen, Forschungsstand und Praxeologie, }Ed. Zeitpunkt Musik,
Reichert Verlag Wiesbaden 2016

\bibitem{key-3}Hans-Helmut Decker-Voigt, S\emph{chulen der Musiktherapie,
}Ernst Reinhardt Verlag München Basel
\end{thebibliography}
