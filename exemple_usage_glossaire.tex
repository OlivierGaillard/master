\chapter{Exemple de l'utilisation du glossaire}

Par exemple si tu utilises
le terme de \gls{musicothérapie} il devra être 
défini par dans le glossaire. Comme ce paragraphe
contient une référence à l'entrée du glossaire son
numéro de page sera affiché à côté de l'entrée.

Il est aussi possible de définir les abréviations ou
acronymes. Par exemple le \acrshort{gcd} (sous sa forme
abrégée), ou longue \acrlong{gcd}, voir complète comme
ici \acrfull{gcd} peuvent être utilisée et améliorer
beaucoup la lisibilité du texte en procurant
toujours les mêmes abréviations.

Une fois que ceci est en place il suffit pour obtenir le glossaire:
\begin{enumerate}
\item lualatex master
\item makeglossaries master
\item lualatex master
\end{enumerate}

Actuellement le titre du glossaire apparaît en anglais. Je n'ai pas
encore cherché à modifier ceci. Détail pour l'instant.
