\chapter{Introduction et Hypothèse : }
\begin{quotation}
``\emph{ Par le Son, le Silence du Non-Être vient à l'Être. Je suis
la musique que je fais ou écoute. La musique a la capacité d'harmoniser
les composantes d'une entité psychophysique pour qu'il soit ``bien
dans sa peau'' et ``bien dans son âme''. Jacques Viret }\footnote{\emph{B. A-BA de la musicothérapie,(2007)} éd.Pardès}
\end{quotation}
J'exerce le métier de musicienne professionnelle depuis de nombreuses
années. Par la suite, je suis aussi devenue musicothérapeute, formée
également en tant que consultante à la méthode Tomatis, études suscitées
par mon intérêt croissant au sujet du développement de la personne
à travers l'outil que nous offre la musique : le son.

Dans ma pratique, les deux formations ont successivement pris plus
ou moins d'importance, selon les périodes de travail en clinique psychiatrique,
en réhabilitation à la SUVA, ou selon mon travail dans mon cabinet
privé. En clinique, ce sera exclusivement la musicothérapie. En cabinet
privé, j'aurai la liberté de choix dans l'utilisation des méthodes.
C'est ainsi que, peu à peu, à travers chaque cas particulier du patient,
j'ai vu mes techniques d'approche et de soin se modifier, évoluer,
pour parfois, aller se fusionner. Dans ce cheminement, de nombreuses
interrogations se sont imposées et continuent de m'interpeller.

Ce qui m'a animée? C'est de creuser et d'approfondir mes recherches.
Cette démarche me semble originale, quoique Jacques Bonhomme, spécialiste de la voix, a aussi utilisé ce test dans son travail. Mais le domaine psychiatrique dans lequel je fais cette  recherche est différent.
 Pourrai-je être plus précise dans mes
observations? Pourrai-je clarifier ma manière de travailler qui est celle de mixer les deux méthodes ?

\textbf{Hypothèse}:\\ 
\begin{enumerate}
	\item \textit{Un test d'écoute comme outil musicothérapeutique :}
\end{enumerate}

En partant de l'utilisation d'un test d'écoute spécifique employé
dans la thérapie  Tomatis, l'hypothèse suivante s'est imposée à moi : ce test d'écoute
pourrait-il servir aux musicothérapeutes? Y aurait-il moyen de donner
un champ de \emph{vision de l'écoute }du patient suivi exclusivement
en musicothérapie ? Nous avons ici exclu les traitements des musiques appliquées dans la méthode Tomatis, car nous devions nous limiter à un seul lieu et contexte- une clinique psychiatrique- où l'application de cette méthode n'a pas été possible pour ce travail.  Il y a eu aussi une limitation dans le temps, liée à mon 10%. Ainsi les changements et les fluctuations d'écoute relevées par les tests Il serait s'agit d'une comparaison des différentes utilisations du son.

Ce domaine est très vaste et la façon d'utiliser les sons et de les
traiter l'est aussi. L'objet de cette étude est de faire un constat
qui pourrait donner matière à réflexion tout en ne confrontant pas
les différentes techniques employées - qui pourraient avoir plus ou
moins d'impact quant à l'évolution du travail du patient.

Ma recherche est celle-ci : est-il pertinent d'utiliser ce test d'écoute
Tomatis en musicothérapie? Un support graphique, visible, presque
``palpable'', avec des critères d' interprétations, pourrait-il
donner une ``dessin'', une image utile, utilisable, tangible ? Permettrait-il
de visualiser plus objectivement les changements, la transformation
de l'écoute du patient et la démonstration de ce travail? un travail
qui mettrait à jour une perception différente, une sensibilité nouvelle?

Ne serait-ce pas là une démonstration tangible du travail de la musicothérapie
? La musique est aérienne. En comparaison avec l'art-thérapie, qui
peut nous apporter des supports graphiques, concrets du travail psychique
d'un patient, la musicothérapie pourrait être, d'une certaine façon,
visuellement aussi plus claire, et ce, en priorité dans l'esprit du
patient pour lui-même. 

Comme l'exprime à juste titre André Malraux : ``\emph{Le monde de
l'art n'est pas celui de l'immortalité , c'est celui de la métamorphose.''}
De même, la musique est un art produit par l'homme et qui a un impact
sur lui-même. Les deux interagissent, s'interpénètrent et s'auto-transforment
au cours des siècles. Ce que nous pourrons constater lors de l'aboutissement
d'une thérapie n'est pas de trouver une autre personne mais une transformation
de la perception de celle-ci par rapport au monde qui l'entoure. Selon
ce que nous vivons, nous nous transformons mais continuons à être
soi. Nous continuons à ``être soi'' mais autrement. Nous ne perdons
pas notre identité.

\paragraph{{\tiny{\normalsize  Dans la première partie, j'aborderai l'aspect théorique : le test}}
d'écoute, les différents sortes de tests d'écoute en musicothérapie,(
Verdeau-Paillès, Auriol). Ensuite, nous expliquerons la méthode Tomatis
et puis, beaucoup plus en détail, son test d'écoute. }

\paragraph{En deuxième partie : ce sera l'aspect clinique : les résultats ,
les études de cas, les tests faits en clinique avec deux groupes de
comparaison.}

\paragraph{Et finalement, vérification de l'hypothèse, conclusions et interrogations.}
