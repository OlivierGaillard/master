\chapter{Introduction et Hypothèse : }
\begin{quotation}
\emph{Par le Son, le Silence du Non-Être vient à l'Être. Je suis
la musique que je fais ou écoute. La musique a la capacité d'harmoniser
les composantes d'une entité psychophysique pour qu'il soit ``bien
dans sa peau'' et ``bien dans son âme''}\footnote{Jacques Viret, \emph{B. A-BA de la musicothérapie}, \cite{Viret2007}.}.
\end{quotation}
\section{Introduction}J'exerce le métier de musicienne professionnelle depuis de nombreuses
années. Par la suite, j'ai complété ma formation en devenant musicothérapeute et également consultante à la méthode Tomatis, études suscitées
par mon intérêt toujours plus croissant au sujet du développement de la personne
à travers l'outil que nous offre la musique : le son.

Tout en restant musicienne et pédagogue, selon les lieux de travail (cabinet privé, cliniques), les deux dernières formations ont pris des valeurs différentes dans la pratique et se sont souvent entremêlées. En clinique, la musicothérapie sera reine. En cabinet
privé, j'aurai la liberté de choix.
C'est ainsi qu'à travers chaque cas particulier,
j'ai vu mes techniques de soins se modifier, évoluer,
pour parfois se fusionner. 
Ce qui m'a animée ? ma curiosité! Se lancer dans l'écriture de ce Master, c'est creuser et approfondir mes recherches. C'est prendre de la distance, observer, se remettre en question. 
Comment y voir plus clair? Pourrai-je être plus précise dans mes observations? Comment 
clarifier et systématiser ma manière de travailler? En tant que thérapeute, on utilise tout ce que l'on a  à disposition, ce qui nous a constitué et construit, sans oublier l'intuition. Mais si  l'esprit de synthèse n'est pas suffisant, la balance  des ingrédients manque d'équilibre. 

 Le doute permet de rester en alerte, vigilant. Du reste, il est salvateur pour le patient!


\section{Hypothèse: un test d'écoute comme outil musicothérapeutique}


En partant de l'utilisation d'un test d'écoute spécifique employé dans
la thérapie Tomatis, l'hypothèse suivante s'est imposée à moi : ce
test d'écoute pourrait-il servir aux musicothérapeutes? Y aurait-il
moyen de donner un champ de \emph{vision de l'écoute} du patient suivi
exclusivement en musicothérapie ? Pourrait-il être un outil
intéressant ?


Ce domaine est très vaste et la façon d'utiliser les sons et de les
traiter l'est aussi. L'objet de cette étude est de faire un constat
qui pourrait donner matière à réflexion tout en ne confrontant pas les
différentes techniques employées --- qui pourraient avoir plus ou % un grand tiret ---
% un tiret entre no de pages: pp. 23--40
moins d'impact quant à l'évolution du travail du patient.

Ma recherche est celle-ci : est-il pertinent d'utiliser ce test
d'écoute Tomatis en musicothérapie sans utiliser les outils propres à
cette méthode ? Il n'y aura pas d'utilisation de l'Oreille
électronique ni des musiques préparées et filtrées.  Un support
graphique, visible, presque ``palpable'', avec des critères
d'interprétations, pourrait-il donner une ``dessin'', une image utile,
utilisable, tangible ? Permettrait-il de visualiser plus objectivement
les changements, la transformation de l'écoute du patient et la
démonstration d'un processus musicothérapeutique ? % quel travail?
une façon de faire émerger  % quel travail
une perception différente, une sensibilité
nouvelle ?

  Il nous semble que ce soit une
question importante car la musicothérapie reste globalement une forme de thérapie qui intrigue, laisse sceptique, voire  interrogateur et parfois peut être mal comprise car
en définitive mal connue. % pourquoi par définition?
La crédibilité d'une thérapie
passe souvent par des preuves. Mais est-ce que la musique, invisible
et légère, qui constitue le médium principal utilisé en musicothérapie, laisse
des traces extérieures, dans le sens de résultats et d'effets mesurables, quantifiables ?
% Une psychothérapie aussi ne laisse pas de traces physiques. Mais on peut
% en mesurer les effets.
\paragraph{La musique} est utilisée dans le but d'un processus
thérapeutique, pour entrer en communication avec soi-même et pouvoir ensuite mieux percevoir le monde qui nous
entoure, communiquer et s'exprimer\footnote{%
Voir ce lien sur \href{http://www.musictherapy.ch/fr/musicotherapie/quest-ce-que-la-musicotherapie/}{musictherapy.ch}}.
L'aspect intemporel du son, de la musique, de ce médium volatil par
définition, ne pourra apporter, comme en art-thérapie, le
même aspect concret que peuvent témoigner des supports graphiques,
visuels, reflets d'un espace-temps lors d'un travail d'élaboration
psychique d'un patient. Nous gageons et faisons l'hypothèse que l'action et l'impact de la
musicothérapie pourraient être perçus, d'une certaine façon plus
clairement, saisis comme dans l'\oe il de l'objectif d'un appareil
photographique,\textsl{ objectivés},vus et constatés, et ce, en priorité dans l'esprit du
patient pour et par lui-même.

En tant que musicothérapeute, nous ne pouvons pas  nous servir des outils
scientifiques tel que l'IRMfct; par contre, toutes ces recherches en
neurosciences appuient et renforcent la crédibilité de l'action
majeure du son sur notre cerveau.  Emmanuel Bigand, professeur de
psychologie cognitive à l'Université de Bourgogne, relève l'aspect
paradoxal de la musique, la complexité de sa structure sonore sans
fonction biologique précise mais faisant réagir fortement l'être
humain.\footnote{\cite{AuteurInconnu2011}, chap.~3 p.~35, "Vous avez l'oreille musicale".}.  Notre cerveau peut être activé autant par
la musique que par la nourriture ou la drogue. Et pourtant la musique,
qui est un élément artificiel en soi, n'a aucun rôle dans notre survie ni dans
notre nutrition.


\section{Plan du travail}

Dans la première partie, j'aborderai l'aspect théorique : l'écoute, le test d'écoute, les différents sortes de tests d'écoute en musicothérapie. Nous aborderons aussi brièvement l'anatomie de l'oreille. Ensuite, nous expliquerons la méthode Tomatis
et puis, beaucoup plus en détail, son test d'écoute.

En deuxième partie ce sera l'aspect clinique : les tests faits en clinique avec deux groupes de
comparaison.

Et finalement suivront la vérification de l'hypothèse, les conclusions et interrogations.
