\chapter{L'écoute :}

\section{Ecouter ou entendre : une différence}

Nous commencerons par la définition du verbe ``entendre'' et du
verbe ``écouter'', faite par le dictionnaire Hachette, édition 2012.
Ceci nous paraît opportun par le fait qu'il y a souvent confusion,
mélange et utilisation des deux termes de manière indistincte : 
\begin{enumerate}
\item \emph{Entendre }: percevoir des sons, saisir par l'ouïe
\item \emph{Ecouter} : a).prêter l'oreille pour entendre b).prêter attention
à l'avis de quelqu'un, suivre un avis.c). fig suivre une impulsion,
une inspiration.
\end{enumerate}
La première action est en soi, passive, involontaire, non sélective. 

Tandis que la deuxième est active, implique la volonté, permet une
forme de décodage : Il s'agit d'un acte, d'une action, d'une capacité.\emph{
}Lorsque nous lisons attentivement, nous faisons abstraction des bruits
environnants, nous les entendons parfaitement mais nous n'y prêtons
pas attention. Nous parvenons à couper les sons parasites pour nous
concentrer que sur les sons pertinents.

\emph{Entendre} est une attitude passive par rapport au monde sonore
qui nous entoure. Nous recevons les sons sans les interpréter et cela
ne demande aucun effort.

Tandis qu'\emph{écouter} est une opération de tout autre nature puisqu'elle
suppose une participation active dans le choix du message ou dans
la sélection d'une voix.

Bernard Auriol, cité plus bas (note 6) ouvre son livre ``La Clef
des sons'' par cet en-tête . ``L'écoute est une action'' et par
ces deux phrases : ``Entendre suppose un son (physique), une oreille
pour le capter, un système nerveux pour le recevoir. Ecouter est un
processus actif supposant préférences et répulsions pour tel son ou
telle séquence sonore.''

Entendre et écouter sont donc des processus bien différents, \footnote{Extrait de l'entretien réalisé par Bernard Auriol avec Alfred Tomatis,1973 }``deux
fonctions essentiellement distinctes bien qu'évoluant apparemment
sur des terrains identiques.'' Tomatis met l'accent sur `` l'élément
conscient, facteur essentiel sur lequel repose toute la différence
entre ces deux activités''.

Pour exemple concret, nous savons qu' un enfant autiste peut entendre
parfaitement bien, souffrant même d'hypersensibilité aux sons mais
qui malgré tout n'écoute pas.\emph{ }

\emph{``Ecouter} se base certes sur une stimulation prenant sa source
à I'extérieur mais \emph{devant êre intérieurement, intentionnellement
recherché}e.'' 

Tomatis fait également un parallèle très clair avec ``voir et vouloir
voir''. Ce sont deux mécanismes totalement différents, le second
utilisant le premier. \emph{``Vouloir voir, c'est viser. }Il en est
de même pour entendre et écouter.\emph{ L\textquoteright écoute résulte
du vouloir entendre }et est l\textquoteright équivalent de la visée.\emph{
``}

Et puis, lorsque l'on considère au sens figuré c) la définition d'écouter
: suivre une impulsion, une inspiration, ne correspondrait-elle pas
au but ultime de la démarche d'ouverture imbriquée dans la notion
même de l'écoute ? Ne serait-ce pas aussi, dans cette démarche, celle
de s'ouvrir à une autre forme d'écoute, donnée par l'inspiration ou
induite par une impulsion? 

D'autre part, il est fort intéressant de se poser la question de l'objectivité
ou de la subjectivité de notre écoute. Nous avons tous anatomiquement
parlant, à peu de choses près, la même oreille. Logiquement, nous
devrions entendre et écouter la même chose lors d'une même information
diffusée. Pourtant, cela ne semble pas se passer ainsi. Chacun n'entend
pas de la même manière les mêmes informations, en somme, tout un chacun
entend ce qu'il veut bien entendre. Nous approfondirons plus loin
le chapitre lié à la méthode Tomatis et du rôle important que semble
jouer notre cerveau à notre écoute. 

Ainsi l'écoute est une fonction exceptionnelle, innée en l'homme mais
qui semble pas si simple car elle demeure souvent enfouie et occultée.

\section{Le test d'écoute :}

Qu'est-ce qu'un test d'écoute ?

Dans le milieu médical, il s'agit d'un test, nommé audiogramme qui
sert à mesurer les seuils d'audition des sujets, grâce à l'audiomètre,
appareil français qui avait été mis au point en 1933. Les Américains
ont repris ces travaux pendant la dernière guerre pour pouvoir dépister
les dommages subis par ceux qui conduisaient des engins bruyants comme
des avions.

C'est une épreuve d'ordre physiologique, voire anatomique.\emph{ }Elle
est pour \marginpar{otologie : branche de la médecine qui étudie l'oreille et ses maladies}l\textquoteright otologiste
un examen fondamental à partir duquel se dessinent les données \footnote{étiologie : étude des causes d'une maladie}étiologiques
d'un trouble de la fonction auditive. D'elle dépend, en outre, le
pronostic qui va orienter le mode de thérapie médicale ou chirurgicale,
ou bien encore prothétique, voire rééducative. 

Existe-t-il d'autres formes de tests d'écoute?

Dans nos recherches sur internet ( décembre 2016) , il nous est proposé
de tests d'écoute en majorité présentés sous une forme verbale tels
que par exemple celui que Roger Lanteri ou de Bruno Daigle, qui mettent
l'accent sur la communication ou une capacité d'empathie. Ils sont
donc de nature purement psychologique. Cela va de pair avec la définition
même du verbe écouter citée plus haut. Mais ces tests sont des tests
d'écoute sans élément sonore à déceler. Est.-ce que l'élément sonore
apporterait davantage d'informations sur le patient ?Pour quelles
raisons la matière sonore ne pourrait-elle pas directement servir
d'élément d'évaluation du sujet pour les musicothérapeutes ? 

Nous avons étendu nos recherches de tests d'écoute qui seraient actuellement
utilisés dans cet optique.

Avec la psychanalyste Jacqueline Verdeau-Paillès qui a étudié et intégré
la psychanalyse avec le son, le sonore est introduit sous forme réceptive,
avec un test d'audition d'oeuvres. Elle est définie comme une technique
complémentaire de musicothérapie qui permet d\textquoteright entreprendre
une relation analytique. 
\begin{itemize}
\item Test de Verdeau-Paillès :\footnote{\emph{``Le bilan psycho-musical et la personnalité}''Dr.Jacqueline
Verdeau-Paillès.Ed.Fuzeau} Celui-ci permet d'étudier la place qu'occupent la musique et le sonore
dans la vie du patient : le type de réceptivité à la musique et les
possibilités de communication à l'aide de la musique et des sons.
C'est une démarche en trois volets qui développe un entretien, un
test d'audition d'oeuvres et un texte actif pour répondre à trois
objectifs : une meilleure connaissance de soi, l'aide à l'établissement
d'un projet thérapeutique et le travail psychopédagogique. Les recherches
de Benenzon\footnote{Rolando Omar Benenzon,'' \emph{La musicothérapie, la part oubliée
de la personnalité''}} ont été reprises par Verdeau-Paillès pour l'élaboration de ce test.
En un peu plus de détails, voici ce qu'il en est :
\end{itemize}
\begin{quote}
``La technique du montage en U : Comparable aux effets de la sophrologie
et de la relaxation en général, cette technique est surtout utilisée
dans le traitement de la douleur, de l\textquoteright anxiété et de
la dépression. Il est généralement recommandé de procéder à des séances
de durée variable de 20 à 45 minutes et décomposées en plusieurs phases
de 5 à 6 morceaux. Ces morceaux de 3 à 4 minutes chacun, fondus et
enchaînés, amènent progressivement le patient à la détente. L\textquoteright implication
et la coopération du patient sont primordiales. La détente, le détournement
de l\textquoteright attention, la relaxation profonde, et la qualité
de la relation patient-soignant sont des facteurs certains d\textquoteright amélioration.(...).\footnote{(Source : ASSOCIATION AMARC, Association de musicothérapie, recherches
cliniques et applications).}''
\end{quote}
Cette technique analytique, de type individuel, est considéré, selon
cet article, comme complémentaire à la musicothérapie par le pouvoir
même de la musique, déclencheuse d' émotions. Elle est de type individuel
dans le sens qu'elle se base sur un entretien-questionnaire à la première
séance. Lors de l'audition des musiques choisies par le thérapeute
et/ou par le patient, le patient va en verbaliser le vécu et le musicothérapeute
va recevoir et analyser ce qui en émerge. On peut considérer qu'il
s'agit d'une relation tripolaire patient-thérapeute-musique. Elle
favorise''\emph{ l\textquoteright expression et le développement
de la pensée}''et (...) va ``permettre la \emph{prise de conscience
des processus pathologiques développés.'' }

De même avec Fern Nevjinsky, le diagnostic se fera avec des morceaux
de musique et ce, particulièrement en association libre : 
\begin{itemize}
\item Le test de Rorschach(visuel) associé au test musical, de Fern Jevjinsky:
\footnote{``\emph{Adolescence, musique, Rorschach}'' de Fern Nevjinsky, publication
de l'Université de Rouen n\textdegree 215. } La consigne évolue de l'identification de sons à l'association libre
car l'auteur remarque que (...) ``la portée diagnostique du test
fait avec des sons purs, en se limitant à l'identification, est insuffisante
; mais, si la consigne est libre -dire ce que le son signifie- toutes
les perceptions erronnées sont le point de départ d'une expression
fantasmatique en relation avec le passé du sujet, ses souvenirs. En
définitive, cette technique nous permet de nous interroger sur la
valeur priviligiée du son comme éveil des affects liés à des conflits
qui n'apparaissent pas dans l'entretien ou dans les tests visuels.(...
)A un niveau psychanalytique, par le biais de la régression, elle
peut amener le sujet à abandonner une partie de sa vigilance défensive.'' 
\end{itemize}
Cette technique est donc un ``plus'', un élément universel et non
anxiogène, utilisée en test. Fern Nevjinsky utilise ainsi le test
musical en complémentarité de celui de Rorschach.

En définitive, nous revenons donc, avec d'autres façons d'intervenir,
à ce qui a été déjà formulé plus haut dans la technique de Verdeau-Paillès,
à savoir, la musique est un outil déclencheur des expressions et provoque
l'éveil des affects avec leur verbalisation.

Bernard Auriol\footnote{Médecin psychiatre, psychothérapeute, né en 1938, a écrit plusieurs
ouvrages, dont : \emph{Le son au subjectif présent}, Ed. du Non Verbal
(AMBx), 1996, ISBN 978-2906274198\emph{ La clef des sons} {[}archive{]},
2e édition. \emph{Éléments de psychosonique,} Erès, coll. « Études
sociales », 1996, ISBN 978-2865861798, traduction en russe. \emph{Méditation
et psychothérapie} {[}archive{]}, Jean-Marc Mantel, Brigitte Kashtan,
Jacques Castermane, Bernard Auriol, Albin Michel, coll. « Espaces
libres », 2006, ISBN 978-2226149244\emph{ méthode TOMATIS :} }a étendu ses recherches sur le son, la psychosonie, tout en s'inspirant
des travaux d'Alfred Tomatis, avec lequel il s'est également formé.
\begin{itemize}
\item Le terme psychosonique a été créé en 1991 par Bernard Auriol pour
désigner la discipline qui cherche à évaluer et décrire les effets
du son sur l'être vivant (spécialement humain) ainsi que les éléments
subjectifs manifestés par l'expression sonore, en particulier la voix.
Il convient de distinguer la psychosonique de la psychoacoustique
qui se situe davantage du côté de la psychophysique que d'une approche
psychodynamique. La psychoacoustique se préoccupe des conditions acoustiques
et neuro-psycho-physiologiques de l'audition, alors que la psychosonique
tente d'étendre le point de vue aux éléments symboliques, psychodynamiques,
inconscients et subjectifs du processus d'écoute. Bernard Auriol a
mis en place un test des chakras.
\end{itemize}
La psychosonique est, selon cette source internet, très proche de
la musicothérapie, avec une acceptation plus large. 
\begin{itemize}
\item Le test d'écoute Tomatis : Ce test est un outil de diagnostic élaboré
par lui-même. Il est aussi psychologique mais basé sur une chaîne
de sons précis et pareils pour chaque patient. Ils doivent être identifiés
par chaque sujet avec un protocole très clair et des consignes précises
. Il semble s'apparenter à un audiogramme dont le but est de déceler
un trouble de l'audition et son origine. Mais ce n'est pas pareil.
La technique de passation du test ne se déroule pas de la même manière
et, de plus en se fondant sur ces résultats, il est possible de savoir
si le patient désire ou non se servir des sons qu'il a à disposition.
C'est en cela que se situe la différence existant entre un audiogramme
et le test Tomatis. Le patient a peut-être la possibilité d'entendre
un large spectre de sons mais ne souhaite pas, pour diverses raisons
( traumatismes, expériences négatives), forcément les écouter. Son
cerveau aura le pouvoir de les assourdir, puis de les faire disparaître
peu à peu de son champ d'écoute. En résumé, par protection, il choisit
de les annihiler alors que l'oreille peut les collecter. Le cerveau
crée ce que l'on appelle des distortions d'écoute. \footnote{Professeur Tomatis \char`\"{}Education et Dyslexie\char`\"{} Editions
ESF Collection \char`\"{}Sciences de l'éducation\char`\"{}.}
\end{itemize}
\begin{quote}
\emph{``Le test d'écoute sait intégrer ces renseignements dans le
cadre d'un processus psychologique qui va permettre de déceler si
le sujet désire ou non se servir des matériaux qu'il a à sa disposition
sur le plan perceptif. (...)Il est avant tout un test psychologique
et} les données psychologiques vont permettre d'établir un\emph{ diagnostic}
et d'orienter un mode d'action.''
\end{quote}
Est-ce que cette théorie tient la route ? est-ce vraiment possible
? N'est-ce pas un peu paradoxal ? on peut entendre des sons que l'on
n'entend pas!ou que l'on n'écoute pas! Nous allons tenter d'en savoir
d'avantage. Notre cerveau comporte encore beaucoup de mystère et le
domaine des neurosciences est un vaste monde à découvrir. 

Il est à relever que cette façon de procéder et d'aborder un test
d'écoute est originale en comparaison avec les différents tests cités
plus haut.

C'est cette forme d'objectivité \footnote{L'objectivité et la subjectivité ne sont pas des notions simples lorsque
l'on parle du son.}dans le test - la mise en évidence des seuils d'écoute- et en même
temps cette possibilité -d'analyser le potentiel d'écoute de chaque
patient- qui nous ont incité à vouloir en savoir davantage et de porter
notre choix dans l'approfondissement de ce sujet. Les réponses des
patients donnent des indices, des renseignements sur eux-même, et
ce de manière involontaire, sans qu'ils puissent les contrôler d'une
quelconque façon ou les influencer intellectuellement. Ces réponses
sont immédiates et résultent d'une adéquation très simple qui est
celle de signaler un son dès qu'ils l'entendent. Ce sont ces éléments
que nous avons rassemblés lors de notre recherche. Nous y reviendrons
en détails dans le chapitre 3 ; mais voyons dans l'immédiat les caractéristiques
du son, données indispensables pour l'analyse d'un test d'écoute.
\begin{quote}
\footnote{``Considérations sur le test d'écoute. Propos recueillis au cours
du IIIème congrès international d'audio-psycho-phonologie ( Anvers
1973) à la suite d'un entretien avec le professeur Tomatis .Dans son
ouvrage \char`\"{}Éducation et Dyslexie\char`\"{}, le Professeur Tomatis
a présenté le test d'écoute comme étant le test le plus important
du bilan Audio-Psycho-Phonologique et comme devant déterminer les
possibilités d\textquoteright écoute du sujet : auto-écoute et écoute
de l'autre.}
\end{quote}

\section{Le son :}

Le son possède plusieurs caractéristiques physiques. Il peut être
défini très précisément par un ensemble d'unités physiques chiffrées
: les décibels et les hertz. 
\begin{itemize}
\item Un décibel est l'unité de mesure de l'intensité du son. Un décibel
est égal à 1/10 de bel ; une augmentation de l'intensité égale à 1
bel équivaut à peu près à un doublement de l'intensité sonore. 
\item Un hertz est une unité de fréquence\footnote{la fréquence est le nombre de vibrations par unité de temps dans un
phénomène périodique} (symbole : Hz). Équivalent à 1 s-1. Fréquence d'un phénomène périodique
dont la période est une seconde. Ses multiples sont, entre autres,
le kilohertz (kHz), le mégahertz (MHz) et le gigahertz (Ghz). Cette
unité vient du savant allemand Heinrich Hertz, pionnier de la radioélectricité.
\end{itemize}
Le son peut être défini de deux manières : 
\begin{itemize}
\item d'une manière objective tout d'abord : c'est le phénomène physique
d'origine mécanique consistant en une variation de pression (très
faible), de vitesse vibratoire ou de densité du fluide, qui se propage
en modifiant progressivement l'état de chaque élément du milieu considéré,
donnant ainsi naissance à une onde acoustique (la propagation des
ronds dans l'eau suite à un ébranlement de la surface donne une bonne
représentation de ce phénomène) ; 
\item d'une manière subjective également : il s'agit de la sensation procurée
par cette onde, qui est reçue par l'oreille, puis transmise au cerveau
et déchiffrée par celui-ci.\index{http://www.futura-sciences.com/sante/dossiers/medecine-bruit-effets-sante-259/page/3}
\end{itemize}
De plus, il y a de nombreux paramètres en prendre en compte : par
exemple : l'impression de force sonore : la sensibilité de l'oreille
est une variable de la fréquence. Il faut 1000 fois moins de pression
acoustique pour avoir une sensation auditive à 4000 hertz qu'à 50
hertz. Notre oreille n'a donc pas la même sensibilité pour toutes
les fréquences audibles. Il en est de même pour la sensation auditive
des basses fréquences et pour la dynamique. 

Nous voyons bien qu'il s'agit d'un domaine passionnant mais très complexe.
\begin{quote}
``(...) Entendre n'implique pas pour autant la présence d'un champ
conscient.\emph{ Entendre, c\textquoteright est en quelque sorte subir
un son }ou un message qui nous est adressé. \emph{Ecouter, c'est désirer
appréhender ce son }ou ce message . (...)'' 

\footnote{Professeur Tomatis \char`\"{}Education et Dyslexie\char`\"{} Editions
ESF Collection \char`\"{}Sciences de l'éducation\char`\"{}.}
\end{quote}
