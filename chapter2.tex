\chapter{L'écoute}

\section{Ecouter ou entendre : une différence}

La définition du verbe ``entendre'' et du verbe ``écouter'', faite par
le dictionnaire Hachette, édition 2012 nous paraît opportun par le % opportune?
fait qu'il y a souvent confusion, mélange des deux termes :
\begin{description}
\item[Entendre] c'est  percevoir des sons, saisir par l'ouïe
\item[Ecouter] a trois sens: 
\begin{enumerate}
	\item prêter l'oreille pour entendre
	\item prêter attention
	à l'avis de quelqu'un, suivre un avis
	\item \emph{fig}: suivre une impulsion,
	une inspiration.
\end{enumerate}
\end{description}

La première action, entendre, est en soi, passive, involontaire et non
sélective.

Tandis que la deuxième, écouter, est active, implique la volonté,
permet une forme de décodage: il s'agit d'un acte, d'une action,
d'une capacité. Dans un milieu sonore important, dans un café par
exemple, lorsque nous lisons attentivement, nous faisons abstraction
des bruits environnants, nous les entendons parfaitement mais nous n'y
prêtons pas attention. Nous parvenons à couper les sons parasites pour
nous concentrer que sur les sons pertinents.

\emph{Entendre} est une attitude passive par rapport au monde sonore
qui nous entoure. Nous recevons les sons sans les interpréter et cela
ne demande aucun effort.

Tandis qu'\emph{écouter} est une opération de tout autre nature
puisqu'elle suppose une participation active dans le choix du message
ou dans la sélection d'une voix.

Bernard Auriol ouvre son livre \emph{La Clef
des sons} par cet en-tête :
\begin{quote}
\og L'écoute est une action». % touche 'alt gr' + y produit «  et 'alt gr' + x produit »
\end{quote} % pour des phrases courtes on met les guillemets. 
et par ces deux phrases :
\begin{quotation} % mais avec plus de 2 phrases cela suffit
	% de mettre le texte en retrait
	% (ce que fait l'environnement quotation)
  Entendre suppose un son (physique), une oreille
pour le capter, un système nerveux pour le recevoir. Ecouter est un
processus actif supposant préférences et répulsions pour tel son ou
telle séquence sonore\footnote{\cite{Auri96:clesons}.}.
\end{quotation}
 
Entendre et écouter sont «deux
fonctions essentiellement distinctes bien qu'évoluant apparemment sur
des terrains identiques\footnote{Extrait de l'entretien réalisé par
  Bernard Auriol avec Alfred Tomatis, 1973.}.» ... avec «l'élément conscient, facteur essentiel sur lequel repose toute la
différence entre ces deux activités».



\emph{Ecouter} se base certes sur une stimulation prenant sa source à
l'extérieur mais \emph{devant être intérieurement, intentionnellement
  recherchée}.





% je réécrirais le paragraphe précédent ainsi:
Car \emph{Vouloir voir, c'est viser.}  Vouloir entendre dans le but d'écouter est comparable  à
la visée de l'oeil lorsque l'on veut collecter une information.  
En définitive,\textbf{ l'audition est la capacité perceptive du système auditif et l'écoute, c'est ce qu'on en fait.}
% mais je ne parviens pas à reformuler la dernière phrase:
% la démarche = écouter-viser, je suppose ou plus général?
% ... s'ouvrir à une écoute guidée par l'inspiration, induite par une impulsion?


 % mini-correction de français
\subparagraph{L'objectivité de l'écoute}.
Il est plausible de se poser la question de
l'objectivité ou de la subjectivité de notre écoute. Nous avons tous
anatomiquement parlant, à peu de choses près, la même
oreille. Logiquement, nous devrions entendre et écouter la même chose
lors d'une même information diffusée. Pourtant, cela ne semble pas se
passer ainsi. Chacun n'entend pas de la même manière les mêmes
informations, en somme, tout un chacun entend ce qu'il veut bien
entendre. 
Notons qu' il y a des cas particuliers, comme dans l'autisme. Un enfant autiste peut avoir la capacité d'entendre 
parfaitement bien, même excessivement bien en souffrant d'hypersensibilité aux sons, mais il n'écoute pas car il n'a pas la faculté de le faire; en définitive il est sans moyen de trier les informations qui lui arrivent dans l'oreille. 
Nous approfondirons plus loin du rôle important que semble jouer notre cerveau à notre
écoute.



\section{Le test d'écoute} % les ':' sont inutiles ou: Qu'est-ce qu'un
                           % test d'écoute? dans  le titre de la section.

Qu'est-ce qu'un test d'écoute ?

Dans le milieu médical, un test d'écoute objectif se nomme audiogramme et
sert à mesurer les seuils d'audition des sujets, grâce à l'audiomètre. Cet 
appareil français avait été mis au point en 1933. Les Américains
ont repris ces travaux pendant la dernière guerre pour pouvoir dépister
les dommages subis par ceux qui conduisaient des engins bruyants comme
des avions.

L'audiogramme est une épreuve d'ordre physiologique, anatomique qui
est pratiquée en otologie\footnote{\emph{otologie}: branche de la médecine
  qui étudie l'oreille et ses maladies, « examen à partir duquel se
  dessinent les données étiologiques.» étiologie : étude des causes
  d'une maladie.} pour détecter un trouble de la fonction auditive. Un pronostic pourra définir le mode de thérapie
médicale, chirurgicale, prothétique ou rééducative.


Existe-t-il d'autres formes de tests d'écoute?

Lors de  nos recherches, la majorité proposée était sous forme verbale
( Roger Lanteri, Bruno Daigle)qui
mettent l'accent sur la communication ou une capacité d'empathie. Ils
sont donc de nature  psychologique sans lien avec un élément sonore à déceler.

% j'aère en créant un paragraphe.
Est-ce que cet élément sonore apporterait davantage d'informations sur
le patient? La matière sonore apporte, il va se soi, aux musicothérapeutes des éléments
 d'évaluation du sujet. C'est une évidence dans le sens subjectif du terme mais non objectif. Comment extraire des informations dites objectives de la personne par un test sonore ?
\subsection{La musicothérapie et la psychothérapie}
	Les musicothérapeutes sont  issus soit du domaine musical professionnel  ou/et soit de  la psychologie et de la psychiatrie. Certains, tels Rolando Benenzon, Edith Lecourt,  ont fait fusionner les deux dans leur pratique  en utilisant le son comme élément facilitant l'exploration psychique. Ils ont élaboré des techniques, des façons de procéder, en soulignant l'importance d'un tel support de communication ou d'introspection. \paragraph{Rolando Benenzon} Le professeur et docteur Rolando Omar Benenzon structura à partir de 1969 un modèle de musicothérapie en se basant sur Freud, Jung, Winnicott, Watzlawick, influencé par le concept de l'objet sonore notamment avec P.Schaeffer et C.Sachs ainsi que par les grands pédagogues musicaux comme Willems, Dalcroze ou Kodaly. Sa définition de la musicothérapie est celle d'une musico-psychothérapie  \emph{\textsl{qui utilise les expressions corporo-sonoro-non verbales.}}, centrée sur le concept d'identité sonore.
	 \paragraph*{ Edith Lecourt, }( Docteur ès lettres et sciences humaines, psychanalyste et musicienne à l'Université René Descarte-- Paris V), et lui-même, tout en recherchant la place qu'occupe le sonore dans la vie d'un patient, ont perçu l'idée générale et conductrice de \textsl{la méthode projective}, en terme 
	d'" investigation dynamique et "holistique" de la personnalité". {Les  tests projectifs } sont devenus à partir de 1939 un des instruments très utilisés en psychologie clinique. Ils  réunissaient trois épreuves : le test d'association de mots de Jung (1904), le test des taches d'encre de Rorschach (1920) et le "TAT"( test d'histoires à inventer) de Murray (1935). @book{ (Catherine Chabert et Didier Anzieu (" Les méthodes projectives" Chap.1,p.13,Ed.Puf, Collection Quadrige))
,
		{Catherine Chabert, Didier Anzieu},
		{Les méthodes projectives},
		{mars 2004},
		{Puff, Collection Quadrige},
		
	
	}
	
Inspirés par ces divers courants,Helen Bonny, Jacqueline Verdeau-Paillès et Fern Nevjinsky ont  mis au point au fil de leur pratique des modèles et des tests spécifiques. 


\paragraph{Helen Bonny } ( USA) était une musicothérapeute,
musicienne et psychothérapeute, qui a mis au point dans les années 70
une technique particulière, le GIM, \og Guided Imagery and Music\fg
l'imagerie guidée et de la musique. Selon GIM
Trainings\footnote{Site visité le 2.1.18 / \href{\#gimsite}{GIM}.}, la
musique associée à la thérapie libère par l'émotion en reliant le
conscient à l'inconscient.(The Evolution of Guided Imagery and Music, by Helen Bonny, Ed. by Lisa Summer(2002), p.7)
 C'est une forme réceptive de travail
en musicothérapie, avec comme principales influences Carl Rogers, Abraham Maslow et Carl Jung,  qui consiste en une longue anamnèse avec le
patient et  qui permettra de cibler le programme de musiques appropriées. 
(des \oe uvres de compositeurs tels Beethoven, Brahms, Debussy,
Mozart, Rachmaninov ou Vivaldi. )
Il n'y a  pas de
tests d'écoute, de notes ou de sons spécifiques à proprement parlé à déterminer ou à localiser.

% on écrit \oe uvre pour avoir un o lié avec le e.
\paragraph{Jacqueline Verdeau-Paillès}De même, la psychanalyste Jacqueline Verdeau-Paillès a étudié et
intégré en 1985 la psychanalyse avec le son.  Le sonore est  introduit
sous forme réceptive avec un test d'audition d'oeuvres pour réaliser
une relation analytique\footnote{\emph{``Le bilan psycho-musical et la
    personnalité}'' Dr. Jacqueline Verdeau-Paillès,
  \cite{verdeau-pailles}.}.

% pourquoi une énumération itemize?  je la commente.
%\begin{itemize}
Quelle est la place qu'occupe la musique et le sonore dans la vie d'un patient ? Son test avec un entretien, un test d'audition d'\oe uvres et un texte actif permet d'évaluer la réceptivité et les possibilités de communication par ce médium, ce qui va permettre d'établir un projet thérapeutique.

Les recherches de Benenzon\footnote{Rolando Omar
  Benenzon, \emph{La musicothérapie, la part oubliée de la
    personnalité}, \cite{Benenzon2007}.} ont été reprises par
Verdeau-Paillès pour l'élaboration de ce test. Il consiste en la technique du montage en U qui débute avec 5 à 6 morceaux de 3 à 4 minutes chacun en fondus enchaînés, amenant progressivement le patient à la détente;   l'entretien-questionnaire à la première
séance;  et lors de l'audition des musiques choisies par le thérapeute
et/ou par le patient, la verbalisation du vécu. Le musicothérapeute
va recevoir et analyser ce qui en émerge. 
La musique favorise  ''\emph{l'expression et le développement
	de la pensée}'' et (\ldots) va ``permettre la \emph{prise de conscience
	des processus pathologiques développés.}''
  (\ldots)\footnote{Source : ASSOCIATION AMARC,
  Association de musicothérapie, recherches cliniques et
  applications). \typeout{TODO: rajouter les references de l'article AMARC}}.


 \paragraph{Fern Nevjinsky et le test de Rorschach}
 De son côté, Fern Nevjinsky a développé à partir du  test de Rorschach un test psycho-musical avec des morceaux
de musique en association libre. Il utilise ainsi le test
musical en complémentarité de celui de Rorschach.

% pas la peine de commencer une liste (itemize) si il n'y a qu'un élément
 
\footnote{Fern Nevjinsky,maître de conférences à l'Université de Rouen, musicien, psycho-analyste ``\emph{Adolescence, musique, Rorschach}'' ,Comparaison des modalités de projection et d'expression au test de Rorschach et à un test psycho-musical pour des adolescents de 13 à 16 ans,
  \cite{Nevjinsky1996}.} Il nous dit  que  «[\ldots] la
portée diagnostique du test fait avec des sons purs, en se limitant à
l'identification, est insuffisante; mais, si la consigne est libre ---
dire ce que le son signifie --- toutes les perceptions erronnées sont
le point de départ d'une expression fantasmatique en relation avec le
passé du sujet, ses souvenirs. (...) Il prouve  la valeur privilégiée du son comme éveil
des affects liés à des conflits qui n'apparaissent pas dans
l'entretien ou dans les tests visuels.  [\ldots] A un niveau
psychanalytique, par le biais de la régression, elle peut amener le sujet à abandonner une partie de sa vigilance défensive.»

En définitive, nous revenons donc, avec d'autres façons d'intervenir,
à ce qui a été déjà formulé plus haut dans la technique de Verdeau-Paillès,
à savoir : la musique est un outil non-anxiogène, déclencheur des expressions qui provoque
l'éveil des affects dans  leur verbalisation.


\subparagraph{Bernard Auriol}\footnote{Médecin psychiatre, psychothérapeute, né en
  1938, a écrit plusieurs ouvrages, dont : \emph{Le son au subjectif
    présent}, \cite{auriol96sonausubjectifpresent}.
%Ed. du Non Verbal
%(AMBx), 1996, ISBN 978-2906274198
\emph{ La clef des sons} \cite{Auri96:clesons}, \emph{Éléments de
  psychosonique} Erès, coll. « Études sociales », 1996, ISBN
978-2865861798, traduction en russe. \emph{Méditation et
  psychothérapie} {[}archive{]}, Jean-Marc Mantel, Brigitte Kashtan,
Jacques Castermane, Bernard Auriol, Albin Michel, coll. « Espaces
libres », 2006, ISBN 978-2226149244\emph{ méthode TOMATIS :} }a étendu
ses recherches sur le son, la psychosonie, tout en s'inspirant des
travaux d'Alfred Tomatis, avec lequel il s'est également formé.

Le terme psychosonique a été créé en 1991 par Bernard Auriol pour
désigner la discipline qui cherche à évaluer et décrire les effets du
son sur l'être vivant, l'homme, ainsi que les éléments
subjectifs manifestés par l'expression sonore, en particulier la voix.
Il convient de distinguer la psychosonique de la psychoacoustique qui
se situe davantage du côté de la psychophysique que d'une approche
psychodynamique. La psychoacoustique se préoccupe des conditions
acoustiques et neuro-psycho-physiologiques de l'audition, alors que la
psychosonique tente d'étendre le point de vue aux éléments
symboliques, psychodynamiques, inconscients et subjectifs du processus
d'écoute ;  en ce sens, elle est très proche de la musicothérapie.
  
\paragraph{Alfred Tomatis et le test d'écoute}
  Alfred Tomatis a créé un test d'écoute; c'est un outil qui permet d'objectiver la qualité de l'écoute.
  Le test est  basé sur une chaîne régulière de sons précis dans un même ordre à identifier, des sons purs dont on varie le volume (de très faible à fort). Dans son ouvrage \emph{Éducation et Dyslexie},le professeur Tomatis
  a présenté le test d'écoute comme étant le test le plus important du
  bilan, dénommé audio-psycho-phonologique et devant déterminer les
  possibilités d'écoute du sujet : auto-écoute et écoute de
  l'autre\footnote{``Considérations sur le test d'écoute. Propos
  	recueillis au cours du IIIème congrès international
  	d'audio-psycho-phonologie ( Anvers 1973) à la suite d'un entretien
  	avec le professeur Tomatis.}. Ce n'est pas un audiogramme dont
  le but serait de déceler l'origine d'un trouble de l'audition. Notons quelques différences se trouvant dans la façon de faire passer le test;  les hauteurs de sons à détecter; l'emploi du masking (un bruit dans l'oreille opposée
  au son à reconnaître); une série de
  mots
  à répéter, d'intensités différentes, de moyen,
  fort, faible à très faible. ( 30dB). 
  
  Une des grandes différences 
  existantes  est  qu'il est
  possible de savoir, selon Tomatis, si le patient désire ou non se servir des sons
  qu'il a à sa disposition. Il a peut-être la possibilité d'entendre un large spectre de
  sons mais ne souhaite pas, ne veut pas les écouter. Les raisons sont multiples et en général d'ordre psychologique ( traumatismes,
  expériences négatives). Le cerveau aura le
  pouvoir d' assourdir certaines fréquences, de les masquer puis de les faire disparaître peu à peu de
  son champ d'écoute. Par protection, par réflexe de survie, il choisit de les
  annihiler alors que les sons sont là, réels, et que  l'oreille peut physiquement les collecter. Le cerveau crée ce
  que l'on appelle des distorsions d'écoute. \footnote{Professeur
    Tomatis \emph{Education et Dyslexie},  Editions ESF
    Collection Sciences de l'éducation.}. % TODO: ajouter en bibliographie}

\begin{quote}
\emph{``Le test d'écoute sait intégrer ces renseignements dans le
cadre d'un processus psychologique qui va permettre de déceler si
le sujet\textbf{ désire ou non se servir des matériaux }qu'il a à sa disposition
sur le plan perceptif. (...)Il est avant tout un test psychologique
et} les données psychologiques vont permettre d'établir un\emph{ diagnostic}
et d'orienter un mode d'action.''
\end{quote}

Est-ce vraisemblable ?
Il est paradoxal de pouvoir  entendre des sons que l'on
n'entend pas puisque l'on ne veut pas, en somme, les écouter. Parallèlement à cette
adéquation très simple qui est celle de signaler un son dès qu'il est 
entendu, en un même temps, les
réponses donnent des indices, des renseignements sur soi sans possibilité de 
contrôle intellectuel.


Cette forme d'objectivité \footnote{L'objectivité et la
  subjectivité, notions très complexes lorsque l'on parle du
  son.} dans le test - la mise en évidence des seuils d'écoute- et en
même temps cette possibilité -d'analyser par les résultats le potentiel d'écoute de
chaque patient- nous ont intrigués et poussés à pousser nos investigations. . Nous  reviendrons plus en détails dans le chapitre 3 sur ces éléments rassemblés.
Mais voyons dans l'immédiat le concept du son, caractéristiques 
indispensables pour l'analyse d'un test d'écoute.

% je ne vois pas où commence et finit la citation ici.
 

\section{Le son}

Le son possède plusieurs caractéristiques physiques. Il peut être
défini très précisément par un ensemble d'unités physiques chiffrées
: les décibels et les hertz. 
\begin{itemize}
\item Un décibel est l'unité de mesure de l'intensité du son. Un décibel
est égal à 1/10 de bel ; une augmentation de l'intensité égale à 1
bel équivaut à peu près à un doublement de l'intensité sonore. 
\item Un hertz est une unité de fréquence\footnote{la fréquence est le nombre de vibrations par unité de temps dans un
phénomène périodique} (symbole : Hz). Équivalent à 1 s-1. Fréquence d'un phénomène périodique
dont la période est une seconde. Ses multiples sont, entre autres,
le kilohertz (kHz), le mégahertz (MHz) et le gigahertz (Ghz). Cette
unité vient du savant allemand Heinrich Hertz, pionnier de la radioélectricité.
\end{itemize}
Le son peut être défini de deux manières : 
\begin{itemize}
\item d'une manière objective tout d'abord : c'est le phénomène physique
d'origine mécanique consistant en une variation de pression (très
faible), de vitesse vibratoire ou de densité du fluide, qui se propage
en modifiant progressivement l'état de chaque élément du milieu considéré,
donnant ainsi naissance à une onde acoustique (la propagation des
ronds dans l'eau suite à un ébranlement de la surface donne une bonne
représentation de ce phénomène) ; 
\item d'une manière subjective également : il s'agit de la sensation procurée
par cette onde, qui est reçue par l'oreille, puis transmise au cerveau
et déchiffrée par celui-ci.\index{http://www.futura-sciences.com/sante/dossiers/medecine-bruit-effets-sante-259/page/3}
\end{itemize}
De plus, il y a de nombreux paramètres en prendre en compte : par
exemple : l'impression de force sonore : la sensibilité de l'oreille
est une variable de la fréquence. Il faut 1000 fois moins de pression
acoustique pour avoir une sensation auditive à 4000 hertz qu'à 50
hertz. Notre oreille n'a donc pas la même sensibilité pour toutes
les fréquences audibles. Il en est de même pour la sensation auditive
des basses fréquences et pour la dynamique. 

\begin{quote}
``(...) Entendre n'implique pas pour autant la présence d'un champ
conscient.\emph{ Entendre, c\textquoteright est en quelque sorte subir
un son }ou un message qui nous est adressé. \emph{Ecouter, c'est désirer
appréhender ce son }ou ce message . (...)'' 

\footnote{Professeur Tomatis \char`\"{}Education et Dyslexie\char`\"{} Editions
ESF Collection \char`\"{}Sciences de l'éducation\char`\"{}.}
\end{quote}
