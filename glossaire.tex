\newglossaryentry{computer}
{
  name=computer,
  description={is a programmable machine that receives input,
               stores and manipulates data, and provides
               output in a useful format}
}

\newglossaryentry{Rorschach}
{
  name=musicothérapie,
  description={test projectif}
}

\newacronym{gcd}{GCD}{Greatest Common Divisor}

\newglossaryentry{réticulée}
{
  name=formation réticulée,
  description={la formation 
réticulée est la partie centrale de la substance grise du tronc cérébral, 
constituée de nombreuses cellules nerveuses qui communiquent entre elles par de 
multiples jonctions appelées synapses.}
}

\newglossaryentry{rémanence}
{
  name=rémanence,
  description={persistance partielle d'un phénomène après disparition
  de sa cause; spécialement de l'aimantation après retrait de
  l'influence magnétique; rémanence ou persistance des images
  visuelles, auditives, phénomènes sur lesquels sont fondés le cinéma
  et l'audition; l'hystérésis: du grec ``usterein= être en retard'':
  c'est un retard de l'effet sur la cause dans le comportement des
  corps soumis à une action (électrique ou magnétique) croissante ou
  décroissante; on parle de cycle d'hystérésis (phys.)}
}
	
\newglossaryentry{PNI}
 {
 name= PNI,
  description={ étudie l’impact des événements psychiques sur le système immunitaire. Elle repose sur la mise en évidence d’interrelations entre le système nerveux central, le système neuroendo- crinien et le système immunitaire. C’est une approche interdisciplinaire incorporant des données de la psychologie, de la neuroscience, de la neurologie, dont l’endocrinologie et l’immunologie.}
}
	